    
    
    
    

    

    \hypertarget{opuxe9rations-logiques}{%
\section{Opérations logiques}\label{opuxe9rations-logiques}}

    \hypertarget{compluxe9ment---niveau-basique}{%
\subsection{Complément - niveau
basique}\label{compluxe9ment---niveau-basique}}

    Même si les tableaux contiennent habituellement des nombres, on peut
être amenés à faire des opérations logiques et du coup à manipuler des
tableaux de booléens. Nous allons voir quelques éléments à ce sujet.

    \begin{Verbatim}[commandchars=\\\{\}]
{\color{incolor}In [{\color{incolor}1}]:} \PY{k+kn}{import} \PY{n+nn}{numpy} \PY{k}{as} \PY{n+nn}{np}
\end{Verbatim}


    \hypertarget{opuxe9rations-logiques}{%
\subsubsection{Opérations logiques}\label{opuxe9rations-logiques}}

    On peut faire des opérations logiques entre tableaux exactement comme on
fait des opérations arithmétiques.

    On va partir de deux tableaux presque identiques. J'en profite pour vous
signaler qu'on peut copier un tableau avec, tout simplement,
\texttt{np.copy}~:

    \begin{Verbatim}[commandchars=\\\{\}]
{\color{incolor}In [{\color{incolor}2}]:} \PY{n}{a} \PY{o}{=} \PY{n}{np}\PY{o}{.}\PY{n}{arange}\PY{p}{(}\PY{l+m+mi}{25}\PY{p}{)}\PY{o}{.}\PY{n}{reshape}\PY{p}{(}\PY{l+m+mi}{5}\PY{p}{,} \PY{l+m+mi}{5}\PY{p}{)}
        \PY{n+nb}{print}\PY{p}{(}\PY{n}{a}\PY{p}{)}
\end{Verbatim}


    \begin{Verbatim}[commandchars=\\\{\}]
[[ 0  1  2  3  4]
 [ 5  6  7  8  9]
 [10 11 12 13 14]
 [15 16 17 18 19]
 [20 21 22 23 24]]

    \end{Verbatim}

    \begin{Verbatim}[commandchars=\\\{\}]
{\color{incolor}In [{\color{incolor}3}]:} \PY{n}{b} \PY{o}{=} \PY{n}{np}\PY{o}{.}\PY{n}{copy}\PY{p}{(}\PY{n}{a}\PY{p}{)}
        \PY{n}{b}\PY{p}{[}\PY{l+m+mi}{2}\PY{p}{,} \PY{l+m+mi}{2}\PY{p}{]} \PY{o}{=} \PY{l+m+mi}{1000}
        \PY{n+nb}{print}\PY{p}{(}\PY{n}{b}\PY{p}{)}
\end{Verbatim}


    \begin{Verbatim}[commandchars=\\\{\}]
[[   0    1    2    3    4]
 [   5    6    7    8    9]
 [  10   11 1000   13   14]
 [  15   16   17   18   19]
 [  20   21   22   23   24]]

    \end{Verbatim}

    Dans la lignée de ce qu'on a vu jusqu'ici en matière de programmation
vectorielle, une opération logique va ici aussi nous retourner un
tableau de la même taille~:

    \begin{Verbatim}[commandchars=\\\{\}]
{\color{incolor}In [{\color{incolor}4}]:} \PY{c+c1}{\PYZsh{} la comparaison par == ne nous}
        \PY{c+c1}{\PYZsh{} retourne pas directement un booléen}
        \PY{c+c1}{\PYZsh{} mais un tableau de la même taille que a et b}
        \PY{n+nb}{print}\PY{p}{(}\PY{n}{a} \PY{o}{==} \PY{n}{b}\PY{p}{)}
\end{Verbatim}


    \begin{Verbatim}[commandchars=\\\{\}]
[[ True  True  True  True  True]
 [ True  True  True  True  True]
 [ True  True False  True  True]
 [ True  True  True  True  True]
 [ True  True  True  True  True]]

    \end{Verbatim}

    \hypertarget{all-et-any}{%
\subsubsection{\texorpdfstring{\texttt{all} et
\texttt{any}}{all et any}}\label{all-et-any}}

    Si votre intention est de vérifier que les deux tableaux sont
entièrement identiques, utilisez \texttt{np.all}~- et non pas le
\emph{built-in} natif \texttt{all} de Python~- qui va vérifier que tous
les éléments du tableau sont vrais~:

    \begin{Verbatim}[commandchars=\\\{\}]
{\color{incolor}In [{\color{incolor}5}]:} \PY{c+c1}{\PYZsh{} oui}
        \PY{n}{np}\PY{o}{.}\PY{n}{all}\PY{p}{(}\PY{n}{a} \PY{o}{==} \PY{n}{a}\PY{p}{)}
\end{Verbatim}


\begin{Verbatim}[commandchars=\\\{\}]
{\color{outcolor}Out[{\color{outcolor}5}]:} True
\end{Verbatim}
            
    \begin{Verbatim}[commandchars=\\\{\}]
{\color{incolor}In [{\color{incolor}6}]:} \PY{c+c1}{\PYZsh{} oui}
        \PY{n}{np}\PY{o}{.}\PY{n}{all}\PY{p}{(}\PY{n}{a} \PY{o}{==} \PY{n}{b}\PY{p}{)}
\end{Verbatim}


\begin{Verbatim}[commandchars=\\\{\}]
{\color{outcolor}Out[{\color{outcolor}6}]:} False
\end{Verbatim}
            
    \begin{Verbatim}[commandchars=\\\{\}]
{\color{incolor}In [{\color{incolor}7}]:} \PY{c+c1}{\PYZsh{} oui}
        \PY{c+c1}{\PYZsh{} on peut faire aussi bien}
        \PY{c+c1}{\PYZsh{}   np.all(x)}
        \PY{c+c1}{\PYZsh{} ou}
        \PY{c+c1}{\PYZsh{}   x.all()}
        \PY{p}{(}\PY{n}{a} \PY{o}{==} \PY{n}{a}\PY{p}{)}\PY{o}{.}\PY{n}{all}\PY{p}{(}\PY{p}{)}
\end{Verbatim}


\begin{Verbatim}[commandchars=\\\{\}]
{\color{outcolor}Out[{\color{outcolor}7}]:} True
\end{Verbatim}
            
    \begin{Verbatim}[commandchars=\\\{\}]
{\color{incolor}In [{\color{incolor}8}]:} \PY{c+c1}{\PYZsh{} par contre : non !}
        \PY{c+c1}{\PYZsh{} ceci n\PYZsq{}est pas conseillé}
        \PY{c+c1}{\PYZsh{} même si ça peut parfois fonctionner}
        \PY{k}{try}\PY{p}{:}
            \PY{n+nb}{all}\PY{p}{(}\PY{n}{a} \PY{o}{==} \PY{n}{a}\PY{p}{)}
        \PY{k}{except} \PY{n+ne}{Exception} \PY{k}{as} \PY{n}{e}\PY{p}{:}
            \PY{n+nb}{print}\PY{p}{(}\PY{n}{f}\PY{l+s+s1}{\PYZsq{}}\PY{l+s+s1}{OOPS }\PY{l+s+s1}{\PYZob{}}\PY{l+s+s1}{type(e)\PYZcb{} }\PY{l+s+si}{\PYZob{}e\PYZcb{}}\PY{l+s+s1}{\PYZsq{}}\PY{p}{)}
\end{Verbatim}


    \begin{Verbatim}[commandchars=\\\{\}]
OOPS <class 'ValueError'> The truth value of an array with more than one element is ambiguous. Use a.any() or a.all()

    \end{Verbatim}

    C'est bien sûr la même chose pour \texttt{any} qui va vérifier qu'il y a
au moins un élément vrai. Comme en Python natif, un nombre qui est nul
est considéré comme faux~:

    \begin{Verbatim}[commandchars=\\\{\}]
{\color{incolor}In [{\color{incolor}9}]:} \PY{n}{np}\PY{o}{.}\PY{n}{zeros}\PY{p}{(}\PY{l+m+mi}{5}\PY{p}{)}\PY{o}{.}\PY{n}{any}\PY{p}{(}\PY{p}{)}
\end{Verbatim}


\begin{Verbatim}[commandchars=\\\{\}]
{\color{outcolor}Out[{\color{outcolor}9}]:} False
\end{Verbatim}
            
    \begin{Verbatim}[commandchars=\\\{\}]
{\color{incolor}In [{\color{incolor}10}]:} \PY{n}{np}\PY{o}{.}\PY{n}{ones}\PY{p}{(}\PY{l+m+mi}{5}\PY{p}{)}\PY{o}{.}\PY{n}{any}\PY{p}{(}\PY{p}{)}
\end{Verbatim}


\begin{Verbatim}[commandchars=\\\{\}]
{\color{outcolor}Out[{\color{outcolor}10}]:} True
\end{Verbatim}
            
    \hypertarget{masques}{%
\subsubsection{Masques}\label{masques}}

    Mais en général, c'est rare qu'on ait besoin de consolider de la sorte
un booléen sur tout un tableau, on utilise plutôt les tableaux logiques
comme des masques, pour faire ou non des opérations sur un autre
tableau.

    J'en profite pour introduire une fonction de \texttt{matplotlib} qui
s'appelle \texttt{imshow} et qui permet d'afficher une image~:

    \begin{Verbatim}[commandchars=\\\{\}]
{\color{incolor}In [{\color{incolor}11}]:} \PY{k+kn}{import} \PY{n+nn}{matplotlib}\PY{n+nn}{.}\PY{n+nn}{pyplot} \PY{k}{as} \PY{n+nn}{plt}
         \PY{o}{\PYZpc{}}\PY{k}{matplotlib} inline
         \PY{n}{plt}\PY{o}{.}\PY{n}{ion}\PY{p}{(}\PY{p}{)}
\end{Verbatim}


    \begin{Verbatim}[commandchars=\\\{\}]
{\color{incolor}In [{\color{incolor}12}]:} \PY{c+c1}{\PYZsh{} construisons un disque centré au milieu de l\PYZsq{}image}
         
         \PY{n}{width} \PY{o}{=} \PY{l+m+mi}{128}
         \PY{n}{center} \PY{o}{=} \PY{n}{width} \PY{o}{/} \PY{l+m+mi}{2}
         
         \PY{n}{ix}\PY{p}{,} \PY{n}{iy} \PY{o}{=} \PY{n}{np}\PY{o}{.}\PY{n}{indices}\PY{p}{(}\PY{p}{(}\PY{n}{width}\PY{p}{,} \PY{n}{width}\PY{p}{)}\PY{p}{)}
         \PY{n}{image} \PY{o}{=} \PY{p}{(}\PY{n}{ix}\PY{o}{\PYZhy{}}\PY{n}{center}\PY{p}{)}\PY{o}{*}\PY{o}{*}\PY{l+m+mi}{2} \PY{o}{+} \PY{p}{(}\PY{n}{iy}\PY{o}{\PYZhy{}}\PY{n}{center}\PY{p}{)}\PY{o}{*}\PY{o}{*}\PY{l+m+mi}{2}
         \PY{c+c1}{\PYZsh{} pour afficher l\PYZsq{}image en niveaux de gris}
         \PY{n}{plt}\PY{o}{.}\PY{n}{imshow}\PY{p}{(}\PY{n}{image}\PY{p}{,} \PY{n}{cmap}\PY{o}{=}\PY{l+s+s1}{\PYZsq{}}\PY{l+s+s1}{gray}\PY{l+s+s1}{\PYZsq{}}\PY{p}{)}\PY{p}{;}
\end{Verbatim}


    \begin{center}
    \adjustimage{max size={0.9\linewidth}{0.9\paperheight}}{w7-s05-c3-operations-logiques_files/w7-s05-c3-operations-logiques_25_0.png}
    \end{center}
    { \hspace*{\fill} \\}
    
    Maintenant je peux créer un masque qui produise des rayures en
diagonale, donc selon la valeur de \texttt{(i+j)}. Par exemple~:

    \begin{Verbatim}[commandchars=\\\{\}]
{\color{incolor}In [{\color{incolor}13}]:} \PY{c+c1}{\PYZsh{} pour faire des rayures}
         \PY{c+c1}{\PYZsh{} de 6 pixels de large}
         \PY{n}{rayures} \PY{o}{=} \PY{p}{(}\PY{n}{ix} \PY{o}{+} \PY{n}{iy}\PY{p}{)} \PY{o}{\PYZpc{}} \PY{l+m+mi}{8} \PY{o}{\PYZlt{}}\PY{o}{=} \PY{l+m+mi}{5}
         \PY{n}{plt}\PY{o}{.}\PY{n}{imshow}\PY{p}{(}\PY{n}{rayures}\PY{p}{,} \PY{n}{cmap}\PY{o}{=}\PY{l+s+s1}{\PYZsq{}}\PY{l+s+s1}{gray}\PY{l+s+s1}{\PYZsq{}}\PY{p}{)}\PY{p}{;}
\end{Verbatim}


    \begin{center}
    \adjustimage{max size={0.9\linewidth}{0.9\paperheight}}{w7-s05-c3-operations-logiques_files/w7-s05-c3-operations-logiques_27_0.png}
    \end{center}
    { \hspace*{\fill} \\}
    
    \begin{Verbatim}[commandchars=\\\{\}]
{\color{incolor}In [{\color{incolor}14}]:} \PY{c+c1}{\PYZsh{} en fait c\PYZsq{}est bien sûr}
         \PY{c+c1}{\PYZsh{} un tableau de booléens}
         \PY{n+nb}{print}\PY{p}{(}\PY{n}{rayures}\PY{p}{)}
\end{Verbatim}


    \begin{Verbatim}[commandchars=\\\{\}]
[[ True  True  True {\ldots}  True False False]
 [ True  True  True {\ldots} False False  True]
 [ True  True  True {\ldots} False  True  True]
 {\ldots}
 [ True False False {\ldots}  True  True  True]
 [False False  True {\ldots}  True  True  True]
 [False  True  True {\ldots}  True  True False]]

    \end{Verbatim}

    je vous montre aussi comment inverser un masque parce que c'est un peu
abscons~:

    \begin{Verbatim}[commandchars=\\\{\}]
{\color{incolor}In [{\color{incolor}15}]:} \PY{c+c1}{\PYZsh{} on ne peut pas faire }
         \PY{k}{try}\PY{p}{:}
             \PY{n}{anti\PYZus{}rayures} \PY{o}{=} \PY{o+ow}{not} \PY{n}{rayures}
         \PY{k}{except} \PY{n+ne}{Exception} \PY{k}{as} \PY{n}{e}\PY{p}{:}
             \PY{n+nb}{print}\PY{p}{(}\PY{n}{f}\PY{l+s+s2}{\PYZdq{}}\PY{l+s+s2}{OOPS \PYZhy{} }\PY{l+s+s2}{\PYZob{}}\PY{l+s+s2}{type(e)\PYZcb{} \PYZhy{} }\PY{l+s+si}{\PYZob{}e\PYZcb{}}\PY{l+s+s2}{\PYZdq{}}\PY{p}{)}
\end{Verbatim}


    \begin{Verbatim}[commandchars=\\\{\}]
OOPS - <class 'ValueError'> - The truth value of an array with more than one element is ambiguous. Use a.any() or a.all()

    \end{Verbatim}

    \begin{Verbatim}[commandchars=\\\{\}]
{\color{incolor}In [{\color{incolor}16}]:} \PY{c+c1}{\PYZsh{} on ne peut pas non plus faire}
         \PY{c+c1}{\PYZsh{}   rayures.not() }
         \PY{c+c1}{\PYZsh{}   parce not est un mot clé}
         \PY{c+c1}{\PYZsh{} et on ne peut pas non plus faire}
         \PY{c+c1}{\PYZsh{}   rayures.logical\PYZus{}not() }
         \PY{c+c1}{\PYZsh{}   et ça c\PYZsq{}est plutôt un défaut}
         
         \PY{n}{anti\PYZus{}rayures} \PY{o}{=} \PY{n}{np}\PY{o}{.}\PY{n}{logical\PYZus{}not}\PY{p}{(}\PY{n}{rayures}\PY{p}{)}
         \PY{n}{plt}\PY{o}{.}\PY{n}{imshow}\PY{p}{(}\PY{n}{anti\PYZus{}rayures}\PY{p}{,} 
                    \PY{n}{cmap}\PY{o}{=}\PY{l+s+s1}{\PYZsq{}}\PY{l+s+s1}{gray}\PY{l+s+s1}{\PYZsq{}}\PY{p}{)}\PY{p}{;}
\end{Verbatim}


    \begin{center}
    \adjustimage{max size={0.9\linewidth}{0.9\paperheight}}{w7-s05-c3-operations-logiques_files/w7-s05-c3-operations-logiques_31_0.png}
    \end{center}
    { \hspace*{\fill} \\}
    
    \begin{Verbatim}[commandchars=\\\{\}]
{\color{incolor}In [{\color{incolor}17}]:} \PY{c+c1}{\PYZsh{} lorsque vous avez de vrais }
         \PY{c+c1}{\PYZsh{} booléens, vous pouvez}
         \PY{c+c1}{\PYZsh{} utiliser l\PYZsq{}opérateur \PYZti{} }
         \PY{c+c1}{\PYZsh{} qui fait un not bitwise}
         \PY{n}{anti\PYZus{}rayures} \PY{o}{=} \PY{o}{\PYZti{}}\PY{n}{rayures}
         \PY{n}{plt}\PY{o}{.}\PY{n}{imshow}\PY{p}{(}\PY{n}{anti\PYZus{}rayures}\PY{p}{,}
                    \PY{n}{cmap}\PY{o}{=}\PY{l+s+s1}{\PYZsq{}}\PY{l+s+s1}{gray}\PY{l+s+s1}{\PYZsq{}}\PY{p}{)}\PY{p}{;}
\end{Verbatim}


    \begin{center}
    \adjustimage{max size={0.9\linewidth}{0.9\paperheight}}{w7-s05-c3-operations-logiques_files/w7-s05-c3-operations-logiques_32_0.png}
    \end{center}
    { \hspace*{\fill} \\}
    
    Maintenant je peux utiliser le masque \texttt{rayures} pour faire des
choses sur l'image. Par exemple simplement~:

    \begin{Verbatim}[commandchars=\\\{\}]
{\color{incolor}In [{\color{incolor}18}]:} \PY{c+c1}{\PYZsh{} pour effacer les rayures}
         \PY{n}{plt}\PY{o}{.}\PY{n}{imshow}\PY{p}{(}\PY{n}{image}\PY{o}{*}\PY{n}{rayures}\PY{p}{,} \PY{n}{cmap}\PY{o}{=}\PY{l+s+s1}{\PYZsq{}}\PY{l+s+s1}{gray}\PY{l+s+s1}{\PYZsq{}}\PY{p}{)}\PY{p}{;}
\end{Verbatim}


    \begin{center}
    \adjustimage{max size={0.9\linewidth}{0.9\paperheight}}{w7-s05-c3-operations-logiques_files/w7-s05-c3-operations-logiques_34_0.png}
    \end{center}
    { \hspace*{\fill} \\}
    
    \begin{Verbatim}[commandchars=\\\{\}]
{\color{incolor}In [{\color{incolor}19}]:} \PY{c+c1}{\PYZsh{} ou garder l\PYZsq{}autre moitié}
         \PY{n}{plt}\PY{o}{.}\PY{n}{imshow}\PY{p}{(}\PY{n}{image}\PY{o}{*}\PY{n}{anti\PYZus{}rayures}\PY{p}{,} \PY{n}{cmap}\PY{o}{=}\PY{l+s+s1}{\PYZsq{}}\PY{l+s+s1}{gray}\PY{l+s+s1}{\PYZsq{}}\PY{p}{)}\PY{p}{;}
\end{Verbatim}


    \begin{center}
    \adjustimage{max size={0.9\linewidth}{0.9\paperheight}}{w7-s05-c3-operations-logiques_files/w7-s05-c3-operations-logiques_35_0.png}
    \end{center}
    { \hspace*{\fill} \\}
    
    \begin{Verbatim}[commandchars=\\\{\}]
{\color{incolor}In [{\color{incolor}20}]:} \PY{n}{image}
\end{Verbatim}


\begin{Verbatim}[commandchars=\\\{\}]
{\color{outcolor}Out[{\color{outcolor}20}]:} array([[8192., 8065., 7940., {\ldots}, 7817., 7940., 8065.],
                [8065., 7938., 7813., {\ldots}, 7690., 7813., 7938.],
                [7940., 7813., 7688., {\ldots}, 7565., 7688., 7813.],
                {\ldots},
                [7817., 7690., 7565., {\ldots}, 7442., 7565., 7690.],
                [7940., 7813., 7688., {\ldots}, 7565., 7688., 7813.],
                [8065., 7938., 7813., {\ldots}, 7690., 7813., 7938.]])
\end{Verbatim}
            
    \begin{Verbatim}[commandchars=\\\{\}]
{\color{incolor}In [{\color{incolor}21}]:} \PY{n}{np}\PY{o}{.}\PY{n}{logical\PYZus{}not}\PY{p}{(}\PY{n}{image}\PY{p}{)}
\end{Verbatim}


\begin{Verbatim}[commandchars=\\\{\}]
{\color{outcolor}Out[{\color{outcolor}21}]:} array([[False, False, False, {\ldots}, False, False, False],
                [False, False, False, {\ldots}, False, False, False],
                [False, False, False, {\ldots}, False, False, False],
                {\ldots},
                [False, False, False, {\ldots}, False, False, False],
                [False, False, False, {\ldots}, False, False, False],
                [False, False, False, {\ldots}, False, False, False]])
\end{Verbatim}
            
    \hypertarget{expression-conditionnelle-et-np.where}{%
\subsubsection{\texorpdfstring{Expression conditionnelle et
\texttt{np.where}}{Expression conditionnelle et np.where}}\label{expression-conditionnelle-et-np.where}}

    Nous avons vu en Python natif l'expression conditionnelle~:

    \begin{Verbatim}[commandchars=\\\{\}]
{\color{incolor}In [{\color{incolor}22}]:} \PY{l+m+mi}{3} \PY{k}{if} \PY{k+kc}{True} \PY{k}{else} \PY{l+m+mi}{2}
\end{Verbatim}


\begin{Verbatim}[commandchars=\\\{\}]
{\color{outcolor}Out[{\color{outcolor}22}]:} 3
\end{Verbatim}
            
    Pour reproduire cette construction en \texttt{numpy} vous avez à votre
disposition \texttt{np.where}. Pour l'illustrer nous allons construire
deux images facilement discernables. Et, pour cela, on va utiliser
\texttt{np.isclose}, qui est très utile pour comparer que deux nombres
sont suffisamment proches, surtout pour les calculs flottants en fait,
mais ça nous convient très bien ici aussi~:

    \begin{Verbatim}[commandchars=\\\{\}]
{\color{incolor}In [{\color{incolor}23}]:} np.isclose\PY{o}{?}
\end{Verbatim}


    Pour élaborer une image qui contient un grand cercle, je vais dire que
la distance au centre (je rappelle que c'est le contenu de
\texttt{image}) est suffisamment proche de \(64^2\), ce que vaut
\texttt{image} au milieu de chaque bord~:

    \begin{Verbatim}[commandchars=\\\{\}]
{\color{incolor}In [{\color{incolor}24}]:} \PY{n}{big\PYZus{}circle} \PY{o}{=} \PY{n}{np}\PY{o}{.}\PY{n}{isclose}\PY{p}{(}\PY{n}{image}\PY{p}{,} \PY{l+m+mi}{64} \PY{o}{*}\PY{o}{*}\PY{l+m+mi}{2}\PY{p}{,} \PY{l+m+mi}{10}\PY{o}{/}\PY{l+m+mi}{100}\PY{p}{)}
         \PY{n}{plt}\PY{o}{.}\PY{n}{imshow}\PY{p}{(}\PY{n}{big\PYZus{}circle}\PY{p}{,} \PY{n}{cmap}\PY{o}{=}\PY{l+s+s1}{\PYZsq{}}\PY{l+s+s1}{gray}\PY{l+s+s1}{\PYZsq{}}\PY{p}{)}\PY{p}{;}
\end{Verbatim}


    \begin{center}
    \adjustimage{max size={0.9\linewidth}{0.9\paperheight}}{w7-s05-c3-operations-logiques_files/w7-s05-c3-operations-logiques_44_0.png}
    \end{center}
    { \hspace*{\fill} \\}
    
    \begin{Verbatim}[commandchars=\\\{\}]
{\color{incolor}In [{\color{incolor}25}]:} \PY{n}{small\PYZus{}circle} \PY{o}{=} \PY{n}{np}\PY{o}{.}\PY{n}{isclose}\PY{p}{(}\PY{n}{image}\PY{p}{,} \PY{l+m+mi}{32} \PY{o}{*}\PY{o}{*}\PY{l+m+mi}{2}\PY{p}{,} \PY{l+m+mi}{10}\PY{o}{/}\PY{l+m+mi}{100}\PY{p}{)}
         \PY{n}{plt}\PY{o}{.}\PY{n}{imshow}\PY{p}{(}\PY{n}{small\PYZus{}circle}\PY{p}{,} \PY{n}{cmap}\PY{o}{=}\PY{l+s+s1}{\PYZsq{}}\PY{l+s+s1}{gray}\PY{l+s+s1}{\PYZsq{}}\PY{p}{)}\PY{p}{;}
\end{Verbatim}


    \begin{center}
    \adjustimage{max size={0.9\linewidth}{0.9\paperheight}}{w7-s05-c3-operations-logiques_files/w7-s05-c3-operations-logiques_45_0.png}
    \end{center}
    { \hspace*{\fill} \\}
    
    En utilisant \texttt{np.where}, je peux simuler quelque chose comme
ceci~:

\begin{Shaded}
\begin{Highlighting}[]
\NormalTok{mixed }\OperatorTok{=}\NormalTok{ big_circle }\ControlFlowTok{if}\NormalTok{ rayures }\ControlFlowTok{else}\NormalTok{ small_circle}
\end{Highlighting}
\end{Shaded}

    \begin{Verbatim}[commandchars=\\\{\}]
{\color{incolor}In [{\color{incolor}26}]:} \PY{c+c1}{\PYZsh{} sauf que ça se présente en fait comme ceci :}
         \PY{n}{mixed} \PY{o}{=} \PY{n}{np}\PY{o}{.}\PY{n}{where}\PY{p}{(}\PY{n}{rayures}\PY{p}{,} \PY{n}{big\PYZus{}circle}\PY{p}{,} \PY{n}{small\PYZus{}circle}\PY{p}{)}
         \PY{n}{plt}\PY{o}{.}\PY{n}{imshow}\PY{p}{(}\PY{n}{mixed}\PY{p}{,} \PY{n}{cmap}\PY{o}{=}\PY{l+s+s1}{\PYZsq{}}\PY{l+s+s1}{gray}\PY{l+s+s1}{\PYZsq{}}\PY{p}{)}\PY{p}{;}
\end{Verbatim}


    \begin{center}
    \adjustimage{max size={0.9\linewidth}{0.9\paperheight}}{w7-s05-c3-operations-logiques_files/w7-s05-c3-operations-logiques_47_0.png}
    \end{center}
    { \hspace*{\fill} \\}
    
    Remarquez enfin qu'on peut aussi faire la même chose en tirant profit
que \texttt{True} == 1 et \texttt{False} == 0~:

    \begin{Verbatim}[commandchars=\\\{\}]
{\color{incolor}In [{\color{incolor}27}]:} \PY{n}{mixed2} \PY{o}{=} \PY{n}{rayures} \PY{o}{*} \PY{n}{big\PYZus{}circle} \PY{o}{+} \PY{p}{(}\PY{l+m+mi}{1}\PY{o}{\PYZhy{}}\PY{n}{rayures}\PY{p}{)} \PY{o}{*} \PY{n}{small\PYZus{}circle}
         \PY{n}{plt}\PY{o}{.}\PY{n}{imshow}\PY{p}{(}\PY{n}{mixed2}\PY{p}{,} \PY{n}{cmap}\PY{o}{=}\PY{l+s+s1}{\PYZsq{}}\PY{l+s+s1}{gray}\PY{l+s+s1}{\PYZsq{}}\PY{p}{)}\PY{p}{;}
\end{Verbatim}


    \begin{center}
    \adjustimage{max size={0.9\linewidth}{0.9\paperheight}}{w7-s05-c3-operations-logiques_files/w7-s05-c3-operations-logiques_49_0.png}
    \end{center}
    { \hspace*{\fill} \\}
    

    % Add a bibliography block to the postdoc
    
    
    
