    
    
    
    

    

    \hypertarget{objets-mutables-et-objets-immuables}{%
\section{Objets mutables et objets
immuables}\label{objets-mutables-et-objets-immuables}}

    \hypertarget{compluxe9ment---niveau-basique}{%
\subsection{Complément - niveau
basique}\label{compluxe9ment---niveau-basique}}

    \hypertarget{les-chauxeenes-sont-des-objets-immuables}{%
\subsubsection{Les chaînes sont des objets
immuables}\label{les-chauxeenes-sont-des-objets-immuables}}

    Voici un exemple d'un fragment de code qui illustre le caractère
immuable des chaînes de caractères. Nous l'exécutons sous
\href{pythontutor.com}{pythontutor}, afin de bien illustrer les
relations entre variables et objets.

    \begin{Verbatim}[commandchars=\\\{\},frame=single,framerule=0.3mm,rulecolor=\color{cellframecolor}]
{\color{incolor}In [{\color{incolor}1}]:} \PY{c+c1}{\PYZsh{} il vous faut charger cette cellule}
        \PY{c+c1}{\PYZsh{} pour pouvoir utiliser les suivantes}
        \PY{o}{\PYZpc{}}\PY{k}{load\PYZus{}ext} ipythontutor
\end{Verbatim}


    \textbf{Note}~: une fois que vous avez évalué la cellule avec
\texttt{\%\%ipythontutor}, vous devez cliquer sur le bouton
\texttt{Forward} pour voir pas à pas le comportement du programme.

    Le scénario est très simple, on crée deux variables \texttt{s1} et
\texttt{s2} vers le même objet
\texttt{\textquotesingle{}abc\textquotesingle{}}, puis on fait une
opération \texttt{+=} sur la variable \texttt{s1}.

Comme l'objet est une chaîne, il est donc immuable, on ne \textbf{peut
pas modifier l'objet} directement~; pour obtenir l'effet recherché (à
savoir que \texttt{s1} s'allonge de
\texttt{\textquotesingle{}def\textquotesingle{}}), Python \textbf{crée
un deuxième objet}, comme on le voit bien sous pythontutor~:

    \begin{Verbatim}[commandchars=\\\{\},frame=single,framerule=0.3mm,rulecolor=\color{cellframecolor}]
{\color{incolor}In [{\color{incolor} }]:} \PY{c+c1}{\PYZsh{} NOTE}
        \PY{c+c1}{\PYZsh{} auto\PYZhy{}exec\PYZhy{}for\PYZhy{}latex has skipped execution of this cell}
        
        \PY{o}{\PYZpc{}\PYZpc{}}\PY{k}{ipythontutor} heapPrimitives=true
        \PYZsh{} deux variables vers le même objet
        s1 = \PYZsq{}abc\PYZsq{}
        s2 = s1
        \PYZsh{} on essaie de modifier l\PYZsq{}objet
        s1 += \PYZsq{}def\PYZsq{}
        \PYZsh{} pensez à cliquer sur `Forward`
\end{Verbatim}


    \hypertarget{les-listes-sont-des-objets-mutables}{%
\subsubsection{Les listes sont des objets
mutables}\label{les-listes-sont-des-objets-mutables}}

    Voici ce qu'on obtient par contraste pour le même scénario mais qui
cette fois utilise des listes, qui sont des objets mutables~:

    \begin{Verbatim}[commandchars=\\\{\},frame=single,framerule=0.3mm,rulecolor=\color{cellframecolor}]
{\color{incolor}In [{\color{incolor} }]:} \PY{c+c1}{\PYZsh{} NOTE}
        \PY{c+c1}{\PYZsh{} auto\PYZhy{}exec\PYZhy{}for\PYZhy{}latex has skipped execution of this cell}
        
        \PY{o}{\PYZpc{}\PYZpc{}}\PY{k}{ipythontutor} heapPrimitives=true ratio=0.8
        \PYZsh{} deux variables vers le même objet
        liste1 = [\PYZsq{}a\PYZsq{}, \PYZsq{}b\PYZsq{}, \PYZsq{}c\PYZsq{}]
        liste2 = liste1
        \PYZsh{} on modifie l\PYZsq{}objet
        liste1 += [\PYZsq{}d\PYZsq{}, \PYZsq{}e\PYZsq{}, \PYZsq{}f\PYZsq{}]
        \PYZsh{} pensez à cliquer sur `Forward`
\end{Verbatim}


    \hypertarget{conclusion}{%
\subsubsection{Conclusion}\label{conclusion}}

    Ce comportement n'est pas propre à l'usage de l'opérateur \texttt{+=} et
pour cette raison nous avons tendance à le déconseiller.

Les objets mutables et immuables ont par essence un comportement
différent, il est très important d'avoir ceci présent à l'esprit.

Nous aurons notamment l'occasion d'approfondir cela dans la séquence
consacrée aux références partagées, en semaine 3.


    % Add a bibliography block to the postdoc
    
    
    
