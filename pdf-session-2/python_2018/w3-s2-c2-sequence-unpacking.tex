    
    
    
    

    

    \hypertarget{sequence-unpacking}{%
\section{Sequence unpacking}\label{sequence-unpacking}}

    \hypertarget{compluxe9ment---niveau-basique}{%
\subsection{Complément - niveau
basique}\label{compluxe9ment---niveau-basique}}

    \textbf{Remarque préliminaire}~: nous avons vainement cherché une
traduction raisonnable pour ce trait du langage, connue en anglais sous
le nom de \emph{sequence unpacking} ou encore parfois \emph{tuple
unpacking}, aussi pour éviter de créer de la confusion nous avons
finalement décidé de conserver le terme anglais à l'identique.

    \hypertarget{duxe9juxe0-rencontruxe9}{%
\subsubsection{Déjà rencontré}\label{duxe9juxe0-rencontruxe9}}

    L'affectation dans Python peut concerner plusieurs variables à la fois.
En fait nous en avons déjà vu un exemple en Semaine 1, avec la fonction
\texttt{fibonacci} dans laquelle il y avait ce fragment~:

\begin{Shaded}
\begin{Highlighting}[]
\ControlFlowTok{for}\NormalTok{ i }\KeywordTok{in} \BuiltInTok{range}\NormalTok{(}\DecValTok{2}\NormalTok{, n }\OperatorTok{+} \DecValTok{1}\NormalTok{):}
\NormalTok{    f2, f1 }\OperatorTok{=}\NormalTok{ f1, f1 }\OperatorTok{+}\NormalTok{ f2}
\end{Highlighting}
\end{Shaded}

Nous allons dans ce complément décortiquer les mécanismes derrière cette
phrase qui a probablement excité votre curiosité. :)

    \hypertarget{un-exemple-simple}{%
\subsubsection{Un exemple simple}\label{un-exemple-simple}}

    Commençons par un exemple simple à base de tuple. Imaginons que l'on
dispose d'un tuple \texttt{couple} dont on sait qu'il a deux éléments~:

    \begin{Verbatim}[commandchars=\\\{\}]
{\color{incolor}In [{\color{incolor}1}]:} \PY{n}{couple} \PY{o}{=} \PY{p}{(}\PY{l+m+mi}{100}\PY{p}{,} \PY{l+s+s1}{\PYZsq{}}\PY{l+s+s1}{spam}\PY{l+s+s1}{\PYZsq{}}\PY{p}{)}
\end{Verbatim}


    On souhaite à présent extraire les deux valeurs, et les affecter à deux
variables distinctes. Une solution naïve consiste bien sûr à faire
simplement~:

    \begin{Verbatim}[commandchars=\\\{\}]
{\color{incolor}In [{\color{incolor}2}]:} \PY{n}{gauche} \PY{o}{=} \PY{n}{couple}\PY{p}{[}\PY{l+m+mi}{0}\PY{p}{]}
        \PY{n}{droite} \PY{o}{=} \PY{n}{couple}\PY{p}{[}\PY{l+m+mi}{1}\PY{p}{]}
        \PY{n+nb}{print}\PY{p}{(}\PY{l+s+s1}{\PYZsq{}}\PY{l+s+s1}{gauche}\PY{l+s+s1}{\PYZsq{}}\PY{p}{,} \PY{n}{gauche}\PY{p}{,} \PY{l+s+s1}{\PYZsq{}}\PY{l+s+s1}{droite}\PY{l+s+s1}{\PYZsq{}}\PY{p}{,} \PY{n}{droite}\PY{p}{)}
\end{Verbatim}


    \begin{Verbatim}[commandchars=\\\{\}]
gauche 100 droite spam

    \end{Verbatim}

    Cela fonctionne naturellement très bien, mais n'est pas très pythonique
- comme on dit ;) Vous devez toujours garder en tête qu'il est rare en
Python de manipuler des indices. Dès que vous voyez des indices dans
votre code, vous devez vous demander si votre code est pythonique.

On préfèrera la formulation équivalente suivante~:

    \begin{Verbatim}[commandchars=\\\{\}]
{\color{incolor}In [{\color{incolor}3}]:} \PY{p}{(}\PY{n}{gauche}\PY{p}{,} \PY{n}{droite}\PY{p}{)} \PY{o}{=} \PY{n}{couple}
        \PY{n+nb}{print}\PY{p}{(}\PY{l+s+s1}{\PYZsq{}}\PY{l+s+s1}{gauche}\PY{l+s+s1}{\PYZsq{}}\PY{p}{,} \PY{n}{gauche}\PY{p}{,} \PY{l+s+s1}{\PYZsq{}}\PY{l+s+s1}{droite}\PY{l+s+s1}{\PYZsq{}}\PY{p}{,} \PY{n}{droite}\PY{p}{)}
\end{Verbatim}


    \begin{Verbatim}[commandchars=\\\{\}]
gauche 100 droite spam

    \end{Verbatim}

    La logique ici consiste à dire : affecter les deux variables de sorte
que le tuple \texttt{(gauche,\ droite)} soit égal à \texttt{couple}. On
voit ici la supériorité de cette notion d'unpacking sur la manipulation
d'indices~: vous avez maintenant des variables qui expriment la nature
de l'objet manipulé, votre code devient expressif, c'est-à-dire
auto-documenté.

Remarquons que les parenthèses ici sont optionnelles - comme lorsque
l'on construit un tuple - et on peut tout aussi bien écrire, et c'est le
cas d'usage le plus fréquent d'omission des parenthèses pour le tuple~:

    \begin{Verbatim}[commandchars=\\\{\}]
{\color{incolor}In [{\color{incolor}4}]:} \PY{n}{gauche}\PY{p}{,} \PY{n}{droite} \PY{o}{=} \PY{n}{couple}
        \PY{n+nb}{print}\PY{p}{(}\PY{l+s+s1}{\PYZsq{}}\PY{l+s+s1}{gauche}\PY{l+s+s1}{\PYZsq{}}\PY{p}{,} \PY{n}{gauche}\PY{p}{,} \PY{l+s+s1}{\PYZsq{}}\PY{l+s+s1}{droite}\PY{l+s+s1}{\PYZsq{}}\PY{p}{,} \PY{n}{droite}\PY{p}{)}
\end{Verbatim}


    \begin{Verbatim}[commandchars=\\\{\}]
gauche 100 droite spam

    \end{Verbatim}

    \hypertarget{autres-types}{%
\subsubsection{Autres types}\label{autres-types}}

    Cette technique fonctionne aussi bien avec d'autres types. Par exemple,
on peut utiliser~:

\begin{itemize}
\tightlist
\item
  une syntaxe de liste à gauche du \texttt{=}~;
\item
  une liste comme expression à droite du \texttt{=}.
\end{itemize}

    \begin{Verbatim}[commandchars=\\\{\}]
{\color{incolor}In [{\color{incolor}5}]:} \PY{c+c1}{\PYZsh{} comme ceci}
        \PY{n}{liste} \PY{o}{=} \PY{p}{[}\PY{l+m+mi}{1}\PY{p}{,} \PY{l+m+mi}{2}\PY{p}{,} \PY{l+m+mi}{3}\PY{p}{]}
        \PY{p}{[}\PY{n}{gauche}\PY{p}{,} \PY{n}{milieu}\PY{p}{,} \PY{n}{droit}\PY{p}{]} \PY{o}{=} \PY{n}{liste}
        \PY{n+nb}{print}\PY{p}{(}\PY{l+s+s1}{\PYZsq{}}\PY{l+s+s1}{gauche}\PY{l+s+s1}{\PYZsq{}}\PY{p}{,} \PY{n}{gauche}\PY{p}{,} \PY{l+s+s1}{\PYZsq{}}\PY{l+s+s1}{milieu}\PY{l+s+s1}{\PYZsq{}}\PY{p}{,} \PY{n}{milieu}\PY{p}{,} \PY{l+s+s1}{\PYZsq{}}\PY{l+s+s1}{droit}\PY{l+s+s1}{\PYZsq{}}\PY{p}{,} \PY{n}{droit}\PY{p}{)}
\end{Verbatim}


    \begin{Verbatim}[commandchars=\\\{\}]
gauche 1 milieu 2 droit 3

    \end{Verbatim}

    Et on n'est même pas obligés d'avoir le même type à gauche et à droite
du signe \texttt{=}, comme ici~:

    \begin{Verbatim}[commandchars=\\\{\}]
{\color{incolor}In [{\color{incolor}6}]:} \PY{c+c1}{\PYZsh{} membre droit: une liste}
        \PY{n}{liste} \PY{o}{=} \PY{p}{[}\PY{l+m+mi}{1}\PY{p}{,} \PY{l+m+mi}{2}\PY{p}{,} \PY{l+m+mi}{3}\PY{p}{]}
        \PY{c+c1}{\PYZsh{} membre gauche : un tuple}
        \PY{n}{gauche}\PY{p}{,} \PY{n}{milieu}\PY{p}{,} \PY{n}{droit} \PY{o}{=} \PY{n}{liste}
        \PY{n+nb}{print}\PY{p}{(}\PY{l+s+s1}{\PYZsq{}}\PY{l+s+s1}{gauche}\PY{l+s+s1}{\PYZsq{}}\PY{p}{,} \PY{n}{gauche}\PY{p}{,} \PY{l+s+s1}{\PYZsq{}}\PY{l+s+s1}{milieu}\PY{l+s+s1}{\PYZsq{}}\PY{p}{,} \PY{n}{milieu}\PY{p}{,} \PY{l+s+s1}{\PYZsq{}}\PY{l+s+s1}{droit}\PY{l+s+s1}{\PYZsq{}}\PY{p}{,} \PY{n}{droit}\PY{p}{)}
\end{Verbatim}


    \begin{Verbatim}[commandchars=\\\{\}]
gauche 1 milieu 2 droit 3

    \end{Verbatim}

    En réalité, les seules contraintes fixées par Python sont que~:

\begin{itemize}
\tightlist
\item
  le terme à droite du signe \texttt{=} soit un \emph{itérable} (tuple,
  liste, string, etc.)~;
\item
  le terme à gauche soit écrit comme un tuple ou une liste - notons tout
  de même que l'utilisation d'une liste à gauche est rare et peu
  pythonique~;
\item
  les deux termes aient la même longueur - en tout cas avec les concepts
  que l'on a vus jusqu'ici, mais voir aussi plus bas l'utilisation de
  \texttt{*arg} avec le \emph{extended unpacking}.
\end{itemize}

    La plupart du temps le terme de gauche est écrit comme un tuple. C'est
pour cette raison que les deux termes \emph{tuple unpacking} et
\emph{sequence unpacking} sont en vigueur.

    \hypertarget{la-fauxe7on-pythonique-duxe9changer-deux-variables}{%
\subsubsection{\texorpdfstring{La façon \emph{pythonique} d'échanger
deux
variables}{La façon pythonique d'échanger deux variables}}\label{la-fauxe7on-pythonique-duxe9changer-deux-variables}}

    Une caractéristique intéressante de l'affectation par \emph{sequence
unpacking} est qu'elle est sûre~; on n'a pas à se préoccuper d'un
éventuel ordre d'évaluation, les valeurs \textbf{à droite} de
l'affectation sont \textbf{toutes} évaluées en premier, et ainsi on peut
par exemple échanger deux variables comme ceci~:

    \begin{Verbatim}[commandchars=\\\{\}]
{\color{incolor}In [{\color{incolor}7}]:} \PY{n}{a} \PY{o}{=} \PY{l+m+mi}{1}
        \PY{n}{b} \PY{o}{=} \PY{l+m+mi}{2}
        \PY{n}{a}\PY{p}{,} \PY{n}{b} \PY{o}{=} \PY{n}{b}\PY{p}{,} \PY{n}{a}
        \PY{n+nb}{print}\PY{p}{(}\PY{l+s+s1}{\PYZsq{}}\PY{l+s+s1}{a}\PY{l+s+s1}{\PYZsq{}}\PY{p}{,} \PY{n}{a}\PY{p}{,} \PY{l+s+s1}{\PYZsq{}}\PY{l+s+s1}{b}\PY{l+s+s1}{\PYZsq{}}\PY{p}{,} \PY{n}{b}\PY{p}{)}
\end{Verbatim}


    \begin{Verbatim}[commandchars=\\\{\}]
a 2 b 1

    \end{Verbatim}

    \hypertarget{extended-unpacking}{%
\subsubsection{\texorpdfstring{\emph{Extended
unpacking}}{Extended unpacking}}\label{extended-unpacking}}

    Le \emph{extended unpacking} a été introduit en Python 3~; commençons
par en voir un exemple~:

    \begin{Verbatim}[commandchars=\\\{\}]
{\color{incolor}In [{\color{incolor}8}]:} \PY{n}{reference} \PY{o}{=} \PY{p}{[}\PY{l+m+mi}{1}\PY{p}{,} \PY{l+m+mi}{2}\PY{p}{,} \PY{l+m+mi}{3}\PY{p}{,} \PY{l+m+mi}{4}\PY{p}{,} \PY{l+m+mi}{5}\PY{p}{]}
        \PY{n}{a}\PY{p}{,} \PY{o}{*}\PY{n}{b}\PY{p}{,} \PY{n}{c} \PY{o}{=} \PY{n}{reference}
        \PY{n+nb}{print}\PY{p}{(}\PY{n}{f}\PY{l+s+s2}{\PYZdq{}}\PY{l+s+s2}{a=}\PY{l+s+si}{\PYZob{}a\PYZcb{}}\PY{l+s+s2}{ b=}\PY{l+s+si}{\PYZob{}b\PYZcb{}}\PY{l+s+s2}{ c=}\PY{l+s+si}{\PYZob{}c\PYZcb{}}\PY{l+s+s2}{\PYZdq{}}\PY{p}{)}
\end{Verbatim}


    \begin{Verbatim}[commandchars=\\\{\}]
a=1 b=[2, 3, 4] c=5

    \end{Verbatim}

    Comme vous le voyez, le mécanisme ici est une extension de
\emph{sequence unpacking}~; Python vous autorise à mentionner
\textbf{une seule fois}, parmi les variables qui apparaissent à gauche
de l'affectation, une variable \textbf{précédée de \texttt{*}}, ici
\texttt{*b}.

Cette variable est interprétée comme une \textbf{liste de longueur
quelconque} des éléments de \texttt{reference}. On aurait donc aussi
bien pu écrire~:

    \begin{Verbatim}[commandchars=\\\{\}]
{\color{incolor}In [{\color{incolor}9}]:} \PY{n}{reference} \PY{o}{=} \PY{n+nb}{range}\PY{p}{(}\PY{l+m+mi}{20}\PY{p}{)}
        \PY{n}{a}\PY{p}{,} \PY{o}{*}\PY{n}{b}\PY{p}{,} \PY{n}{c} \PY{o}{=} \PY{n}{reference}
        \PY{n+nb}{print}\PY{p}{(}\PY{n}{f}\PY{l+s+s2}{\PYZdq{}}\PY{l+s+s2}{a=}\PY{l+s+si}{\PYZob{}a\PYZcb{}}\PY{l+s+s2}{ b=}\PY{l+s+si}{\PYZob{}b\PYZcb{}}\PY{l+s+s2}{ c=}\PY{l+s+si}{\PYZob{}c\PYZcb{}}\PY{l+s+s2}{\PYZdq{}}\PY{p}{)}
\end{Verbatim}


    \begin{Verbatim}[commandchars=\\\{\}]
a=0 b=[1, 2, 3, 4, 5, 6, 7, 8, 9, 10, 11, 12, 13, 14, 15, 16, 17, 18] c=19

    \end{Verbatim}

    Ce trait peut s'avérer pratique, lorsque par exemple on s'intéresse
seulement aux premiers éléments d'une structure~:

    \begin{Verbatim}[commandchars=\\\{\}]
{\color{incolor}In [{\color{incolor}10}]:} \PY{c+c1}{\PYZsh{} si on sait que data contient prenom, nom, }
         \PY{c+c1}{\PYZsh{} et un nombre inconnu d\PYZsq{}autres informations}
         \PY{n}{data} \PY{o}{=} \PY{p}{[} \PY{l+s+s1}{\PYZsq{}}\PY{l+s+s1}{Jean}\PY{l+s+s1}{\PYZsq{}}\PY{p}{,} \PY{l+s+s1}{\PYZsq{}}\PY{l+s+s1}{Dupont}\PY{l+s+s1}{\PYZsq{}}\PY{p}{,} \PY{l+s+s1}{\PYZsq{}}\PY{l+s+s1}{061234567}\PY{l+s+s1}{\PYZsq{}}\PY{p}{,} \PY{l+s+s1}{\PYZsq{}}\PY{l+s+s1}{12}\PY{l+s+s1}{\PYZsq{}}\PY{p}{,} \PY{l+s+s1}{\PYZsq{}}\PY{l+s+s1}{rue du four}\PY{l+s+s1}{\PYZsq{}}\PY{p}{,} \PY{l+s+s1}{\PYZsq{}}\PY{l+s+s1}{57000}\PY{l+s+s1}{\PYZsq{}}\PY{p}{,} \PY{l+s+s1}{\PYZsq{}}\PY{l+s+s1}{METZ}\PY{l+s+s1}{\PYZsq{}}\PY{p}{,} \PY{p}{]}
         
         \PY{c+c1}{\PYZsh{} on peut utiliser la variable \PYZus{}}
         \PY{c+c1}{\PYZsh{} ce n\PYZsq{}est pas une variable spéciale dans le langage,}
         \PY{c+c1}{\PYZsh{} mais cela indique au lecteur que l\PYZsq{}on ne va pas s\PYZsq{}en servir}
         \PY{n}{prenom}\PY{p}{,} \PY{n}{nom}\PY{p}{,} \PY{o}{*}\PY{n}{\PYZus{}} \PY{o}{=} \PY{n}{data}
         \PY{n+nb}{print}\PY{p}{(}\PY{n}{f}\PY{l+s+s2}{\PYZdq{}}\PY{l+s+s2}{prenom=}\PY{l+s+si}{\PYZob{}prenom\PYZcb{}}\PY{l+s+s2}{ nom=}\PY{l+s+si}{\PYZob{}nom\PYZcb{}}\PY{l+s+s2}{\PYZdq{}}\PY{p}{)}
\end{Verbatim}


    \begin{Verbatim}[commandchars=\\\{\}]
prenom=Jean nom=Dupont

    \end{Verbatim}

    \hypertarget{compluxe9ment---niveau-intermuxe9diaire}{%
\subsection{Complément - niveau
intermédiaire}\label{compluxe9ment---niveau-intermuxe9diaire}}

    On a vu les principaux cas d'utilisation de la \emph{sequence
unpacking}, voyons à présent quelques subtilités.

    \hypertarget{plusieurs-occurrences-dune-muxeame-variable}{%
\subsubsection{Plusieurs occurrences d'une même
variable}\label{plusieurs-occurrences-dune-muxeame-variable}}

    On peut utiliser \textbf{plusieurs fois} la même variable dans la partie
gauche de l'affectation~:

    \begin{Verbatim}[commandchars=\\\{\}]
{\color{incolor}In [{\color{incolor}11}]:} \PY{c+c1}{\PYZsh{} ceci en toute rigueur est légal}
         \PY{c+c1}{\PYZsh{} mais en pratique on évite de le faire}
         \PY{n}{entree} \PY{o}{=} \PY{p}{[}\PY{l+m+mi}{1}\PY{p}{,} \PY{l+m+mi}{2}\PY{p}{,} \PY{l+m+mi}{3}\PY{p}{]}
         \PY{n}{a}\PY{p}{,} \PY{n}{a}\PY{p}{,} \PY{n}{a} \PY{o}{=} \PY{n}{entree}
         \PY{n+nb}{print}\PY{p}{(}\PY{n}{f}\PY{l+s+s2}{\PYZdq{}}\PY{l+s+s2}{a = }\PY{l+s+si}{\PYZob{}a\PYZcb{}}\PY{l+s+s2}{\PYZdq{}}\PY{p}{)}
\end{Verbatim}


    \begin{Verbatim}[commandchars=\\\{\}]
a = 3

    \end{Verbatim}

    \textbf{Attention} toutefois, comme on le voit ici, Python
\textbf{n'impose pas} que les différentes occurrences de \texttt{a}
correspondent \textbf{à des valeurs identiques} (en langage savant, on
dirait que cela ne permet pas de faire de l'unification). De manière
beaucoup plus pragmatique, l'interpréteur se contente de faire comme
s'il faisait l'affectation plusieurs fois de gauche à droite,
c'est-à-dire comme s'il faisait~:

    \begin{Verbatim}[commandchars=\\\{\}]
{\color{incolor}In [{\color{incolor}12}]:} \PY{n}{a} \PY{o}{=} \PY{l+m+mi}{1}\PY{p}{;} \PY{n}{a} \PY{o}{=} \PY{l+m+mi}{2}\PY{p}{;} \PY{n}{a} \PY{o}{=} \PY{l+m+mi}{3}
\end{Verbatim}


    Cette technique n'est utilisée en pratique que pour les parties de la
structure dont on n'a que faire dans le contexte. Dans ces cas-là, il
arrive qu'on utilise le nom de variable \texttt{\_}, dont on rappelle
qu'il est légal, ou tout autre nom comme \texttt{ignored} pour
manifester le fait que cette partie de la structure ne sera pas
utilisée, par exemple~:

    \begin{Verbatim}[commandchars=\\\{\}]
{\color{incolor}In [{\color{incolor}13}]:} \PY{n}{entree} \PY{o}{=} \PY{p}{[}\PY{l+m+mi}{1}\PY{p}{,} \PY{l+m+mi}{2}\PY{p}{,} \PY{l+m+mi}{3}\PY{p}{]}
         
         \PY{n}{\PYZus{}}\PY{p}{,} \PY{n}{milieu}\PY{p}{,} \PY{n}{\PYZus{}} \PY{o}{=} \PY{n}{entree}
         \PY{n+nb}{print}\PY{p}{(}\PY{l+s+s1}{\PYZsq{}}\PY{l+s+s1}{milieu}\PY{l+s+s1}{\PYZsq{}}\PY{p}{,} \PY{n}{milieu}\PY{p}{)}
         
         \PY{n}{ignored}\PY{p}{,} \PY{n}{ignored}\PY{p}{,} \PY{n}{right} \PY{o}{=} \PY{n}{entree}
         \PY{n+nb}{print}\PY{p}{(}\PY{l+s+s1}{\PYZsq{}}\PY{l+s+s1}{right}\PY{l+s+s1}{\PYZsq{}}\PY{p}{,} \PY{n}{right}\PY{p}{)}
\end{Verbatim}


    \begin{Verbatim}[commandchars=\\\{\}]
milieu 2
right 3

    \end{Verbatim}

    \hypertarget{en-profondeur}{%
\subsubsection{En profondeur}\label{en-profondeur}}

    Le \emph{sequence unpacking} ne se limite pas au premier niveau dans les
structures, on peut extraire des données plus profondément imbriquées
dans la structure de départ~; par exemple avec en entrée la liste~:

    \begin{Verbatim}[commandchars=\\\{\}]
{\color{incolor}In [{\color{incolor}14}]:} \PY{n}{structure} \PY{o}{=} \PY{p}{[}\PY{l+s+s1}{\PYZsq{}}\PY{l+s+s1}{abc}\PY{l+s+s1}{\PYZsq{}}\PY{p}{,} \PY{p}{[}\PY{p}{(}\PY{l+m+mi}{1}\PY{p}{,} \PY{l+m+mi}{2}\PY{p}{)}\PY{p}{,} \PY{p}{(}\PY{p}{[}\PY{l+m+mi}{3}\PY{p}{]}\PY{p}{,} \PY{l+m+mi}{4}\PY{p}{)}\PY{p}{]}\PY{p}{,} \PY{l+m+mi}{5}\PY{p}{]}
\end{Verbatim}


    Si on souhaite extraire la valeur qui se trouve à l'emplacement du
\texttt{3}, on peut écrire~:

    \begin{Verbatim}[commandchars=\\\{\}]
{\color{incolor}In [{\color{incolor}15}]:} \PY{p}{(}\PY{n}{a}\PY{p}{,} \PY{p}{(}\PY{n}{b}\PY{p}{,} \PY{p}{(}\PY{p}{(}\PY{n}{trois}\PY{p}{,}\PY{p}{)}\PY{p}{,} \PY{n}{c}\PY{p}{)}\PY{p}{)}\PY{p}{,} \PY{n}{d}\PY{p}{)} \PY{o}{=} \PY{n}{structure}
         \PY{n+nb}{print}\PY{p}{(}\PY{l+s+s1}{\PYZsq{}}\PY{l+s+s1}{trois}\PY{l+s+s1}{\PYZsq{}}\PY{p}{,} \PY{n}{trois}\PY{p}{)}
\end{Verbatim}


    \begin{Verbatim}[commandchars=\\\{\}]
trois 3

    \end{Verbatim}

    Ou encore, sans doute un peu plus lisible~:

    \begin{Verbatim}[commandchars=\\\{\}]
{\color{incolor}In [{\color{incolor}16}]:} \PY{p}{(}\PY{n}{a}\PY{p}{,} \PY{p}{(}\PY{n}{b}\PY{p}{,} \PY{p}{(}\PY{p}{[}\PY{n}{trois}\PY{p}{]}\PY{p}{,} \PY{n}{c}\PY{p}{)}\PY{p}{)}\PY{p}{,} \PY{n}{d}\PY{p}{)} \PY{o}{=} \PY{n}{structure}
         \PY{n+nb}{print}\PY{p}{(}\PY{l+s+s1}{\PYZsq{}}\PY{l+s+s1}{trois}\PY{l+s+s1}{\PYZsq{}}\PY{p}{,} \PY{n}{trois}\PY{p}{)}
\end{Verbatim}


    \begin{Verbatim}[commandchars=\\\{\}]
trois 3

    \end{Verbatim}

    Naturellement on aurait aussi bien pu écrire ici quelque chose comme~:

    \begin{Verbatim}[commandchars=\\\{\}]
{\color{incolor}In [{\color{incolor}17}]:} \PY{n}{trois} \PY{o}{=} \PY{n}{structure}\PY{p}{[}\PY{l+m+mi}{1}\PY{p}{]}\PY{p}{[}\PY{l+m+mi}{1}\PY{p}{]}\PY{p}{[}\PY{l+m+mi}{0}\PY{p}{]}\PY{p}{[}\PY{l+m+mi}{0}\PY{p}{]}
         \PY{n+nb}{print}\PY{p}{(}\PY{l+s+s1}{\PYZsq{}}\PY{l+s+s1}{trois}\PY{l+s+s1}{\PYZsq{}}\PY{p}{,} \PY{n}{trois}\PY{p}{)}
\end{Verbatim}


    \begin{Verbatim}[commandchars=\\\{\}]
trois 3

    \end{Verbatim}

    Affaire de goût évidemment. Mais n'oublions pas une des phrases du Zen
de Python \(\textit{Flat is better than nested}\), ce qui veut dire que
ce n'est pas parce que vous pouvez faire des structures imbriquées
complexes que vous devez le faire. Bien souvent, cela rend la lecture et
la maintenance du code complexe, j'espère que l'exemple précédent vous
en a convaincu.

    \hypertarget{extended-unpacking-et-profondeur}{%
\subsubsection{\texorpdfstring{\emph{Extended unpacking} et
profondeur}{Extended unpacking et profondeur}}\label{extended-unpacking-et-profondeur}}

    On peut naturellement ajouter de l'\emph{extended unpacking} à n'importe
quel étage d'un \emph{unpacking} imbriqué~:

    \begin{Verbatim}[commandchars=\\\{\}]
{\color{incolor}In [{\color{incolor}18}]:} \PY{c+c1}{\PYZsh{} un exemple très alambiqué }
         \PY{n}{tree} \PY{o}{=} \PY{p}{[}\PY{l+m+mi}{1}\PY{p}{,} \PY{l+m+mi}{2}\PY{p}{,} \PY{p}{[}\PY{p}{(}\PY{l+m+mi}{3}\PY{p}{,} \PY{l+m+mi}{33}\PY{p}{,} \PY{l+s+s1}{\PYZsq{}}\PY{l+s+s1}{three}\PY{l+s+s1}{\PYZsq{}}\PY{p}{,} \PY{l+s+s1}{\PYZsq{}}\PY{l+s+s1}{thirty\PYZhy{}three}\PY{l+s+s1}{\PYZsq{}}\PY{p}{)}\PY{p}{]}\PY{p}{,}
                 \PY{p}{(} \PY{p}{[}\PY{l+m+mi}{4}\PY{p}{,} \PY{l+m+mi}{44}\PY{p}{,} \PY{p}{(}\PY{l+s+s1}{\PYZsq{}}\PY{l+s+s1}{forty}\PY{l+s+s1}{\PYZsq{}}\PY{p}{,} \PY{l+s+s1}{\PYZsq{}}\PY{l+s+s1}{forty\PYZhy{}four}\PY{l+s+s1}{\PYZsq{}}\PY{p}{)}\PY{p}{]}\PY{p}{)}\PY{p}{]}
         \PY{n}{tree}
\end{Verbatim}


\begin{Verbatim}[commandchars=\\\{\}]
{\color{outcolor}Out[{\color{outcolor}18}]:} [1, 2, [(3, 33, 'three', 'thirty-three')], [4, 44, ('forty', 'forty-four')]]
\end{Verbatim}
            
    \begin{Verbatim}[commandchars=\\\{\}]
{\color{incolor}In [{\color{incolor}19}]:} \PY{c+c1}{\PYZsh{} unpacking avec plusieurs variables *extended}
         \PY{o}{*}\PY{n}{\PYZus{}}\PY{p}{,}  \PY{p}{(}\PY{p}{(}\PY{n}{\PYZus{}}\PY{p}{,} \PY{o}{*}\PY{n}{x3}\PY{p}{,} \PY{n}{\PYZus{}}\PY{p}{)}\PY{p}{,}\PY{p}{)}\PY{p}{,} \PY{p}{(}\PY{o}{*}\PY{n}{\PYZus{}}\PY{p}{,} \PY{n}{x4}\PY{p}{)} \PY{o}{=} \PY{n}{tree}
         \PY{n+nb}{print}\PY{p}{(}\PY{n}{f}\PY{l+s+s2}{\PYZdq{}}\PY{l+s+s2}{x3=}\PY{l+s+si}{\PYZob{}x3\PYZcb{}}\PY{l+s+s2}{, x4=}\PY{l+s+si}{\PYZob{}x4\PYZcb{}}\PY{l+s+s2}{\PYZdq{}}\PY{p}{)}
\end{Verbatim}


    \begin{Verbatim}[commandchars=\\\{\}]
x3=[33, 'three'], x4=('forty', 'forty-four')

    \end{Verbatim}

    Dans ce cas, la limitation d'avoir une seule variable de la forme
\texttt{*extended} s'applique toujours, naturellement, mais à chaque
niveau dans l'imbrication, comme on le voit sur cet exemple.

    \hypertarget{pour-en-savoir-plus}{%
\subsection{Pour en savoir plus}\label{pour-en-savoir-plus}}

\begin{itemize}
\tightlist
\item
  \href{https://www.python.org/dev/peps/pep-3132/}{Le PEP (en anglais)
  qui introduit le \emph{extended unpacking}}.
\end{itemize}


    % Add a bibliography block to the postdoc
    
    
    
