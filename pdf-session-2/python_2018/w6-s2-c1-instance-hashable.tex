    
    
    
    

    

    \hypertarget{manipuler-des-ensembles-dinstances}{%
\section{Manipuler des ensembles
d'instances}\label{manipuler-des-ensembles-dinstances}}

    \hypertarget{compluxe9ment---niveau-intermuxe9diaire}{%
\subsection{Complément - niveau
intermédiaire}\label{compluxe9ment---niveau-intermuxe9diaire}}

    Souvenez-vous de ce qu'on avait dit en semaine 3 séquence 4, concernant
les clés dans un dictionnaire ou les éléments dans un ensemble. Nous
avions vu alors que, pour les types \emph{built-in}, les clés devaient
être des objets immuables et même globalement immuables.

    Nous allons voir dans ce complément quelles sont les règles qui
s'appliquent aux instances de classe.

    \hypertarget{instance-mutable-dans-un-ensemble}{%
\subsubsection{Instance mutable dans un
ensemble}\label{instance-mutable-dans-un-ensemble}}

    Une instance de classe est par défaut un objet mutable. Malgré cela, le
langage vous permet d'insérer une instance dans un ensemble - ou de
l'utiliser comme clé dans un dictionnaire. Nous allons voir ce mécanisme
en action.

    \hypertarget{hachage-par-duxe9faut-basuxe9-sur-id}{%
\subsubsection{\texorpdfstring{Hachage par défaut~: basé sur
\texttt{id()}}{Hachage par défaut~: basé sur id()}}\label{hachage-par-duxe9faut-basuxe9-sur-id}}

    \begin{Verbatim}[commandchars=\\\{\}]
{\color{incolor}In [{\color{incolor}1}]:} \PY{c+c1}{\PYZsh{} une classe Point}
        \PY{k}{class} \PY{n+nc}{Point1}\PY{p}{:}
            \PY{k}{def} \PY{n+nf}{\PYZus{}\PYZus{}init\PYZus{}\PYZus{}}\PY{p}{(}\PY{n+nb+bp}{self}\PY{p}{,} \PY{n}{x}\PY{p}{,} \PY{n}{y}\PY{p}{)}\PY{p}{:}
                \PY{n+nb+bp}{self}\PY{o}{.}\PY{n}{x} \PY{o}{=} \PY{n}{x}
                \PY{n+nb+bp}{self}\PY{o}{.}\PY{n}{y} \PY{o}{=} \PY{n}{y}
                
            \PY{k}{def} \PY{n+nf}{\PYZus{}\PYZus{}repr\PYZus{}\PYZus{}}\PY{p}{(}\PY{n+nb+bp}{self}\PY{p}{)}\PY{p}{:}
                \PY{k}{return} \PY{n}{f}\PY{l+s+s2}{\PYZdq{}}\PY{l+s+s2}{Pt[}\PY{l+s+si}{\PYZob{}self.x\PYZcb{}}\PY{l+s+s2}{, }\PY{l+s+si}{\PYZob{}self.y\PYZcb{}}\PY{l+s+s2}{]}\PY{l+s+s2}{\PYZdq{}}
\end{Verbatim}


    Avec ce code, les instances de \texttt{Point} sont mutables~:

    \begin{Verbatim}[commandchars=\\\{\}]
{\color{incolor}In [{\color{incolor}2}]:} \PY{c+c1}{\PYZsh{} deux instances }
        \PY{n}{p1} \PY{o}{=} \PY{n}{Point1}\PY{p}{(}\PY{l+m+mi}{2}\PY{p}{,} \PY{l+m+mi}{2}\PY{p}{)}
        \PY{n}{p2} \PY{o}{=} \PY{n}{Point1}\PY{p}{(}\PY{l+m+mi}{2}\PY{p}{,} \PY{l+m+mi}{3}\PY{p}{)}
\end{Verbatim}


    \begin{Verbatim}[commandchars=\\\{\}]
{\color{incolor}In [{\color{incolor}3}]:} \PY{c+c1}{\PYZsh{} objets mutables}
        \PY{n}{p1}\PY{o}{.}\PY{n}{y} \PY{o}{=} \PY{l+m+mi}{3}
\end{Verbatim}


    Mais par contre soyez attentifs, car il faut savoir que pour la classe
\texttt{Point1}, où nous n'avons rien redéfini, la fonction de hachage
sur une instance de \texttt{Point1} ne dépend que de la valeur de
\texttt{id()} sur cet objet.

Ce qui, dit autrement, signifie que deux objets qui sont distincts au
sens de \texttt{id()} sont considérés comme différents, et donc peuvent
coexister dans un ensemble (ou dans un dictionnaire)~:

    \begin{Verbatim}[commandchars=\\\{\}]
{\color{incolor}In [{\color{incolor}4}]:} \PY{c+c1}{\PYZsh{} nos deux objets se ressemblent}
        \PY{n}{p1}\PY{p}{,} \PY{n}{p2}
\end{Verbatim}


\begin{Verbatim}[commandchars=\\\{\}]
{\color{outcolor}Out[{\color{outcolor}4}]:} (Pt[2, 3], Pt[2, 3])
\end{Verbatim}
            
    \begin{Verbatim}[commandchars=\\\{\}]
{\color{incolor}In [{\color{incolor}5}]:} \PY{c+c1}{\PYZsh{} mais peuvent coexister }
        \PY{c+c1}{\PYZsh{} dans un ensemble}
        \PY{c+c1}{\PYZsh{} qui a alors 2 éléments}
        \PY{n}{s} \PY{o}{=} \PY{p}{\PYZob{}} \PY{n}{p1}\PY{p}{,} \PY{n}{p2} \PY{p}{\PYZcb{}}
        \PY{n+nb}{len}\PY{p}{(}\PY{n}{s}\PY{p}{)}
\end{Verbatim}


\begin{Verbatim}[commandchars=\\\{\}]
{\color{outcolor}Out[{\color{outcolor}5}]:} 2
\end{Verbatim}
            
    Si on recherche un de ces deux objets on le trouve~:

    \begin{Verbatim}[commandchars=\\\{\}]
{\color{incolor}In [{\color{incolor}6}]:} \PY{n}{p1} \PY{o+ow}{in} \PY{n}{s}
\end{Verbatim}


\begin{Verbatim}[commandchars=\\\{\}]
{\color{outcolor}Out[{\color{outcolor}6}]:} True
\end{Verbatim}
            
    \begin{Verbatim}[commandchars=\\\{\}]
{\color{incolor}In [{\color{incolor}7}]:} \PY{c+c1}{\PYZsh{} mais pas un troisième}
        \PY{n}{p3} \PY{o}{=} \PY{n}{Point1}\PY{p}{(}\PY{l+m+mi}{2}\PY{p}{,} \PY{l+m+mi}{4}\PY{p}{)}
        \PY{n}{p3} \PY{o+ow}{in} \PY{n}{s}
\end{Verbatim}


\begin{Verbatim}[commandchars=\\\{\}]
{\color{outcolor}Out[{\color{outcolor}7}]:} False
\end{Verbatim}
            
    Cette possibilité de gérer des ensembles d'objets selon cette stratégie
est très commode et peut apporter de gros gains de performance,
notamment lorsqu'on a souvent besoin de faire des tests d'appartenance.

En pratique, lorsqu'un modèle de données définit une relation de type
``1-n'', je vous recommande d'envisager d'utiliser un ensemble plutôt
qu'une liste.

    Par exemple envisagez :

\begin{verbatim}
class Animal:
    # blabla
   
class Zoo:
    def __init__(self):
        self.animals = set()
\end{verbatim}

    Plutôt que :

\begin{verbatim}
class Animal:
    # blabla
   
class Zoo:
    def __init__(self):
        self.animals = []
\end{verbatim}

    \hypertarget{compluxe9ment---niveau-avancuxe9}{%
\subsection{Complément - niveau
avancé}\label{compluxe9ment---niveau-avancuxe9}}

    \hypertarget{ce-nest-pas-ce-que-vous-voulez}{%
\subsubsection{Ce n'est pas ce que vous voulez
?}\label{ce-nest-pas-ce-que-vous-voulez}}

    Le comportement qu'on vient de voir pour les instances de
\texttt{Point1} dans les tables de hachage est raisonnable, si on admet
que deux points ne sont égaux que s'ils sont \textbf{le même objet} au
sens de \texttt{is}.

    Mais imaginons que vous voulez au contraire considérer que deux points
son égaux lorsqu'ils coincident sur le plan. Avec ce modèle de données,
vous voudriez que~:

\begin{itemize}
\tightlist
\item
  un ensemble dans lequel on insère \texttt{p1} et \texttt{p2} ne
  contienne qu'un élément,
\item
  et qu'on trouve \texttt{p3} quand on le cherche dans cet ensemble.
\end{itemize}

    \hypertarget{le-protocole-hashable-__hash__-et-__eq__}{%
\subsubsection{\texorpdfstring{Le protocole \emph{hashable}:
\texttt{\_\_hash\_\_} et
\texttt{\_\_eq\_\_}}{Le protocole hashable: \_\_hash\_\_ et \_\_eq\_\_}}\label{le-protocole-hashable-__hash__-et-__eq__}}

    Le langage nous permet de faire cela, grâce au protocole
\emph{hashable}; pour cela il nous faut définir deux méthodes~:

\begin{itemize}
\tightlist
\item
  \texttt{\_\_eq\_\_} qui, sans grande surprise, va servir à évaluer
  \texttt{p\ ==\ q}~;
\item
  \texttt{\_\_hash\_\_} qui va retourner la clé de hachage sur un objet.
\end{itemize}

La subtilité étant bien entendu que ces deux méthodes doivent être
cohérentes, si deux objets sont égaux, il faut que leurs hashs soient
égaux~; de bon sens, si l'égalité se base sur nos deux attributs
\texttt{x} et \texttt{y}, il faudra bien entendu que la fonction de
hachage utilise elle aussi ces deux attributs. Voir la documentation de
\href{https://docs.python.org/3/reference/datamodel.html?highlight=__hash__\#object.__hash__}{\texttt{\_\_hash\_\_}}.

    Voyons cela sur une sous-classe de \texttt{Point1}, dans laquelle nous
définissons ces deux méthodes~:

    \begin{Verbatim}[commandchars=\\\{\}]
{\color{incolor}In [{\color{incolor}8}]:} \PY{k}{class} \PY{n+nc}{Point2}\PY{p}{(}\PY{n}{Point1}\PY{p}{)}\PY{p}{:}
        
            \PY{c+c1}{\PYZsh{} l\PYZsq{}égalité va se baser naturellement sur x et y}
            \PY{k}{def} \PY{n+nf}{\PYZus{}\PYZus{}eq\PYZus{}\PYZus{}}\PY{p}{(}\PY{n+nb+bp}{self}\PY{p}{,} \PY{n}{other}\PY{p}{)}\PY{p}{:}
                \PY{k}{return} \PY{n+nb+bp}{self}\PY{o}{.}\PY{n}{x} \PY{o}{==} \PY{n}{other}\PY{o}{.}\PY{n}{x} \PY{o+ow}{and} \PY{n+nb+bp}{self}\PY{o}{.}\PY{n}{y} \PY{o}{==} \PY{n}{other}\PY{o}{.}\PY{n}{y}
        
            \PY{c+c1}{\PYZsh{} du coup la fonction de hachage }
            \PY{c+c1}{\PYZsh{} dépend aussi de x et de y}
            \PY{k}{def} \PY{n+nf}{\PYZus{}\PYZus{}hash\PYZus{}\PYZus{}}\PY{p}{(}\PY{n+nb+bp}{self}\PY{p}{)}\PY{p}{:}
                \PY{k}{return} \PY{p}{(}\PY{l+m+mi}{11} \PY{o}{*} \PY{n+nb+bp}{self}\PY{o}{.}\PY{n}{x} \PY{o}{+} \PY{n+nb+bp}{self}\PY{o}{.}\PY{n}{y}\PY{p}{)} \PY{o}{/}\PY{o}{/} \PY{l+m+mi}{16}
\end{Verbatim}


    On peut vérifier que cette fois les choses fonctionnent correctement~:

    \begin{Verbatim}[commandchars=\\\{\}]
{\color{incolor}In [{\color{incolor}9}]:} \PY{n}{q1} \PY{o}{=} \PY{n}{Point2}\PY{p}{(}\PY{l+m+mi}{2}\PY{p}{,} \PY{l+m+mi}{3}\PY{p}{)}
        \PY{n}{q2} \PY{o}{=} \PY{n}{Point2}\PY{p}{(}\PY{l+m+mi}{2}\PY{p}{,} \PY{l+m+mi}{3}\PY{p}{)}
\end{Verbatim}


    Nos deux objets sont distincts pour \texttt{id()}/\texttt{is}, mais
égaux pour \texttt{==}~:

    \begin{Verbatim}[commandchars=\\\{\}]
{\color{incolor}In [{\color{incolor}10}]:} \PY{n+nb}{print}\PY{p}{(}\PY{n}{f}\PY{l+s+s2}{\PYZdq{}}\PY{l+s+s2}{is → }\PY{l+s+s2}{\PYZob{}}\PY{l+s+s2}{q1 is q2\PYZcb{} }\PY{l+s+se}{\PYZbs{}n}\PY{l+s+s2}{== → }\PY{l+s+s2}{\PYZob{}}\PY{l+s+s2}{q1 == q2\PYZcb{}}\PY{l+s+s2}{\PYZdq{}}\PY{p}{)}
\end{Verbatim}


    \begin{Verbatim}[commandchars=\\\{\}]
is → False 
== → True

    \end{Verbatim}

    Et un ensemble contenant les deux points n'en contient qu'un~:

    \begin{Verbatim}[commandchars=\\\{\}]
{\color{incolor}In [{\color{incolor}11}]:} \PY{n}{s} \PY{o}{=} \PY{p}{\PYZob{}}\PY{n}{q1}\PY{p}{,} \PY{n}{q2}\PY{p}{\PYZcb{}}
         \PY{n+nb}{len}\PY{p}{(}\PY{n}{s}\PY{p}{)}
\end{Verbatim}


\begin{Verbatim}[commandchars=\\\{\}]
{\color{outcolor}Out[{\color{outcolor}11}]:} 1
\end{Verbatim}
            
    \begin{Verbatim}[commandchars=\\\{\}]
{\color{incolor}In [{\color{incolor}12}]:} \PY{n}{q3} \PY{o}{=} \PY{n}{Point2}\PY{p}{(}\PY{l+m+mi}{2}\PY{p}{,} \PY{l+m+mi}{3}\PY{p}{)}
         \PY{n}{q3} \PY{o+ow}{in} \PY{n}{s}
\end{Verbatim}


\begin{Verbatim}[commandchars=\\\{\}]
{\color{outcolor}Out[{\color{outcolor}12}]:} True
\end{Verbatim}
            
    Comme les ensembles et les dictionnaires reposent sur le même mécanisme
de table de hachage, on peut aussi indifféremment utiliser n'importe
lequel de nos 3 points pour indexer un dictionnaire~:

    \begin{Verbatim}[commandchars=\\\{\}]
{\color{incolor}In [{\color{incolor}13}]:} \PY{n}{d} \PY{o}{=} \PY{p}{\PYZob{}}\PY{p}{\PYZcb{}}
         \PY{n}{d}\PY{p}{[}\PY{n}{q1}\PY{p}{]} \PY{o}{=} \PY{l+m+mi}{1}
         \PY{n}{d}\PY{p}{[}\PY{n}{q2}\PY{p}{]}
\end{Verbatim}


\begin{Verbatim}[commandchars=\\\{\}]
{\color{outcolor}Out[{\color{outcolor}13}]:} 1
\end{Verbatim}
            
    \begin{Verbatim}[commandchars=\\\{\}]
{\color{incolor}In [{\color{incolor}14}]:} \PY{c+c1}{\PYZsh{} les clés q1, q2 et q3 sont}
         \PY{c+c1}{\PYZsh{} les mêmes pour le dictionnaire}
         \PY{n}{d}\PY{p}{[}\PY{n}{q3}\PY{p}{]} \PY{o}{=} \PY{l+m+mi}{10000}
         \PY{n}{d}
\end{Verbatim}


\begin{Verbatim}[commandchars=\\\{\}]
{\color{outcolor}Out[{\color{outcolor}14}]:} \{Pt[2, 3]: 10000\}
\end{Verbatim}
            
    \hypertarget{attention}{%
\subsubsection{Attention !}\label{attention}}

    Tout ceci semble très bien fonctionner; sauf qu'en fait, il y a une
\textbf{grosse faille}, c'est que nos objets \texttt{Point2} sont
\textbf{mutables}. Du coup on peut maintenant imaginer un scénario comme
celui-ci~:

    \begin{Verbatim}[commandchars=\\\{\}]
{\color{incolor}In [{\color{incolor}15}]:} \PY{n}{t1}\PY{p}{,} \PY{n}{t2} \PY{o}{=} \PY{n}{Point2}\PY{p}{(}\PY{l+m+mi}{10}\PY{p}{,} \PY{l+m+mi}{10}\PY{p}{)}\PY{p}{,} \PY{n}{Point2}\PY{p}{(}\PY{l+m+mi}{10}\PY{p}{,} \PY{l+m+mi}{10}\PY{p}{)}
         \PY{n}{s} \PY{o}{=} \PY{p}{\PYZob{}}\PY{n}{t1}\PY{p}{,} \PY{n}{t2}\PY{p}{\PYZcb{}}
         \PY{n}{s}
\end{Verbatim}


\begin{Verbatim}[commandchars=\\\{\}]
{\color{outcolor}Out[{\color{outcolor}15}]:} \{Pt[10, 10]\}
\end{Verbatim}
            
    \begin{Verbatim}[commandchars=\\\{\}]
{\color{incolor}In [{\color{incolor}16}]:} \PY{n}{t1} \PY{o+ow}{in} \PY{n}{s}\PY{p}{,} \PY{n}{t2} \PY{o+ow}{in} \PY{n}{s}
\end{Verbatim}


\begin{Verbatim}[commandchars=\\\{\}]
{\color{outcolor}Out[{\color{outcolor}16}]:} (True, True)
\end{Verbatim}
            
    Mais si maintenant je change un des deux objets:

    \begin{Verbatim}[commandchars=\\\{\}]
{\color{incolor}In [{\color{incolor}17}]:} \PY{n}{t1}\PY{o}{.}\PY{n}{x} \PY{o}{=} \PY{l+m+mi}{100}
\end{Verbatim}


    \begin{Verbatim}[commandchars=\\\{\}]
{\color{incolor}In [{\color{incolor}18}]:} \PY{n}{s}
\end{Verbatim}


\begin{Verbatim}[commandchars=\\\{\}]
{\color{outcolor}Out[{\color{outcolor}18}]:} \{Pt[100, 10]\}
\end{Verbatim}
            
    \begin{Verbatim}[commandchars=\\\{\}]
{\color{incolor}In [{\color{incolor}19}]:} \PY{n}{t1} \PY{o+ow}{in} \PY{n}{s}
\end{Verbatim}


\begin{Verbatim}[commandchars=\\\{\}]
{\color{outcolor}Out[{\color{outcolor}19}]:} False
\end{Verbatim}
            
    \begin{Verbatim}[commandchars=\\\{\}]
{\color{incolor}In [{\color{incolor}20}]:} \PY{n}{t2} \PY{o+ow}{in} \PY{n}{s}
\end{Verbatim}


\begin{Verbatim}[commandchars=\\\{\}]
{\color{outcolor}Out[{\color{outcolor}20}]:} False
\end{Verbatim}
            
    Évidemment cela n'est pas correct. Ce qui se passe ici c'est qu'on a

\begin{itemize}
\tightlist
\item
  d'abord inséré \texttt{t1} dans \texttt{s}, avec un indice de hachage
  calculé à partir de \texttt{10,\ 10}
\item
  pas inséré \texttt{t2} dans \texttt{s} parce qu'on a déterminé qu'il
  existait déjà.
\end{itemize}

Après avoir modifié \texttt{t1} qui est le seul élément de \texttt{s}: À
ce stade:

\begin{itemize}
\tightlist
\item
  lorsqu'on cherche \texttt{t1} dans \texttt{s}, on le fait avec un
  indice de hachage calculé à partir de \texttt{100,\ 10} et du coup on
  ne le trouve pas,
\item
  lorsqu'on cherche \texttt{t2} dans \texttt{s}, on utilise le bon
  indice de hachage, mais ensuite le seul élément qui pourrait faire
  l'affaire n'est pas égal à \texttt{t2}.
\end{itemize}

    \hypertarget{conclusion}{%
\subsubsection{Conclusion}\label{conclusion}}

    La
\href{https://docs.python.org/3/reference/datamodel.html\#object.__hash__}{documentation
de Python sur ce sujet} indique ceci~:

\begin{quote}
If a class defines mutable objects and implements an
\texttt{\_\_eq\_\_}() method, it should not implement
\texttt{\_\_hash\_\_}(), since the implementation of hashable
collections requires that a key's hash value is immutable (if the
object's hash value changes, it will be in the wrong hash bucket).
\end{quote}

    Notre classe \texttt{Point2} illustre bien cette limitation. Pour
qu'elle soit utilisable en pratique, il faut \textbf{rendre ses
instances immutables}. Cela peut se faire de plusieurs façons, dont deux
que nous aborderons dans la prochaine séquence et qui sont - entre
autres~:

\begin{itemize}
\tightlist
\item
  le \texttt{namedtuple}
\item
  et la \texttt{dataclass} (nouveau en 3.7).
\end{itemize}


    % Add a bibliography block to the postdoc
    
    
    
