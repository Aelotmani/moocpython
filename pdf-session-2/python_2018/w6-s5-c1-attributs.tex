    
    
    
    

    

    \hypertarget{les-attributs}{%
\section{Les attributs}\label{les-attributs}}

    \hypertarget{compluxe9ments---niveau-basique}{%
\subsection{Compléments - niveau
basique}\label{compluxe9ments---niveau-basique}}

    \hypertarget{la-notation-.-et-les-attributs}{%
\subsubsection{\texorpdfstring{La notation \texttt{.} et les
attributs}{La notation . et les attributs}}\label{la-notation-.-et-les-attributs}}

    La notation \texttt{module.variable} que nous avons vue dans la vidéo
est un cas particulier de la notion d'attribut, qui permet d'étendre un
objet, ou si on préfère de lui accrocher des données.

Nous avons déjà rencontré ceci de nombreuses fois à présent, c'est
exactement le même mécanisme d'attribut qui est utilisé pour les
méthodes~; pour le système d'attribut il n'y a pas de différence entre
\texttt{module.variable}, \texttt{module.fonction},
\texttt{objet.methode}, etc.

Nous verrons très bientôt que ce mécanisme est massivement utilisé
également dans les instances de classe.

    \hypertarget{les-fonctions-de-gestion-des-attributs}{%
\subsubsection{Les fonctions de gestion des
attributs}\label{les-fonctions-de-gestion-des-attributs}}

    Pour accéder programmativement aux attributs d'un objet, on dispose des
3 fonctions \emph{built-in} \texttt{getattr}, \texttt{setattr}, et
\texttt{hasattr}, que nous allons illustrer tout de suite.

    \hypertarget{lire-un-attribut}{%
\subparagraph{Lire un attribut}\label{lire-un-attribut}}

    \begin{Verbatim}[commandchars=\\\{\},frame=single,framerule=0.3mm,rulecolor=\color{cellframecolor}]
{\color{incolor}In [{\color{incolor}1}]:} \PY{k+kn}{import} \PY{n+nn}{math}
        \PY{c+c1}{\PYZsh{} nous savons lire un attribut comme ceci }
        \PY{c+c1}{\PYZsh{} qui lit l\PYZsq{}attribut de nom \PYZsq{}pi\PYZsq{} dans le module math}
        \PY{n}{math}\PY{o}{.}\PY{n}{pi}
\end{Verbatim}


\begin{Verbatim}[commandchars=\\\{\},frame=single,framerule=0.3mm,rulecolor=\color{cellframecolor}]
{\color{outcolor}Out[{\color{outcolor}1}]:} 3.141592653589793
\end{Verbatim}
            
    La
\href{https://docs.python.org/3/library/functions.html\#getattr}{fonction
\emph{built-in} \texttt{getattr}} permet de lire un attribut
programmativement~:

    \begin{Verbatim}[commandchars=\\\{\},frame=single,framerule=0.3mm,rulecolor=\color{cellframecolor}]
{\color{incolor}In [{\color{incolor}2}]:} \PY{c+c1}{\PYZsh{} si on part d\PYZsq{}une chaîne qui désigne le nom de l\PYZsq{}attribut}
        \PY{c+c1}{\PYZsh{} la formule équivalente est alors}
        \PY{n+nb}{getattr}\PY{p}{(}\PY{n}{math}\PY{p}{,} \PY{l+s+s1}{\PYZsq{}}\PY{l+s+s1}{pi}\PY{l+s+s1}{\PYZsq{}}\PY{p}{)}
\end{Verbatim}


\begin{Verbatim}[commandchars=\\\{\},frame=single,framerule=0.3mm,rulecolor=\color{cellframecolor}]
{\color{outcolor}Out[{\color{outcolor}2}]:} 3.141592653589793
\end{Verbatim}
            
    \begin{Verbatim}[commandchars=\\\{\},frame=single,framerule=0.3mm,rulecolor=\color{cellframecolor}]
{\color{incolor}In [{\color{incolor}3}]:} \PY{c+c1}{\PYZsh{} on peut utiliser les attributs avec la plupart des objets}
        \PY{c+c1}{\PYZsh{} ici nous allons le faire sur une fonction}
        \PY{k}{def} \PY{n+nf}{foo}\PY{p}{(}\PY{p}{)}\PY{p}{:} 
            \PY{l+s+s2}{\PYZdq{}}\PY{l+s+s2}{une fonction vide}\PY{l+s+s2}{\PYZdq{}}
            \PY{k}{pass}
        
        \PY{c+c1}{\PYZsh{} on a déjà vu certains attributs des fonctions}
        \PY{n+nb}{print}\PY{p}{(}\PY{n}{f}\PY{l+s+s2}{\PYZdq{}}\PY{l+s+s2}{nom=}\PY{l+s+si}{\PYZob{}foo.\PYZus{}\PYZus{}name\PYZus{}\PYZus{}\PYZcb{}}\PY{l+s+s2}{, docstring=`}\PY{l+s+si}{\PYZob{}foo.\PYZus{}\PYZus{}doc\PYZus{}\PYZus{}\PYZcb{}}\PY{l+s+s2}{`}\PY{l+s+s2}{\PYZdq{}}\PY{p}{)}
\end{Verbatim}


    \begin{Verbatim}[commandchars=\\\{\},frame=single,framerule=0.3mm,rulecolor=\color{cellframecolor}]
nom=foo, docstring=`une fonction vide`
\end{Verbatim}

    \begin{Verbatim}[commandchars=\\\{\},frame=single,framerule=0.3mm,rulecolor=\color{cellframecolor}]
{\color{incolor}In [{\color{incolor}4}]:} \PY{c+c1}{\PYZsh{} on peut préciser une valeur par défaut pour le cas où l\PYZsq{}attribut}
        \PY{c+c1}{\PYZsh{} n\PYZsq{}existe pas}
        \PY{n+nb}{getattr}\PY{p}{(}\PY{n}{foo}\PY{p}{,} \PY{l+s+s2}{\PYZdq{}}\PY{l+s+s2}{attribut\PYZus{}inexistant}\PY{l+s+s2}{\PYZdq{}}\PY{p}{,} \PY{l+s+s1}{\PYZsq{}}\PY{l+s+s1}{valeur\PYZus{}par\PYZus{}defaut}\PY{l+s+s1}{\PYZsq{}}\PY{p}{)}
\end{Verbatim}


\begin{Verbatim}[commandchars=\\\{\},frame=single,framerule=0.3mm,rulecolor=\color{cellframecolor}]
{\color{outcolor}Out[{\color{outcolor}4}]:} 'valeur\_par\_defaut'
\end{Verbatim}
            
    \hypertarget{uxe9crire-un-attribut}{%
\subparagraph{Écrire un attribut}\label{uxe9crire-un-attribut}}

    \begin{Verbatim}[commandchars=\\\{\},frame=single,framerule=0.3mm,rulecolor=\color{cellframecolor}]
{\color{incolor}In [{\color{incolor}5}]:} \PY{c+c1}{\PYZsh{} on peut ajouter un attribut arbitraire (toujours sur l\PYZsq{}objet fonction)}
        \PY{n}{foo}\PY{o}{.}\PY{n}{hauteur} \PY{o}{=} \PY{l+m+mi}{100}
        
        \PY{n}{foo}\PY{o}{.}\PY{n}{hauteur}
\end{Verbatim}


\begin{Verbatim}[commandchars=\\\{\},frame=single,framerule=0.3mm,rulecolor=\color{cellframecolor}]
{\color{outcolor}Out[{\color{outcolor}5}]:} 100
\end{Verbatim}
            
    Comme pour la lecture on peut écrire un attribut programmativement avec
la
\href{https://docs.python.org/3/library/functions.html\#setattr}{fonction
\emph{built-in} \texttt{setattr}}~:

    \begin{Verbatim}[commandchars=\\\{\},frame=single,framerule=0.3mm,rulecolor=\color{cellframecolor}]
{\color{incolor}In [{\color{incolor}6}]:} \PY{c+c1}{\PYZsh{} écrire un attribut avec setattr}
        \PY{n+nb}{setattr}\PY{p}{(}\PY{n}{foo}\PY{p}{,} \PY{l+s+s2}{\PYZdq{}}\PY{l+s+s2}{largeur}\PY{l+s+s2}{\PYZdq{}}\PY{p}{,} \PY{l+m+mi}{200}\PY{p}{)}
        
        \PY{c+c1}{\PYZsh{} on peut bien sûr le lire indifféremment}
        \PY{c+c1}{\PYZsh{} directement comme ici, ou avec getattr}
        \PY{n}{foo}\PY{o}{.}\PY{n}{largeur}
\end{Verbatim}


\begin{Verbatim}[commandchars=\\\{\},frame=single,framerule=0.3mm,rulecolor=\color{cellframecolor}]
{\color{outcolor}Out[{\color{outcolor}6}]:} 200
\end{Verbatim}
            
    \hypertarget{liste-des-attributs}{%
\subparagraph{Liste des attributs}\label{liste-des-attributs}}

    La
\href{https://docs.python.org/3/library/functions.html\#hasattr}{fonction
\emph{built-in} \texttt{hasattr}} permet de savoir si un objet possède
ou pas un attribut~:

    \begin{Verbatim}[commandchars=\\\{\},frame=single,framerule=0.3mm,rulecolor=\color{cellframecolor}]
{\color{incolor}In [{\color{incolor}7}]:} \PY{c+c1}{\PYZsh{} pour savoir si un attribut existe}
        \PY{n+nb}{hasattr}\PY{p}{(}\PY{n}{math}\PY{p}{,} \PY{l+s+s1}{\PYZsq{}}\PY{l+s+s1}{pi}\PY{l+s+s1}{\PYZsq{}}\PY{p}{)}
\end{Verbatim}


\begin{Verbatim}[commandchars=\\\{\},frame=single,framerule=0.3mm,rulecolor=\color{cellframecolor}]
{\color{outcolor}Out[{\color{outcolor}7}]:} True
\end{Verbatim}
            
    Ce qui peut aussi être retrouvé autrement, avec la
\href{https://docs.python.org/3/library/functions.html\#vars}{fonction
\emph{built-in} \texttt{vars}}~:

    \begin{Verbatim}[commandchars=\\\{\},frame=single,framerule=0.3mm,rulecolor=\color{cellframecolor}]
{\color{incolor}In [{\color{incolor}8}]:} \PY{n+nb}{vars}\PY{p}{(}\PY{n}{foo}\PY{p}{)}
\end{Verbatim}


\begin{Verbatim}[commandchars=\\\{\},frame=single,framerule=0.3mm,rulecolor=\color{cellframecolor}]
{\color{outcolor}Out[{\color{outcolor}8}]:} \{'hauteur': 100, 'largeur': 200\}
\end{Verbatim}
            
    \hypertarget{sur-quels-objets}{%
\subsubsection{Sur quels objets}\label{sur-quels-objets}}

    Il n'est pas possible d'ajouter des attributs sur les types de base, car
ce sont des classes immuables~:

    \begin{Verbatim}[commandchars=\\\{\},frame=single,framerule=0.3mm,rulecolor=\color{cellframecolor}]
{\color{incolor}In [{\color{incolor}9}]:} \PY{k}{for} \PY{n}{builtin\PYZus{}type} \PY{o+ow}{in} \PY{p}{(}\PY{n+nb}{int}\PY{p}{,} \PY{n+nb}{str}\PY{p}{,} \PY{n+nb}{float}\PY{p}{,} \PY{n+nb}{complex}\PY{p}{,} \PY{n+nb}{tuple}\PY{p}{,} \PY{n+nb}{dict}\PY{p}{,} \PY{n+nb}{set}\PY{p}{,} \PY{n+nb}{frozenset}\PY{p}{)}\PY{p}{:}
            \PY{n}{obj} \PY{o}{=} \PY{n}{builtin\PYZus{}type}\PY{p}{(}\PY{p}{)}
            \PY{k}{try}\PY{p}{:} 
                \PY{n}{obj}\PY{o}{.}\PY{n}{foo} \PY{o}{=} \PY{l+s+s1}{\PYZsq{}}\PY{l+s+s1}{bar}\PY{l+s+s1}{\PYZsq{}}
            \PY{k}{except} \PY{n+ne}{AttributeError} \PY{k}{as} \PY{n}{e}\PY{p}{:} 
                \PY{n+nb}{print}\PY{p}{(}\PY{n}{f}\PY{l+s+s2}{\PYZdq{}}\PY{l+s+si}{\PYZob{}builtin\PYZus{}type.\PYZus{}\PYZus{}name\PYZus{}\PYZus{}:\PYZgt{}10\PYZcb{}}\PY{l+s+s2}{ → exception }\PY{l+s+s2}{\PYZob{}}\PY{l+s+s2}{type(e)\PYZcb{} \PYZhy{} }\PY{l+s+si}{\PYZob{}e\PYZcb{}}\PY{l+s+s2}{\PYZdq{}}\PY{p}{)}
\end{Verbatim}


    \begin{Verbatim}[commandchars=\\\{\},frame=single,framerule=0.3mm,rulecolor=\color{cellframecolor}]
       int → exception <class 'AttributeError'> - 'int' object has no attribute 'foo'
       str → exception <class 'AttributeError'> - 'str' object has no attribute 'foo'
     float → exception <class 'AttributeError'> - 'float' object has no attribute 'foo'
   complex → exception <class 'AttributeError'> - 'complex' object has no attribute 'foo'
     tuple → exception <class 'AttributeError'> - 'tuple' object has no attribute 'foo'
      dict → exception <class 'AttributeError'> - 'dict' object has no attribute 'foo'
       set → exception <class 'AttributeError'> - 'set' object has no attribute 'foo'
 frozenset → exception <class 'AttributeError'> - 'frozenset' object has no attribute 'foo'
\end{Verbatim}

    C'est par contre possible sur virtuellement tout le reste, et notamment
là où c'est très utile, c'est-à-dire pour ce qui nous concerne sur les~:

\begin{itemize}
\tightlist
\item
  modules
\item
  packages
\item
  fonctions
\item
  classes
\item
  instances
\end{itemize}


    % Add a bibliography block to the postdoc
    
    
    
