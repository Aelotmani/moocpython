    
    
    
    

    

    \hypertarget{fichiers-et-utilitaires}{%
\section{Fichiers et utilitaires}\label{fichiers-et-utilitaires}}

    \hypertarget{compluxe9ment---niveau-basique}{%
\subsection{Complément - niveau
basique}\label{compluxe9ment---niveau-basique}}

    Outre les objets fichiers créés avec la fonction \texttt{open}, comme on
l'a vu dans la vidéo, et qui servent à lire et écrire à un endroit
précis, une application a besoin d'un minimum d'utilitaires pour
\textbf{parcourir l'arborescence de répertoires et fichiers}, c'est
notre propos dans ce complément.

    \hypertarget{le-module-os.path-obsoluxe8te}{%
\subsubsection{\texorpdfstring{Le module \texttt{os.path}
(obsolète)}{Le module os.path (obsolète)}}\label{le-module-os.path-obsoluxe8te}}

    Avant la version python-3.4, la librairie standard offrait une
conjonction d'outils pour ce type de fonctionnalités:

\begin{itemize}
\tightlist
\item
  le module \texttt{os.path}, pour faire des calculs sur les les chemins
  et noms de fichiers
  \href{https://docs.python.org/3/library/os.html}{doc},
\item
  le module \texttt{os} pour certaines fonctions complémentaires comme
  renommer ou détruire un fichier
  \href{https://docs.python.org/3/library/os.path.html}{doc},
\item
  et enfin le module \texttt{glob} pour la recherche de fichiers, par
  exemple pour trouver tous les fichiers en \texttt{*.txt}
  \href{https://docs.python.org/3/library/glob.html}{doc}.
\end{itemize}

    Cet ensemble un peu disparate a été remplacé par une \textbf{librairie
unique \texttt{pathlib}}, qui fournit toutes ces fonctionnalités sous un
interface unique et moderne, que nous \textbf{recommandons} évidemment
d'utiliser pour \textbf{du nouveau code}.

Avant d'aborder \texttt{pathlib}, voici un très bref aperçu de ces trois
anciens modules, pour le cas - assez probable - où vous les
rencontreriez dans du code existant; tous les noms qui suivent
correspondent à des \textbf{fonctions} - par opposition à
\texttt{pathlib} qui, comme nous allons le voir, offre une interface
orientée objet:

    \begin{itemize}
\tightlist
\item
  \texttt{os.path.join} ajoute `/' ou '' entre deux morceaux de chemin,
  selon l'OS
\item
  \texttt{os.path.basename} trouve le nom de fichier dans un chemin
\item
  \texttt{os.path.dirname} trouve le nom du directory dans un chemin
\item
  \texttt{os.path.abspath} calcule un chemin absolu, c'est-à-dire à
  partir de la racine du filesystem
\end{itemize}

    \begin{itemize}
\tightlist
\item
  \texttt{os.path.exists} pour savoir si un chemin existe ou pas
  (fichier ou répertoire)
\item
  \texttt{os.path.isfile} (et \texttt{isdir}) pour savoir si un chemin
  est un fichier (et un répertoire)
\item
  \texttt{os.path.getsize} pour obtenir la taille du fichier
\item
  \texttt{os.path.getatime} et aussi \texttt{getmtime} et
  \texttt{getctime} pour obtenir les dates de création/modification d'un
  fichier
\end{itemize}

    \begin{itemize}
\tightlist
\item
  \texttt{os.remove} (ou son ancien nom \texttt{os.unlink}), qui permet
  de supprimer un fichier
\item
  \texttt{os.rmdir} pour supprimer un répertoire (mais qui doit être
  vide)
\item
  \texttt{os.removedirs} pour supprimer tout un répertoire avec son
  contenu, récursivement si nécessaire
\item
  \texttt{os.rename} pour renommer un fichier
\end{itemize}

    \begin{itemize}
\tightlist
\item
  \texttt{glob.glob} comme dans par exemple \texttt{glob.glob("*.txt")}
\end{itemize}

    \hypertarget{le-module-pathlib}{%
\subsubsection{\texorpdfstring{Le module
\texttt{pathlib}}{Le module pathlib}}\label{le-module-pathlib}}

    C'est la méthode recommandée aujourd'hui pour travailler sur les
fichiers et répertoires.

    \hypertarget{orientuxe9-objet}{%
\subparagraph{Orienté Objet}\label{orientuxe9-objet}}

    Comme on l'a mentionné \texttt{pathlib} offre une interface orientée
objet; mais qu'est-ce que ça veut dire au juste ?

Ceci nous donne un prétexte pour une première application pratique des
notions de module (que nous avons introduits en fin de semaine 2) et de
classe (que nous allons voir en fin de semaine).

    De même que le langage nous propose les types \emph{builtin}
\texttt{int} et \texttt{str}, le module \texttt{pathlib} nous expose
\textbf{un type} (on dira plutôt \textbf{une classe}) qui s'appelle
\texttt{Path}, que nous allons importer comme ceci:

    \begin{Verbatim}[commandchars=\\\{\}]
{\color{incolor}In [{\color{incolor}1}]:} \PY{k+kn}{from} \PY{n+nn}{pathlib} \PY{k}{import} \PY{n}{Path}
\end{Verbatim}


    Nous allons faire tourner un petit scénario qui va créer un fichier:

    \begin{Verbatim}[commandchars=\\\{\}]
{\color{incolor}In [{\color{incolor}2}]:} \PY{c+c1}{\PYZsh{} le nom de notre fichier jouet }
        \PY{n}{nom} \PY{o}{=} \PY{l+s+s1}{\PYZsq{}}\PY{l+s+s1}{fichier\PYZhy{}temoin}\PY{l+s+s1}{\PYZsq{}}
\end{Verbatim}


    Pour commencer, nous allons vérifier si le fichier en question existe.

Pour ça nous créons un \textbf{objet} qui est une \textbf{instance} de
la classe \texttt{Path}, comme ceci:

    \begin{Verbatim}[commandchars=\\\{\}]
{\color{incolor}In [{\color{incolor}3}]:} \PY{c+c1}{\PYZsh{} on crée un objet de la classe Path, associé au nom de fichier}
        \PY{n}{path} \PY{o}{=} \PY{n}{Path}\PY{p}{(}\PY{n}{nom}\PY{p}{)}
\end{Verbatim}


    Vous remarquez que c'est cohérent avec par exemple:

    \begin{Verbatim}[commandchars=\\\{\}]
{\color{incolor}In [{\color{incolor}4}]:} \PY{c+c1}{\PYZsh{} transformer un float en int}
        \PY{n}{i} \PY{o}{=} \PY{n+nb}{int}\PY{p}{(}\PY{l+m+mf}{3.5}\PY{p}{)}
\end{Verbatim}


    en ce sens que le type (\texttt{int} ou \texttt{Path}) se comporte comme
une usine pour créer des objets du type en question.

    Quoi qu'il en soit, cet objet \texttt{path} offre un certain nombre de
méthodes; pour les voir puisque nous sommes dans un notebook, je vous
invite dans la cellule suivante à utiliser l'aide en ligne en appuyant
sur la touche `Tabulation' après avoir ajouté un \texttt{.} comme si
vous alliez envoyer une méthode à cet objet

\begin{verbatim}
path.[taper la touche TAB]
\end{verbatim}

et le notebook vous montrera la liste des méthodes disponibles.

    \begin{Verbatim}[commandchars=\\\{\}]
{\color{incolor}In [{\color{incolor}5}]:} \PY{c+c1}{\PYZsh{} ajouter un . et utilisez la touche \PYZlt{}Tabulation\PYZgt{}}
        \PY{n}{path}
\end{Verbatim}


\begin{Verbatim}[commandchars=\\\{\}]
{\color{outcolor}Out[{\color{outcolor}5}]:} PosixPath('fichier-temoin')
\end{Verbatim}
            
    Ainsi par exemple on peut savoir si le fichier existe avec la méthode
\texttt{exists()}

    \begin{Verbatim}[commandchars=\\\{\}]
{\color{incolor}In [{\color{incolor}6}]:} \PY{c+c1}{\PYZsh{} au départ le fichier n\PYZsq{}existe pas}
        \PY{n}{path}\PY{o}{.}\PY{n}{exists}\PY{p}{(}\PY{p}{)}
\end{Verbatim}


\begin{Verbatim}[commandchars=\\\{\}]
{\color{outcolor}Out[{\color{outcolor}6}]:} False
\end{Verbatim}
            
    \begin{Verbatim}[commandchars=\\\{\}]
{\color{incolor}In [{\color{incolor}7}]:} \PY{c+c1}{\PYZsh{} si j\PYZsq{}écris dedans je le crée}
        \PY{k}{with} \PY{n+nb}{open}\PY{p}{(}\PY{n}{nom}\PY{p}{,} \PY{l+s+s1}{\PYZsq{}}\PY{l+s+s1}{w}\PY{l+s+s1}{\PYZsq{}}\PY{p}{,} \PY{n}{encoding}\PY{o}{=}\PY{l+s+s1}{\PYZsq{}}\PY{l+s+s1}{utf\PYZhy{}8}\PY{l+s+s1}{\PYZsq{}}\PY{p}{)} \PY{k}{as} \PY{n}{output}\PY{p}{:}
            \PY{n}{output}\PY{o}{.}\PY{n}{write}\PY{p}{(}\PY{l+s+s1}{\PYZsq{}}\PY{l+s+s1}{0123456789}\PY{l+s+se}{\PYZbs{}n}\PY{l+s+s1}{\PYZsq{}}\PY{p}{)}
\end{Verbatim}


    \begin{Verbatim}[commandchars=\\\{\}]
{\color{incolor}In [{\color{incolor}8}]:} \PY{c+c1}{\PYZsh{} et maintenant il existe}
        \PY{n}{path}\PY{o}{.}\PY{n}{exists}\PY{p}{(}\PY{p}{)}
\end{Verbatim}


\begin{Verbatim}[commandchars=\\\{\}]
{\color{outcolor}Out[{\color{outcolor}8}]:} True
\end{Verbatim}
            
    \hypertarget{muxe9tadonnuxe9es}{%
\subparagraph{Métadonnées}\label{muxe9tadonnuxe9es}}

    Voici quelques exemples qui montrent comment accéder aux métadonnées de
ce fichier:

    \begin{Verbatim}[commandchars=\\\{\}]
{\color{incolor}In [{\color{incolor}9}]:} \PY{c+c1}{\PYZsh{} cette méthode retourne (en un seul appel système) les métadonnées agrégées}
        \PY{n}{path}\PY{o}{.}\PY{n}{stat}\PY{p}{(}\PY{p}{)}
\end{Verbatim}


\begin{Verbatim}[commandchars=\\\{\}]
{\color{outcolor}Out[{\color{outcolor}9}]:} os.stat\_result(st\_mode=33188, st\_ino=17121245, st\_dev=16777220, st\_nlink=1, st\_uid=501, st\_gid=20, st\_size=11, st\_atime=1540558080, st\_mtime=1540558080, st\_ctime=1540558080)
\end{Verbatim}
            
    Pour ceux que ça intéresse, l'objet retourné par cette méthode
\texttt{stat} est un \texttt{namedtuple}, que l'on va voir très bientôt.

On accède aux différentes informations comme ceci:

    \begin{Verbatim}[commandchars=\\\{\}]
{\color{incolor}In [{\color{incolor}10}]:} \PY{c+c1}{\PYZsh{} la taille du fichier en octets est de 11 }
         \PY{c+c1}{\PYZsh{} car il faut compter un caractère \PYZdq{}newline\PYZdq{} en fin de ligne }
         \PY{n}{path}\PY{o}{.}\PY{n}{stat}\PY{p}{(}\PY{p}{)}\PY{o}{.}\PY{n}{st\PYZus{}size}
\end{Verbatim}


\begin{Verbatim}[commandchars=\\\{\}]
{\color{outcolor}Out[{\color{outcolor}10}]:} 11
\end{Verbatim}
            
    \begin{Verbatim}[commandchars=\\\{\}]
{\color{incolor}In [{\color{incolor}11}]:} \PY{c+c1}{\PYZsh{} la date de dernière modification, sous forme d\PYZsq{}un nombre}
         \PY{c+c1}{\PYZsh{} c\PYZsq{}est le nombre de secondes depuis le 1er Janvier 1970}
         \PY{n}{mtime} \PY{o}{=} \PY{n}{path}\PY{o}{.}\PY{n}{stat}\PY{p}{(}\PY{p}{)}\PY{o}{.}\PY{n}{st\PYZus{}mtime}
         \PY{n}{mtime}
\end{Verbatim}


\begin{Verbatim}[commandchars=\\\{\}]
{\color{outcolor}Out[{\color{outcolor}11}]:} 1540558080.0
\end{Verbatim}
            
    \begin{Verbatim}[commandchars=\\\{\}]
{\color{incolor}In [{\color{incolor}12}]:} \PY{c+c1}{\PYZsh{} que je peux rendre lisible comme ceci}
         \PY{c+c1}{\PYZsh{} en anticipant sur le module datetime}
         \PY{k+kn}{from} \PY{n+nn}{datetime} \PY{k}{import} \PY{n}{datetime}
         \PY{n}{mtime\PYZus{}datetime} \PY{o}{=} \PY{n}{datetime}\PY{o}{.}\PY{n}{fromtimestamp}\PY{p}{(}\PY{n}{mtime}\PY{p}{)}
         \PY{n}{mtime\PYZus{}datetime}
\end{Verbatim}


\begin{Verbatim}[commandchars=\\\{\}]
{\color{outcolor}Out[{\color{outcolor}12}]:} datetime.datetime(2018, 10, 26, 14, 48)
\end{Verbatim}
            
    \begin{Verbatim}[commandchars=\\\{\}]
{\color{incolor}In [{\color{incolor}13}]:} \PY{c+c1}{\PYZsh{} ou encore, si je formatte pour n\PYZsq{}obtenir que}
         \PY{c+c1}{\PYZsh{} l\PYZsq{}heure et la minute}
         \PY{n}{f}\PY{l+s+s2}{\PYZdq{}}\PY{l+s+s2}{\PYZob{}}\PY{l+s+s2}{mtime\PYZus{}datetime:}\PY{l+s+s2}{\PYZpc{}}\PY{l+s+s2}{H:}\PY{l+s+s2}{\PYZpc{}}\PY{l+s+s2}{M\PYZcb{}}\PY{l+s+s2}{\PYZdq{}}
\end{Verbatim}


\begin{Verbatim}[commandchars=\\\{\}]
{\color{outcolor}Out[{\color{outcolor}13}]:} '14:48'
\end{Verbatim}
            
    \hypertarget{duxe9truire-un-fichier}{%
\subparagraph{Détruire un fichier}\label{duxe9truire-un-fichier}}

    \begin{Verbatim}[commandchars=\\\{\}]
{\color{incolor}In [{\color{incolor}14}]:} \PY{c+c1}{\PYZsh{} je peux maintenant détruire le fichier}
         \PY{n}{path}\PY{o}{.}\PY{n}{unlink}\PY{p}{(}\PY{p}{)}
\end{Verbatim}


    \begin{Verbatim}[commandchars=\\\{\}]
{\color{incolor}In [{\color{incolor}15}]:} \PY{c+c1}{\PYZsh{} ou encore mieux, si je veux détruire }
         \PY{c+c1}{\PYZsh{} seulement dans le cas où il existe je peux aussi faire}
         \PY{k}{try}\PY{p}{:} 
             \PY{n}{path}\PY{o}{.}\PY{n}{unlink}\PY{p}{(}\PY{p}{)}
         \PY{k}{except} \PY{n+ne}{FileNotFoundError}\PY{p}{:}
             \PY{n+nb}{print}\PY{p}{(}\PY{l+s+s2}{\PYZdq{}}\PY{l+s+s2}{no need to remove}\PY{l+s+s2}{\PYZdq{}}\PY{p}{)}
\end{Verbatim}


    \begin{Verbatim}[commandchars=\\\{\}]
no need to remove

    \end{Verbatim}

    \begin{Verbatim}[commandchars=\\\{\}]
{\color{incolor}In [{\color{incolor}16}]:} \PY{c+c1}{\PYZsh{} et maintenant il n\PYZsq{}existe plus}
         \PY{n}{path}\PY{o}{.}\PY{n}{exists}\PY{p}{(}\PY{p}{)}
\end{Verbatim}


\begin{Verbatim}[commandchars=\\\{\}]
{\color{outcolor}Out[{\color{outcolor}16}]:} False
\end{Verbatim}
            
    \begin{Verbatim}[commandchars=\\\{\}]
{\color{incolor}In [{\color{incolor}17}]:} \PY{c+c1}{\PYZsh{} je peux aussi retrouver le nom du fichier comme ceci}
         \PY{c+c1}{\PYZsh{} attention ce n\PYZsq{}est pas une méthode mais un attribut }
         \PY{c+c1}{\PYZsh{} c\PYZsq{}est pourquoi il n\PYZsq{}y a pas de parenthèses}
         \PY{n}{path}\PY{o}{.}\PY{n}{name}
\end{Verbatim}


\begin{Verbatim}[commandchars=\\\{\}]
{\color{outcolor}Out[{\color{outcolor}17}]:} 'fichier-temoin'
\end{Verbatim}
            
    \hypertarget{recherche-de-fichiers}{%
\subparagraph{Recherche de fichiers}\label{recherche-de-fichiers}}

    Maintenant je voudrais connaître la liste des fichiers de nom
\texttt{*.json} dans le directory \texttt{data}.

La méthode la plus naturelle consiste à créer une instance de
\texttt{Path} associée au directory lui-même:

    \begin{Verbatim}[commandchars=\\\{\}]
{\color{incolor}In [{\color{incolor}18}]:} \PY{n}{dirpath} \PY{o}{=} \PY{n}{Path}\PY{p}{(}\PY{l+s+s1}{\PYZsq{}}\PY{l+s+s1}{./data/}\PY{l+s+s1}{\PYZsq{}}\PY{p}{)}
\end{Verbatim}


    Sur cet objet la méthode \texttt{glob} nous retourne un itérable qui
contient ce qu'on veut:

    \begin{Verbatim}[commandchars=\\\{\}]
{\color{incolor}In [{\color{incolor}19}]:} \PY{c+c1}{\PYZsh{} tous les fichiers *.json dans le répertoire data/}
         \PY{k}{for} \PY{n}{json} \PY{o+ow}{in} \PY{n}{dirpath}\PY{o}{.}\PY{n}{glob}\PY{p}{(}\PY{l+s+s2}{\PYZdq{}}\PY{l+s+s2}{*.json}\PY{l+s+s2}{\PYZdq{}}\PY{p}{)}\PY{p}{:}
             \PY{n+nb}{print}\PY{p}{(}\PY{n}{json}\PY{p}{)}
\end{Verbatim}


    \begin{Verbatim}[commandchars=\\\{\}]
data/cities\_europe.json
data/cities\_foo.json
data/cities\_france.json
data/cities\_idf.json
data/cities\_nice.json
data/cities\_world.json
data/marine-abb.json
data/marine-e1-abb.json
data/marine-e1-ext.json
data/marine-e2-abb.json
data/marine-e2-ext.json
data/marine-ext.json
data/meteo\_europe.json
data/meteo\_france.json
data/meteo\_world.json

    \end{Verbatim}

    \hypertarget{documentation-compluxe8te}{%
\subparagraph{Documentation complète}\label{documentation-compluxe8te}}

    Voyez \href{https://docs.python.org/3/library/pathlib.html}{la
documentation complète ici}

    \hypertarget{compluxe9ment---niveau-avancuxe9}{%
\subsection{Complément - niveau
avancé}\label{compluxe9ment---niveau-avancuxe9}}

    Pour ceux qui sont déjà familiers avec les classes, j'en profite pour
vous faire remarquer le type de notre objet path

    \begin{Verbatim}[commandchars=\\\{\}]
{\color{incolor}In [{\color{incolor}20}]:} \PY{n+nb}{type}\PY{p}{(}\PY{n}{path}\PY{p}{)}
\end{Verbatim}


\begin{Verbatim}[commandchars=\\\{\}]
{\color{outcolor}Out[{\color{outcolor}20}]:} pathlib.PosixPath
\end{Verbatim}
            
    qui n'est pas \texttt{Path}, mais en fait une sous-classe de
\texttt{Path} qui est - sur la plateforme du MOOC au moins, qui
fonctionne sous linux - un objet de type \texttt{PosixPath}, qui est une
sous-classe de \texttt{Path}, comme vous pouvez le voir:

    \begin{Verbatim}[commandchars=\\\{\}]
{\color{incolor}In [{\color{incolor}21}]:} \PY{k+kn}{from} \PY{n+nn}{pathlib} \PY{k}{import} \PY{n}{PosixPath}
         \PY{n+nb}{issubclass}\PY{p}{(}\PY{n}{PosixPath}\PY{p}{,} \PY{n}{Path}\PY{p}{)}
\end{Verbatim}


\begin{Verbatim}[commandchars=\\\{\}]
{\color{outcolor}Out[{\color{outcolor}21}]:} True
\end{Verbatim}
            
    Ce qui fait que mécaniquement, path est bien une instance de
\texttt{Path}

    \begin{Verbatim}[commandchars=\\\{\}]
{\color{incolor}In [{\color{incolor}22}]:} \PY{n+nb}{isinstance}\PY{p}{(}\PY{n}{path}\PY{p}{,} \PY{n}{Path}\PY{p}{)}
\end{Verbatim}


\begin{Verbatim}[commandchars=\\\{\}]
{\color{outcolor}Out[{\color{outcolor}22}]:} True
\end{Verbatim}
            
    ce qui est heureux puisqu'on avait utilisé \texttt{Path()} pour
construire l'objet \texttt{path} au départ :)


    % Add a bibliography block to the postdoc
    
    
    
