    
    
    
    

    

    \hypertarget{exercice---niveau-basique}{%
\subsection{Exercice - niveau basique}\label{exercice---niveau-basique}}

    \hypertarget{tri-de-plusieurs-listes}{%
\subsubsection{Tri de plusieurs listes}\label{tri-de-plusieurs-listes}}

    \begin{Verbatim}[commandchars=\\\{\},frame=single,framerule=0.3mm,rulecolor=\color{cellframecolor}]
{\color{incolor}In [{\color{incolor}1}]:} \PY{c+c1}{\PYZsh{} pour charger l\PYZsq{}exercice}
        \PY{k+kn}{from} \PY{n+nn}{corrections}\PY{n+nn}{.}\PY{n+nn}{exo\PYZus{}multi\PYZus{}tri} \PY{k}{import} \PY{n}{exo\PYZus{}multi\PYZus{}tri}
\end{Verbatim}


    Écrivez une fonction qui~:

\begin{itemize}
\tightlist
\item
  accepte en argument une liste de listes,
\item
  et qui retourne \textbf{la même liste}, mais avec toutes les
  sous-listes \textbf{triées en place}.
\end{itemize}

    \begin{Verbatim}[commandchars=\\\{\},frame=single,framerule=0.3mm,rulecolor=\color{cellframecolor}]
{\color{incolor}In [{\color{incolor}2}]:} \PY{c+c1}{\PYZsh{} voici un exemple de ce qui est attendu}
        \PY{n}{exo\PYZus{}multi\PYZus{}tri}\PY{o}{.}\PY{n}{example}\PY{p}{(}\PY{p}{)}
\end{Verbatim}


\begin{Verbatim}[commandchars=\\\{\},frame=single,framerule=0.3mm,rulecolor=\color{cellframecolor}]
{\color{outcolor}Out[{\color{outcolor}2}]:} <IPython.core.display.HTML object>
\end{Verbatim}
            
    Écrivez votre code ici~:

    \begin{Verbatim}[commandchars=\\\{\},frame=single,framerule=0.3mm,rulecolor=\color{cellframecolor}]
{\color{incolor}In [{\color{incolor}3}]:} \PY{k}{def} \PY{n+nf}{multi\PYZus{}tri}\PY{p}{(}\PY{n}{listes}\PY{p}{)}\PY{p}{:} 
            \PY{l+s+s2}{\PYZdq{}}\PY{l+s+s2}{\PYZlt{}votre\PYZus{}code\PYZgt{}}\PY{l+s+s2}{\PYZdq{}}
\end{Verbatim}


    \begin{Verbatim}[commandchars=\\\{\},frame=single,framerule=0.3mm,rulecolor=\color{cellframecolor}]
{\color{incolor}In [{\color{incolor} }]:} \PY{c+c1}{\PYZsh{} NOTE}
        \PY{c+c1}{\PYZsh{} auto\PYZhy{}exec\PYZhy{}for\PYZhy{}latex has skipped execution of this cell}
        
        \PY{c+c1}{\PYZsh{} pour vérifier votre code}
        \PY{n}{exo\PYZus{}multi\PYZus{}tri}\PY{o}{.}\PY{n}{correction}\PY{p}{(}\PY{n}{multi\PYZus{}tri}\PY{p}{)}
\end{Verbatim}


    \hypertarget{exercice---niveau-intermuxe9diaire}{%
\subsection{Exercice - niveau
intermédiaire}\label{exercice---niveau-intermuxe9diaire}}

    \hypertarget{tri-de-plusieurs-listes-dans-des-directions-diffuxe9rentes}{%
\subsubsection{Tri de plusieurs listes, dans des directions
différentes}\label{tri-de-plusieurs-listes-dans-des-directions-diffuxe9rentes}}

    \begin{Verbatim}[commandchars=\\\{\},frame=single,framerule=0.3mm,rulecolor=\color{cellframecolor}]
{\color{incolor}In [{\color{incolor}4}]:} \PY{c+c1}{\PYZsh{} pour charger l\PYZsq{}exercice}
        \PY{k+kn}{from} \PY{n+nn}{corrections}\PY{n+nn}{.}\PY{n+nn}{exo\PYZus{}multi\PYZus{}tri\PYZus{}reverse} \PY{k}{import} \PY{n}{exo\PYZus{}multi\PYZus{}tri\PYZus{}reverse}
\end{Verbatim}


    Modifiez votre code pour qu'il accepte cette fois \textbf{deux}
arguments listes que l'on suppose de tailles égales.

Comme tout à l'heure le premier argument est une liste de listes à
trier.

À présent le second argument est une liste (ou un tuple) de booléens, de
même cardinal que le premier argument, et qui indiquent l'ordre dans
lequel on veut trier la liste d'entrée de même rang. \texttt{True}
signifie un tri descendant, \texttt{False} un tri ascendant.

Comme dans l'exercice \texttt{multi\_tri}, il s'agit de modifier en
place les données en entrée, et de retourner la liste de départ.

    \begin{Verbatim}[commandchars=\\\{\},frame=single,framerule=0.3mm,rulecolor=\color{cellframecolor}]
{\color{incolor}In [{\color{incolor}5}]:} \PY{c+c1}{\PYZsh{} Pour être un peu plus clair, voici à quoi on s\PYZsq{}attend}
        \PY{n}{exo\PYZus{}multi\PYZus{}tri\PYZus{}reverse}\PY{o}{.}\PY{n}{example}\PY{p}{(}\PY{p}{)}
\end{Verbatim}


\begin{Verbatim}[commandchars=\\\{\},frame=single,framerule=0.3mm,rulecolor=\color{cellframecolor}]
{\color{outcolor}Out[{\color{outcolor}5}]:} <IPython.core.display.HTML object>
\end{Verbatim}
            
    À vous de jouer~:

    \begin{Verbatim}[commandchars=\\\{\},frame=single,framerule=0.3mm,rulecolor=\color{cellframecolor}]
{\color{incolor}In [{\color{incolor}6}]:} \PY{k}{def} \PY{n+nf}{multi\PYZus{}tri\PYZus{}reverse}\PY{p}{(}\PY{n}{listes}\PY{p}{,} \PY{n}{reverses}\PY{p}{)}\PY{p}{:}
            \PY{l+s+s2}{\PYZdq{}}\PY{l+s+s2}{\PYZlt{}votre\PYZus{}code\PYZgt{}}\PY{l+s+s2}{\PYZdq{}}
\end{Verbatim}


    \begin{Verbatim}[commandchars=\\\{\},frame=single,framerule=0.3mm,rulecolor=\color{cellframecolor}]
{\color{incolor}In [{\color{incolor} }]:} \PY{c+c1}{\PYZsh{} NOTE}
        \PY{c+c1}{\PYZsh{} auto\PYZhy{}exec\PYZhy{}for\PYZhy{}latex has skipped execution of this cell}
        
        \PY{c+c1}{\PYZsh{} et pour vérifier votre code}
        \PY{n}{exo\PYZus{}multi\PYZus{}tri\PYZus{}reverse}\PY{o}{.}\PY{n}{correction}\PY{p}{(}\PY{n}{multi\PYZus{}tri\PYZus{}reverse}\PY{p}{)}
\end{Verbatim}



    % Add a bibliography block to the postdoc
    
    
    
