    \hypertarget{calculer-le-pgcd}{%
\section{Calculer le PGCD}\label{calculer-le-pgcd}}

    \hypertarget{exercice---niveau-basique}{%
\subsection{Exercice - niveau basique}\label{exercice---niveau-basique}}

    \begin{Verbatim}[commandchars=\\\{\}]
{\color{incolor}In [{\color{incolor} }]:} \PY{c+c1}{\PYZsh{} chargement de l\PYZsq{}exercice}
        \PY{k+kn}{from} \PY{n+nn}{corrections}\PY{n+nn}{.}\PY{n+nn}{exo\PYZus{}pgcd} \PY{k}{import} \PY{n}{exo\PYZus{}pgcd}
\end{Verbatim}


    On vous demande d'écrire une fonction qui calcule le PGCD de deux
entiers, en utilisant
\href{http://fr.wikipedia.org/wiki/Algorithme_d'Euclide}{l'algorithme
d'Euclide}.\\

    Les deux paramètres sont supposés être des entiers positifs ou nuls (pas
la peine de le vérifier).\\

Dans le cas où un des deux paramètres est nul, le PGCD vaut l'autre
paramètre. Ainsi par exemple:

    \begin{Verbatim}[commandchars=\\\{\}]
{\color{incolor}In [{\color{incolor} }]:} \PY{n}{exo\PYZus{}pgcd}\PY{o}{.}\PY{n}{example}\PY{p}{(}\PY{p}{)}
\end{Verbatim}


    \textbf{Remarque} on peut tout à fait utiliser une fonction récursive
pour implémenter l'algorithme d'Euclide. Par exemple cette version de
\texttt{pgcd} fonctionne très bien aussi (en supposant
a\textgreater{}=b)

\begin{verbatim}
def pgcd(a, b):
   "Le PGCD avec une fonction récursive"
   if not b:
       return a
   return pgcd(b, a % b)
\end{verbatim}

Cependant, il vous est demandé ici d'utiliser une boucle \texttt{while},
qui est le sujet de la séquence, pour implémenter \texttt{pgcd}.

    \begin{Verbatim}[commandchars=\\\{\}]
{\color{incolor}In [{\color{incolor} }]:} \PY{c+c1}{\PYZsh{} à vous de jouer}
        \PY{k}{def} \PY{n+nf}{pgcd}\PY{p}{(}\PY{n}{a}\PY{p}{,} \PY{n}{b}\PY{p}{)}\PY{p}{:}
            \PY{l+s+s2}{\PYZdq{}}\PY{l+s+s2}{\PYZlt{}votre code\PYZgt{}}\PY{l+s+s2}{\PYZdq{}}
\end{Verbatim}


    \begin{Verbatim}[commandchars=\\\{\}]
{\color{incolor}In [{\color{incolor} }]:} \PY{c+c1}{\PYZsh{} pour vérifier votre code}
        \PY{n}{exo\PYZus{}pgcd}\PY{o}{.}\PY{n}{correction}\PY{p}{(}\PY{n}{pgcd}\PY{p}{)}
\end{Verbatim}