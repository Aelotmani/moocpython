    
    
    
    

    

    \hypertarget{listes}{%
\section{Listes}\label{listes}}

    \hypertarget{exercice---niveau-basique}{%
\subsection{Exercice - niveau basique}\label{exercice---niveau-basique}}

    \begin{Verbatim}[commandchars=\\\{\}]
{\color{incolor}In [{\color{incolor}1}]:} \PY{k+kn}{from} \PY{n+nn}{corrections}\PY{n+nn}{.}\PY{n+nn}{exo\PYZus{}laccess} \PY{k}{import} \PY{n}{exo\PYZus{}laccess}
\end{Verbatim}


    Vous devez écrire une fonction \texttt{laccess} qui prend en argument
une liste, et qui retourne~:

\begin{itemize}
\tightlist
\item
  \texttt{None} si la liste est vide~;
\item
  sinon le dernier élément de la liste si elle est de taille paire~;
\item
  et sinon l'élément du milieu.
\end{itemize}

    \begin{Verbatim}[commandchars=\\\{\}]
{\color{incolor}In [{\color{incolor}2}]:} \PY{n}{exo\PYZus{}laccess}\PY{o}{.}\PY{n}{example}\PY{p}{(}\PY{p}{)}
\end{Verbatim}


\begin{Verbatim}[commandchars=\\\{\}]
{\color{outcolor}Out[{\color{outcolor}2}]:} <IPython.core.display.HTML object>
\end{Verbatim}
            
    \begin{Verbatim}[commandchars=\\\{\}]
{\color{incolor}In [{\color{incolor}3}]:} \PY{c+c1}{\PYZsh{} écrivez votre code ici}
        \PY{k}{def} \PY{n+nf}{laccess}\PY{p}{(}\PY{n}{liste}\PY{p}{)}\PY{p}{:}
            \PY{k}{return} \PY{l+s+s2}{\PYZdq{}}\PY{l+s+s2}{votre code}\PY{l+s+s2}{\PYZdq{}}
\end{Verbatim}


    \begin{Verbatim}[commandchars=\\\{\}]
{\color{incolor}In [{\color{incolor} }]:} \PY{c+c1}{\PYZsh{} NOTE}
        \PY{c+c1}{\PYZsh{} auto\PYZhy{}exec\PYZhy{}for\PYZhy{}latex has skipped execution of this cell}
        
        \PY{c+c1}{\PYZsh{} pour le corriger}
        \PY{n}{exo\PYZus{}laccess}\PY{o}{.}\PY{n}{correction}\PY{p}{(}\PY{n}{laccess}\PY{p}{)}
\end{Verbatim}


    Une fois que votre code fonctionne, vous pouvez regarder si par hasard
il marcherait aussi avec des chaînes~:

    \begin{Verbatim}[commandchars=\\\{\}]
{\color{incolor}In [{\color{incolor}4}]:} \PY{k+kn}{from} \PY{n+nn}{corrections}\PY{n+nn}{.}\PY{n+nn}{exo\PYZus{}laccess} \PY{k}{import} \PY{n}{exo\PYZus{}laccess\PYZus{}strings}
\end{Verbatim}


    \begin{Verbatim}[commandchars=\\\{\}]
{\color{incolor}In [{\color{incolor} }]:} \PY{c+c1}{\PYZsh{} NOTE}
        \PY{c+c1}{\PYZsh{} auto\PYZhy{}exec\PYZhy{}for\PYZhy{}latex has skipped execution of this cell}
        
        \PY{n}{exo\PYZus{}laccess\PYZus{}strings}\PY{o}{.}\PY{n}{correction}\PY{p}{(}\PY{n}{laccess}\PY{p}{)}
\end{Verbatim}



    % Add a bibliography block to the postdoc
    
    
    
