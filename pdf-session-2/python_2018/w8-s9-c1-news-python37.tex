    
    
    
    

    

    \hypertarget{nouveautuxe9s-de-python-3.7-dans-asyncio}{%
\section{\texorpdfstring{Nouveautés de Python-3.7 dans
\texttt{asyncio}}{Nouveautés de Python-3.7 dans asyncio}}\label{nouveautuxe9s-de-python-3.7-dans-asyncio}}

    \hypertarget{compluxe9ment---niveau-intermuxe9diaire}{%
\subsection{Complément - niveau
intermédiaire}\label{compluxe9ment---niveau-intermuxe9diaire}}

    Comme on l'a signalé au début de la semaine, \texttt{asyncio} a subi
quelques modifications dans Python-3.7, que nous allons rapidement
illustrer dans ce complément.

Nous verrons aussi par ailleurs une curiosité liée à la version de

    \hypertarget{documentation}{%
\subsubsection{Documentation}\label{documentation}}

    L'évolution la plus radicale est une refonte totale de la documentation.

C'est une très bonne nouvelle, car de ll'aveu même de Guido van Rossum,
la documentation en place pour les versions 3.5 et 3.6 était
particulièrement obscure;
\href{https://twitter.com/gvanrossum/status/1041889574052429826?lang=en}{voici
comment il l'a annoncé}~:

\begin{quote}
Finally the asyncio docs are not an embarrassment to us all.
\end{quote}

Si vous avez déjà eu l'occasion de parcourir ces anciennes
documentations, et que vous les avez trouvées imbittables, sachez que
vous n'êtes pas seul dans ce cas ;) Dans tous les cas je vous invite à
\href{https://docs.python.org/3/library/asyncio.html}{parcourir la
nouvelle version}, qui a le mérite d'apporter plus de réponses qu'elle
ne soulève d'interrogations. Ce n'était pas vraiment le cas avant.

    \hypertarget{accuxe8s-plus-facile}{%
\subsubsection{Accès plus facile}\label{accuxe8s-plus-facile}}

    Un certain nombre de changements ont été apportés à la librairie pour en
rendre l'accès plus facile.

\textbf{XXX}

    \hypertarget{pas-de-changement-de-fond}{%
\subsubsection{Pas de changement de
fond}\label{pas-de-changement-de-fond}}

    Par ailleurs en terme de l'utilisation de la librairie,

\textbf{XXX}

    \hypertarget{await-dans-ipython-7}{%
\subsubsection{\texorpdfstring{\texttt{await} dans
ipython-7}{await dans ipython-7}}\label{await-dans-ipython-7}}

Cette section ne s'applique pas \emph{stricto sensu} à Python-3.7, mais
à la version 7 de IPython.

Le sujet, c'est ici encore de raccourcir le \emph{boilerplate}
nécessaire pour faire tourner une coroutine.

    \hypertarget{python-standard}{%
\subparagraph{Python standard}\label{python-standard}}

    Voici d'abord ce qui se passe avec l'interpréteur Python standard~:

    \begin{Shaded}
\begin{Highlighting}[frame=lines,framerule=0.6mm,rulecolor=\color{asisframecolor}]
\NormalTok{$ python3}
\NormalTok{Python }\DecValTok{3}\NormalTok{.}\FloatTok{7.0}\NormalTok{ (default, Jun }\DecValTok{29} \DecValTok{2018}\NormalTok{, }\DecValTok{20}\NormalTok{:}\DecValTok{14}\NormalTok{:}\DecValTok{27}\NormalTok{)}
\NormalTok{[Clang }\DecValTok{9}\NormalTok{.}\FloatTok{0.0}\NormalTok{ (clang}\DecValTok{-900}\NormalTok{.}\DecValTok{0}\NormalTok{.}\FloatTok{39.2}\NormalTok{)] on darwin}
\NormalTok{Type }\StringTok{"help"}\NormalTok{, }\StringTok{"copyright"}\NormalTok{, }\StringTok{"credits"} \KeywordTok{or} \StringTok{"license"} \ControlFlowTok{for}\NormalTok{ more information.}
\OperatorTok{>>>} \ImportTok{import}\NormalTok{ asyncio}
\OperatorTok{>>>} \ControlFlowTok{async} \KeywordTok{def}\NormalTok{ foo(message, duree):}
\NormalTok{...    }\ControlFlowTok{await}\NormalTok{ asyncio.sleep(duree)}
\NormalTok{...    }\BuiltInTok{print}\NormalTok{(message)}
\NormalTok{...}
\OperatorTok{>>>} \ControlFlowTok{await}\NormalTok{(foo(}\StringTok{"hello"}\NormalTok{, }\DecValTok{1}\NormalTok{))}
\NormalTok{  File }\StringTok{"<stdin>"}\NormalTok{, line }\DecValTok{1}
\PreprocessorTok{SyntaxError}\NormalTok{: }\StringTok{'await'}\NormalTok{ outside function}
\end{Highlighting}
\end{Shaded}

    La syntaxe de Python nous interdit en effet d'utiliser \texttt{await} en
dehors du code d'une coroutine, on l'a vu dans une des vidéos.

    \hypertarget{ipython-7}{%
\subparagraph{IPython-7}\label{ipython-7}}

    Pour \textbf{simplifier encore} la mise en place de code asynchrone,
depuis ipython-7, on peut carrément déclencher une coroutine en
invoquant \texttt{await} dans la boucle principale de l'interpréteur :

    \begin{Shaded}
\begin{Highlighting}[frame=lines,framerule=0.6mm,rulecolor=\color{asisframecolor}]
\NormalTok{$ ipython3}
\NormalTok{Python }\DecValTok{3}\NormalTok{.}\FloatTok{7.0}\NormalTok{ (default, Jun }\DecValTok{29} \DecValTok{2018}\NormalTok{, }\DecValTok{20}\NormalTok{:}\DecValTok{14}\NormalTok{:}\DecValTok{27}\NormalTok{)}
\NormalTok{Type }\StringTok{'copyright'}\NormalTok{, }\StringTok{'credits'} \KeywordTok{or} \StringTok{'license'} \ControlFlowTok{for}\NormalTok{ more information}
\NormalTok{IPython }\DecValTok{7}\NormalTok{.}\FloatTok{0.1} \OperatorTok{--}\NormalTok{ An enhanced Interactive Python. Type }\StringTok{'?'} \ControlFlowTok{for} \BuiltInTok{help}\NormalTok{.}

\NormalTok{In [}\DecValTok{1}\NormalTok{]: }\ImportTok{import}\NormalTok{ asyncio}

\NormalTok{In [}\DecValTok{2}\NormalTok{]: }\ControlFlowTok{async} \KeywordTok{def}\NormalTok{ foo(message, duree):}
\NormalTok{   ...:     }\ControlFlowTok{await}\NormalTok{ asyncio.sleep(duree)}
\NormalTok{   ...:     }\BuiltInTok{print}\NormalTok{(message)}

\NormalTok{In [}\DecValTok{3}\NormalTok{]: }\ControlFlowTok{await}\NormalTok{(foo(}\StringTok{"hello"}\NormalTok{, }\FloatTok{0.5}\NormalTok{))}
\NormalTok{hello}

\NormalTok{In [}\DecValTok{4}\NormalTok{]: }\ControlFlowTok{await}\NormalTok{(asyncio.gather(foo(}\StringTok{"hello"}\NormalTok{, }\FloatTok{0.5}\NormalTok{), foo(}\StringTok{"world"}\NormalTok{, }\DecValTok{1}\NormalTok{)))}
\NormalTok{hello}
\NormalTok{world}
\NormalTok{Out[}\DecValTok{4}\NormalTok{]: [}\VariableTok{None}\NormalTok{, }\VariableTok{None}\NormalTok{]}
\end{Highlighting}
\end{Shaded}

    Du coup cette façon de faire fonctionnera aussi dans un notebook, si
vous avez la bonne version de IPython en dessous de Jupyter.

    \hypertarget{pour-en-savoir-plus}{%
\subsubsection{Pour en savoir plus}\label{pour-en-savoir-plus}}

https://docs.python.org/3/whatsnew/3.7.html\#whatsnew37-asyncio


    % Add a bibliography block to the postdoc
    
    
    
