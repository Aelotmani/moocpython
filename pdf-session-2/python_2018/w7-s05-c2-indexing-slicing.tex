    
    
    
    

    

    \hypertarget{index-et-slices}{%
\section{Index et slices}\label{index-et-slices}}

    \hypertarget{compluxe9ment---niveau-basique}{%
\subsection{Complément - niveau
basique}\label{compluxe9ment---niveau-basique}}

    \begin{Verbatim}[commandchars=\\\{\}]
{\color{incolor}In [{\color{incolor}1}]:} \PY{k+kn}{import} \PY{n+nn}{numpy} \PY{k}{as} \PY{n+nn}{np}
        \PY{k+kn}{import} \PY{n+nn}{matplotlib}\PY{n+nn}{.}\PY{n+nn}{pyplot} \PY{k}{as} \PY{n+nn}{plt}
        \PY{o}{\PYZpc{}}\PY{k}{matplotlib} inline
        \PY{n}{plt}\PY{o}{.}\PY{n}{ion}\PY{p}{(}\PY{p}{)}
\end{Verbatim}


    J'espère que vous êtes à présent convaincus qu'il est possible de faire
énormément de choses avec \texttt{numpy} en faisant des opérations entre
tableaux, et sans aller référencer un par un les éléments des tableaux,
ni faire de boucle \texttt{for}.

Il est temps maintenant de voir que l'on peut \emph{aussi} manipuler les
tableaux \texttt{numpy} avec des index.

    \hypertarget{indexation-par-des-entiers-et-tuples}{%
\subsubsection{Indexation par des entiers et
tuples}\label{indexation-par-des-entiers-et-tuples}}

    La façon la plus naturelle d'utiliser un tableau est habituellement à
l'aide des indices. On peut aussi bien sûr accéder aux éléments d'un
tableau \texttt{numpy} par des indices~:

    \begin{Verbatim}[commandchars=\\\{\}]
{\color{incolor}In [{\color{incolor}2}]:} \PY{c+c1}{\PYZsh{} une fonction qui crée un tableau}
        \PY{c+c1}{\PYZsh{} tab[i, j] = i + 10 * j}
        \PY{k}{def} \PY{n+nf}{background}\PY{p}{(}\PY{n}{n}\PY{p}{)}\PY{p}{:}
            \PY{n}{i} \PY{o}{=} \PY{n}{np}\PY{o}{.}\PY{n}{arange}\PY{p}{(}\PY{n}{n}\PY{p}{)}
            \PY{n}{j} \PY{o}{=} \PY{n}{i}\PY{o}{.}\PY{n}{reshape}\PY{p}{(}\PY{p}{(}\PY{n}{n}\PY{p}{,} \PY{l+m+mi}{1}\PY{p}{)}\PY{p}{)}
            \PY{k}{return} \PY{n}{i} \PY{o}{+} \PY{l+m+mi}{10} \PY{o}{*} \PY{n}{j}
\end{Verbatim}


    \begin{Verbatim}[commandchars=\\\{\}]
{\color{incolor}In [{\color{incolor}3}]:} \PY{n}{a5} \PY{o}{=} \PY{n}{background}\PY{p}{(}\PY{l+m+mi}{5}\PY{p}{)}
        \PY{n+nb}{print}\PY{p}{(}\PY{n}{a5}\PY{p}{)}
\end{Verbatim}


    \begin{Verbatim}[commandchars=\\\{\}]
[[ 0  1  2  3  4]
 [10 11 12 13 14]
 [20 21 22 23 24]
 [30 31 32 33 34]
 [40 41 42 43 44]]

    \end{Verbatim}

    Avec un seul index on obtient naturellement une ligne~:

    \begin{Verbatim}[commandchars=\\\{\}]
{\color{incolor}In [{\color{incolor}4}]:} \PY{n}{a5}\PY{p}{[}\PY{l+m+mi}{1}\PY{p}{]}
\end{Verbatim}


\begin{Verbatim}[commandchars=\\\{\}]
{\color{outcolor}Out[{\color{outcolor}4}]:} array([10, 11, 12, 13, 14])
\end{Verbatim}
            
    \begin{Verbatim}[commandchars=\\\{\}]
{\color{incolor}In [{\color{incolor}5}]:} \PY{c+c1}{\PYZsh{} que l\PYZsq{}on peut à nouveau indexer}
        \PY{n}{a5}\PY{p}{[}\PY{l+m+mi}{1}\PY{p}{]}\PY{p}{[}\PY{l+m+mi}{2}\PY{p}{]}
\end{Verbatim}


\begin{Verbatim}[commandchars=\\\{\}]
{\color{outcolor}Out[{\color{outcolor}5}]:} 12
\end{Verbatim}
            
    \begin{Verbatim}[commandchars=\\\{\}]
{\color{incolor}In [{\color{incolor}6}]:} \PY{c+c1}{\PYZsh{} ou plus simplement indexer par un tuple}
        \PY{n}{a5}\PY{p}{[}\PY{l+m+mi}{1}\PY{p}{,} \PY{l+m+mi}{2}\PY{p}{]}
\end{Verbatim}


\begin{Verbatim}[commandchars=\\\{\}]
{\color{outcolor}Out[{\color{outcolor}6}]:} 12
\end{Verbatim}
            
    \begin{Verbatim}[commandchars=\\\{\}]
{\color{incolor}In [{\color{incolor}7}]:} \PY{c+c1}{\PYZsh{} naturellement on peut affecter une case}
        \PY{c+c1}{\PYZsh{} individuellement}
        \PY{n}{a5}\PY{p}{[}\PY{l+m+mi}{2}\PY{p}{]}\PY{p}{[}\PY{l+m+mi}{1}\PY{p}{]} \PY{o}{=} \PY{l+m+mi}{221}
        \PY{n}{a5}\PY{p}{[}\PY{l+m+mi}{3}\PY{p}{,} \PY{l+m+mi}{2}\PY{p}{]} \PY{o}{+}\PY{o}{=} \PY{l+m+mi}{300}
        \PY{n+nb}{print}\PY{p}{(}\PY{n}{a5}\PY{p}{)}
\end{Verbatim}


    \begin{Verbatim}[commandchars=\\\{\}]
[[  0   1   2   3   4]
 [ 10  11  12  13  14]
 [ 20 221  22  23  24]
 [ 30  31 332  33  34]
 [ 40  41  42  43  44]]

    \end{Verbatim}

    \begin{Verbatim}[commandchars=\\\{\}]
{\color{incolor}In [{\color{incolor}8}]:} \PY{c+c1}{\PYZsh{} ou toute une ligne}
        \PY{n}{a5}\PY{p}{[}\PY{l+m+mi}{1}\PY{p}{]} \PY{o}{=} \PY{n}{np}\PY{o}{.}\PY{n}{arange}\PY{p}{(}\PY{l+m+mi}{100}\PY{p}{,} \PY{l+m+mi}{105}\PY{p}{)}
        \PY{n+nb}{print}\PY{p}{(}\PY{n}{a5}\PY{p}{)}
\end{Verbatim}


    \begin{Verbatim}[commandchars=\\\{\}]
[[  0   1   2   3   4]
 [100 101 102 103 104]
 [ 20 221  22  23  24]
 [ 30  31 332  33  34]
 [ 40  41  42  43  44]]

    \end{Verbatim}

    \begin{Verbatim}[commandchars=\\\{\}]
{\color{incolor}In [{\color{incolor}9}]:} \PY{c+c1}{\PYZsh{} et on on peut aussi changer}
        \PY{c+c1}{\PYZsh{} toute une ligne par broadcasting}
        \PY{n}{a5}\PY{p}{[}\PY{l+m+mi}{4}\PY{p}{]} \PY{o}{=} \PY{l+m+mi}{400}
        \PY{n+nb}{print}\PY{p}{(}\PY{n}{a5}\PY{p}{)}
\end{Verbatim}


    \begin{Verbatim}[commandchars=\\\{\}]
[[  0   1   2   3   4]
 [100 101 102 103 104]
 [ 20 221  22  23  24]
 [ 30  31 332  33  34]
 [400 400 400 400 400]]

    \end{Verbatim}

    \hypertarget{slicing}{%
\section{Slicing}\label{slicing}}

    Grâce au slicing on peut aussi référencer une colonne~:

    \begin{Verbatim}[commandchars=\\\{\}]
{\color{incolor}In [{\color{incolor}10}]:} \PY{n}{a5} \PY{o}{=} \PY{n}{background}\PY{p}{(}\PY{l+m+mi}{5}\PY{p}{)}
         \PY{n+nb}{print}\PY{p}{(}\PY{n}{a5}\PY{p}{)}
\end{Verbatim}


    \begin{Verbatim}[commandchars=\\\{\}]
[[ 0  1  2  3  4]
 [10 11 12 13 14]
 [20 21 22 23 24]
 [30 31 32 33 34]
 [40 41 42 43 44]]

    \end{Verbatim}

    \begin{Verbatim}[commandchars=\\\{\}]
{\color{incolor}In [{\color{incolor}11}]:} \PY{n}{a5}\PY{p}{[}\PY{p}{:}\PY{p}{,} \PY{l+m+mi}{3}\PY{p}{]}
\end{Verbatim}


\begin{Verbatim}[commandchars=\\\{\}]
{\color{outcolor}Out[{\color{outcolor}11}]:} array([ 3, 13, 23, 33, 43])
\end{Verbatim}
            
    C'est un tableau à une dimension, mais vous pouvez tout de même modifier
la colonne par une affectation~:

    \begin{Verbatim}[commandchars=\\\{\}]
{\color{incolor}In [{\color{incolor}12}]:} \PY{n}{a5}\PY{p}{[}\PY{p}{:}\PY{p}{,} \PY{l+m+mi}{3}\PY{p}{]} \PY{o}{=} \PY{n+nb}{range}\PY{p}{(}\PY{l+m+mi}{300}\PY{p}{,} \PY{l+m+mi}{305}\PY{p}{)}
         \PY{n+nb}{print}\PY{p}{(}\PY{n}{a5}\PY{p}{)}
\end{Verbatim}


    \begin{Verbatim}[commandchars=\\\{\}]
[[  0   1   2 300   4]
 [ 10  11  12 301  14]
 [ 20  21  22 302  24]
 [ 30  31  32 303  34]
 [ 40  41  42 304  44]]

    \end{Verbatim}

    Ou, ici également bien sûr, par broadcasting~:

    \begin{Verbatim}[commandchars=\\\{\}]
{\color{incolor}In [{\color{incolor}13}]:} \PY{c+c1}{\PYZsh{} on affecte un scalaire à une colonne}
         \PY{n}{a5}\PY{p}{[}\PY{p}{:}\PY{p}{,} \PY{l+m+mi}{2}\PY{p}{]} \PY{o}{=} \PY{l+m+mi}{200}
         \PY{n+nb}{print}\PY{p}{(}\PY{n}{a5}\PY{p}{)}
\end{Verbatim}


    \begin{Verbatim}[commandchars=\\\{\}]
[[  0   1 200 300   4]
 [ 10  11 200 301  14]
 [ 20  21 200 302  24]
 [ 30  31 200 303  34]
 [ 40  41 200 304  44]]

    \end{Verbatim}

    \begin{Verbatim}[commandchars=\\\{\}]
{\color{incolor}In [{\color{incolor}14}]:} \PY{c+c1}{\PYZsh{} ou on ajoute un scalaire à une colonne}
         \PY{n}{a5}\PY{p}{[}\PY{p}{:}\PY{p}{,} \PY{l+m+mi}{4}\PY{p}{]} \PY{o}{+}\PY{o}{=} \PY{l+m+mi}{400}
         \PY{n+nb}{print}\PY{p}{(}\PY{n}{a5}\PY{p}{)}
\end{Verbatim}


    \begin{Verbatim}[commandchars=\\\{\}]
[[  0   1 200 300 404]
 [ 10  11 200 301 414]
 [ 20  21 200 302 424]
 [ 30  31 200 303 434]
 [ 40  41 200 304 444]]

    \end{Verbatim}

    Les slices peuvent prendre une forme générale~:

    \begin{Verbatim}[commandchars=\\\{\}]
{\color{incolor}In [{\color{incolor}15}]:} \PY{n}{a8} \PY{o}{=} \PY{n}{background}\PY{p}{(}\PY{l+m+mi}{8}\PY{p}{)}
         \PY{n+nb}{print}\PY{p}{(}\PY{n}{a8}\PY{p}{)}
\end{Verbatim}


    \begin{Verbatim}[commandchars=\\\{\}]
[[ 0  1  2  3  4  5  6  7]
 [10 11 12 13 14 15 16 17]
 [20 21 22 23 24 25 26 27]
 [30 31 32 33 34 35 36 37]
 [40 41 42 43 44 45 46 47]
 [50 51 52 53 54 55 56 57]
 [60 61 62 63 64 65 66 67]
 [70 71 72 73 74 75 76 77]]

    \end{Verbatim}

    \begin{Verbatim}[commandchars=\\\{\}]
{\color{incolor}In [{\color{incolor}16}]:} \PY{c+c1}{\PYZsh{} toutes les lignes de rang 1, 4, 7}
         \PY{n}{a8}\PY{p}{[}\PY{l+m+mi}{1}\PY{p}{:}\PY{p}{:}\PY{l+m+mi}{3}\PY{p}{]}
\end{Verbatim}


\begin{Verbatim}[commandchars=\\\{\}]
{\color{outcolor}Out[{\color{outcolor}16}]:} array([[10, 11, 12, 13, 14, 15, 16, 17],
                [40, 41, 42, 43, 44, 45, 46, 47],
                [70, 71, 72, 73, 74, 75, 76, 77]])
\end{Verbatim}
            
    \begin{Verbatim}[commandchars=\\\{\}]
{\color{incolor}In [{\color{incolor}17}]:} \PY{c+c1}{\PYZsh{} toutes les colonnes de rang 1, 5, 9}
         \PY{n}{a8}\PY{p}{[}\PY{p}{:}\PY{p}{,} \PY{l+m+mi}{1}\PY{p}{:}\PY{p}{:}\PY{l+m+mi}{4}\PY{p}{]}
\end{Verbatim}


\begin{Verbatim}[commandchars=\\\{\}]
{\color{outcolor}Out[{\color{outcolor}17}]:} array([[ 1,  5],
                [11, 15],
                [21, 25],
                [31, 35],
                [41, 45],
                [51, 55],
                [61, 65],
                [71, 75]])
\end{Verbatim}
            
    \begin{Verbatim}[commandchars=\\\{\}]
{\color{incolor}In [{\color{incolor}18}]:} \PY{c+c1}{\PYZsh{} et on peut bien sûr les modifier}
         \PY{n}{a8}\PY{p}{[}\PY{p}{:}\PY{p}{,} \PY{l+m+mi}{1}\PY{p}{:}\PY{p}{:}\PY{l+m+mi}{4}\PY{p}{]} \PY{o}{=} \PY{l+m+mi}{0}
         \PY{n+nb}{print}\PY{p}{(}\PY{n}{a8}\PY{p}{)}
\end{Verbatim}


    \begin{Verbatim}[commandchars=\\\{\}]
[[ 0  0  2  3  4  0  6  7]
 [10  0 12 13 14  0 16 17]
 [20  0 22 23 24  0 26 27]
 [30  0 32 33 34  0 36 37]
 [40  0 42 43 44  0 46 47]
 [50  0 52 53 54  0 56 57]
 [60  0 62 63 64  0 66 67]
 [70  0 72 73 74  0 76 77]]

    \end{Verbatim}

    Du coup, le slicing peut servir à extraire des blocs~:

    \begin{Verbatim}[commandchars=\\\{\}]
{\color{incolor}In [{\color{incolor}19}]:} \PY{c+c1}{\PYZsh{} un bloc au hasard dans a8}
         \PY{n+nb}{print}\PY{p}{(}\PY{n}{a8}\PY{p}{[}\PY{l+m+mi}{5}\PY{p}{:}\PY{l+m+mi}{8}\PY{p}{,} \PY{l+m+mi}{2}\PY{p}{:}\PY{l+m+mi}{5}\PY{p}{]}\PY{p}{)}
\end{Verbatim}


    \begin{Verbatim}[commandchars=\\\{\}]
[[52 53 54]
 [62 63 64]
 [72 73 74]]

    \end{Verbatim}

    \hypertarget{newaxis}{%
\subsubsection{\texorpdfstring{\texttt{newaxis}}{newaxis}}\label{newaxis}}

    On peut utiliser également le symbole spécial \texttt{np.newaxis} en
conjonction avec un slice pour ``décaler'' les dimensions~:

    \begin{Verbatim}[commandchars=\\\{\}]
{\color{incolor}In [{\color{incolor}20}]:} \PY{n}{X} \PY{o}{=} \PY{n}{np}\PY{o}{.}\PY{n}{arange}\PY{p}{(}\PY{l+m+mi}{1}\PY{p}{,} \PY{l+m+mi}{7}\PY{p}{)}
         \PY{n+nb}{print}\PY{p}{(}\PY{n}{X}\PY{p}{)}
\end{Verbatim}


    \begin{Verbatim}[commandchars=\\\{\}]
[1 2 3 4 5 6]

    \end{Verbatim}

    \begin{Verbatim}[commandchars=\\\{\}]
{\color{incolor}In [{\color{incolor}21}]:} \PY{n}{X}\PY{o}{.}\PY{n}{shape}
\end{Verbatim}


\begin{Verbatim}[commandchars=\\\{\}]
{\color{outcolor}Out[{\color{outcolor}21}]:} (6,)
\end{Verbatim}
            
    \begin{Verbatim}[commandchars=\\\{\}]
{\color{incolor}In [{\color{incolor}22}]:} \PY{n}{Y} \PY{o}{=} \PY{n}{X}\PY{p}{[}\PY{p}{:}\PY{p}{,} \PY{n}{np}\PY{o}{.}\PY{n}{newaxis}\PY{p}{]}
         \PY{n+nb}{print}\PY{p}{(}\PY{n}{Y}\PY{p}{)}
\end{Verbatim}


    \begin{Verbatim}[commandchars=\\\{\}]
[[1]
 [2]
 [3]
 [4]
 [5]
 [6]]

    \end{Verbatim}

    \begin{Verbatim}[commandchars=\\\{\}]
{\color{incolor}In [{\color{incolor}23}]:} \PY{n}{Y}\PY{o}{.}\PY{n}{shape}
\end{Verbatim}


\begin{Verbatim}[commandchars=\\\{\}]
{\color{outcolor}Out[{\color{outcolor}23}]:} (6, 1)
\end{Verbatim}
            
    Et ainsi de suite~:

    \begin{Verbatim}[commandchars=\\\{\}]
{\color{incolor}In [{\color{incolor}24}]:} \PY{n}{Z} \PY{o}{=} \PY{n}{Y}\PY{p}{[}\PY{p}{:}\PY{p}{,} \PY{n}{np}\PY{o}{.}\PY{n}{newaxis}\PY{p}{]}
         \PY{n}{Z}
\end{Verbatim}


\begin{Verbatim}[commandchars=\\\{\}]
{\color{outcolor}Out[{\color{outcolor}24}]:} array([[[1]],
         
                [[2]],
         
                [[3]],
         
                [[4]],
         
                [[5]],
         
                [[6]]])
\end{Verbatim}
            
    \begin{Verbatim}[commandchars=\\\{\}]
{\color{incolor}In [{\color{incolor}25}]:} \PY{n}{Z}\PY{o}{.}\PY{n}{shape}
\end{Verbatim}


\begin{Verbatim}[commandchars=\\\{\}]
{\color{outcolor}Out[{\color{outcolor}25}]:} (6, 1, 1)
\end{Verbatim}
            
    De cette façon, par exemple, en combinant le slicing pour créer X et Y,
et le broadcasting pour créer leur somme, je peux créer facilement la
table de tous les tirages de 2 dés à 6 faces~:

    \begin{Verbatim}[commandchars=\\\{\}]
{\color{incolor}In [{\color{incolor}26}]:} \PY{n}{dice2} \PY{o}{=} \PY{n}{X} \PY{o}{+} \PY{n}{Y}
         \PY{n+nb}{print}\PY{p}{(}\PY{n}{dice2}\PY{p}{)}
\end{Verbatim}


    \begin{Verbatim}[commandchars=\\\{\}]
[[ 2  3  4  5  6  7]
 [ 3  4  5  6  7  8]
 [ 4  5  6  7  8  9]
 [ 5  6  7  8  9 10]
 [ 6  7  8  9 10 11]
 [ 7  8  9 10 11 12]]

    \end{Verbatim}

    Ou tous les tirages à trois dés~:

    \begin{Verbatim}[commandchars=\\\{\}]
{\color{incolor}In [{\color{incolor}27}]:} \PY{n}{dice3} \PY{o}{=} \PY{n}{X} \PY{o}{+} \PY{n}{Y} \PY{o}{+} \PY{n}{Z}
         \PY{n+nb}{print}\PY{p}{(}\PY{n}{dice3}\PY{p}{)}
\end{Verbatim}


    \begin{Verbatim}[commandchars=\\\{\}]
[[[ 3  4  5  6  7  8]
  [ 4  5  6  7  8  9]
  [ 5  6  7  8  9 10]
  [ 6  7  8  9 10 11]
  [ 7  8  9 10 11 12]
  [ 8  9 10 11 12 13]]

 [[ 4  5  6  7  8  9]
  [ 5  6  7  8  9 10]
  [ 6  7  8  9 10 11]
  [ 7  8  9 10 11 12]
  [ 8  9 10 11 12 13]
  [ 9 10 11 12 13 14]]

 [[ 5  6  7  8  9 10]
  [ 6  7  8  9 10 11]
  [ 7  8  9 10 11 12]
  [ 8  9 10 11 12 13]
  [ 9 10 11 12 13 14]
  [10 11 12 13 14 15]]

 [[ 6  7  8  9 10 11]
  [ 7  8  9 10 11 12]
  [ 8  9 10 11 12 13]
  [ 9 10 11 12 13 14]
  [10 11 12 13 14 15]
  [11 12 13 14 15 16]]

 [[ 7  8  9 10 11 12]
  [ 8  9 10 11 12 13]
  [ 9 10 11 12 13 14]
  [10 11 12 13 14 15]
  [11 12 13 14 15 16]
  [12 13 14 15 16 17]]

 [[ 8  9 10 11 12 13]
  [ 9 10 11 12 13 14]
  [10 11 12 13 14 15]
  [11 12 13 14 15 16]
  [12 13 14 15 16 17]
  [13 14 15 16 17 18]]]

    \end{Verbatim}

    J'en profite pour introduire un utilitaire qui n'a rien à voir, mais
avec \texttt{np.unique}, vous pourriez calculer le nombre d'occurrences
dans le tableau, et ainsi calculer les probabilités d'apparition de tous
les nombres entre 3 et 18~:

    \begin{Verbatim}[commandchars=\\\{\}]
{\color{incolor}In [{\color{incolor}28}]:} \PY{n}{np}\PY{o}{.}\PY{n}{unique}\PY{p}{(}\PY{n}{dice3}\PY{p}{,} \PY{n}{return\PYZus{}counts}\PY{o}{=}\PY{k+kc}{True}\PY{p}{)}
\end{Verbatim}


\begin{Verbatim}[commandchars=\\\{\}]
{\color{outcolor}Out[{\color{outcolor}28}]:} (array([ 3,  4,  5,  6,  7,  8,  9, 10, 11, 12, 13, 14, 15, 16, 17, 18]),
          array([ 1,  3,  6, 10, 15, 21, 25, 27, 27, 25, 21, 15, 10,  6,  3,  1]))
\end{Verbatim}
            
    \hypertarget{diffuxe9rences-avec-les-listes}{%
\subsubsection{Différences avec les
listes}\label{diffuxe9rences-avec-les-listes}}

    Avec l'indexation et le slicing, on peut créer des tableaux qui sont des
vues sur des fragments d'un tableau~; on peut également déformer leur
dimension grâce à \texttt{newaxis}~; on peut modifier ces fragments, en
utilisant un scalaire, un tableau, ou une slice sur un autre tableau.
Les possibilités sont infinies.

    Il est cependant utile de souligner quelques différences entre les
tableaux \texttt{numpy} et, les listes natives, pour ce qui concerne les
indexations et le \emph{slicing}.

    \hypertarget{on-ne-peut-pas-changer-la-taille-dun-tableau-avec-le-slicing}{%
\paragraph{On ne peut pas changer la taille d'un tableau avec le
slicing}\label{on-ne-peut-pas-changer-la-taille-dun-tableau-avec-le-slicing}}

La taille d'un objet \texttt{numpy} est par définition constante~; cela
signifie qu'on ne peut pas, par exemple, modifier sa taille totale avec
du slicing ; c'est à mettre en contraste avec, si vous vous souvenez~:

    \hypertarget{listes}{%
\subparagraph{Listes}\label{listes}}

    \begin{Verbatim}[commandchars=\\\{\}]
{\color{incolor}In [{\color{incolor}29}]:} \PY{c+c1}{\PYZsh{} on peut faire ceci}
         \PY{n}{liste} \PY{o}{=} \PY{p}{[}\PY{l+m+mi}{0}\PY{p}{,} \PY{l+m+mi}{1}\PY{p}{,} \PY{l+m+mi}{2}\PY{p}{]}
         \PY{n}{liste}\PY{p}{[}\PY{l+m+mi}{1}\PY{p}{:}\PY{l+m+mi}{2}\PY{p}{]} \PY{o}{=} \PY{p}{[}\PY{l+m+mi}{100}\PY{p}{,} \PY{l+m+mi}{102}\PY{p}{,} \PY{l+m+mi}{102}\PY{p}{]}
         \PY{n}{liste}
\end{Verbatim}


\begin{Verbatim}[commandchars=\\\{\}]
{\color{outcolor}Out[{\color{outcolor}29}]:} [0, 100, 102, 102, 2]
\end{Verbatim}
            
    \hypertarget{tableaux}{%
\subparagraph{Tableaux}\label{tableaux}}

    \begin{Verbatim}[commandchars=\\\{\}]
{\color{incolor}In [{\color{incolor}30}]:} \PY{c+c1}{\PYZsh{} on ne peut pas faire cela}
         \PY{n}{array} \PY{o}{=} \PY{n}{np}\PY{o}{.}\PY{n}{array}\PY{p}{(}\PY{p}{[}\PY{l+m+mi}{0}\PY{p}{,} \PY{l+m+mi}{1}\PY{p}{,} \PY{l+m+mi}{2}\PY{p}{]}\PY{p}{)}
         \PY{k}{try}\PY{p}{:}
             \PY{n}{array}\PY{p}{[}\PY{l+m+mi}{1}\PY{p}{:}\PY{l+m+mi}{2}\PY{p}{]} \PY{o}{=} \PY{n}{np}\PY{o}{.}\PY{n}{array}\PY{p}{(}\PY{p}{[}\PY{l+m+mi}{100}\PY{p}{,} \PY{l+m+mi}{102}\PY{p}{,} \PY{l+m+mi}{102}\PY{p}{]}\PY{p}{)}
         \PY{k}{except} \PY{n+ne}{Exception} \PY{k}{as} \PY{n}{e}\PY{p}{:}
             \PY{n+nb}{print}\PY{p}{(}\PY{n}{f}\PY{l+s+s2}{\PYZdq{}}\PY{l+s+s2}{OOPS, }\PY{l+s+s2}{\PYZob{}}\PY{l+s+s2}{type(e)\PYZcb{}, }\PY{l+s+si}{\PYZob{}e\PYZcb{}}\PY{l+s+s2}{\PYZdq{}}\PY{p}{)}
\end{Verbatim}


    \begin{Verbatim}[commandchars=\\\{\}]
OOPS, <class 'ValueError'>, could not broadcast input array from shape (3) into shape (1)

    \end{Verbatim}

    \hypertarget{on-peut-modifier-un-tableau-en-modifiant-une-slice}{%
\subparagraph{On peut modifier un tableau en modifiant une
slice}\label{on-peut-modifier-un-tableau-en-modifiant-une-slice}}

Une slice sur un objet \texttt{numpy} renvoie une \textbf{vue} sur un
extrait du tableau, et en changeant la vue on change le tableau~; ici
encore c'est à mettre en contraste avec ce qui se passe sur les listes~:

    \hypertarget{listes}{%
\subparagraph{Listes}\label{listes}}

    \begin{Verbatim}[commandchars=\\\{\}]
{\color{incolor}In [{\color{incolor}31}]:} \PY{c+c1}{\PYZsh{} une slice d\PYZsq{}une liste est une shallow copy}
         \PY{n}{liste} \PY{o}{=} \PY{p}{[}\PY{l+m+mi}{0}\PY{p}{,} \PY{l+m+mi}{1}\PY{p}{,} \PY{l+m+mi}{2}\PY{p}{]}
         \PY{n}{liste}\PY{p}{[}\PY{l+m+mi}{1}\PY{p}{:}\PY{l+m+mi}{2}\PY{p}{]}
\end{Verbatim}


\begin{Verbatim}[commandchars=\\\{\}]
{\color{outcolor}Out[{\color{outcolor}31}]:} [1]
\end{Verbatim}
            
    \begin{Verbatim}[commandchars=\\\{\}]
{\color{incolor}In [{\color{incolor}32}]:} \PY{c+c1}{\PYZsh{} en modifiant la slice,}
         \PY{c+c1}{\PYZsh{} on ne modifie pas la liste}
         \PY{n}{liste}\PY{p}{[}\PY{l+m+mi}{1}\PY{p}{:}\PY{l+m+mi}{2}\PY{p}{]}\PY{p}{[}\PY{l+m+mi}{0}\PY{p}{]} \PY{o}{=} \PY{l+m+mi}{999999}
         \PY{n}{liste}
\end{Verbatim}


\begin{Verbatim}[commandchars=\\\{\}]
{\color{outcolor}Out[{\color{outcolor}32}]:} [0, 1, 2]
\end{Verbatim}
            
    \hypertarget{tableaux}{%
\subparagraph{Tableaux}\label{tableaux}}

    \begin{Verbatim}[commandchars=\\\{\}]
{\color{incolor}In [{\color{incolor}33}]:} \PY{c+c1}{\PYZsh{} une slice d\PYZsq{}un tableau numpy est un extrait du tableau}
         \PY{n}{array} \PY{o}{=} \PY{n}{np}\PY{o}{.}\PY{n}{array}\PY{p}{(}\PY{p}{[}\PY{l+m+mi}{0}\PY{p}{,} \PY{l+m+mi}{1}\PY{p}{,} \PY{l+m+mi}{2}\PY{p}{]}\PY{p}{)}
         \PY{n}{array}\PY{p}{[}\PY{l+m+mi}{1}\PY{p}{:}\PY{l+m+mi}{2}\PY{p}{]}
\end{Verbatim}


\begin{Verbatim}[commandchars=\\\{\}]
{\color{outcolor}Out[{\color{outcolor}33}]:} array([1])
\end{Verbatim}
            
    \begin{Verbatim}[commandchars=\\\{\}]
{\color{incolor}In [{\color{incolor}34}]:} \PY{n}{array}\PY{p}{[}\PY{l+m+mi}{1}\PY{p}{:}\PY{l+m+mi}{2}\PY{p}{]}\PY{p}{[}\PY{l+m+mi}{0}\PY{p}{]} \PY{o}{=} \PY{l+m+mi}{100}
         \PY{n}{array}
\end{Verbatim}


\begin{Verbatim}[commandchars=\\\{\}]
{\color{outcolor}Out[{\color{outcolor}34}]:} array([  0, 100,   2])
\end{Verbatim}
            

    % Add a bibliography block to the postdoc
    
    
    
