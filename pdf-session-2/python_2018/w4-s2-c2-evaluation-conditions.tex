    
    
    
    

    

    \hypertarget{uxe9valuation-des-tests}{%
\section{Évaluation des tests}\label{uxe9valuation-des-tests}}

    \hypertarget{compluxe9ment---niveau-basique}{%
\subsection{Complément - niveau
basique}\label{compluxe9ment---niveau-basique}}

    \hypertarget{quels-tests-sont-uxe9valuuxe9s}{%
\subsubsection{Quels tests sont évalués
?}\label{quels-tests-sont-uxe9valuuxe9s}}

    On a vu dans la vidéo que l'instruction conditionnelle \texttt{if}
permet d'implémenter simplement des branchements à plusieurs choix,
comme dans cet exemple~:

    \begin{Verbatim}[commandchars=\\\{\}]
{\color{incolor}In [{\color{incolor}1}]:} \PY{n}{s} \PY{o}{=} \PY{l+s+s1}{\PYZsq{}}\PY{l+s+s1}{berlin}\PY{l+s+s1}{\PYZsq{}}
        \PY{k}{if} \PY{l+s+s1}{\PYZsq{}}\PY{l+s+s1}{a}\PY{l+s+s1}{\PYZsq{}} \PY{o+ow}{in} \PY{n}{s}\PY{p}{:}
            \PY{n+nb}{print}\PY{p}{(}\PY{l+s+s1}{\PYZsq{}}\PY{l+s+s1}{avec a}\PY{l+s+s1}{\PYZsq{}}\PY{p}{)}
        \PY{k}{elif} \PY{l+s+s1}{\PYZsq{}}\PY{l+s+s1}{b}\PY{l+s+s1}{\PYZsq{}} \PY{o+ow}{in} \PY{n}{s}\PY{p}{:}
            \PY{n+nb}{print}\PY{p}{(}\PY{l+s+s1}{\PYZsq{}}\PY{l+s+s1}{avec b}\PY{l+s+s1}{\PYZsq{}}\PY{p}{)}
        \PY{k}{elif} \PY{l+s+s1}{\PYZsq{}}\PY{l+s+s1}{c}\PY{l+s+s1}{\PYZsq{}} \PY{o+ow}{in} \PY{n}{s}\PY{p}{:}
            \PY{n+nb}{print}\PY{p}{(}\PY{l+s+s1}{\PYZsq{}}\PY{l+s+s1}{avec c}\PY{l+s+s1}{\PYZsq{}}\PY{p}{)}
        \PY{k}{else}\PY{p}{:}
            \PY{n+nb}{print}\PY{p}{(}\PY{l+s+s1}{\PYZsq{}}\PY{l+s+s1}{sans a ni b ni c}\PY{l+s+s1}{\PYZsq{}}\PY{p}{)}
\end{Verbatim}


    \begin{Verbatim}[commandchars=\\\{\}]
avec b

    \end{Verbatim}

    Comme on s'en doute, les expressions conditionnelles \textbf{sont
évaluées jusqu'à obtenir un résultat vrai} - ou considéré comme vrai -,
et le bloc correspondant est alors exécuté. Le point important ici est
qu'\textbf{une fois qu'on a obtenu un résultat vrai}, on sort de
l'expression conditionnelle \textbf{sans évaluer les autres conditions}.
En termes savants, on parle d'évaluation paresseuse : on s'arrête dès
qu'on peut.

    Dans notre exemple, on aura évalué à la sortie
\texttt{\textquotesingle{}a\textquotesingle{}\ in\ s}, et aussi
\texttt{\textquotesingle{}b\textquotesingle{}\ in\ s}, mais pas
\texttt{\textquotesingle{}c\textquotesingle{}\ in\ s}

    \hypertarget{pourquoi-cest-important}{%
\subsubsection{Pourquoi c'est important
?}\label{pourquoi-cest-important}}

    C'est important de bien comprendre quels sont les tests qui sont
réellement évalués pour deux raisons~:

\begin{itemize}
\tightlist
\item
  d'abord, pour des raisons de performance~; comme on n'évalue que les
  tests nécessaires, si un des tests prend du temps, il est peut-être
  préférable de le faire en dernier~;
\item
  mais aussi et surtout, il se peut tout à fait qu'un test fasse des
  \textbf{effets de bord}, c'est-à-dire qu'il modifie un ou plusieurs
  objets.
\end{itemize}

    Dans notre premier exemple, les conditions elles-mêmes sont
inoffensives~; la valeur de \texttt{s} reste \emph{identique}, que l'on
\emph{évalue ou non} les différentes conditions.

Mais nous allons voir ci-dessous qu'il est relativement facile d'écrire
des conditions qui \textbf{modifient} par \textbf{effet de bord} les
objets mutables sur lesquelles elles opèrent, et dans ce cas il est
crucial de bien assimiler la règle des évaluations des expressions dans
un \texttt{if}.

    \hypertarget{compluxe9ment---niveau-intermuxe9diaire}{%
\subsection{Complément - niveau
intermédiaire}\label{compluxe9ment---niveau-intermuxe9diaire}}

    \hypertarget{rappel-sur-la-muxe9thode-pop}{%
\subsubsection{\texorpdfstring{Rappel sur la méthode
\texttt{pop}}{Rappel sur la méthode pop}}\label{rappel-sur-la-muxe9thode-pop}}

    Pour illustrer la notion d'\textbf{effet de bord}, nous revenons sur la
méthode de liste \texttt{pop()} qui, on le rappelle, renvoie un élément
de liste \textbf{après l'avoir effacé} de la liste.

    \begin{Verbatim}[commandchars=\\\{\}]
{\color{incolor}In [{\color{incolor}2}]:} \PY{c+c1}{\PYZsh{} on se donne une liste}
        \PY{n}{liste} \PY{o}{=} \PY{p}{[}\PY{l+s+s1}{\PYZsq{}}\PY{l+s+s1}{premier}\PY{l+s+s1}{\PYZsq{}}\PY{p}{,} \PY{l+s+s1}{\PYZsq{}}\PY{l+s+s1}{deuxieme}\PY{l+s+s1}{\PYZsq{}}\PY{p}{,} \PY{l+s+s1}{\PYZsq{}}\PY{l+s+s1}{troisieme}\PY{l+s+s1}{\PYZsq{}}\PY{p}{]}
        \PY{n+nb}{print}\PY{p}{(}\PY{n}{f}\PY{l+s+s2}{\PYZdq{}}\PY{l+s+s2}{liste=}\PY{l+s+si}{\PYZob{}liste\PYZcb{}}\PY{l+s+s2}{\PYZdq{}}\PY{p}{)}
\end{Verbatim}


    \begin{Verbatim}[commandchars=\\\{\}]
liste=['premier', 'deuxieme', 'troisieme']

    \end{Verbatim}

    \begin{Verbatim}[commandchars=\\\{\}]
{\color{incolor}In [{\color{incolor}3}]:} \PY{c+c1}{\PYZsh{} pop(0) renvoie le premier élément de la liste, et raccourcit la liste}
        \PY{n}{element} \PY{o}{=} \PY{n}{liste}\PY{o}{.}\PY{n}{pop}\PY{p}{(}\PY{l+m+mi}{0}\PY{p}{)}
        \PY{n+nb}{print}\PY{p}{(}\PY{n}{f}\PY{l+s+s2}{\PYZdq{}}\PY{l+s+s2}{après pop(0), element=}\PY{l+s+si}{\PYZob{}element\PYZcb{}}\PY{l+s+s2}{ et liste=}\PY{l+s+si}{\PYZob{}liste\PYZcb{}}\PY{l+s+s2}{\PYZdq{}}\PY{p}{)}
\end{Verbatim}


    \begin{Verbatim}[commandchars=\\\{\}]
après pop(0), element=premier et liste=['deuxieme', 'troisieme']

    \end{Verbatim}

    \begin{Verbatim}[commandchars=\\\{\}]
{\color{incolor}In [{\color{incolor}4}]:} \PY{c+c1}{\PYZsh{} et ainsi de suite}
        \PY{n}{element} \PY{o}{=} \PY{n}{liste}\PY{o}{.}\PY{n}{pop}\PY{p}{(}\PY{l+m+mi}{0}\PY{p}{)}
        \PY{n+nb}{print}\PY{p}{(}\PY{n}{f}\PY{l+s+s2}{\PYZdq{}}\PY{l+s+s2}{après pop(0), element=}\PY{l+s+si}{\PYZob{}element\PYZcb{}}\PY{l+s+s2}{ et liste=}\PY{l+s+si}{\PYZob{}liste\PYZcb{}}\PY{l+s+s2}{\PYZdq{}}\PY{p}{)}
\end{Verbatim}


    \begin{Verbatim}[commandchars=\\\{\}]
après pop(0), element=deuxieme et liste=['troisieme']

    \end{Verbatim}

    \hypertarget{conditions-avec-effet-de-bord}{%
\subsubsection{Conditions avec effet de
bord}\label{conditions-avec-effet-de-bord}}

    Une fois ce rappel fait, voyons maintenant l'exemple suivant~:

    \begin{Verbatim}[commandchars=\\\{\}]
{\color{incolor}In [{\color{incolor}5}]:} \PY{n}{liste} \PY{o}{=} \PY{n+nb}{list}\PY{p}{(}\PY{n+nb}{range}\PY{p}{(}\PY{l+m+mi}{5}\PY{p}{)}\PY{p}{)}
        \PY{n+nb}{print}\PY{p}{(}\PY{l+s+s1}{\PYZsq{}}\PY{l+s+s1}{liste en entree:}\PY{l+s+s1}{\PYZsq{}}\PY{p}{,} \PY{n}{liste}\PY{p}{,} \PY{l+s+s1}{\PYZsq{}}\PY{l+s+s1}{de taille}\PY{l+s+s1}{\PYZsq{}}\PY{p}{,} \PY{n+nb}{len}\PY{p}{(}\PY{n}{liste}\PY{p}{)}\PY{p}{)}
\end{Verbatim}


    \begin{Verbatim}[commandchars=\\\{\}]
liste en entree: [0, 1, 2, 3, 4] de taille 5

    \end{Verbatim}

    \begin{Verbatim}[commandchars=\\\{\}]
{\color{incolor}In [{\color{incolor}6}]:} \PY{k}{if} \PY{n}{liste}\PY{o}{.}\PY{n}{pop}\PY{p}{(}\PY{l+m+mi}{0}\PY{p}{)} \PY{o}{\PYZlt{}}\PY{o}{=} \PY{l+m+mi}{0}\PY{p}{:}
            \PY{n+nb}{print}\PY{p}{(}\PY{l+s+s1}{\PYZsq{}}\PY{l+s+s1}{cas 1}\PY{l+s+s1}{\PYZsq{}}\PY{p}{)}
        \PY{k}{elif} \PY{n}{liste}\PY{o}{.}\PY{n}{pop}\PY{p}{(}\PY{l+m+mi}{0}\PY{p}{)} \PY{o}{\PYZlt{}}\PY{o}{=} \PY{l+m+mi}{1}\PY{p}{:}
            \PY{n+nb}{print}\PY{p}{(}\PY{l+s+s1}{\PYZsq{}}\PY{l+s+s1}{cas 2}\PY{l+s+s1}{\PYZsq{}}\PY{p}{)}
        \PY{k}{elif} \PY{n}{liste}\PY{o}{.}\PY{n}{pop}\PY{p}{(}\PY{l+m+mi}{0}\PY{p}{)} \PY{o}{\PYZlt{}}\PY{o}{=} \PY{l+m+mi}{2}\PY{p}{:}
            \PY{n+nb}{print}\PY{p}{(}\PY{l+s+s1}{\PYZsq{}}\PY{l+s+s1}{cas 3}\PY{l+s+s1}{\PYZsq{}}\PY{p}{)}
        \PY{k}{else}\PY{p}{:}
            \PY{n+nb}{print}\PY{p}{(}\PY{l+s+s1}{\PYZsq{}}\PY{l+s+s1}{cas 4}\PY{l+s+s1}{\PYZsq{}}\PY{p}{)}
        \PY{n+nb}{print}\PY{p}{(}\PY{l+s+s1}{\PYZsq{}}\PY{l+s+s1}{liste en sortie de taille}\PY{l+s+s1}{\PYZsq{}}\PY{p}{,} \PY{n+nb}{len}\PY{p}{(}\PY{n}{liste}\PY{p}{)}\PY{p}{)}
\end{Verbatim}


    \begin{Verbatim}[commandchars=\\\{\}]
cas 1
liste en sortie de taille 4

    \end{Verbatim}

    Avec cette entrée, le premier test est vrai (car \texttt{pop(0)} renvoie
0), aussi on n'exécute en tout \texttt{pop()} qu'\textbf{une seule
fois}, et donc à la sortie la liste n'a été raccourcie que d'un élément.

    Exécutons à présent le même code avec une entrée différente~:

    \begin{Verbatim}[commandchars=\\\{\}]
{\color{incolor}In [{\color{incolor}7}]:} \PY{n}{liste} \PY{o}{=} \PY{n+nb}{list}\PY{p}{(}\PY{n+nb}{range}\PY{p}{(}\PY{l+m+mi}{5}\PY{p}{,} \PY{l+m+mi}{10}\PY{p}{)}\PY{p}{)}
        \PY{n+nb}{print}\PY{p}{(}\PY{l+s+s1}{\PYZsq{}}\PY{l+s+s1}{en entree: liste=}\PY{l+s+s1}{\PYZsq{}}\PY{p}{,} \PY{n}{liste}\PY{p}{,} \PY{l+s+s1}{\PYZsq{}}\PY{l+s+s1}{de taille}\PY{l+s+s1}{\PYZsq{}}\PY{p}{,} \PY{n+nb}{len}\PY{p}{(}\PY{n}{liste}\PY{p}{)}\PY{p}{)}
\end{Verbatim}


    \begin{Verbatim}[commandchars=\\\{\}]
en entree: liste= [5, 6, 7, 8, 9] de taille 5

    \end{Verbatim}

    \begin{Verbatim}[commandchars=\\\{\}]
{\color{incolor}In [{\color{incolor}8}]:} \PY{k}{if} \PY{n}{liste}\PY{o}{.}\PY{n}{pop}\PY{p}{(}\PY{l+m+mi}{0}\PY{p}{)} \PY{o}{\PYZlt{}}\PY{o}{=} \PY{l+m+mi}{0}\PY{p}{:}
            \PY{n+nb}{print}\PY{p}{(}\PY{l+s+s1}{\PYZsq{}}\PY{l+s+s1}{cas 1}\PY{l+s+s1}{\PYZsq{}}\PY{p}{)}
        \PY{k}{elif} \PY{n}{liste}\PY{o}{.}\PY{n}{pop}\PY{p}{(}\PY{l+m+mi}{0}\PY{p}{)} \PY{o}{\PYZlt{}}\PY{o}{=} \PY{l+m+mi}{1}\PY{p}{:}
            \PY{n+nb}{print}\PY{p}{(}\PY{l+s+s1}{\PYZsq{}}\PY{l+s+s1}{cas 2}\PY{l+s+s1}{\PYZsq{}}\PY{p}{)}
        \PY{k}{elif} \PY{n}{liste}\PY{o}{.}\PY{n}{pop}\PY{p}{(}\PY{l+m+mi}{0}\PY{p}{)} \PY{o}{\PYZlt{}}\PY{o}{=} \PY{l+m+mi}{2}\PY{p}{:}
            \PY{n+nb}{print}\PY{p}{(}\PY{l+s+s1}{\PYZsq{}}\PY{l+s+s1}{cas 3}\PY{l+s+s1}{\PYZsq{}}\PY{p}{)}
        \PY{k}{else}\PY{p}{:}
            \PY{n+nb}{print}\PY{p}{(}\PY{l+s+s1}{\PYZsq{}}\PY{l+s+s1}{cas 4}\PY{l+s+s1}{\PYZsq{}}\PY{p}{)}
        \PY{n+nb}{print}\PY{p}{(}\PY{l+s+s1}{\PYZsq{}}\PY{l+s+s1}{en sortie: liste=}\PY{l+s+s1}{\PYZsq{}}\PY{p}{,} \PY{n}{liste}\PY{p}{,} \PY{l+s+s1}{\PYZsq{}}\PY{l+s+s1}{de taille}\PY{l+s+s1}{\PYZsq{}}\PY{p}{,} \PY{n+nb}{len}\PY{p}{(}\PY{n}{liste}\PY{p}{)}\PY{p}{)}
\end{Verbatim}


    \begin{Verbatim}[commandchars=\\\{\}]
cas 4
en sortie: liste= [8, 9] de taille 2

    \end{Verbatim}

    On observe que cette fois la liste a été \textbf{raccourcie 3 fois}, car
les trois tests se sont révélés faux.

    Cet exemple vous montre qu'il faut être attentif avec des conditions qui
font des effets de bord. Bien entendu, ce type de pratique est de
manière générale à utiliser avec beaucoup de discernement.

    \hypertarget{court-circuit-short-circuit}{%
\subsubsection{\texorpdfstring{Court-circuit
(\emph{short-circuit})}{Court-circuit (short-circuit)}}\label{court-circuit-short-circuit}}

    La logique que l'on vient de voir est celle qui s'applique aux
différentes branches d'un \texttt{if}~; c'est la même logique qui est à
l'œuvre aussi lorsque python évalue une condition logique à base de
\texttt{and} et \texttt{or}. C'est ici aussi une forme d'évaluation
paresseuse.

    Pour illustrer cela, nous allons nous définir deux fonctions toutes
simples qui renvoient \texttt{True} et \texttt{False} mais avec une
impression de sorte qu'on voit lorsqu'elles sont exécutées~:

    \begin{Verbatim}[commandchars=\\\{\}]
{\color{incolor}In [{\color{incolor}9}]:} \PY{k}{def} \PY{n+nf}{true}\PY{p}{(}\PY{p}{)}\PY{p}{:}
            \PY{n+nb}{print}\PY{p}{(}\PY{l+s+s1}{\PYZsq{}}\PY{l+s+s1}{true}\PY{l+s+s1}{\PYZsq{}}\PY{p}{)}
            \PY{k}{return} \PY{k+kc}{True}
\end{Verbatim}


    \begin{Verbatim}[commandchars=\\\{\}]
{\color{incolor}In [{\color{incolor}10}]:} \PY{k}{def} \PY{n+nf}{false}\PY{p}{(}\PY{p}{)}\PY{p}{:}
             \PY{n+nb}{print}\PY{p}{(}\PY{l+s+s1}{\PYZsq{}}\PY{l+s+s1}{false}\PY{l+s+s1}{\PYZsq{}}\PY{p}{)}
             \PY{k}{return} \PY{k+kc}{False}
\end{Verbatim}


    \begin{Verbatim}[commandchars=\\\{\}]
{\color{incolor}In [{\color{incolor}11}]:} \PY{n}{true}\PY{p}{(}\PY{p}{)}
\end{Verbatim}


    \begin{Verbatim}[commandchars=\\\{\}]
true

    \end{Verbatim}

\begin{Verbatim}[commandchars=\\\{\}]
{\color{outcolor}Out[{\color{outcolor}11}]:} True
\end{Verbatim}
            
    Ceci va nous permettre d'illustrer notre point, qui est que lorsque
python évalue un \texttt{and} ou un \texttt{or}, il \textbf{n'évalue la
deuxième condition que si c'est nécessaire}. Ainsi par exemple~:

    \begin{Verbatim}[commandchars=\\\{\}]
{\color{incolor}In [{\color{incolor}12}]:} \PY{n}{false}\PY{p}{(}\PY{p}{)} \PY{o+ow}{and} \PY{n}{true}\PY{p}{(}\PY{p}{)}
\end{Verbatim}


    \begin{Verbatim}[commandchars=\\\{\}]
false

    \end{Verbatim}

\begin{Verbatim}[commandchars=\\\{\}]
{\color{outcolor}Out[{\color{outcolor}12}]:} False
\end{Verbatim}
            
    Dans ce cas, python évalue la première partie du \texttt{and} - qui
provoque l'impression de \texttt{false} - et comme le résultat est faux,
il n'est \textbf{pas nécessaire} d'évaluer la seconde condition, on sait
que de toute façon le résultat du \texttt{and} est forcément faux. C'est
pourquoi vous ne voyez pas l'impression de \texttt{true}.

    De manière symétrique avec un \texttt{or}~:

    \begin{Verbatim}[commandchars=\\\{\}]
{\color{incolor}In [{\color{incolor}13}]:} \PY{n}{true}\PY{p}{(}\PY{p}{)} \PY{o+ow}{or} \PY{n}{false}\PY{p}{(}\PY{p}{)}
\end{Verbatim}


    \begin{Verbatim}[commandchars=\\\{\}]
true

    \end{Verbatim}

\begin{Verbatim}[commandchars=\\\{\}]
{\color{outcolor}Out[{\color{outcolor}13}]:} True
\end{Verbatim}
            
    À nouveau ici il n'est pas nécessaire d'évaluer \texttt{false()}, et
donc seul \texttt{true} est imprimé à l'évaluation.

    À titre d'exercice, essayez de dire combien d'impressions sont émises
lorsqu'on évalue cette expression un peu plus compliquée~:

    \begin{Verbatim}[commandchars=\\\{\}]
{\color{incolor}In [{\color{incolor}14}]:} \PY{n}{true}\PY{p}{(}\PY{p}{)} \PY{o+ow}{and} \PY{p}{(}\PY{n}{false}\PY{p}{(}\PY{p}{)} \PY{o+ow}{or} \PY{n}{true}\PY{p}{(}\PY{p}{)}\PY{p}{)} \PY{o+ow}{or} \PY{p}{(}\PY{n}{true} \PY{p}{(}\PY{p}{)} \PY{o+ow}{and} \PY{n}{false}\PY{p}{(}\PY{p}{)}\PY{p}{)}
\end{Verbatim}


    \begin{Verbatim}[commandchars=\\\{\}]
true
false
true

    \end{Verbatim}

\begin{Verbatim}[commandchars=\\\{\}]
{\color{outcolor}Out[{\color{outcolor}14}]:} True
\end{Verbatim}
            

    % Add a bibliography block to the postdoc
    
    
    
