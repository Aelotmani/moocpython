    
    
    
    

    

    \hypertarget{modes-dexuxe9cution}{%
\section{Modes d'exécution}\label{modes-dexuxe9cution}}

    Nous avons donc à notre disposition plusieurs façons d'exécuter un
programme Python. Nous allons les étudier plus en détail~:

    \begin{longtable}[]{@{}ll@{}}
\toprule
Quoi & Avec quel outil\tabularnewline
\midrule
\endhead
fichier complet &
\texttt{python3\ \textless{}fichier\textgreater{}.py}\tabularnewline
ligne à ligne & \texttt{python3} en mode interactif\tabularnewline
~ & ou sous \texttt{ipython3}\tabularnewline
~ & ou avec IDLE\tabularnewline
par fragments & dans un notebook\tabularnewline
\bottomrule
\end{longtable}

    Pour cela nous allons voir le comportement d'un tout petit programme
Python lorsqu'on l'exécute sous ces différents environnements.

On veut surtout expliquer une petite différence quant au niveau de
détail de ce qui se trouve imprimé.

    Essentiellement, lorsqu'on utilise l'interpréteur en mode interactif -
ou sous IDLE - à chaque fois que l'on tape une ligne, le résultat est
\textbf{calculé} (on dit aussi \textbf{évalué}) puis \textbf{imprimé}.

Par contre, lorsqu'on écrit tout un programme, on ne peut plus imprimer
le résultat de toutes les lignes, cela produirait un flot d'impression
beaucoup trop important. Par conséquent, si vous ne déclenchez pas une
impression avec, par exemple, la fonction \texttt{print}, rien ne
s'affichera.

Enfin, en ce qui concerne le notebook, le comportement est un peu
hybride entre les deux, en ce sens que seul le \textbf{dernier résultat}
de la cellule est imprimé.

    \hypertarget{linterpruxe9teur-python-interactif}{%
\subsubsection{L'interpréteur Python
interactif}\label{linterpruxe9teur-python-interactif}}

    Le programme choisi est très simple, c'est le suivant~:

    \begin{Shaded}
\begin{Highlighting}[]
\DecValTok{10} \OperatorTok{*} \DecValTok{10}
\DecValTok{20} \OperatorTok{*} \DecValTok{20}
\DecValTok{30} \OperatorTok{*} \DecValTok{30}
\end{Highlighting}
\end{Shaded}

    Voici comment se comporte l'interpréteur interactif quand on lui soumet
ces instructions~:

    \begin{verbatim}
$ python3
Python 3.5.1 (v3.5.1:37a07cee5969, Dec  5 2015, 21:12:44)
[GCC 4.2.1 (Apple Inc. build 5666) (dot 3)] on darwin
Type "help", "copyright", "credits" or "license" for more information.
>>> 10 * 10
100
>>> 20 * 20
400
>>> 30 * 30
900
>>> exit()
$
\end{verbatim}

    Notez que pour terminer la session, il nous faut ``sortir'' de
l'interpréteur en tapant \texttt{exit()}.

On peut aussi taper \texttt{Control-D} sous Linux ou macOS.

    Comme on le voit ici, l'interpréteur imprime \textbf{le résultat de
chaque ligne}. On voit bien apparaître toutes les valeurs calculées,
100, 400, puis enfin 900.

    \hypertarget{sous-forme-de-programme-constituuxe9}{%
\subsubsection{Sous forme de programme
constitué}\label{sous-forme-de-programme-constituuxe9}}

    Voyons à présent ce que donne cette même séquence de calculs dans un
programme complet. Pour cela, il nous faut tout d'abord fabriquer un
fichier avec un suffixe en \texttt{.py}, en utilisant par exemple un
éditeur de fichier. Le résultat doit ressembler à ceci~:

    \begin{verbatim}
$ cat foo.py
10 * 10
20 * 20
30 * 30
$
\end{verbatim}

    Exécutons à présent ce programme~:

    \begin{verbatim}
$ python3 foo.py
$
\end{verbatim}

    On constate donc que ce programme \textbf{ne fait rien~!} En tout cas,
selon toute apparence.

En réalité, les 3 valeurs 100, 400 et 900 sont bien calculées, mais
comme aucune instruction \texttt{print} n'est présente, rien n'est
imprimé et le programme se termine sans signe apparent d'avoir
réellement fonctionné.

Ce comportement peut paraître un peu déroutant au début, mais comme nous
l'avons mentionné c'est tout à fait délibéré. Un programme fonctionnel
faisant facilement plusieurs milliers de lignes, voire beaucoup plus, il
ne serait pas du tout réaliste que chaque ligne produise une impression,
comme c'est le cas en mode interactif.

    \hypertarget{dans-un-notebook}{%
\subsubsection{Dans un notebook}\label{dans-un-notebook}}

    Voici à présent le même programme dans un notebook~:

    \begin{Verbatim}[commandchars=\\\{\}]
{\color{incolor}In [{\color{incolor}1}]:} \PY{l+m+mi}{10} \PY{o}{*} \PY{l+m+mi}{10}
        \PY{l+m+mi}{20} \PY{o}{*} \PY{l+m+mi}{20}
        \PY{l+m+mi}{30} \PY{o}{*} \PY{l+m+mi}{30}
\end{Verbatim}


\begin{Verbatim}[commandchars=\\\{\}]
{\color{outcolor}Out[{\color{outcolor}1}]:} 900
\end{Verbatim}
            
    Lorsqu'on exécute cette cellule (rappel~: sélectionner la cellule, et
utiliser le bouton en forme de flèche vers la droite, ou entrer
\textbf{``Shift+Enter''} au clavier), on obtient une seule valeur dans
la rubrique \texttt{Out{[}{]}}, 900, qui correspond \textbf{au résultat
de la dernière ligne.}

    \hypertarget{utiliser-print}{%
\subsubsection{\texorpdfstring{Utiliser
\texttt{print}}{Utiliser print}}\label{utiliser-print}}

    Ainsi, pour afficher un résultat intermédiaire, on utilise l'instruction
\texttt{print}. Nous verrons cette instruction en détail dans les
semaines qui viennent, mais en guise d'introduction disons seulement que
c'est une fonction comme les autres en Python 3.

    \begin{Verbatim}[commandchars=\\\{\}]
{\color{incolor}In [{\color{incolor}2}]:} \PY{n}{a} \PY{o}{=} \PY{l+m+mi}{10}
        \PY{n}{b} \PY{o}{=} \PY{l+m+mi}{20}
        
        \PY{n+nb}{print}\PY{p}{(}\PY{n}{a}\PY{p}{,} \PY{n}{b}\PY{p}{)}
\end{Verbatim}


    \begin{Verbatim}[commandchars=\\\{\}]
10 20

    \end{Verbatim}

    On peut naturellement mélanger des objets de plusieurs types, et donc
mélanger des chaînes de caractères et des nombres pour obtenir un
résultat un peu plus lisible. En effet, lorsque le programme devient
gros, il est important de savoir à quoi correspond une ligne dans le
flot de toutes les impressions. Aussi on préfèrera quelque chose comme~:

    \begin{Verbatim}[commandchars=\\\{\}]
{\color{incolor}In [{\color{incolor}3}]:} \PY{n+nb}{print}\PY{p}{(}\PY{l+s+s2}{\PYZdq{}}\PY{l+s+s2}{a =}\PY{l+s+s2}{\PYZdq{}}\PY{p}{,} \PY{n}{a}\PY{p}{,} \PY{l+s+s2}{\PYZdq{}}\PY{l+s+s2}{et b =}\PY{l+s+s2}{\PYZdq{}}\PY{p}{,} \PY{n}{b}\PY{p}{)}
\end{Verbatim}


    \begin{Verbatim}[commandchars=\\\{\}]
a = 10 et b = 20

    \end{Verbatim}

    \begin{Verbatim}[commandchars=\\\{\}]
{\color{incolor}In [{\color{incolor}4}]:} \PY{c+c1}{\PYZsh{} ou encore, équivalente mais avec un f\PYZhy{}string}
        \PY{n+nb}{print}\PY{p}{(}\PY{n}{f}\PY{l+s+s2}{\PYZdq{}}\PY{l+s+s2}{a = }\PY{l+s+si}{\PYZob{}a\PYZcb{}}\PY{l+s+s2}{ et b = }\PY{l+s+si}{\PYZob{}b\PYZcb{}}\PY{l+s+s2}{\PYZdq{}}\PY{p}{)}
\end{Verbatim}


    \begin{Verbatim}[commandchars=\\\{\}]
a = 10 et b = 20

    \end{Verbatim}

    Une pratique courante consiste d'ailleurs à utiliser les commentaires
pour laisser dans le code les instructions \texttt{print} qui
correspondent à du debug (c'est-à-dire qui ont pu être utiles lors de la
mise au point et qu'on veut pouvoir réactiver rapidement).

    \hypertarget{utiliser-print-pour-sous-titrer-une-affectation}{%
\subsubsection{\texorpdfstring{Utiliser \texttt{print} pour
``sous-titrer'' une
affectation}{Utiliser print pour ``sous-titrer'' une affectation}}\label{utiliser-print-pour-sous-titrer-une-affectation}}

    Remarquons enfin que l'affectation à une variable ne retourne aucun
résultat.

C'est-à-dire, en pratique, que si on écrit~:

    \begin{Verbatim}[commandchars=\\\{\}]
{\color{incolor}In [{\color{incolor}5}]:} \PY{n}{a} \PY{o}{=} \PY{l+m+mi}{100}
\end{Verbatim}


    même une fois l'expression évaluée par l'interpréteur, aucune ligne
\texttt{Out{[}{]}} n'est ajoutée.

    C'est pourquoi, il nous arrivera parfois d'écrire, notamment lorsque
l'expression est complexe et pour rendre explicite la valeur qui vient
d'être affectée~:

    \begin{Verbatim}[commandchars=\\\{\}]
{\color{incolor}In [{\color{incolor}6}]:} \PY{n}{a} \PY{o}{=} \PY{l+m+mi}{100}\PY{p}{;} \PY{n+nb}{print}\PY{p}{(}\PY{n}{a}\PY{p}{)}
\end{Verbatim}


    \begin{Verbatim}[commandchars=\\\{\}]
100

    \end{Verbatim}

    Notez bien que cette technique est uniquement pédagogique, et n'a
absolument aucun autre intérêt dans la pratique~; il n'est \textbf{pas
recommandé} de l'utiliser en dehors de ce contexte.


    % Add a bibliography block to the postdoc
    
    
    
