    \hypertarget{uxe9numuxe9rations}{%
\section{Énumérations}\label{uxe9numuxe9rations}}

    \hypertarget{compluxe9ment---niveau-basique}{%
\subsection{Complément - niveau
basique}\label{compluxe9ment---niveau-basique}}

    On trouve dans d'autres langages la notion de types énumérés.

    L'usage habituel, c'est typiquement un code d'erreur qui peut prendre
certaines valeurs précises. Pensez par exemple aux
\href{https://fr.wikipedia.org/wiki/Liste_des_codes_HTTP}{codes prévus
par le protocole HTTP}. Le protocole prévoit un code de retour qui peut
prendre un ensemble fini de valeurs, comme par exemple 200, 301, 302,
404, 500, mais pas 90 ni 110.\\

    On veut pouvoir utiliser des noms parlants dans les programmes qui
gèrent ce type de valeurs, c'est une application typique des types
énumérés.\\

    La bibliothèque standard offre depuis Python-3.4 un module qui s'appelle
sans grande surprise \texttt{enum}, et qui expose entre autres une
classe \texttt{Enum}. On l'utiliserait comme ceci, dans un cas d'usage
plus simple~:

    \begin{Verbatim}[commandchars=\\\{\}]
{\color{incolor}In [{\color{incolor}1}]:} \PY{k+kn}{from} \PY{n+nn}{enum} \PY{k}{import} \PY{n}{Enum}
\end{Verbatim}


    \begin{Verbatim}[commandchars=\\\{\}]
{\color{incolor}In [{\color{incolor}2}]:} \PY{k}{class} \PY{n+nc}{Flavour}\PY{p}{(}\PY{n}{Enum}\PY{p}{)}\PY{p}{:}
            \PY{n}{CHOCOLATE} \PY{o}{=} \PY{l+m+mi}{1}
            \PY{n}{VANILLA} \PY{o}{=} \PY{l+m+mi}{2}
            \PY{n}{PEAR} \PY{o}{=} \PY{l+m+mi}{3}
\end{Verbatim}


    \begin{Verbatim}[commandchars=\\\{\}]
{\color{incolor}In [{\color{incolor}3}]:} \PY{n}{vanilla} \PY{o}{=} \PY{n}{Flavour}\PY{o}{.}\PY{n}{VANILLA}
\end{Verbatim}


    Un premier avantage est que les représentations textuelles sont plus
parlantes~:

    \begin{Verbatim}[commandchars=\\\{\}]
{\color{incolor}In [{\color{incolor}4}]:} \PY{n+nb}{str}\PY{p}{(}\PY{n}{vanilla}\PY{p}{)}
\end{Verbatim}


\begin{Verbatim}[commandchars=\\\{\}]
{\color{outcolor}Out[{\color{outcolor}4}]:} 'Flavour.VANILLA'
\end{Verbatim}
            
    \begin{Verbatim}[commandchars=\\\{\}]
{\color{incolor}In [{\color{incolor}5}]:} \PY{n+nb}{repr}\PY{p}{(}\PY{n}{vanilla}\PY{p}{)}
\end{Verbatim}


\begin{Verbatim}[commandchars=\\\{\}]
{\color{outcolor}Out[{\color{outcolor}5}]:} '<Flavour.VANILLA: 2>'
\end{Verbatim}
            
    Vous pouvez aussi retrouver une valeur par son nom~:

    \begin{Verbatim}[commandchars=\\\{\}]
{\color{incolor}In [{\color{incolor}6}]:} \PY{n}{chocolate} \PY{o}{=} \PY{n}{Flavour}\PY{p}{[}\PY{l+s+s1}{\PYZsq{}}\PY{l+s+s1}{CHOCOLATE}\PY{l+s+s1}{\PYZsq{}}\PY{p}{]}
        \PY{n}{chocolate}
\end{Verbatim}


\begin{Verbatim}[commandchars=\\\{\}]
{\color{outcolor}Out[{\color{outcolor}6}]:} <Flavour.CHOCOLATE: 1>
\end{Verbatim}
            
    \begin{Verbatim}[commandchars=\\\{\}]
{\color{incolor}In [{\color{incolor}7}]:} \PY{n}{Flavour}\PY{o}{.}\PY{n}{CHOCOLATE}
\end{Verbatim}


\begin{Verbatim}[commandchars=\\\{\}]
{\color{outcolor}Out[{\color{outcolor}7}]:} <Flavour.CHOCOLATE: 1>
\end{Verbatim}
            
    Et réciproquement~:

    \begin{Verbatim}[commandchars=\\\{\}]
{\color{incolor}In [{\color{incolor}8}]:} \PY{n}{chocolate}\PY{o}{.}\PY{n}{name}
\end{Verbatim}


\begin{Verbatim}[commandchars=\\\{\}]
{\color{outcolor}Out[{\color{outcolor}8}]:} 'CHOCOLATE'
\end{Verbatim}
            
    \hypertarget{intenum}{%
\subsubsection{\texorpdfstring{\texttt{IntEnum}}{IntEnum}}\label{intenum}}

    En fait, le plus souvent on préfère utiliser \texttt{IntEnum}, une
sous-classe de \texttt{Enum} qui permet également de faire des
comparaisons. Pour reprendre le cas des codes d'erreur HTTP~:

    \begin{Verbatim}[commandchars=\\\{\}]
{\color{incolor}In [{\color{incolor}9}]:} \PY{k+kn}{from} \PY{n+nn}{enum} \PY{k}{import} \PY{n}{IntEnum}
        
        \PY{k}{class} \PY{n+nc}{HttpError}\PY{p}{(}\PY{n}{IntEnum}\PY{p}{)}\PY{p}{:}
        
            \PY{n}{OK} \PY{o}{=} \PY{l+m+mi}{200}
            \PY{n}{REDIRECT} \PY{o}{=} \PY{l+m+mi}{301}
            \PY{n}{REDIRECT\PYZus{}TMP} \PY{o}{=} \PY{l+m+mi}{302}
            \PY{n}{NOT\PYZus{}FOUND} \PY{o}{=} \PY{l+m+mi}{404}
            \PY{n}{INTERNAL\PYZus{}ERROR} \PY{o}{=} \PY{l+m+mi}{500}
        
            \PY{c+c1}{\PYZsh{} avec un IntEnum on peut faire des comparaisons}
            \PY{k}{def} \PY{n+nf}{is\PYZus{}redirect}\PY{p}{(}\PY{n+nb+bp}{self}\PY{p}{)}\PY{p}{:}
                \PY{k}{return} \PY{l+m+mi}{300} \PY{o}{\PYZlt{}}\PY{o}{=} \PY{n+nb+bp}{self}\PY{o}{.}\PY{n}{value} \PY{o}{\PYZlt{}}\PY{o}{=} \PY{l+m+mi}{399}
\end{Verbatim}


    \begin{Verbatim}[commandchars=\\\{\}]
{\color{incolor}In [{\color{incolor}10}]:} \PY{n}{code} \PY{o}{=} \PY{n}{HttpError}\PY{o}{.}\PY{n}{REDIRECT\PYZus{}TMP}
\end{Verbatim}


    \begin{Verbatim}[commandchars=\\\{\}]
{\color{incolor}In [{\color{incolor}11}]:} \PY{n}{code}\PY{o}{.}\PY{n}{is\PYZus{}redirect}\PY{p}{(}\PY{p}{)}
\end{Verbatim}


\begin{Verbatim}[commandchars=\\\{\}]
{\color{outcolor}Out[{\color{outcolor}11}]:} True
\end{Verbatim}
            
    \hypertarget{pour-en-savoir-plus}{%
\subsubsection{Pour en savoir plus}\label{pour-en-savoir-plus}}

Consultez \href{https://docs.python.org/3/library/enum.html}{la
documentation officielle du module \texttt{enum}}.