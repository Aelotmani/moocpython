    \hypertarget{typages-statique-et-dynamique}{%
\section{Typages statique et
dynamique}\label{typages-statique-et-dynamique}}

    \hypertarget{compluxe9ment---niveau-intermuxe9diaire}{%
\subsection{Complément - niveau
intermédiaire}\label{compluxe9ment---niveau-intermuxe9diaire}}

    Parmi les langages typés, on distingue les langages à typage statique et
ceux à typage dynamique. Ce notebook tente d'éclaircir ces notions pour
ceux qui n'y sont pas familiers.

    \hypertarget{typage-statique}{%
\subsubsection{Typage statique}\label{typage-statique}}

    À une extrémité du spectre, on trouve les langages compilés, dits à
typage statique, comme par exemple C ou C++.

En C on écrira, par exemple, une version simpliste de la fonction
factoriel comme ceci~:

    \begin{Shaded}
\begin{Highlighting}[]
\DataTypeTok{int}\NormalTok{ factoriel(}\DataTypeTok{int}\NormalTok{ n) \{}
    \DataTypeTok{int}\NormalTok{ result = }\DecValTok{1}\NormalTok{;}
    \ControlFlowTok{for}\NormalTok{ (}\DataTypeTok{int}\NormalTok{ loop = }\DecValTok{1}\NormalTok{; loop <= n; loop++)}
\NormalTok{        result *= loop;}
    \ControlFlowTok{return}\NormalTok{ result;}
\NormalTok{\}}
\end{Highlighting}
\end{Shaded}

    Comme vous pouvez le voir - ou le deviner - toutes les
\textbf{variables} utilisées ici (comme par exemple \texttt{n},
\texttt{result} et \texttt{loop}) sont typées~:

\begin{itemize}
\tightlist
\item
  on doit appeler \texttt{factoriel} avec un argument \texttt{n} qui
  doit être un entier (\texttt{int} est le nom du type entier)~;
\item
  les variables internes \texttt{result} et \texttt{loop} sont de type
  entier~;
\item
  \texttt{factoriel} retourne une valeur de type entier.
\end{itemize}

    Ces informations de type ont essentiellement trois fonctions~:

\begin{itemize}
\tightlist
\item
  en premier lieu, elles sont nécessaires au compilateur. En C si le
  programmeur ne précisait pas que \texttt{result} est de type entier,
  le compilateur n'aurait pas suffisamment d'éléments pour générer le
  code assembleur correspondant~;
\item
  en contrepartie, le programmeur a un contrôle très fin de l'usage
  qu'il fait de la mémoire et du matériel. Il peut choisir d'utiliser un
  entier sur 32 ou 64 bits, signé ou pas, ou construire avec
  \texttt{struct} et \texttt{union} un arrangement de ses données~;
\item
  enfin, et surtout, ces informations de type permettent de faire un
  contrôle \emph{a priori} de la validité du programme, par exemple, si
  à un autre endroit dans le code on trouve~:
\end{itemize}

    \begin{Shaded}
\begin{Highlighting}[]
\PreprocessorTok{#include }\ImportTok{<stdio.h>}

\DataTypeTok{int}\NormalTok{ main(}\DataTypeTok{int}\NormalTok{ argc, }\DataTypeTok{char}\NormalTok{ *argv[]) \{}
    \CommentTok{/* le premier argument de la ligne de commande est argv[1] */}
    \DataTypeTok{char}\NormalTok{ *input = argv[}\DecValTok{1}\NormalTok{];}
    \CommentTok{/* calculer son factoriel et afficher le résultat */}
\NormalTok{    printf(}\StringTok{"Factoriel (%s) = %d}\SpecialCharTok{\textbackslash{}n}\StringTok{"}\NormalTok{, input, factoriel(input));}
    \CommentTok{/*                                               ^^^^^}
\CommentTok{     * ici on appelle factoriel avec une entrée de type 'chaîne de caractères' */}
\NormalTok{\}}
\end{Highlighting}
\end{Shaded}

    alors le compilateur va remarquer qu'on essaie d'appeler
\texttt{factoriel} avec comme argument \texttt{input} qui, pour faire
simple, est une chaîne de caractères et comme \texttt{factoriel}
s'attend à recevoir un entier, ce programme n'a aucune chance de
compiler.\\

On parle alors de \textbf{typage statique}, en ce sens que chaque
\textbf{variable} a exactement un type qui est défini par le programmeur
une bonne fois pour toutes.\\

    C'est ce qu'on appelle le \textbf{contrôle de type}, ou
\emph{type-checking} en anglais. Si on ignore le point sur le contrôle
fin de la mémoire, qui n'est pas crucial à notre sujet, ce modèle de
contrôle de type présente~:

\begin{itemize}
\tightlist
\item
  l'\textbf{inconvénient} de demander davantage au programmeur (je fais
  abstraction, à ce stade et pour simplifier, de
  \href{https://en.wikipedia.org/wiki/Type_inference}{langages à
  inférence de types} comme ML et Haskell)~;
\item
  et l'\textbf{avantage} de permettre un contrôle étendu, et surtout
  précoce (avant même de l'exécuter), de la bonne correction du
  programme.
\end{itemize}

    Cela étant dit, le typage statique en C n'empêche pas le programmeur
débutant d'essayer d'écrire dans la mémoire à partir d'un pointeur
\texttt{NULL} - et le programme de s'interrompre brutalement. Il faut
être conscient des limites du typage statique.

    \hypertarget{typage-dynamique}{%
\subsubsection{Typage dynamique}\label{typage-dynamique}}

    À l'autre bout du spectre, on trouve des langages comme, eh bien,
Python.\\

    Pour comprendre cette notion de typage dynamique, regardons la fonction
suivante \texttt{somme}.

    \begin{Verbatim}[commandchars=\\\{\}]
{\color{incolor}In [{\color{incolor}1}]:} \PY{k}{def} \PY{n+nf}{somme}\PY{p}{(}\PY{o}{*}\PY{n}{largs}\PY{p}{)}\PY{p}{:}
            \PY{l+s+s2}{\PYZdq{}}\PY{l+s+s2}{retourne la somme de tous ses arguments}\PY{l+s+s2}{\PYZdq{}}
            \PY{k}{if} \PY{o+ow}{not} \PY{n}{largs}\PY{p}{:}
                \PY{k}{return} \PY{l+m+mi}{0}
            \PY{n}{result} \PY{o}{=} \PY{n}{largs}\PY{p}{[}\PY{l+m+mi}{0}\PY{p}{]}
            \PY{k}{for} \PY{n}{i} \PY{o+ow}{in} \PY{n+nb}{range}\PY{p}{(}\PY{l+m+mi}{1}\PY{p}{,} \PY{n+nb}{len}\PY{p}{(}\PY{n}{largs}\PY{p}{)}\PY{p}{)}\PY{p}{:}
                \PY{n}{result} \PY{o}{+}\PY{o}{=} \PY{n}{largs}\PY{p}{[}\PY{n}{i}\PY{p}{]}
            \PY{k}{return} \PY{n}{result}
\end{Verbatim}


    Naturellement, vous n'êtes pas à ce stade en mesure de comprendre le
fonctionnement intime de la fonction. Mais vous pouvez tout de même
l'utiliser~:

    \begin{Verbatim}[commandchars=\\\{\}]
{\color{incolor}In [{\color{incolor}2}]:} \PY{n}{somme}\PY{p}{(}\PY{l+m+mi}{12}\PY{p}{,} \PY{l+m+mi}{14}\PY{p}{,} \PY{l+m+mi}{300}\PY{p}{)}
\end{Verbatim}


\begin{Verbatim}[commandchars=\\\{\}]
{\color{outcolor}Out[{\color{outcolor}2}]:} 326
\end{Verbatim}
            
    \begin{Verbatim}[commandchars=\\\{\}]
{\color{incolor}In [{\color{incolor}3}]:} \PY{n}{liste1} \PY{o}{=} \PY{p}{[}\PY{l+s+s1}{\PYZsq{}}\PY{l+s+s1}{a}\PY{l+s+s1}{\PYZsq{}}\PY{p}{,} \PY{l+s+s1}{\PYZsq{}}\PY{l+s+s1}{b}\PY{l+s+s1}{\PYZsq{}}\PY{p}{,} \PY{l+s+s1}{\PYZsq{}}\PY{l+s+s1}{c}\PY{l+s+s1}{\PYZsq{}}\PY{p}{]}
        \PY{n}{liste2} \PY{o}{=} \PY{p}{[}\PY{l+m+mi}{0}\PY{p}{,} \PY{l+m+mi}{20}\PY{p}{,} \PY{l+m+mi}{30}\PY{p}{]}
        \PY{n}{liste3} \PY{o}{=} \PY{p}{[}\PY{l+s+s1}{\PYZsq{}}\PY{l+s+s1}{spam}\PY{l+s+s1}{\PYZsq{}}\PY{p}{,} \PY{l+s+s1}{\PYZsq{}}\PY{l+s+s1}{eggs}\PY{l+s+s1}{\PYZsq{}}\PY{p}{]}
        \PY{n}{somme}\PY{p}{(}\PY{n}{liste1}\PY{p}{,} \PY{n}{liste2}\PY{p}{,} \PY{n}{liste3}\PY{p}{)}
\end{Verbatim}


\begin{Verbatim}[commandchars=\\\{\}]
{\color{outcolor}Out[{\color{outcolor}3}]:} ['a', 'b', 'c', 0, 20, 30, 'spam', 'eggs']
\end{Verbatim}
            
    Vous pouvez donc constater que \texttt{somme} peut fonctionner avec des
objets de types différents. En fait, telle qu'elle est écrite, elle va
fonctionner s'il est possible de faire \texttt{+} entre ses arguments.
Ainsi, par exemple, on pourrait même faire~:

    \begin{Verbatim}[commandchars=\\\{\}]
{\color{incolor}In [{\color{incolor}4}]:} \PY{c+c1}{\PYZsh{} Python sait faire + entre deux chaînes de caractères}
        \PY{n}{somme}\PY{p}{(}\PY{l+s+s1}{\PYZsq{}}\PY{l+s+s1}{abc}\PY{l+s+s1}{\PYZsq{}}\PY{p}{,} \PY{l+s+s1}{\PYZsq{}}\PY{l+s+s1}{def}\PY{l+s+s1}{\PYZsq{}}\PY{p}{)}
\end{Verbatim}


\begin{Verbatim}[commandchars=\\\{\}]
{\color{outcolor}Out[{\color{outcolor}4}]:} 'abcdef'
\end{Verbatim}
            
    Mais par contre on ne pourrait pas faire

    \begin{Verbatim}[commandchars=\\\{\}]
{\color{incolor}In [{\color{incolor}5}]:} \PY{c+c1}{\PYZsh{} ceci va déclencher une exception à l\PYZsq{}exécution}
        \PY{n}{somme}\PY{p}{(}\PY{l+m+mi}{12}\PY{p}{,} \PY{p}{[}\PY{l+m+mi}{1}\PY{p}{,} \PY{l+m+mi}{2}\PY{p}{,} \PY{l+m+mi}{3}\PY{p}{]}\PY{p}{)}
\end{Verbatim}


    \begin{Verbatim}[commandchars=\\\{\}]

        ---------------------------------------------------------------------------

        TypeError                                 Traceback (most recent call last)

        <ipython-input-5-1b5269e9e129> in <module>()
          1 \# ceci va déclencher une exception à l'exécution
    ----> 2 somme(12, [1, 2, 3])
    

        <ipython-input-1-29005c25d5bb> in somme(*largs)
          5     result = largs[0]
          6     for i in range(1, len(largs)):
    ----> 7         result += largs[i]
          8     return result


        TypeError: unsupported operand type(s) for +=: 'int' and 'list'

    \end{Verbatim}

    Il est utile de remarquer que le typage de Python, qui existe bel et
bien comme on le verra, est qualifié de dynamique parce que le type est
attaché \textbf{à un objet} et non à la variable qui le référence. On
aura bien entendu l'occasion d'approfondir tout ça dans le cours.\\

    En Python, on fait souvent référence au typage sous l'appellation
\emph{duck typing}, de manière imagée~:\\

\begin{quote}
If it looks like a duck and quacks like a duck, it's a duck.\\
\end{quote}

    On voit qu'on se trouve dans une situation très différente de celle du
programmeur C/C++, en ce sens que~:

\begin{itemize}
\tightlist
\item
  à l'écriture du programme, il n'y aucun des surcoûts qu'on trouve avec
  C ou C++ en matière de définition de type~;
\item
  aucun contrôle de type n'est effectué \emph{a priori} par le langage
  au moment de la définition de la fonction \texttt{somme}~;
\item
  par contre au moment de l'exécution, s'il s'avère qu'on tente de faire
  une somme entre deux types qui ne peuvent pas être additionnés, comme
  ci-dessus avec un entier et une liste, le programme ne pourra pas se
  dérouler correctement.
\end{itemize}

    Il y a deux points de vue vis-à-vis de la question du typage.\\

Les gens habitués au \emph{typage statique} se plaignent du typage
dynamique en disant qu'on peut écrire des programmes faux et qu'on s'en
rend compte trop tard - à l'exécution.\\

À l'inverse les gens habitués au \emph{typage dynamique} font valoir que
le typage statique est très partiel, par exemple, en C si on essaie
d'écrire dans un pointeur \texttt{NULL}, le système d'exploitation ne le
permet pas et le programme sort tout aussi brutalement.\\

    Bref, selon le point de vue, le typage dynamique est vécu comme un
inconvénient (pas assez de bonnes propriétés détectées par le langage)
ou comme un avantage (pas besoin de passer du temps à déclarer le type
des variables, ni à faire des conversions pour satisfaire le
compilateur).\\

Vous remarquerez cependant à l'usage, qu'en matière de vitesse de
développement, les inconvénients du typage dynamique sont très largement
compensés par ses avantages.

    \hypertarget{type-hints}{%
\subsubsection{\texorpdfstring{\emph{Type
hints}}{Type hints}}\label{type-hints}}

    Signalons enfin que depuis python-3.5, il est \textbf{possible}
d'ajouter des annotations de type, pour expliciter les suppositions qui
sont faites par le programmeur pour le bon fonctionnement du code.\\

Nous aurons là encore l'occasion de détailler ce point dans le cours,
signalons simplement que ces annotations sont totalement optionnelles,
et que même lorsqu'elles sont présentes elles ne sont pas utilisées à
l'exécution par l'interpréteur. L'idée est plutôt de permettre à des
outils externes, \href{http://www.mypy-lang.org}{comme par exemple
\texttt{mypy}}, d'effectuer des contrôles plus poussés concernant la
correction du programme.