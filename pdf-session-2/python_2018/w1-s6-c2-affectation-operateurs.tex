    \hypertarget{affectations-et-opuxe9rations-uxe0-la}{%
\section{\texorpdfstring{Affectations et Opérations (à la
\texttt{+=})}{Affectations et Opérations (à la +=)}}\label{affectations-et-opuxe9rations-uxe0-la}}

    \hypertarget{compluxe9ment---niveau-intermuxe9diaire}{%
\subsection{Complément - niveau
intermédiaire}\label{compluxe9ment---niveau-intermuxe9diaire}}

    Il existe en Python toute une famille d'opérateurs dérivés de
l'affectation qui permettent de faire en une fois une opération et une
affectation. En voici quelques exemples.

    \hypertarget{incruxe9mentation}{%
\subsubsection{Incrémentation}\label{incruxe9mentation}}

    On peut facilement augmenter la valeur d'une variable numérique comme
ceci~:

    \begin{Verbatim}[commandchars=\\\{\}]
{\color{incolor}In [{\color{incolor}1}]:} \PY{n}{entier} \PY{o}{=} \PY{l+m+mi}{10}
        
        \PY{n}{entier} \PY{o}{+}\PY{o}{=} \PY{l+m+mi}{2}
        \PY{n+nb}{print}\PY{p}{(}\PY{l+s+s1}{\PYZsq{}}\PY{l+s+s1}{entier}\PY{l+s+s1}{\PYZsq{}}\PY{p}{,} \PY{n}{entier}\PY{p}{)}
\end{Verbatim}


    \begin{Verbatim}[commandchars=\\\{\}]
entier 12

    \end{Verbatim}

    Comme on le devine peut-être, ceci est équivalent à~:

    \begin{Verbatim}[commandchars=\\\{\}]
{\color{incolor}In [{\color{incolor}2}]:} \PY{n}{entier} \PY{o}{=} \PY{l+m+mi}{10}
        
        \PY{n}{entier} \PY{o}{=} \PY{n}{entier} \PY{o}{+} \PY{l+m+mi}{2}
        \PY{n+nb}{print}\PY{p}{(}\PY{l+s+s1}{\PYZsq{}}\PY{l+s+s1}{entier}\PY{l+s+s1}{\PYZsq{}}\PY{p}{,} \PY{n}{entier}\PY{p}{)}
\end{Verbatim}


    \begin{Verbatim}[commandchars=\\\{\}]
entier 12

    \end{Verbatim}

    \hypertarget{autres-opuxe9rateurs-courants}{%
\subsubsection{Autres opérateurs
courants}\label{autres-opuxe9rateurs-courants}}

    Cette forme, qui combine opération sur une variable et réaffectation du
résultat à la même variable, est disponible avec tous les opérateurs
courants~:

    \begin{Verbatim}[commandchars=\\\{\}]
{\color{incolor}In [{\color{incolor}3}]:} \PY{n}{entier} \PY{o}{\PYZhy{}}\PY{o}{=} \PY{l+m+mi}{4}
        \PY{n+nb}{print}\PY{p}{(}\PY{l+s+s1}{\PYZsq{}}\PY{l+s+s1}{après décrément}\PY{l+s+s1}{\PYZsq{}}\PY{p}{,} \PY{n}{entier}\PY{p}{)}
        \PY{n}{entier} \PY{o}{*}\PY{o}{=} \PY{l+m+mi}{2}
        \PY{n+nb}{print}\PY{p}{(}\PY{l+s+s1}{\PYZsq{}}\PY{l+s+s1}{après doublement}\PY{l+s+s1}{\PYZsq{}}\PY{p}{,} \PY{n}{entier}\PY{p}{)}
        \PY{n}{entier} \PY{o}{/}\PY{o}{=} \PY{l+m+mi}{2}
        \PY{n+nb}{print}\PY{p}{(}\PY{l+s+s1}{\PYZsq{}}\PY{l+s+s1}{mis à moitié}\PY{l+s+s1}{\PYZsq{}}\PY{p}{,} \PY{n}{entier}\PY{p}{)}
\end{Verbatim}


    \begin{Verbatim}[commandchars=\\\{\}]
après décrément 8
après doublement 16
mis à moitié 8.0

    \end{Verbatim}

    \hypertarget{types-non-numuxe9riques}{%
\subsubsection{Types non numériques}\label{types-non-numuxe9riques}}

    En réalité cette construction est disponible sur tous les types qui
supportent l'opérateur en question. Par exemple, les listes (que nous
verrons bientôt) peuvent être additionnées entre elles~:

    \begin{Verbatim}[commandchars=\\\{\}]
{\color{incolor}In [{\color{incolor}4}]:} \PY{n}{liste} \PY{o}{=} \PY{p}{[}\PY{l+m+mi}{0}\PY{p}{,} \PY{l+m+mi}{3}\PY{p}{,} \PY{l+m+mi}{5}\PY{p}{]}
        \PY{n+nb}{print}\PY{p}{(}\PY{l+s+s1}{\PYZsq{}}\PY{l+s+s1}{liste}\PY{l+s+s1}{\PYZsq{}}\PY{p}{,} \PY{n}{liste}\PY{p}{)}
        
        \PY{n}{liste} \PY{o}{+}\PY{o}{=} \PY{p}{[}\PY{l+s+s1}{\PYZsq{}}\PY{l+s+s1}{a}\PY{l+s+s1}{\PYZsq{}}\PY{p}{,} \PY{l+s+s1}{\PYZsq{}}\PY{l+s+s1}{b}\PY{l+s+s1}{\PYZsq{}}\PY{p}{]}
        \PY{n+nb}{print}\PY{p}{(}\PY{l+s+s1}{\PYZsq{}}\PY{l+s+s1}{après ajout}\PY{l+s+s1}{\PYZsq{}}\PY{p}{,} \PY{n}{liste}\PY{p}{)}
\end{Verbatim}


    \begin{Verbatim}[commandchars=\\\{\}]
liste [0, 3, 5]
après ajout [0, 3, 5, 'a', 'b']

    \end{Verbatim}

    Beaucoup de types supportent l'opérateur \texttt{+}, qui est sans doute
de loin celui qui est le plus utilisé avec cette construction.

    \hypertarget{opuxe9rateurs-plus-abscons}{%
\subsubsection{Opérateurs plus
abscons}\label{opuxe9rateurs-plus-abscons}}

    Signalons enfin que l'on trouve aussi cette construction avec d'autres
opérateurs moins fréquents, par exemple~:

    \begin{Verbatim}[commandchars=\\\{\}]
{\color{incolor}In [{\color{incolor}5}]:} \PY{n}{entier} \PY{o}{=} \PY{l+m+mi}{2}
        \PY{n+nb}{print}\PY{p}{(}\PY{l+s+s1}{\PYZsq{}}\PY{l+s+s1}{entier:}\PY{l+s+s1}{\PYZsq{}}\PY{p}{,} \PY{n}{entier}\PY{p}{)}
        \PY{n}{entier} \PY{o}{*}\PY{o}{*}\PY{o}{=} \PY{l+m+mi}{10}
        \PY{n+nb}{print}\PY{p}{(}\PY{l+s+s1}{\PYZsq{}}\PY{l+s+s1}{à la puissance dix:}\PY{l+s+s1}{\PYZsq{}}\PY{p}{,} \PY{n}{entier}\PY{p}{)}
        \PY{n}{entier} \PY{o}{\PYZpc{}}\PY{o}{=} \PY{l+m+mi}{5}
        \PY{n+nb}{print}\PY{p}{(}\PY{l+s+s1}{\PYZsq{}}\PY{l+s+s1}{modulo 5:}\PY{l+s+s1}{\PYZsq{}}\PY{p}{,} \PY{n}{entier}\PY{p}{)}
\end{Verbatim}


    \begin{Verbatim}[commandchars=\\\{\}]
entier: 2
à la puissance dix: 1024
modulo 5: 4

    \end{Verbatim}

    Et pour ceux qui connaissent déjà un peu Python, on peut même le faire
avec des opérateurs de décalage, que nous verrons très bientôt~:

    \begin{Verbatim}[commandchars=\\\{\}]
{\color{incolor}In [{\color{incolor}6}]:} \PY{n}{entier} \PY{o}{\PYZlt{}\PYZlt{}}\PY{o}{=} \PY{l+m+mi}{2}
        \PY{n+nb}{print}\PY{p}{(}\PY{l+s+s1}{\PYZsq{}}\PY{l+s+s1}{double décalage gauche:}\PY{l+s+s1}{\PYZsq{}}\PY{p}{,} \PY{n}{entier}\PY{p}{)}
\end{Verbatim}


    \begin{Verbatim}[commandchars=\\\{\}]
double décalage gauche: 16

    \end{Verbatim}