    
    
    
    

    

    \hypertarget{les-fonctions-globals-et-locals}{%
\section{\texorpdfstring{Les fonctions \texttt{globals} et
\texttt{locals}}{Les fonctions globals et locals}}\label{les-fonctions-globals-et-locals}}

    \hypertarget{compluxe9ment---niveau-intermuxe9diaire}{%
\subsection{Complément - niveau
intermédiaire}\label{compluxe9ment---niveau-intermuxe9diaire}}

    \hypertarget{un-exemple}{%
\subsubsection{Un exemple}\label{un-exemple}}

    python fournit un accès à la liste des noms et valeurs des variables
visibles à cet endroit du code. Dans le jargon des langages de
programmation on appelle ceci \textbf{l'environnement}.

Cela est fait grâce aux fonctions \emph{built-in} \texttt{globals} et
\texttt{locals}, que nous allons commencer par essayer sur quelques
exemples. Nous avons pour cela écrit un module dédié~:

    \begin{Verbatim}[commandchars=\\\{\},frame=single,framerule=0.3mm,rulecolor=\color{cellframecolor}]
{\color{incolor}In [{\color{incolor}1}]:} \PY{k+kn}{import} \PY{n+nn}{env\PYZus{}locals\PYZus{}globals}
\end{Verbatim}


    dont voici le code

    \begin{Verbatim}[commandchars=\\\{\},frame=single,framerule=0.3mm,rulecolor=\color{cellframecolor}]
{\color{incolor}In [{\color{incolor}2}]:} \PY{k+kn}{from} \PY{n+nn}{modtools} \PY{k}{import} \PY{n}{show\PYZus{}module}
        \PY{n}{show\PYZus{}module}\PY{p}{(}\PY{n}{env\PYZus{}locals\PYZus{}globals}\PY{p}{)}
\end{Verbatim}


    \begin{Verbatim}[commandchars=\\\{\},frame=single,framerule=0.3mm,rulecolor=\color{cellframecolor}]
Fichier /Users/tparment/git/flotpython/modules/env\_locals\_globals.py
----------------------------------------
01|"""
02|un module pour illustrer les fonctions globals et locals
03|"""
04|
05|globale = "variable globale au module"
06|
07|def display\_env(env):
08|    """
09|    affiche un environnement
10|    on affiche juste le nom et le type de chaque variable
11|    """
12|    for variable, valeur in sorted(env.items()):
13|        print("\{:>20\} → \{\}".format(variable, type(valeur).\_\_name\_\_))
14|
15|def temoin(x):
16|    "la fonction témoin"
17|    y = x ** 2
18|    print(20 * '-', 'globals:')
19|    display\_env(globals())
20|    print(20 * '-', 'locals:')
21|    display\_env(locals())
22|
23|class Foo:
24|    "une classe vide"
\end{Verbatim}

    et voici ce qu'on obtient lorsqu'on appelle

    \begin{Verbatim}[commandchars=\\\{\},frame=single,framerule=0.3mm,rulecolor=\color{cellframecolor}]
{\color{incolor}In [{\color{incolor}3}]:} \PY{n}{env\PYZus{}locals\PYZus{}globals}\PY{o}{.}\PY{n}{temoin}\PY{p}{(}\PY{l+m+mi}{10}\PY{p}{)}
\end{Verbatim}


    \begin{Verbatim}[commandchars=\\\{\},frame=single,framerule=0.3mm,rulecolor=\color{cellframecolor}]
-------------------- globals:
                 Foo → type
        \_\_builtins\_\_ → dict
          \_\_cached\_\_ → str
             \_\_doc\_\_ → str
            \_\_file\_\_ → str
          \_\_loader\_\_ → SourceFileLoader
            \_\_name\_\_ → str
         \_\_package\_\_ → str
            \_\_spec\_\_ → ModuleSpec
         display\_env → function
             globale → str
              temoin → function
-------------------- locals:
                   x → int
                   y → int
\end{Verbatim}

    \hypertarget{interpruxe9tation}{%
\subsubsection{Interprétation}\label{interpruxe9tation}}

    Que nous montre cet exemple~?

\begin{itemize}
\item
  D'une part la fonction \textbf{\texttt{globals}} nous donne la liste
  des symboles définis au niveau de \textbf{l'espace de noms du module}.
  Il s'agit évidemment du module dans lequel est définie la fonction,
  pas celui dans lequel elle est appelée. Vous remarquerez que ceci
  englobe \textbf{tous} les symboles du module
  \texttt{env\_locals\_globals}, et non pas seulement ceux définis avant
  \texttt{temoin}, c'est-à-dire la variable \texttt{globale}, les deux
  fonctions \texttt{display\_env} et \texttt{temoin}, et la classe
  \texttt{Foo}.
\item
  D'autre part \textbf{\texttt{locals}} nous donne les variables locales
  qui sont accessibles \textbf{à cet endroit du code}, comme le montre
  ce second exemple qui se concentre sur \texttt{locals} à différents
  points d'une même fonction.
\end{itemize}

    \begin{Verbatim}[commandchars=\\\{\},frame=single,framerule=0.3mm,rulecolor=\color{cellframecolor}]
{\color{incolor}In [{\color{incolor}4}]:} \PY{k+kn}{import} \PY{n+nn}{env\PYZus{}locals}
\end{Verbatim}


    \begin{Verbatim}[commandchars=\\\{\},frame=single,framerule=0.3mm,rulecolor=\color{cellframecolor}]
{\color{incolor}In [{\color{incolor}5}]:} \PY{c+c1}{\PYZsh{} le code de ce module }
        \PY{n}{show\PYZus{}module}\PY{p}{(}\PY{n}{env\PYZus{}locals}\PY{p}{)}
\end{Verbatim}


    \begin{Verbatim}[commandchars=\\\{\},frame=single,framerule=0.3mm,rulecolor=\color{cellframecolor}]
Fichier /Users/tparment/git/flotpython/modules/env\_locals.py
----------------------------------------
01|"""
02|un module pour illustrer la fonction locals
03|"""
04|
05|\# pour afficher
06|from env\_locals\_globals import display\_env
07|
08|def temoin(x):
09|    "la fonction témoin"
10|    y = x ** 2
11|    print(20*'-', 'locals - entrée:')
12|    display\_env(locals())
13|
14|    for i in (1,):
15|        for j in (1,):
16|            print(20*'-', 'locals - boucles for:')
17|            display\_env(locals())
18|
\end{Verbatim}

    \begin{Verbatim}[commandchars=\\\{\},frame=single,framerule=0.3mm,rulecolor=\color{cellframecolor}]
{\color{incolor}In [{\color{incolor}6}]:} \PY{n}{env\PYZus{}locals}\PY{o}{.}\PY{n}{temoin}\PY{p}{(}\PY{l+m+mi}{10}\PY{p}{)}
\end{Verbatim}


    \begin{Verbatim}[commandchars=\\\{\},frame=single,framerule=0.3mm,rulecolor=\color{cellframecolor}]
-------------------- locals - entrée:
                   x → int
                   y → int
-------------------- locals - boucles for:
                   i → int
                   j → int
                   x → int
                   y → int
\end{Verbatim}

    \hypertarget{compluxe9ment---niveau-avancuxe9}{%
\subsection{Complément - niveau
avancé}\label{compluxe9ment---niveau-avancuxe9}}

    \textbf{NOTE}: cette section est en pratique devenue obsolète maintenant
que les \emph{f-strings} sont présents dans la version 3.6.

Nous l'avons conservée pour l'instant toutefois, pour ceux d'entre vous
qui ne peuvent pas encore utiliser les \emph{f-strings} en production.
N'hésitez pas à passer si vous n'êtes pas dans ce cas.

    \hypertarget{usage-pour-le-formatage-de-chauxeenes}{%
\subsubsection{Usage pour le formatage de
chaînes}\label{usage-pour-le-formatage-de-chauxeenes}}

    Les deux fonctions \texttt{locals} et \texttt{globals} ne sont pas d'une
utilisation très fréquente. Elles peuvent cependant être utiles dans le
contexte du formatage de chaînes, comme on peut le voir dans les deux
exemples ci-dessous.

    \hypertarget{avec-format}{%
\subparagraph{\texorpdfstring{Avec
\texttt{format}}{Avec format}}\label{avec-format}}

    On peut utiliser \texttt{format} qui s'attend à quelque chose comme~:

    \begin{Verbatim}[commandchars=\\\{\},frame=single,framerule=0.3mm,rulecolor=\color{cellframecolor}]
{\color{incolor}In [{\color{incolor}7}]:} \PY{l+s+s2}{\PYZdq{}}\PY{l+s+si}{\PYZob{}nom\PYZcb{}}\PY{l+s+s2}{\PYZdq{}}\PY{o}{.}\PY{n}{format}\PY{p}{(}\PY{n}{nom}\PY{o}{=}\PY{l+s+s2}{\PYZdq{}}\PY{l+s+s2}{Dupont}\PY{l+s+s2}{\PYZdq{}}\PY{p}{)}
\end{Verbatim}


\begin{Verbatim}[commandchars=\\\{\},frame=single,framerule=0.3mm,rulecolor=\color{cellframecolor}]
{\color{outcolor}Out[{\color{outcolor}7}]:} 'Dupont'
\end{Verbatim}
            
    que l'on peut obtenir de manière équivalente, en anticipant sur la
prochaine vidéo, avec le passage d'arguments en \texttt{**}~:

    \begin{Verbatim}[commandchars=\\\{\},frame=single,framerule=0.3mm,rulecolor=\color{cellframecolor}]
{\color{incolor}In [{\color{incolor}8}]:} \PY{l+s+s2}{\PYZdq{}}\PY{l+s+si}{\PYZob{}nom\PYZcb{}}\PY{l+s+s2}{\PYZdq{}}\PY{o}{.}\PY{n}{format}\PY{p}{(}\PY{o}{*}\PY{o}{*}\PY{p}{\PYZob{}}\PY{l+s+s1}{\PYZsq{}}\PY{l+s+s1}{nom}\PY{l+s+s1}{\PYZsq{}}\PY{p}{:} \PY{l+s+s1}{\PYZsq{}}\PY{l+s+s1}{Dupont}\PY{l+s+s1}{\PYZsq{}}\PY{p}{\PYZcb{}}\PY{p}{)}
\end{Verbatim}


\begin{Verbatim}[commandchars=\\\{\},frame=single,framerule=0.3mm,rulecolor=\color{cellframecolor}]
{\color{outcolor}Out[{\color{outcolor}8}]:} 'Dupont'
\end{Verbatim}
            
    En versant la fonction \texttt{locals} dans cette formule on obtient une
forme relativement élégante

    \begin{Verbatim}[commandchars=\\\{\},frame=single,framerule=0.3mm,rulecolor=\color{cellframecolor}]
{\color{incolor}In [{\color{incolor}9}]:} \PY{k}{def} \PY{n+nf}{format\PYZus{}et\PYZus{}locals}\PY{p}{(}\PY{n}{nom}\PY{p}{,} \PY{n}{prenom}\PY{p}{,} \PY{n}{civilite}\PY{p}{,} \PY{n}{telephone}\PY{p}{)}\PY{p}{:}
            \PY{k}{return} \PY{l+s+s2}{\PYZdq{}}\PY{l+s+si}{\PYZob{}civilite\PYZcb{}}\PY{l+s+s2}{ }\PY{l+s+si}{\PYZob{}prenom\PYZcb{}}\PY{l+s+s2}{ }\PY{l+s+si}{\PYZob{}nom\PYZcb{}}\PY{l+s+s2}{ : Poste }\PY{l+s+si}{\PYZob{}telephone\PYZcb{}}\PY{l+s+s2}{\PYZdq{}}\PY{o}{.}\PY{n}{format}\PY{p}{(}\PY{o}{*}\PY{o}{*}\PY{n+nb}{locals}\PY{p}{(}\PY{p}{)}\PY{p}{)}
        
        \PY{n}{format\PYZus{}et\PYZus{}locals}\PY{p}{(}\PY{l+s+s1}{\PYZsq{}}\PY{l+s+s1}{Dupont}\PY{l+s+s1}{\PYZsq{}}\PY{p}{,} \PY{l+s+s1}{\PYZsq{}}\PY{l+s+s1}{Jean}\PY{l+s+s1}{\PYZsq{}}\PY{p}{,} \PY{l+s+s1}{\PYZsq{}}\PY{l+s+s1}{Mr}\PY{l+s+s1}{\PYZsq{}}\PY{p}{,} \PY{l+s+s1}{\PYZsq{}}\PY{l+s+s1}{7748}\PY{l+s+s1}{\PYZsq{}}\PY{p}{)}
\end{Verbatim}


\begin{Verbatim}[commandchars=\\\{\},frame=single,framerule=0.3mm,rulecolor=\color{cellframecolor}]
{\color{outcolor}Out[{\color{outcolor}9}]:} 'Mr Jean Dupont : Poste 7748'
\end{Verbatim}
            
    \hypertarget{avec-lopuxe9rateur}}{Avec l'opérateur \%}}\label{avec-lopuxe9rateur}}

    De manière similaire, avec l'opérateur \texttt{\%} - dont nous rappelons
qu'il est obsolète - on peut écrire

    \begin{Verbatim}[commandchars=\\\{\},frame=single,framerule=0.3mm,rulecolor=\color{cellframecolor}]
{\color{incolor}In [{\color{incolor}10}]:} \PY{k}{def} \PY{n+nf}{pourcent\PYZus{}et\PYZus{}locals}\PY{p}{(}\PY{n}{nom}\PY{p}{,} \PY{n}{prenom}\PY{p}{,} \PY{n}{civilite}\PY{p}{,} \PY{n}{telephone}\PY{p}{)}\PY{p}{:}
             \PY{k}{return} \PY{l+s+s2}{\PYZdq{}}\PY{l+s+si}{\PYZpc{}(civilite)s}\PY{l+s+s2}{ }\PY{l+s+si}{\PYZpc{}(prenom)s}\PY{l+s+s2}{ }\PY{l+s+si}{\PYZpc{}(nom)s}\PY{l+s+s2}{ : Poste }\PY{l+s+si}{\PYZpc{}(telephone)s}\PY{l+s+s2}{\PYZdq{}}\PY{o}{\PYZpc{}}\PY{k}{locals}()
         
         \PY{n}{pourcent\PYZus{}et\PYZus{}locals}\PY{p}{(}\PY{l+s+s1}{\PYZsq{}}\PY{l+s+s1}{Dupont}\PY{l+s+s1}{\PYZsq{}}\PY{p}{,} \PY{l+s+s1}{\PYZsq{}}\PY{l+s+s1}{Jean}\PY{l+s+s1}{\PYZsq{}}\PY{p}{,} \PY{l+s+s1}{\PYZsq{}}\PY{l+s+s1}{Mr}\PY{l+s+s1}{\PYZsq{}}\PY{p}{,} \PY{l+s+s1}{\PYZsq{}}\PY{l+s+s1}{7748}\PY{l+s+s1}{\PYZsq{}}\PY{p}{)}
\end{Verbatim}


\begin{Verbatim}[commandchars=\\\{\},frame=single,framerule=0.3mm,rulecolor=\color{cellframecolor}]
{\color{outcolor}Out[{\color{outcolor}10}]:} 'Mr Jean Dupont : Poste 7748'
\end{Verbatim}
            
    \hypertarget{avec-un-f-string}{%
\subparagraph{\texorpdfstring{Avec un
\emph{f-string}}{Avec un f-string}}\label{avec-un-f-string}}

    Pour rappel si vous disposez de python 3.6, vous pouvez alors écrire
simplement - et sans avoir recours, donc, à \texttt{locals()} ou autre~:

    \begin{Verbatim}[commandchars=\\\{\},frame=single,framerule=0.3mm,rulecolor=\color{cellframecolor}]
{\color{incolor}In [{\color{incolor}11}]:} \PY{c+c1}{\PYZsh{} attention ceci nécessite python\PYZhy{}3.6}
         \PY{k}{def} \PY{n+nf}{avec\PYZus{}f\PYZus{}string}\PY{p}{(}\PY{n}{nom}\PY{p}{,} \PY{n}{prenom}\PY{p}{,} \PY{n}{civilite}\PY{p}{,} \PY{n}{telephone}\PY{p}{)}\PY{p}{:}
             \PY{k}{return} \PY{n}{f}\PY{l+s+s2}{\PYZdq{}}\PY{l+s+si}{\PYZob{}civilite\PYZcb{}}\PY{l+s+s2}{ }\PY{l+s+si}{\PYZob{}prenom\PYZcb{}}\PY{l+s+s2}{ }\PY{l+s+si}{\PYZob{}nom\PYZcb{}}\PY{l+s+s2}{ : Poste }\PY{l+s+si}{\PYZob{}telephone\PYZcb{}}\PY{l+s+s2}{\PYZdq{}}
         
         \PY{n}{avec\PYZus{}f\PYZus{}string}\PY{p}{(}\PY{l+s+s1}{\PYZsq{}}\PY{l+s+s1}{Dupont}\PY{l+s+s1}{\PYZsq{}}\PY{p}{,} \PY{l+s+s1}{\PYZsq{}}\PY{l+s+s1}{Jean}\PY{l+s+s1}{\PYZsq{}}\PY{p}{,} \PY{l+s+s1}{\PYZsq{}}\PY{l+s+s1}{Mr}\PY{l+s+s1}{\PYZsq{}}\PY{p}{,} \PY{l+s+s1}{\PYZsq{}}\PY{l+s+s1}{7748}\PY{l+s+s1}{\PYZsq{}}\PY{p}{)}
\end{Verbatim}


\begin{Verbatim}[commandchars=\\\{\},frame=single,framerule=0.3mm,rulecolor=\color{cellframecolor}]
{\color{outcolor}Out[{\color{outcolor}11}]:} 'Mr Jean Dupont : Poste 7748'
\end{Verbatim}
            

    % Add a bibliography block to the postdoc
    
    
    
