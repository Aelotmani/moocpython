    
    
    
    

    

    \hypertarget{huxe9riter-des-types-built-in}{%
\section{\texorpdfstring{Hériter des types
\emph{built-in}~?}{Hériter des types built-in~?}}\label{huxe9riter-des-types-built-in}}

    \hypertarget{compluxe9ment---niveau-avancuxe9}{%
\subsection{Complément - niveau
avancé}\label{compluxe9ment---niveau-avancuxe9}}

    Vous vous demandez peut-être s'il est possible d'hériter des types
\emph{built-in}.

    La réponse est oui, et nous allons voir un exemple qui est parfois très
utile en pratique, c'est le type - ou plus exactement la famille de
types - \texttt{namedtuple}

    \hypertarget{la-notion-de-record}{%
\subsubsection{\texorpdfstring{La notion de
\emph{record}}{La notion de record}}\label{la-notion-de-record}}

    On se place dans un contexte voisin de celui de \emph{record} - en
français enregistrement - qu'on a déjà rencontré souvent~; pour ce
notebook nous allons à nouveau prendre le cas du point à deux
coordonnées x et y. Nous avons déjà vu que pour implémenter un point on
peut utiliser~:

    \hypertarget{un-dictionnaire}{%
\subparagraph{un dictionnaire}\label{un-dictionnaire}}

    \begin{Verbatim}[commandchars=\\\{\}]
{\color{incolor}In [{\color{incolor}1}]:} \PY{n}{p1} \PY{o}{=} \PY{p}{\PYZob{}}\PY{l+s+s1}{\PYZsq{}}\PY{l+s+s1}{x}\PY{l+s+s1}{\PYZsq{}}\PY{p}{:} \PY{l+m+mi}{1}\PY{p}{,} \PY{l+s+s1}{\PYZsq{}}\PY{l+s+s1}{y}\PY{l+s+s1}{\PYZsq{}}\PY{p}{:} \PY{l+m+mi}{2}\PY{p}{\PYZcb{}}
        \PY{c+c1}{\PYZsh{} ou de manière équivalente}
        \PY{n}{p1} \PY{o}{=} \PY{n+nb}{dict}\PY{p}{(}\PY{n}{x}\PY{o}{=}\PY{l+m+mi}{1}\PY{p}{,} \PY{n}{y}\PY{o}{=}\PY{l+m+mi}{2}\PY{p}{)}
\end{Verbatim}


    \hypertarget{ou-une-classe}{%
\subparagraph{ou une classe}\label{ou-une-classe}}

    \begin{Verbatim}[commandchars=\\\{\}]
{\color{incolor}In [{\color{incolor}2}]:} \PY{k}{class} \PY{n+nc}{Point}\PY{p}{:}
            \PY{k}{def} \PY{n+nf}{\PYZus{}\PYZus{}init\PYZus{}\PYZus{}}\PY{p}{(}\PY{n+nb+bp}{self}\PY{p}{,} \PY{n}{x}\PY{p}{,} \PY{n}{y}\PY{p}{)}\PY{p}{:}
                \PY{n+nb+bp}{self}\PY{o}{.}\PY{n}{x} \PY{o}{=} \PY{n}{x}
                \PY{n+nb+bp}{self}\PY{o}{.}\PY{n}{y} \PY{o}{=} \PY{n}{y}
        
        \PY{n}{p2} \PY{o}{=} \PY{n}{Point}\PY{p}{(}\PY{l+m+mi}{1}\PY{p}{,} \PY{l+m+mi}{2}\PY{p}{)}
\end{Verbatim}


    Nous allons voir une troisième façon de s'y prendre, qui présente deux
caractéristiques~:

\begin{itemize}
\tightlist
\item
  les objets seront non-mutables (en fait ce sont des tuples)~;
\item
  et accessoirement on pourra accéder aux différents champs par leur nom
  aussi bien que par un index.
\end{itemize}

    Pous faire ça il nous faut donc créer une sous-classe de
\texttt{tuple}~; pour nous simplifier la vie,
\href{https://docs.python.org/3/library/collections.html\#collections.namedtuple}{le
module \texttt{collections} nous offre un utilitaire}~:

    \hypertarget{namedtuple}{%
\subparagraph{\texorpdfstring{\texttt{namedtuple}}{namedtuple}}\label{namedtuple}}

    \begin{Verbatim}[commandchars=\\\{\}]
{\color{incolor}In [{\color{incolor}3}]:} \PY{k+kn}{from} \PY{n+nn}{collections} \PY{k}{import} \PY{n}{namedtuple}
\end{Verbatim}


    Techniquement, il s'agit d'une fonction~:

    \begin{Verbatim}[commandchars=\\\{\}]
{\color{incolor}In [{\color{incolor}4}]:} \PY{n+nb}{type}\PY{p}{(}\PY{n}{namedtuple}\PY{p}{)}
\end{Verbatim}


\begin{Verbatim}[commandchars=\\\{\}]
{\color{outcolor}Out[{\color{outcolor}4}]:} function
\end{Verbatim}
            
    qui \textbf{renvoie une classe} - oui les classes sont des objets comme
les autres~; par exemple pour créer une classe \texttt{TuplePoint}, on
ferait~:

    \begin{Verbatim}[commandchars=\\\{\}]
{\color{incolor}In [{\color{incolor}5}]:} \PY{c+c1}{\PYZsh{} on passe à namedtuple}
        \PY{c+c1}{\PYZsh{}  \PYZhy{} le nom du type qu\PYZsq{}on veut créer}
        \PY{c+c1}{\PYZsh{}  \PYZhy{} la liste ordonnée des composants (champs)}
        \PY{n}{TuplePoint} \PY{o}{=} \PY{n}{namedtuple}\PY{p}{(}\PY{l+s+s1}{\PYZsq{}}\PY{l+s+s1}{TuplePoint}\PY{l+s+s1}{\PYZsq{}}\PY{p}{,} \PY{p}{[}\PY{l+s+s1}{\PYZsq{}}\PY{l+s+s1}{x}\PY{l+s+s1}{\PYZsq{}}\PY{p}{,} \PY{l+s+s1}{\PYZsq{}}\PY{l+s+s1}{y}\PY{l+s+s1}{\PYZsq{}}\PY{p}{]}\PY{p}{)}
\end{Verbatim}


    Et maintenant si je crée un objet~:

    \begin{Verbatim}[commandchars=\\\{\}]
{\color{incolor}In [{\color{incolor}6}]:} \PY{n}{p3} \PY{o}{=} \PY{n}{TuplePoint}\PY{p}{(}\PY{l+m+mi}{1}\PY{p}{,} \PY{l+m+mi}{2}\PY{p}{)}
\end{Verbatim}


    \begin{Verbatim}[commandchars=\\\{\}]
{\color{incolor}In [{\color{incolor}7}]:} \PY{c+c1}{\PYZsh{} cet objet est un tuple}
        \PY{n+nb}{isinstance}\PY{p}{(}\PY{n}{p3}\PY{p}{,} \PY{n+nb}{tuple}\PY{p}{)}
\end{Verbatim}


\begin{Verbatim}[commandchars=\\\{\}]
{\color{outcolor}Out[{\color{outcolor}7}]:} True
\end{Verbatim}
            
    \begin{Verbatim}[commandchars=\\\{\}]
{\color{incolor}In [{\color{incolor}8}]:} \PY{c+c1}{\PYZsh{} auquel je peux accéder par index}
        \PY{c+c1}{\PYZsh{} comme un tuple}
        \PY{n}{p3}\PY{p}{[}\PY{l+m+mi}{0}\PY{p}{]}
\end{Verbatim}


\begin{Verbatim}[commandchars=\\\{\}]
{\color{outcolor}Out[{\color{outcolor}8}]:} 1
\end{Verbatim}
            
    \begin{Verbatim}[commandchars=\\\{\}]
{\color{incolor}In [{\color{incolor}9}]:} \PY{c+c1}{\PYZsh{} mais aussi par nom via un attribut}
        \PY{n}{p3}\PY{o}{.}\PY{n}{x}
\end{Verbatim}


\begin{Verbatim}[commandchars=\\\{\}]
{\color{outcolor}Out[{\color{outcolor}9}]:} 1
\end{Verbatim}
            
    \begin{Verbatim}[commandchars=\\\{\}]
{\color{incolor}In [{\color{incolor}10}]:} \PY{c+c1}{\PYZsh{} et comme c\PYZsq{}est un tuple il est immuable}
         \PY{k}{try}\PY{p}{:}
             \PY{n}{p3}\PY{o}{.}\PY{n}{x} \PY{o}{=} \PY{l+m+mi}{10}
         \PY{k}{except} \PY{n+ne}{Exception} \PY{k}{as} \PY{n}{e}\PY{p}{:}
             \PY{n+nb}{print}\PY{p}{(}\PY{n}{f}\PY{l+s+s2}{\PYZdq{}}\PY{l+s+s2}{OOPS }\PY{l+s+s2}{\PYZob{}}\PY{l+s+s2}{type(e)\PYZcb{} }\PY{l+s+si}{\PYZob{}e\PYZcb{}}\PY{l+s+s2}{\PYZdq{}}\PY{p}{)}
\end{Verbatim}


    \begin{Verbatim}[commandchars=\\\{\}]
OOPS <class 'AttributeError'> can't set attribute

    \end{Verbatim}

    \hypertarget{uxe0-quoi-uxe7a-sert}{%
\subsubsection{À quoi ça sert}\label{uxe0-quoi-uxe7a-sert}}

    Les \texttt{namedtuple} ne sont pas d'un usage fréquent, mais on en a
déjà rencontré un exemple dans le notebook sur le module
\texttt{pathlib}. En effet le type de retour de la méthode
\texttt{Path.stat} est un \texttt{namedtuple}~:

    \begin{Verbatim}[commandchars=\\\{\}]
{\color{incolor}In [{\color{incolor}11}]:} \PY{k+kn}{from} \PY{n+nn}{pathlib} \PY{k}{import} \PY{n}{Path}
         \PY{n}{dot\PYZus{}stat} \PY{o}{=} \PY{n}{Path}\PY{p}{(}\PY{l+s+s1}{\PYZsq{}}\PY{l+s+s1}{.}\PY{l+s+s1}{\PYZsq{}}\PY{p}{)}\PY{o}{.}\PY{n}{stat}\PY{p}{(}\PY{p}{)}
\end{Verbatim}


    \begin{Verbatim}[commandchars=\\\{\}]
{\color{incolor}In [{\color{incolor}12}]:} \PY{n}{dot\PYZus{}stat}
\end{Verbatim}


\begin{Verbatim}[commandchars=\\\{\}]
{\color{outcolor}Out[{\color{outcolor}12}]:} os.stat\_result(st\_mode=16877, st\_ino=1808553, st\_dev=16777220, st\_nlink=77, st\_uid=501, st\_gid=20, st\_size=2618, st\_atime=1540404776, st\_mtime=1540404643, st\_ctime=1540404643)
\end{Verbatim}
            
    \begin{Verbatim}[commandchars=\\\{\}]
{\color{incolor}In [{\color{incolor}13}]:} \PY{n+nb}{isinstance}\PY{p}{(}\PY{n}{dot\PYZus{}stat}\PY{p}{,} \PY{n+nb}{tuple}\PY{p}{)}
\end{Verbatim}


\begin{Verbatim}[commandchars=\\\{\}]
{\color{outcolor}Out[{\color{outcolor}13}]:} True
\end{Verbatim}
            
    \hypertarget{nom}{%
\subsubsection{Nom}\label{nom}}

    Quand on crée une classe avec l'instruction \texttt{class}, on ne
mentionne le nom de la classe qu'une seule fois. Ici vous avez remarqué
qu'il faut en pratique le donner deux fois. Pour être précis, le
paramètre qu'on a passé à \texttt{namedtuple} sert à ranger le nom dans
l'attribut \texttt{\_\_name\_\_} de la classe créée~:

    \begin{Verbatim}[commandchars=\\\{\}]
{\color{incolor}In [{\color{incolor}14}]:} \PY{n}{Foo} \PY{o}{=} \PY{n}{namedtuple}\PY{p}{(}\PY{l+s+s1}{\PYZsq{}}\PY{l+s+s1}{Bar}\PY{l+s+s1}{\PYZsq{}}\PY{p}{,} \PY{p}{[}\PY{l+s+s1}{\PYZsq{}}\PY{l+s+s1}{spam}\PY{l+s+s1}{\PYZsq{}}\PY{p}{,} \PY{l+s+s1}{\PYZsq{}}\PY{l+s+s1}{eggs}\PY{l+s+s1}{\PYZsq{}}\PY{p}{]}\PY{p}{)}
\end{Verbatim}


    \begin{Verbatim}[commandchars=\\\{\}]
{\color{incolor}In [{\color{incolor}15}]:} \PY{c+c1}{\PYZsh{} Foo est le nom de la variable classe}
         \PY{n}{foo} \PY{o}{=} \PY{n}{Foo}\PY{p}{(}\PY{l+m+mi}{1}\PY{p}{,} \PY{l+m+mi}{2}\PY{p}{)}
\end{Verbatim}


    \begin{Verbatim}[commandchars=\\\{\}]
{\color{incolor}In [{\color{incolor}16}]:} \PY{c+c1}{\PYZsh{} mais cette classe a son attribut \PYZus{}\PYZus{}name\PYZus{}\PYZus{} mal positionné}
         \PY{n}{Foo}\PY{o}{.}\PY{n+nv+vm}{\PYZus{}\PYZus{}name\PYZus{}\PYZus{}}
\end{Verbatim}


\begin{Verbatim}[commandchars=\\\{\}]
{\color{outcolor}Out[{\color{outcolor}16}]:} 'Bar'
\end{Verbatim}
            
    Il est donc évidemment préférable d'utiliser deux fois le même nom..

    \hypertarget{muxe9moire}{%
\subsubsection{Mémoire}\label{muxe9moire}}

    À titre de comparaison voici la place prise par chacun de ces objets~;
le \texttt{namedtuple} ne semble pas de ce point de vue spécialement
attractif par rapport à une instance~:

    \begin{Verbatim}[commandchars=\\\{\}]
{\color{incolor}In [{\color{incolor}17}]:} \PY{k+kn}{import} \PY{n+nn}{sys}
         
         \PY{c+c1}{\PYZsh{} p1 = dict / p2 = instance / p3 = namedtuple}
         
         \PY{k}{for} \PY{n}{p} \PY{o+ow}{in} \PY{n}{p1}\PY{p}{,} \PY{n}{p2}\PY{p}{,} \PY{n}{p3}\PY{p}{:}
             \PY{n+nb}{print}\PY{p}{(}\PY{n}{sys}\PY{o}{.}\PY{n}{getsizeof}\PY{p}{(}\PY{n}{p}\PY{p}{)}\PY{p}{)}
\end{Verbatim}


    \begin{Verbatim}[commandchars=\\\{\}]
240
56
64

    \end{Verbatim}

    \hypertarget{duxe9finir-des-muxe9thodes-sur-un-namedtuple}{%
\subsubsection{\texorpdfstring{Définir des méthodes sur un
\texttt{namedtuple}}{Définir des méthodes sur un namedtuple}}\label{duxe9finir-des-muxe9thodes-sur-un-namedtuple}}

    Dans un des compléments de la séquence précédente, intitulé
\emph{``Manipuler des ensembles d'instances''}, nous avions vu comment
redéfinir le protocole \emph{hashable} sur des instances, en mettant en
évidence la nécessité de disposer d'instances non mutables lorsqu'on
veut redéfinir \texttt{\_\_hash\_\_()}.

Voyons ici comment on pourrait tirer parti d'un \texttt{namedtuple} pour
refaire proprement notre classe \texttt{Point2} - souvenez-vous, il
s'agissait de rechercher dans un ensemble de points.

    \begin{Verbatim}[commandchars=\\\{\}]
{\color{incolor}In [{\color{incolor}18}]:} \PY{n}{Point2} \PY{o}{=} \PY{n}{namedtuple}\PY{p}{(}\PY{l+s+s1}{\PYZsq{}}\PY{l+s+s1}{Point2}\PY{l+s+s1}{\PYZsq{}}\PY{p}{,} \PY{p}{[}\PY{l+s+s1}{\PYZsq{}}\PY{l+s+s1}{x}\PY{l+s+s1}{\PYZsq{}}\PY{p}{,} \PY{l+s+s1}{\PYZsq{}}\PY{l+s+s1}{y}\PY{l+s+s1}{\PYZsq{}}\PY{p}{]}\PY{p}{)}
\end{Verbatim}


    Sans utiliser le mot-clé \texttt{class}, il faudrait se livrer à une
petite gymnastique pour redéfinir les méthodes spéciales sur la classe
\texttt{Point2}. Nous allons utiliser l'héritage pour arriver au même
résultat~:

    \begin{Verbatim}[commandchars=\\\{\}]
{\color{incolor}In [{\color{incolor}19}]:} \PY{c+c1}{\PYZsh{} ce code est très proche du code utilisé dans le précédent complément}
         \PY{k}{class} \PY{n+nc}{Point2}\PY{p}{(}\PY{n}{namedtuple}\PY{p}{(}\PY{l+s+s1}{\PYZsq{}}\PY{l+s+s1}{Point2}\PY{l+s+s1}{\PYZsq{}}\PY{p}{,} \PY{p}{[}\PY{l+s+s1}{\PYZsq{}}\PY{l+s+s1}{x}\PY{l+s+s1}{\PYZsq{}}\PY{p}{,} \PY{l+s+s1}{\PYZsq{}}\PY{l+s+s1}{y}\PY{l+s+s1}{\PYZsq{}}\PY{p}{]}\PY{p}{)}\PY{p}{)}\PY{p}{:}
         
             \PY{c+c1}{\PYZsh{} l\PYZsq{}égalité va se baser naturellement sur x et y}
             \PY{k}{def} \PY{n+nf}{\PYZus{}\PYZus{}eq\PYZus{}\PYZus{}}\PY{p}{(}\PY{n+nb+bp}{self}\PY{p}{,} \PY{n}{other}\PY{p}{)}\PY{p}{:}
                 \PY{k}{return} \PY{n+nb+bp}{self}\PY{o}{.}\PY{n}{x} \PY{o}{==} \PY{n}{other}\PY{o}{.}\PY{n}{x} \PY{o+ow}{and} \PY{n+nb+bp}{self}\PY{o}{.}\PY{n}{y} \PY{o}{==} \PY{n}{other}\PY{o}{.}\PY{n}{y}
         
             \PY{c+c1}{\PYZsh{} du coup la fonction de hachage }
             \PY{c+c1}{\PYZsh{} dépend aussi de x et de y}
             \PY{k}{def} \PY{n+nf}{\PYZus{}\PYZus{}hash\PYZus{}\PYZus{}}\PY{p}{(}\PY{n+nb+bp}{self}\PY{p}{)}\PY{p}{:}
                 \PY{k}{return} \PY{p}{(}\PY{l+m+mi}{11} \PY{o}{*} \PY{n+nb+bp}{self}\PY{o}{.}\PY{n}{x} \PY{o}{+} \PY{n+nb+bp}{self}\PY{o}{.}\PY{n}{y}\PY{p}{)} \PY{o}{/}\PY{o}{/} \PY{l+m+mi}{16}
\end{Verbatim}


    Avec ceci en place on peut maintenant faire:

    \begin{Verbatim}[commandchars=\\\{\}]
{\color{incolor}In [{\color{incolor}20}]:} \PY{c+c1}{\PYZsh{} trois points égaux au sens de cette classe}
         \PY{n}{q1}\PY{p}{,} \PY{n}{q2}\PY{p}{,} \PY{n}{q3} \PY{o}{=} \PY{n}{Point2}\PY{p}{(}\PY{l+m+mi}{10}\PY{p}{,} \PY{l+m+mi}{10}\PY{p}{)}\PY{p}{,} \PY{n}{Point2}\PY{p}{(}\PY{l+m+mi}{10}\PY{p}{,} \PY{l+m+mi}{10}\PY{p}{)}\PY{p}{,} \PY{n}{Point2}\PY{p}{(}\PY{l+m+mi}{10}\PY{p}{,} \PY{l+m+mi}{10}\PY{p}{)}
\end{Verbatim}


    \begin{Verbatim}[commandchars=\\\{\}]
{\color{incolor}In [{\color{incolor}21}]:} \PY{c+c1}{\PYZsh{} deux objets distincts}
         \PY{n}{q1} \PY{o+ow}{is} \PY{n}{q2}
\end{Verbatim}


\begin{Verbatim}[commandchars=\\\{\}]
{\color{outcolor}Out[{\color{outcolor}21}]:} False
\end{Verbatim}
            
    \begin{Verbatim}[commandchars=\\\{\}]
{\color{incolor}In [{\color{incolor}22}]:} \PY{c+c1}{\PYZsh{} mais égaux}
         \PY{n}{q1} \PY{o}{==} \PY{n}{q2}
\end{Verbatim}


\begin{Verbatim}[commandchars=\\\{\}]
{\color{outcolor}Out[{\color{outcolor}22}]:} True
\end{Verbatim}
            
    \begin{Verbatim}[commandchars=\\\{\}]
{\color{incolor}In [{\color{incolor}23}]:} \PY{c+c1}{\PYZsh{} ne font qu\PYZsq{}un dans un ensemble}
         \PY{n}{s} \PY{o}{=} \PY{p}{\PYZob{}}\PY{n}{q1}\PY{p}{,} \PY{n}{q2}\PY{p}{\PYZcb{}}
         \PY{n+nb}{len}\PY{p}{(}\PY{n}{s}\PY{p}{)}
\end{Verbatim}


\begin{Verbatim}[commandchars=\\\{\}]
{\color{outcolor}Out[{\color{outcolor}23}]:} 1
\end{Verbatim}
            
    \begin{Verbatim}[commandchars=\\\{\}]
{\color{incolor}In [{\color{incolor}24}]:} \PY{c+c1}{\PYZsh{} et on peut les trouver}
         \PY{c+c1}{\PYZsh{} par le troisiéme}
         \PY{n}{q3} \PY{o+ow}{in} \PY{n}{s}
\end{Verbatim}


\begin{Verbatim}[commandchars=\\\{\}]
{\color{outcolor}Out[{\color{outcolor}24}]:} True
\end{Verbatim}
            
    \begin{Verbatim}[commandchars=\\\{\}]
{\color{incolor}In [{\color{incolor}25}]:} \PY{c+c1}{\PYZsh{} et les instances ne sont pas mutables}
         \PY{k}{try}\PY{p}{:}
             \PY{n}{q1}\PY{o}{.}\PY{n}{x} \PY{o}{=} \PY{l+m+mi}{100}
         \PY{k}{except} \PY{n+ne}{Exception} \PY{k}{as} \PY{n}{e}\PY{p}{:}
             \PY{n+nb}{print}\PY{p}{(}\PY{n}{f}\PY{l+s+s2}{\PYZdq{}}\PY{l+s+s2}{OOPS }\PY{l+s+s2}{\PYZob{}}\PY{l+s+s2}{type(e)\PYZcb{}}\PY{l+s+s2}{\PYZdq{}}\PY{p}{)}
\end{Verbatim}


    \begin{Verbatim}[commandchars=\\\{\}]
OOPS <class 'AttributeError'>

    \end{Verbatim}

    \hypertarget{pour-en-savoir-plus}{%
\section{Pour en savoir plus}\label{pour-en-savoir-plus}}

    Vous pouvez vous reporter
\href{https://docs.python.org/3/library/collections.html\#collections.namedtuple}{à
la documentation officielle}.

    Si vous êtes intéressés de savoir comment on peut bien arriver à rendre
les objets d'une classe immuable, vous pouvez commencer par regarder le
code utilisé par \texttt{namedtuple} pour créer son résultat, en
l'invoquant avec le mode bavard (cette possibilité a disparu,
malheureusement, dans python-3.7).

Vous y remarquerez notamment~:

\begin{itemize}
\item
  une redéfinition de
  \href{https://docs.python.org/3/reference/datamodel.html\#object.__new__}{la
  méthode spéciale \texttt{\_\_new\_\_}},
\item
  et aussi un usage des \texttt{property} que l'on a rencontrés en début
  de semaine.
\end{itemize}

    \begin{Verbatim}[commandchars=\\\{\}]
{\color{incolor}In [{\color{incolor} }]:} \PY{c+c1}{\PYZsh{} NOTE}
        \PY{c+c1}{\PYZsh{} automatic execution has skipped execution of this cell}
        
        \PY{c+c1}{\PYZsh{} exécuter ceci pour voir le détail de ce que fait `namedtuple` }
        \PY{n}{Point} \PY{o}{=} \PY{n}{namedtuple}\PY{p}{(}\PY{l+s+s1}{\PYZsq{}}\PY{l+s+s1}{Point}\PY{l+s+s1}{\PYZsq{}}\PY{p}{,} \PY{p}{[}\PY{l+s+s1}{\PYZsq{}}\PY{l+s+s1}{x}\PY{l+s+s1}{\PYZsq{}}\PY{p}{,} \PY{l+s+s1}{\PYZsq{}}\PY{l+s+s1}{y}\PY{l+s+s1}{\PYZsq{}}\PY{p}{]}\PY{p}{,} \PY{n}{verbose}\PY{o}{=}\PY{k+kc}{True}\PY{p}{)}
\end{Verbatim}



    % Add a bibliography block to the postdoc
    
    
    
