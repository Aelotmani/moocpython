    \hypertarget{dictionnaires-et-listes}{%
\section{Dictionnaires et listes}\label{dictionnaires-et-listes}}

    \hypertarget{exercice---niveau-basique}{%
\subsection{Exercice - niveau basique}\label{exercice---niveau-basique}}

    \begin{Verbatim}[commandchars=\\\{\}]
{\color{incolor}In [{\color{incolor} }]:} \PY{k+kn}{from} \PY{n+nn}{corrections}\PY{n+nn}{.}\PY{n+nn}{exo\PYZus{}graph\PYZus{}dict} \PY{k}{import} \PY{n}{exo\PYZus{}graph\PYZus{}dict}
\end{Verbatim}


    On veut implémenter un petit modèle de graphes. Comme on a les données
dans des fichiers, on veut analyser des fichiers d'entrée qui
ressemblent à ceci~:

    \begin{Verbatim}[commandchars=\\\{\}]
{\color{incolor}In [{\color{incolor} }]:} \PY{o}{!}cat data/graph1.txt
\end{Verbatim}


    qui signifierait~:
    
\begin{itemize}
	\item 
	un graphe à 3 sommets \emph{s1}, \emph{s2} et \emph{s3};
	\item
	et 4 arêtes
	\begin{itemize}
		\item 
		une entre \emph{s1} et \emph{s2} de longueur 10;
		\item
		une entre \emph{s2} et \emph{s3} de longueur 12;
		\item
		etc\ldots{}
	\end{itemize}
\end{itemize}

    On vous demande d'écrire une fonction qui lit un tel fichier texte, et
construit (et retourne) un dictionnaire Python qui représente ce graphe.

Dans cet exercice on choisit~:

\begin{itemize}
	\item
	de modéliser le graphe comme un dictionnaire indexé sur les (noms de) sommets~;
	\item
	et chaque valeur est une liste de tuples de la forme (\emph{suivant}, \emph{longueur}), dans
	l'ordre d'apparition dans le fichier d'entrée.
\end{itemize}

    \begin{Verbatim}[commandchars=\\\{\}]
{\color{incolor}In [{\color{incolor} }]:} \PY{c+c1}{\PYZsh{} voici ce qu\PYZsq{}on obtiendrait par exemple avec les données ci\PYZhy{}dessus}
        \PY{n}{exo\PYZus{}graph\PYZus{}dict}\PY{o}{.}\PY{n}{example}\PY{p}{(}\PY{p}{)}
\end{Verbatim}


    \textbf{Notes}
    
\begin{itemize}
	\item
    Vous remarquerez que l'exemple ci-dessus retourne un
	dictionnaire standard; une solution qui utiliserait \texttt{defaultdict}
	est acceptable également;
	\item
	Notez bien également que dans le résultat,
	la longueur d'un arc est attendue comme un \textbf{\texttt{int}}.
\end{itemize}

    \begin{Verbatim}[commandchars=\\\{\}]
{\color{incolor}In [{\color{incolor} }]:} \PY{c+c1}{\PYZsh{} n\PYZsq{}oubliez pas d\PYZsq{}importer si nécessaire}
        
        \PY{c+c1}{\PYZsh{} à vous de jouer}
        \PY{k}{def} \PY{n+nf}{graph\PYZus{}dict}\PY{p}{(}\PY{n}{filename}\PY{p}{)}\PY{p}{:}
            \PY{l+s+s2}{\PYZdq{}}\PY{l+s+s2}{votre code}\PY{l+s+s2}{\PYZdq{}}
\end{Verbatim}


    \begin{Verbatim}[commandchars=\\\{\}]
{\color{incolor}In [{\color{incolor} }]:} \PY{n}{exo\PYZus{}graph\PYZus{}dict}\PY{o}{.}\PY{n}{correction}\PY{p}{(}\PY{n}{graph\PYZus{}dict}\PY{p}{)}
\end{Verbatim}