    
    
    
    

    

    \hypertarget{exercice-sur-les-ensembles}{%
\section{Exercice sur les ensembles}\label{exercice-sur-les-ensembles}}

    \hypertarget{exercice---niveau-intermuxe9diaire}{%
\subsection{Exercice - niveau
intermédiaire}\label{exercice---niveau-intermuxe9diaire}}

    \begin{Verbatim}[commandchars=\\\{\}]
{\color{incolor}In [{\color{incolor}1}]:} \PY{c+c1}{\PYZsh{} chargement de l\PYZsq{}exercice}
        \PY{k+kn}{from} \PY{n+nn}{corrections}\PY{n+nn}{.}\PY{n+nn}{exo\PYZus{}marine\PYZus{}set} \PY{k}{import} \PY{n}{exo\PYZus{}diff}
\end{Verbatim}


    \hypertarget{les-donnuxe9es}{%
\subsubsection{Les données}\label{les-donnuxe9es}}

    Nous reprenons le même genre de données marines en provenance de
MarineTraffic que nous avons vues dans l'exercice précédent.

    \begin{Verbatim}[commandchars=\\\{\}]
{\color{incolor}In [{\color{incolor}2}]:} \PY{k+kn}{from} \PY{n+nn}{corrections}\PY{n+nn}{.}\PY{n+nn}{exo\PYZus{}marine\PYZus{}set} \PY{k}{import} \PY{n}{abbreviated}\PY{p}{,} \PY{n}{extended}
\end{Verbatim}


    \hypertarget{rappels-sur-les-formats}{%
\subsubsection{Rappels sur les formats}\label{rappels-sur-les-formats}}

    \begin{verbatim}
étendu: [id, latitude, longitude, date_heure, nom_bateau, code_pays...]
abrégé: [id, latitude, longitude, date_heure]
\end{verbatim}

    \begin{Verbatim}[commandchars=\\\{\}]
{\color{incolor}In [{\color{incolor}3}]:} \PY{n+nb}{print}\PY{p}{(}\PY{n}{extended}\PY{p}{[}\PY{l+m+mi}{0}\PY{p}{]}\PY{p}{)}
\end{Verbatim}


    \begin{Verbatim}[commandchars=\\\{\}]
[305880000, 48.95318, -5.629418, '2013-10-08T21:51:00', 'MISTRAL', 'AG', '', 'ANTWERP']

    \end{Verbatim}

    \begin{Verbatim}[commandchars=\\\{\}]
{\color{incolor}In [{\color{incolor}4}]:} \PY{n+nb}{print}\PY{p}{(}\PY{n}{abbreviated}\PY{p}{[}\PY{l+m+mi}{0}\PY{p}{]}\PY{p}{)}
\end{Verbatim}


    \begin{Verbatim}[commandchars=\\\{\}]
[227843000, 47.34981, -5.879823, '2013-10-08T22:58:00']

    \end{Verbatim}

    \hypertarget{but-de-lexercice}{%
\subsubsection{But de l'exercice}\label{but-de-lexercice}}

    \begin{Verbatim}[commandchars=\\\{\}]
{\color{incolor}In [{\color{incolor}5}]:} \PY{c+c1}{\PYZsh{} chargement de l\PYZsq{}exercice}
        \PY{k+kn}{from} \PY{n+nn}{corrections}\PY{n+nn}{.}\PY{n+nn}{exo\PYZus{}marine\PYZus{}set} \PY{k}{import} \PY{n}{exo\PYZus{}diff}
\end{Verbatim}


    Notez bien une différence importante avec l'exercice précédent~: cette
fois \textbf{il n'y a plus correspondance} entre les bateaux rapportés
dans les données étendues et abrégées.

Le but de l'exercice est précisément d'étudier la différence, et pour
cela on vous demande d'écrire une fonction

\begin{Shaded}
\begin{Highlighting}[]
\NormalTok{    diff(extended, abbreviated)}
\end{Highlighting}
\end{Shaded}

qui retourne un tuple à trois éléments~:

\begin{itemize}
\tightlist
\item
  l'ensemble (\texttt{set}) des \textbf{noms} des bateaux présents dans
  \texttt{extended} mais pas dans \texttt{abbreviated}~;
\item
  l'ensemble des \textbf{noms} des bateaux présents dans
  \texttt{extended} et dans \texttt{abbreviated}~;
\item
  l'ensemble des \textbf{id} des bateaux présents dans
  \texttt{abbreviated} mais pas dans \texttt{extended} (par
  construction, les données ne nous permettent pas d'obtenir les noms de
  ces bateaux).
\end{itemize}

    \begin{Verbatim}[commandchars=\\\{\}]
{\color{incolor}In [{\color{incolor}6}]:} \PY{c+c1}{\PYZsh{} le résultat attendu}
        \PY{n}{result} \PY{o}{=} \PY{n}{exo\PYZus{}diff}\PY{o}{.}\PY{n}{resultat}\PY{p}{(}\PY{n}{extended}\PY{p}{,} \PY{n}{abbreviated}\PY{p}{)}
        
        \PY{c+c1}{\PYZsh{} combien de bateaux sont concernés}
        \PY{k}{def} \PY{n+nf}{show\PYZus{}result}\PY{p}{(}\PY{n}{extended}\PY{p}{,} \PY{n}{abbreviated}\PY{p}{,} \PY{n}{result}\PY{p}{)}\PY{p}{:}
            \PY{l+s+sd}{\PYZdq{}\PYZdq{}\PYZdq{}}
        \PY{l+s+sd}{    Affiche divers décomptes sur les arguments}
        \PY{l+s+sd}{    en entrée et en sortie de diff}
        \PY{l+s+sd}{    \PYZdq{}\PYZdq{}\PYZdq{}}
            \PY{n+nb}{print}\PY{p}{(}\PY{l+m+mi}{10}\PY{o}{*}\PY{l+s+s1}{\PYZsq{}}\PY{l+s+s1}{\PYZhy{}}\PY{l+s+s1}{\PYZsq{}}\PY{p}{,} \PY{l+s+s2}{\PYZdq{}}\PY{l+s+s2}{Les entrées}\PY{l+s+s2}{\PYZdq{}}\PY{p}{)}
            \PY{n+nb}{print}\PY{p}{(}\PY{n}{f}\PY{l+s+s2}{\PYZdq{}}\PY{l+s+s2}{Dans extended: }\PY{l+s+s2}{\PYZob{}}\PY{l+s+s2}{len(extended)\PYZcb{} entrées}\PY{l+s+s2}{\PYZdq{}}\PY{p}{)}
            \PY{n+nb}{print}\PY{p}{(}\PY{n}{f}\PY{l+s+s2}{\PYZdq{}}\PY{l+s+s2}{Dans abbreviated: }\PY{l+s+s2}{\PYZob{}}\PY{l+s+s2}{len(abbreviated)\PYZcb{} entrées}\PY{l+s+s2}{\PYZdq{}}\PY{p}{)}
            \PY{n+nb}{print}\PY{p}{(}\PY{l+m+mi}{10}\PY{o}{*}\PY{l+s+s1}{\PYZsq{}}\PY{l+s+s1}{\PYZhy{}}\PY{l+s+s1}{\PYZsq{}}\PY{p}{,} \PY{l+s+s2}{\PYZdq{}}\PY{l+s+s2}{Le résultat du diff}\PY{l+s+s2}{\PYZdq{}}\PY{p}{)}
            \PY{n}{extended\PYZus{}only}\PY{p}{,} \PY{n}{both}\PY{p}{,} \PY{n}{abbreviated\PYZus{}only} \PY{o}{=} \PY{n}{result}
            \PY{n+nb}{print}\PY{p}{(}\PY{n}{f}\PY{l+s+s2}{\PYZdq{}}\PY{l+s+s2}{Dans extended mais pas dans abbreviated }\PY{l+s+s2}{\PYZob{}}\PY{l+s+s2}{len(extended\PYZus{}only)\PYZcb{}}\PY{l+s+s2}{\PYZdq{}}\PY{p}{)}
            \PY{n+nb}{print}\PY{p}{(}\PY{n}{f}\PY{l+s+s2}{\PYZdq{}}\PY{l+s+s2}{Dans les deux }\PY{l+s+s2}{\PYZob{}}\PY{l+s+s2}{len(both)\PYZcb{}}\PY{l+s+s2}{\PYZdq{}}\PY{p}{)}
            \PY{n+nb}{print}\PY{p}{(}\PY{n}{f}\PY{l+s+s2}{\PYZdq{}}\PY{l+s+s2}{Dans abbreviated mais pas dans extended }\PY{l+s+s2}{\PYZob{}}\PY{l+s+s2}{len(abbreviated\PYZus{}only)\PYZcb{}}\PY{l+s+s2}{\PYZdq{}}\PY{p}{)}
        
        \PY{n}{show\PYZus{}result}\PY{p}{(}\PY{n}{extended}\PY{p}{,} \PY{n}{abbreviated}\PY{p}{,} \PY{n}{result}\PY{p}{)}
\end{Verbatim}


    \begin{Verbatim}[commandchars=\\\{\}]
---------- Les entrées
Dans extended: 4 entrées
Dans abbreviated: 4 entrées
---------- Le résultat du diff
Dans extended mais pas dans abbreviated 2
Dans les deux 2
Dans abbreviated mais pas dans extended 2

    \end{Verbatim}

    \hypertarget{votre-code}{%
\subsubsection{Votre code}\label{votre-code}}

    \begin{Verbatim}[commandchars=\\\{\}]
{\color{incolor}In [{\color{incolor}7}]:} \PY{k}{def} \PY{n+nf}{diff}\PY{p}{(}\PY{n}{extended}\PY{p}{,} \PY{n}{abbreviated}\PY{p}{)}\PY{p}{:}
            \PY{l+s+s2}{\PYZdq{}}\PY{l+s+s2}{\PYZlt{}votre\PYZus{}code\PYZgt{}}\PY{l+s+s2}{\PYZdq{}}
        
        \PY{c+c1}{\PYZsh{} NOTE:}
        \PY{c+c1}{\PYZsh{} auto\PYZhy{}exec\PYZhy{}for\PYZhy{}latex has used instead:}
        \PY{c+c1}{\PYZsh{}\PYZsh{}\PYZsh{}\PYZsh{}\PYZsh{}\PYZsh{}\PYZsh{}\PYZsh{}\PYZsh{}\PYZsh{}}
        \PY{n}{diff} \PY{o}{=} \PY{n}{exo\PYZus{}diff}\PY{o}{.}\PY{n}{solution}
        \PY{c+c1}{\PYZsh{}\PYZsh{}\PYZsh{}\PYZsh{}\PYZsh{}\PYZsh{}\PYZsh{}\PYZsh{}\PYZsh{}\PYZsh{}}
\end{Verbatim}


    \hypertarget{validation}{%
\subsubsection{Validation}\label{validation}}

    \begin{Verbatim}[commandchars=\\\{\}]
{\color{incolor}In [{\color{incolor} }]:} \PY{c+c1}{\PYZsh{} NOTE}
        \PY{c+c1}{\PYZsh{} auto\PYZhy{}exec\PYZhy{}for\PYZhy{}latex has skipped execution of this cell}
        
        \PY{n}{exo\PYZus{}diff}\PY{o}{.}\PY{n}{correction}\PY{p}{(}\PY{n}{diff}\PY{p}{,} \PY{n}{extended}\PY{p}{,} \PY{n}{abbreviated}\PY{p}{)}
\end{Verbatim}


    \hypertarget{des-fichiers-de-donnuxe9es-plus-ruxe9alistes}{%
\subsubsection{Des fichiers de données plus
réalistes}\label{des-fichiers-de-donnuxe9es-plus-ruxe9alistes}}

    Comme pour l'exercice précédent, les données fournies ici sont très
simplistes~; vous pouvez, si vous le voulez, essayer votre code avec des
données (un peu) plus réalistes en chargeant des fichiers de données
plus complets~:

\begin{itemize}
\tightlist
\item
  \url{data/marine-e2-ext.json}
\item
  \url{data/marine-e2-abb.json}
\end{itemize}

Ce qui donnerait en Python~:

    \begin{Verbatim}[commandchars=\\\{\}]
{\color{incolor}In [{\color{incolor}8}]:} \PY{c+c1}{\PYZsh{} load data from files}
        \PY{k+kn}{import} \PY{n+nn}{json}
        
        \PY{k}{with} \PY{n+nb}{open}\PY{p}{(}\PY{l+s+s2}{\PYZdq{}}\PY{l+s+s2}{data/marine\PYZhy{}e2\PYZhy{}ext.json}\PY{l+s+s2}{\PYZdq{}}\PY{p}{,} \PY{n}{encoding}\PY{o}{=}\PY{l+s+s2}{\PYZdq{}}\PY{l+s+s2}{utf\PYZhy{}8}\PY{l+s+s2}{\PYZdq{}}\PY{p}{)} \PY{k}{as} \PY{n}{feed}\PY{p}{:}
            \PY{n}{extended\PYZus{}full} \PY{o}{=} \PY{n}{json}\PY{o}{.}\PY{n}{load}\PY{p}{(}\PY{n}{feed}\PY{p}{)}
        
        \PY{k}{with} \PY{n+nb}{open}\PY{p}{(}\PY{l+s+s2}{\PYZdq{}}\PY{l+s+s2}{data/marine\PYZhy{}e2\PYZhy{}abb.json}\PY{l+s+s2}{\PYZdq{}}\PY{p}{,} \PY{n}{encoding}\PY{o}{=}\PY{l+s+s2}{\PYZdq{}}\PY{l+s+s2}{utf\PYZhy{}8}\PY{l+s+s2}{\PYZdq{}}\PY{p}{)} \PY{k}{as} \PY{n}{feed}\PY{p}{:}
            \PY{n}{abbreviated\PYZus{}full} \PY{o}{=} \PY{n}{json}\PY{o}{.}\PY{n}{load}\PY{p}{(}\PY{n}{feed}\PY{p}{)}
\end{Verbatim}


    \begin{Verbatim}[commandchars=\\\{\}]
{\color{incolor}In [{\color{incolor}9}]:} \PY{c+c1}{\PYZsh{} le résultat de votre fonction sur des données plus vastes}
        \PY{c+c1}{\PYZsh{} attention, show\PYZus{}result fait des hypothèses sur le type de votre résultat}
        \PY{c+c1}{\PYZsh{} aussi si vous essayez d\PYZsq{}exécuter ceci avec comme fonction diff}
        \PY{c+c1}{\PYZsh{} la version vide qui est dans le notebook original}
        \PY{c+c1}{\PYZsh{} cela peut provoquer une exception}
        \PY{n}{diff\PYZus{}full} \PY{o}{=} \PY{n}{diff}\PY{p}{(}\PY{n}{extended\PYZus{}full}\PY{p}{,} \PY{n}{abbreviated\PYZus{}full}\PY{p}{)}
        \PY{n}{show\PYZus{}result}\PY{p}{(}\PY{n}{extended\PYZus{}full}\PY{p}{,} \PY{n}{abbreviated\PYZus{}full}\PY{p}{,} \PY{n}{diff\PYZus{}full}\PY{p}{)}
\end{Verbatim}


    \begin{Verbatim}[commandchars=\\\{\}]
---------- Les entrées
Dans extended: 205 entrées
Dans abbreviated: 200 entrées
---------- Le résultat du diff
Dans extended mais pas dans abbreviated 70
Dans les deux 132
Dans abbreviated mais pas dans extended 68

    \end{Verbatim}

    Je signale enfin à propos de ces données plus complètes que~:

\begin{itemize}
\tightlist
\item
  on a supprimé les entrées correspondant à des bateaux différents mais
  de même nom~; cette situation peut arriver dans la réalité (c'est
  pourquoi d'ailleurs les bateaux ont un \emph{id}) mais ici ce n'est
  pas le cas~;
\item
  il se peut par contre qu'un même bateau fasse l'objet de plusieurs
  mesures dans \texttt{extended} et/ou dans \texttt{abbreviated}.
\end{itemize}


    % Add a bibliography block to the postdoc
    
    
    
