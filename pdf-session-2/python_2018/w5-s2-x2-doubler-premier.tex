    \hypertarget{exercice---niveau-intermuxe9diaire}{%
\subsection{Exercice - niveau
intermédiaire}\label{exercice---niveau-intermuxe9diaire}}

    Les deux exercices de ce notebook font référence également à des notions
vues en fin de semaine 4, sur le passage d'arguments aux fonctions.

    \begin{Verbatim}[commandchars=\\\{\}]
{\color{incolor}In [{\color{incolor} }]:} \PY{c+c1}{\PYZsh{} pour charger l\PYZsq{}exercice}
        \PY{k+kn}{from} \PY{n+nn}{corrections}\PY{n+nn}{.}\PY{n+nn}{exo\PYZus{}doubler\PYZus{}premier} \PY{k}{import} \PY{n}{exo\PYZus{}doubler\PYZus{}premier}
\end{Verbatim}


    On vous demande d'écrire une fonction qui prend en argument~:
    
\begin{itemize}
	\item 
	une fonction \texttt{f}, dont vous savez seulement que le premier argument
	est numérique, et qu'elle ne prend \textbf{que des arguments
	positionnels} (sans valeur par défaut)~;
	\item
	un nombre quelconque - mais
	au moins 1 - d'arguments positionnels \texttt{args}, dont on sait qu'ils
	pourraient être passés à \texttt{f}.
\end{itemize}

Et on attend en retour le résultat de \texttt{f} appliqués à tous ces
arguments, mais avec le premier d'entre eux multiplié par deux.\\

Formellement~: doubler\_premier(f, \(x_1\), \(x_2\),\ldots{}, \(x_n\)) =
f(\(2*x_1\), \(x_2\),\ldots{}, \(x_n\))\\

    Voici d'abord quelques exemples de ce qui est attendu. Pour cela on va
utiliser comme fonctions:

\begin{itemize}
\tightlist
\item
  \texttt{add} et \texttt{mul} sont les opérateurs (binaires) du module
  operator;
\item
  et \texttt{distance} est la fonction qu'on a vu dans un exercice
  précédent; pour rappel
\end{itemize}

\(distance\) (\(x_1\), \ldots{}, \(x_n\)) = \(\sqrt{\sum x_i^2}\)

    \begin{Verbatim}[commandchars=\\\{\}]
{\color{incolor}In [{\color{incolor} }]:} \PY{c+c1}{\PYZsh{} rappel sur la fonction distance:}
        \PY{k+kn}{from} \PY{n+nn}{corrections}\PY{n+nn}{.}\PY{n+nn}{exo\PYZus{}distance} \PY{k}{import} \PY{n}{distance}
        \PY{n}{distance}\PY{p}{(}\PY{l+m+mf}{3.0}\PY{p}{,} \PY{l+m+mf}{4.0}\PY{p}{)}
\end{Verbatim}


    \begin{Verbatim}[commandchars=\\\{\}]
{\color{incolor}In [{\color{incolor} }]:} \PY{n}{distance}\PY{p}{(}\PY{l+m+mf}{4.0}\PY{p}{,} \PY{l+m+mf}{4.0}\PY{p}{,} \PY{l+m+mf}{4.0}\PY{p}{,} \PY{l+m+mf}{4.0}\PY{p}{)}
\end{Verbatim}


    \begin{Verbatim}[commandchars=\\\{\}]
{\color{incolor}In [{\color{incolor} }]:} \PY{c+c1}{\PYZsh{} voici donc quelques exemples de ce qui est attendu.}
        \PY{n}{exo\PYZus{}doubler\PYZus{}premier}\PY{o}{.}\PY{n}{example}\PY{p}{(}\PY{p}{)}
\end{Verbatim}


    \begin{Verbatim}[commandchars=\\\{\}]
{\color{incolor}In [{\color{incolor} }]:} \PY{c+c1}{\PYZsh{} ATTENTION vous devez aussi définir les arguments de la fonction}
        \PY{k}{def} \PY{n+nf}{doubler\PYZus{}premier}\PY{p}{(}\PY{n}{votre}\PY{p}{,} \PY{n}{signature}\PY{p}{)}\PY{p}{:}
            \PY{k}{return} \PY{l+s+s2}{\PYZdq{}}\PY{l+s+s2}{votre code}\PY{l+s+s2}{\PYZdq{}}
\end{Verbatim}


    \begin{Verbatim}[commandchars=\\\{\}]
{\color{incolor}In [{\color{incolor} }]:} \PY{n}{exo\PYZus{}doubler\PYZus{}premier}\PY{o}{.}\PY{n}{correction}\PY{p}{(}\PY{n}{doubler\PYZus{}premier}\PY{p}{)}
\end{Verbatim}


    \hypertarget{exercice---niveau-intermuxe9diaire}{%
\subsection{Exercice - niveau
intermédiaire}\label{exercice---niveau-intermuxe9diaire}}

    \begin{Verbatim}[commandchars=\\\{\}]
{\color{incolor}In [{\color{incolor} }]:} \PY{c+c1}{\PYZsh{} Pour charger l\PYZsq{}exercice}
        \PY{k+kn}{from} \PY{n+nn}{corrections}\PY{n+nn}{.}\PY{n+nn}{exo\PYZus{}doubler\PYZus{}premier\PYZus{}kwds} \PY{k}{import} \PY{n}{exo\PYZus{}doubler\PYZus{}premier\PYZus{}kwds}
\end{Verbatim}


    Vous devez maintenant écrire une deuxième version qui peut fonctionner
avec une fonction quelconque (elle peut avoir des arguments nommés avec
valeurs par défaut).\\

La fonction \texttt{doubler\_premier\_kwds} que l'on vous demande
d'écrire maintenant prend donc un premier argument \texttt{f} qui est
une fonction, un second argument positionnel qui est le premier argument
de \texttt{f} (et donc qu'il faut doubler), et le reste des arguments de
f, qui donc, à nouveau, peuvent être nommés ou non.

    \begin{Verbatim}[commandchars=\\\{\}]
{\color{incolor}In [{\color{incolor} }]:} \PY{c+c1}{\PYZsh{} quelques exemples de ce qui est attendu}
        \PY{c+c1}{\PYZsh{} avec ces deux fonctions }
        
        \PY{k}{def} \PY{n+nf}{add3}\PY{p}{(}\PY{n}{x}\PY{p}{,} \PY{n}{y}\PY{o}{=}\PY{l+m+mi}{0}\PY{p}{,} \PY{n}{z}\PY{o}{=}\PY{l+m+mi}{0}\PY{p}{)}\PY{p}{:}
            \PY{k}{return} \PY{n}{x} \PY{o}{+} \PY{n}{y} \PY{o}{+} \PY{n}{z}
        
        \PY{k}{def} \PY{n+nf}{mul3}\PY{p}{(}\PY{n}{x}\PY{o}{=}\PY{l+m+mi}{1}\PY{p}{,} \PY{n}{y}\PY{o}{=}\PY{l+m+mi}{1}\PY{p}{,} \PY{n}{z}\PY{o}{=}\PY{l+m+mi}{1}\PY{p}{)}\PY{p}{:}
            \PY{k}{return} \PY{n}{x} \PY{o}{*} \PY{n}{y} \PY{o}{*} \PY{n}{z}
        
        \PY{n}{exo\PYZus{}doubler\PYZus{}premier\PYZus{}kwds}\PY{o}{.}\PY{n}{example}\PY{p}{(}\PY{p}{)}
\end{Verbatim}


    Vous remarquerez que l'on n'a pas mentionné dans cette liste d'exemples

\begin{verbatim}
doubler_premier_kwds (muln, x=1, y=1)
\end{verbatim}

que l'on ne demande pas de supporter puisqu'il est bien précisé que
doubler\_premier a deux arguments positionnels.

    \begin{Verbatim}[commandchars=\\\{\}]
{\color{incolor}In [{\color{incolor} }]:} \PY{c+c1}{\PYZsh{} ATTENTION vous devez aussi définir les arguments de la fonction}
        \PY{k}{def} \PY{n+nf}{doubler\PYZus{}premier\PYZus{}kwds}\PY{p}{(}\PY{n}{votre}\PY{p}{,} \PY{n}{signature}\PY{p}{)}\PY{p}{:}
            \PY{l+s+s2}{\PYZdq{}}\PY{l+s+s2}{\PYZlt{}votre code\PYZgt{}}\PY{l+s+s2}{\PYZdq{}}
\end{Verbatim}


    \begin{Verbatim}[commandchars=\\\{\}]
{\color{incolor}In [{\color{incolor} }]:} \PY{n}{exo\PYZus{}doubler\PYZus{}premier\PYZus{}kwds}\PY{o}{.}\PY{n}{correction}\PY{p}{(}\PY{n}{doubler\PYZus{}premier\PYZus{}kwds}\PY{p}{)}
\end{Verbatim}