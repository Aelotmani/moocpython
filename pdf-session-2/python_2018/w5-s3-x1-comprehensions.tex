    
    
    
    

    

    \hypertarget{compruxe9hensions}{%
\section{Compréhensions}\label{compruxe9hensions}}

    \hypertarget{exercice---niveau-basique}{%
\subsection{Exercice - niveau basique}\label{exercice---niveau-basique}}

    \begin{Verbatim}[commandchars=\\\{\}]
{\color{incolor}In [{\color{incolor}1}]:} \PY{c+c1}{\PYZsh{} pour charger l\PYZsq{}exercice}
        \PY{k+kn}{from} \PY{n+nn}{corrections}\PY{n+nn}{.}\PY{n+nn}{exo\PYZus{}aplatir} \PY{k}{import} \PY{n}{exo\PYZus{}aplatir}
\end{Verbatim}


    Il vous est demandé d'écrire une fonction \texttt{aplatir} qui prend
\emph{un unique} argument \texttt{l\_conteneurs} qui est une liste (ou
plus généralement un itérable) de conteneurs (ou plus généralement
d'itérables), et qui retourne la liste de tous les éléments de tous les
conteneurs.

    \begin{Verbatim}[commandchars=\\\{\}]
{\color{incolor}In [{\color{incolor}2}]:} \PY{c+c1}{\PYZsh{} par exemple}
        \PY{n}{exo\PYZus{}aplatir}\PY{o}{.}\PY{n}{example}\PY{p}{(}\PY{p}{)}
\end{Verbatim}


\begin{Verbatim}[commandchars=\\\{\}]
{\color{outcolor}Out[{\color{outcolor}2}]:} <IPython.core.display.HTML object>
\end{Verbatim}
            
    \begin{Verbatim}[commandchars=\\\{\}]
{\color{incolor}In [{\color{incolor}3}]:} \PY{k}{def} \PY{n+nf}{aplatir}\PY{p}{(}\PY{n}{conteneurs}\PY{p}{)}\PY{p}{:}
            \PY{l+s+s2}{\PYZdq{}}\PY{l+s+s2}{\PYZlt{}votre\PYZus{}code\PYZgt{}}\PY{l+s+s2}{\PYZdq{}}
\end{Verbatim}


    \begin{Verbatim}[commandchars=\\\{\}]
{\color{incolor}In [{\color{incolor} }]:} \PY{c+c1}{\PYZsh{} NOTE}
        \PY{c+c1}{\PYZsh{} auto\PYZhy{}exec\PYZhy{}for\PYZhy{}latex has skipped execution of this cell}
        
        \PY{c+c1}{\PYZsh{} vérifier votre code}
        \PY{n}{exo\PYZus{}aplatir}\PY{o}{.}\PY{n}{correction}\PY{p}{(}\PY{n}{aplatir}\PY{p}{)}
\end{Verbatim}


    \hypertarget{exercice---niveau-intermuxe9diaire}{%
\subsection{Exercice - niveau
intermédiaire}\label{exercice---niveau-intermuxe9diaire}}

    \begin{Verbatim}[commandchars=\\\{\}]
{\color{incolor}In [{\color{incolor}4}]:} \PY{c+c1}{\PYZsh{} chargement de l\PYZsq{}exercice}
        \PY{k+kn}{from} \PY{n+nn}{corrections}\PY{n+nn}{.}\PY{n+nn}{exo\PYZus{}alternat} \PY{k}{import} \PY{n}{exo\PYZus{}alternat}
\end{Verbatim}


    À présent, on passe en argument deux conteneurs (deux itérables)
\texttt{c1} et \texttt{c2} de même taille à la fonction
\texttt{alternat}, qui doit construire une liste contenant les éléments
pris alternativement dans \texttt{c1} et dans \texttt{c2}.

    \begin{Verbatim}[commandchars=\\\{\}]
{\color{incolor}In [{\color{incolor}5}]:} \PY{c+c1}{\PYZsh{} exemple}
        \PY{n}{exo\PYZus{}alternat}\PY{o}{.}\PY{n}{example}\PY{p}{(}\PY{p}{)}
\end{Verbatim}


\begin{Verbatim}[commandchars=\\\{\}]
{\color{outcolor}Out[{\color{outcolor}5}]:} <IPython.core.display.HTML object>
\end{Verbatim}
            
    \textbf{Indice} pour cet exercice il peut être pertinent de recourir à
la fonction \emph{built-in} \texttt{zip}.

    \begin{Verbatim}[commandchars=\\\{\}]
{\color{incolor}In [{\color{incolor}6}]:} \PY{k}{def} \PY{n+nf}{alternat}\PY{p}{(}\PY{n}{c1}\PY{p}{,} \PY{n}{c2}\PY{p}{)}\PY{p}{:}
            \PY{l+s+s2}{\PYZdq{}}\PY{l+s+s2}{\PYZlt{}votre\PYZus{}code\PYZgt{}}\PY{l+s+s2}{\PYZdq{}}
\end{Verbatim}


    \begin{Verbatim}[commandchars=\\\{\}]
{\color{incolor}In [{\color{incolor} }]:} \PY{c+c1}{\PYZsh{} NOTE}
        \PY{c+c1}{\PYZsh{} auto\PYZhy{}exec\PYZhy{}for\PYZhy{}latex has skipped execution of this cell}
        
        \PY{c+c1}{\PYZsh{} pour vérifier votre code}
        \PY{n}{exo\PYZus{}alternat}\PY{o}{.}\PY{n}{correction}\PY{p}{(}\PY{n}{alternat}\PY{p}{)}
\end{Verbatim}


    \hypertarget{exercice---niveau-intermuxe9diaire}{%
\subsection{Exercice - niveau
intermédiaire}\label{exercice---niveau-intermuxe9diaire}}

    On se donne deux ensembles A et B de tuples de la forme

\begin{verbatim}
(entier, valeur)
\end{verbatim}

On vous demande d'écrire une fonction \texttt{intersect} qui retourne
l'ensemble des objets \texttt{valeur} associés (dans A ou dans B) à un
entier qui soit présent dans (un tuple de) A \emph{et} dans (un tuple
de) B.

    \begin{Verbatim}[commandchars=\\\{\}]
{\color{incolor}In [{\color{incolor}7}]:} \PY{c+c1}{\PYZsh{} un exemple}
        \PY{k+kn}{from} \PY{n+nn}{corrections}\PY{n+nn}{.}\PY{n+nn}{exo\PYZus{}intersect} \PY{k}{import} \PY{n}{exo\PYZus{}intersect}
        \PY{n}{exo\PYZus{}intersect}\PY{o}{.}\PY{n}{example}\PY{p}{(}\PY{p}{)}
\end{Verbatim}


\begin{Verbatim}[commandchars=\\\{\}]
{\color{outcolor}Out[{\color{outcolor}7}]:} <IPython.core.display.HTML object>
\end{Verbatim}
            
    \begin{Verbatim}[commandchars=\\\{\}]
{\color{incolor}In [{\color{incolor}8}]:} \PY{k}{def} \PY{n+nf}{intersect}\PY{p}{(}\PY{n}{A}\PY{p}{,} \PY{n}{B}\PY{p}{)}\PY{p}{:}
            \PY{l+s+s2}{\PYZdq{}}\PY{l+s+s2}{\PYZlt{}votre\PYZus{}code\PYZgt{}}\PY{l+s+s2}{\PYZdq{}}
\end{Verbatim}


    \begin{Verbatim}[commandchars=\\\{\}]
{\color{incolor}In [{\color{incolor} }]:} \PY{c+c1}{\PYZsh{} NOTE}
        \PY{c+c1}{\PYZsh{} auto\PYZhy{}exec\PYZhy{}for\PYZhy{}latex has skipped execution of this cell}
        
        \PY{c+c1}{\PYZsh{} pour vérifier votre code}
        \PY{n}{exo\PYZus{}intersect}\PY{o}{.}\PY{n}{correction}\PY{p}{(}\PY{n}{intersect}\PY{p}{)}
\end{Verbatim}



    % Add a bibliography block to the postdoc
    
    
    
