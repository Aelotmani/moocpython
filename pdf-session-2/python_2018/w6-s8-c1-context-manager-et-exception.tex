    \hypertarget{context-managers-et-exceptions}{%
\section{\texorpdfstring{\emph{Context managers} et
exceptions}{Context managers et exceptions}}\label{context-managers-et-exceptions}}

    \hypertarget{compluxe9ment---niveau-intermuxe9diaire}{%
\subsection{Complément - niveau
intermédiaire}\label{compluxe9ment---niveau-intermuxe9diaire}}

    On a vu jusqu'ici dans la vidéo comment écrire un context manager; on a
vu notamment qu'il était bon pour la méthode \texttt{\_\_exit\_\_()} de
retourner \texttt{False}, de façon à ce que l'exception soit propagée à
l'instruction \texttt{with}:

    \begin{Verbatim}[commandchars=\\\{\}]
{\color{incolor}In [{\color{incolor}1}]:} \PY{k+kn}{import} \PY{n+nn}{time}
        
        \PY{k}{class} \PY{n+nc}{Timer1}\PY{p}{:}
            \PY{k}{def} \PY{n+nf}{\PYZus{}\PYZus{}enter\PYZus{}\PYZus{}}\PY{p}{(}\PY{n+nb+bp}{self}\PY{p}{)}\PY{p}{:}
                \PY{n+nb}{print}\PY{p}{(}\PY{l+s+s2}{\PYZdq{}}\PY{l+s+s2}{Entering Timer1}\PY{l+s+s2}{\PYZdq{}}\PY{p}{)}
                \PY{n+nb+bp}{self}\PY{o}{.}\PY{n}{start} \PY{o}{=} \PY{n}{time}\PY{o}{.}\PY{n}{time}\PY{p}{(}\PY{p}{)}
                \PY{k}{return} \PY{n+nb+bp}{self}
            
            \PY{c+c1}{\PYZsh{} en règle générale on se contente de propager l\PYZsq{}exception }
            \PY{c+c1}{\PYZsh{} à l\PYZsq{}instruction with englobante}
            \PY{k}{def} \PY{n+nf}{\PYZus{}\PYZus{}exit\PYZus{}\PYZus{}}\PY{p}{(}\PY{n+nb+bp}{self}\PY{p}{,} \PY{o}{*}\PY{n}{args}\PY{p}{)}\PY{p}{:}
                \PY{n+nb}{print}\PY{p}{(}\PY{n}{f}\PY{l+s+s2}{\PYZdq{}}\PY{l+s+s2}{Total duration }\PY{l+s+s2}{\PYZob{}}\PY{l+s+s2}{time.time()\PYZhy{}self.start:2f\PYZcb{}}\PY{l+s+s2}{\PYZdq{}}\PY{p}{)}
        
                \PY{c+c1}{\PYZsh{} et pour cela il suffit que \PYZus{}\PYZus{}exit\PYZus{}\PYZus{} retourne False}
                \PY{k}{return} \PY{k+kc}{False}
\end{Verbatim}


    Ainsi si le corps de l'instruction lève une exception, celle-ci est
propagée :

    \begin{Verbatim}[commandchars=\\\{\}]
{\color{incolor}In [{\color{incolor}2}]:} \PY{k+kn}{import} \PY{n+nn}{time}
        \PY{k}{try}\PY{p}{:}
            \PY{k}{with} \PY{n}{Timer1}\PY{p}{(}\PY{p}{)}\PY{p}{:}
                \PY{n}{time}\PY{o}{.}\PY{n}{sleep}\PY{p}{(}\PY{l+m+mf}{0.5}\PY{p}{)}
                \PY{l+m+mi}{1}\PY{o}{/}\PY{l+m+mi}{0}
        \PY{k}{except} \PY{n+ne}{Exception} \PY{k}{as} \PY{n}{e}\PY{p}{:}
            \PY{c+c1}{\PYZsh{} on va bien recevoir cette exception}
            \PY{n+nb}{print}\PY{p}{(}\PY{n}{f}\PY{l+s+s2}{\PYZdq{}}\PY{l+s+s2}{OOPS \PYZhy{}\PYZgt{} }\PY{l+s+s2}{\PYZob{}}\PY{l+s+s2}{type(e)\PYZcb{}}\PY{l+s+s2}{\PYZdq{}}\PY{p}{)}
\end{Verbatim}


    \begin{Verbatim}[commandchars=\\\{\}]
Entering Timer1
Total duration 0.514807
OOPS -> <class 'ZeroDivisionError'>

    \end{Verbatim}

    À la toute première itération de la boucle, on fait une division par 0
qui lève l'exception \texttt{ZeroDivisionError}, qui passe bien à
l'appelant.\\

Il est important, lorsqu'on conçoit un context manager, de bien
\textbf{propager} les exceptions qui ne sont pas liées au fonctionnement
attendu du context manager. Par exemple un objet de type fichier va par
exemple devoir attraper les exceptions liées à la fin du fichier, mais
doit par contre laisser passer une exception comme
\texttt{ZeroDivisionError}.

    \hypertarget{les-paramuxe8tres-de-__exit__}{%
\subsubsection{\texorpdfstring{Les paramètres de
\texttt{\_\_exit\_\_}}{Les paramètres de \_\_exit\_\_}}\label{les-paramuxe8tres-de-__exit__}}

    Si on a besoin de filtrer entre les exceptions - c'est-à-dire en laisser
passer certaines et pas d'autres - il nous faut quelque chose de plus
pour pouvoir faire le tri. Comme
\href{https://docs.python.org/3/reference/datamodel.html\#with-statement-context-managers}{vous
pouvez le retrouver ici}, la méthode \texttt{\_\_exit\_\_} reçoit trois
arguments~:

\begin{verbatim}
def __exit__(self, exc_type, exc_value, traceback):
\end{verbatim}

\begin{itemize}
\tightlist
\item
  si l'on sort du bloc \texttt{with} sans qu'une exception soit levée,
  ces trois arguments valent \texttt{None};
\item
  par contre si une exception est levée, ils permettent d'accéder
  respectivement au type, à la valeur de l'exception, et à l'état de la
  pile lorsque l'exception est levée.
\end{itemize}

    Pour illustrer cela, écrivons une nouvelle version de \texttt{Timer} qui
filtre, disons, l'exception \texttt{ZeroDivisionError} que je choisis au
hasard, c'est uniquement pour illustrer le mécanisme.

    \begin{Verbatim}[commandchars=\\\{\}]
{\color{incolor}In [{\color{incolor}3}]:} \PY{c+c1}{\PYZsh{} une deuxième version de Timer}
        \PY{c+c1}{\PYZsh{} qui propage toutes les exceptions sauf \PYZsq{}OSError\PYZsq{}}
        
        \PY{k}{class} \PY{n+nc}{Timer2}\PY{p}{:}
            \PY{k}{def} \PY{n+nf}{\PYZus{}\PYZus{}enter\PYZus{}\PYZus{}}\PY{p}{(}\PY{n+nb+bp}{self}\PY{p}{)}\PY{p}{:}
                \PY{n+nb}{print}\PY{p}{(}\PY{l+s+s2}{\PYZdq{}}\PY{l+s+s2}{Entering Timer1}\PY{l+s+s2}{\PYZdq{}}\PY{p}{)}
                \PY{n+nb+bp}{self}\PY{o}{.}\PY{n}{start} \PY{o}{=} \PY{n}{time}\PY{o}{.}\PY{n}{time}\PY{p}{(}\PY{p}{)}
                \PY{c+c1}{\PYZsh{} rappel : le retour de \PYZus{}\PYZus{}enter\PYZus{}\PYZus{} est ce qui est passé}
                \PY{c+c1}{\PYZsh{} à la clause `as` du `with`}
                \PY{k}{return} \PY{n+nb+bp}{self}
            
            \PY{k}{def} \PY{n+nf}{\PYZus{}\PYZus{}exit\PYZus{}\PYZus{}}\PY{p}{(}\PY{n+nb+bp}{self}\PY{p}{,} \PY{n}{exc\PYZus{}type}\PY{p}{,} \PY{n}{exc\PYZus{}value}\PY{p}{,} \PY{n}{traceback}\PY{p}{)}\PY{p}{:}
                \PY{k}{if} \PY{n}{exc\PYZus{}type} \PY{o+ow}{is} \PY{k+kc}{None}\PY{p}{:}
                    \PY{c+c1}{\PYZsh{} pas d\PYZsq{}exception levée dans le corps du \PYZsq{}with\PYZsq{}}
                    \PY{n+nb}{print}\PY{p}{(}\PY{n}{f}\PY{l+s+s2}{\PYZdq{}}\PY{l+s+s2}{Total duration }\PY{l+s+s2}{\PYZob{}}\PY{l+s+s2}{time.time()\PYZhy{}self.start:2f\PYZcb{}}\PY{l+s+s2}{\PYZdq{}}\PY{p}{)}
                    \PY{c+c1}{\PYZsh{} dans ce cas la valeur de retour n\PYZsq{}est pas utilisée}
                \PY{k}{else}\PY{p}{:}
                    \PY{c+c1}{\PYZsh{} il y a eu une exception de type \PYZsq{}exc\PYZus{}type\PYZsq{}}
                    \PY{k}{if} \PY{n}{exc\PYZus{}type} \PY{o+ow}{in} \PY{p}{(}\PY{n+ne}{ZeroDivisionError}\PY{p}{,}\PY{p}{)} \PY{p}{:}
                        \PY{n+nb}{print}\PY{p}{(}\PY{l+s+s2}{\PYZdq{}}\PY{l+s+s2}{on étouffe}\PY{l+s+s2}{\PYZdq{}}\PY{p}{)}
                        \PY{c+c1}{\PYZsh{} on peut l\PYZsq{}étouffer en retournant True}
                        \PY{k}{return} \PY{k+kc}{True}
                    \PY{k}{else}\PY{p}{:}
                        \PY{n+nb}{print}\PY{p}{(}\PY{n}{f}\PY{l+s+s2}{\PYZdq{}}\PY{l+s+s2}{OOPS : on propage l}\PY{l+s+s2}{\PYZsq{}}\PY{l+s+s2}{exception }\PY{l+s+si}{\PYZob{}exc\PYZus{}type\PYZcb{}}\PY{l+s+s2}{ \PYZhy{} }\PY{l+s+si}{\PYZob{}exc\PYZus{}value\PYZcb{}}\PY{l+s+s2}{\PYZdq{}}\PY{p}{)}
                        \PY{c+c1}{\PYZsh{} et pour ça il suffit... de ne rien faire du tout}
                        \PY{c+c1}{\PYZsh{} ce qui renverra None }
\end{Verbatim}


    \begin{Verbatim}[commandchars=\\\{\}]
{\color{incolor}In [{\color{incolor}4}]:} \PY{c+c1}{\PYZsh{} commençons avec un code sans souci}
        \PY{k}{try}\PY{p}{:}
            \PY{k}{with} \PY{n}{Timer2}\PY{p}{(}\PY{p}{)}\PY{p}{:}
                \PY{n}{time}\PY{o}{.}\PY{n}{sleep}\PY{p}{(}\PY{l+m+mf}{0.5}\PY{p}{)}
        \PY{k}{except} \PY{n+ne}{Exception} \PY{k}{as} \PY{n}{e}\PY{p}{:}
            \PY{c+c1}{\PYZsh{} on va bien recevoir cette exception}
            \PY{n+nb}{print}\PY{p}{(}\PY{n}{f}\PY{l+s+s2}{\PYZdq{}}\PY{l+s+s2}{OOPS \PYZhy{}\PYZgt{} }\PY{l+s+s2}{\PYZob{}}\PY{l+s+s2}{type(e)\PYZcb{}}\PY{l+s+s2}{\PYZdq{}}\PY{p}{)}
\end{Verbatim}


    \begin{Verbatim}[commandchars=\\\{\}]
Entering Timer1
Total duration 0.514807

    \end{Verbatim}

    \begin{Verbatim}[commandchars=\\\{\}]
{\color{incolor}In [{\color{incolor}5}]:} \PY{c+c1}{\PYZsh{} avec une exception filtrée}
        \PY{k}{try}\PY{p}{:}
            \PY{k}{with} \PY{n}{Timer2}\PY{p}{(}\PY{p}{)}\PY{p}{:}
                \PY{n}{time}\PY{o}{.}\PY{n}{sleep}\PY{p}{(}\PY{l+m+mf}{0.5}\PY{p}{)}
                \PY{l+m+mi}{1}\PY{o}{/}\PY{l+m+mi}{0}
        \PY{k}{except} \PY{n+ne}{Exception} \PY{k}{as} \PY{n}{e}\PY{p}{:}
            \PY{c+c1}{\PYZsh{} on va bien recevoir cette exception}
            \PY{n+nb}{print}\PY{p}{(}\PY{n}{f}\PY{l+s+s2}{\PYZdq{}}\PY{l+s+s2}{OOPS \PYZhy{}\PYZgt{} }\PY{l+s+s2}{\PYZob{}}\PY{l+s+s2}{type(e)\PYZcb{}}\PY{l+s+s2}{\PYZdq{}}\PY{p}{)}
\end{Verbatim}


    \begin{Verbatim}[commandchars=\\\{\}]
Entering Timer1
on étouffe

    \end{Verbatim}

    \begin{Verbatim}[commandchars=\\\{\}]
{\color{incolor}In [{\color{incolor}6}]:} \PY{c+c1}{\PYZsh{} avec une autre exception }
        \PY{k}{try}\PY{p}{:}
            \PY{k}{with} \PY{n}{Timer2}\PY{p}{(}\PY{p}{)}\PY{p}{:}
                \PY{n}{time}\PY{o}{.}\PY{n}{sleep}\PY{p}{(}\PY{l+m+mf}{0.5}\PY{p}{)}
                \PY{k}{raise} \PY{n+ne}{OSError}\PY{p}{(}\PY{p}{)}
        \PY{k}{except} \PY{n+ne}{Exception} \PY{k}{as} \PY{n}{e}\PY{p}{:}
            \PY{c+c1}{\PYZsh{} on va bien recevoir cette exception}
            \PY{n+nb}{print}\PY{p}{(}\PY{n}{f}\PY{l+s+s2}{\PYZdq{}}\PY{l+s+s2}{OOPS \PYZhy{}\PYZgt{} }\PY{l+s+s2}{\PYZob{}}\PY{l+s+s2}{type(e)\PYZcb{}}\PY{l+s+s2}{\PYZdq{}}\PY{p}{)}
\end{Verbatim}


    \begin{Verbatim}[commandchars=\\\{\}]
Entering Timer1
OOPS : on propage l'exception <class 'OSError'> - 
OOPS -> <class 'OSError'>

    \end{Verbatim}

    \hypertarget{la-bibliothuxe8que-contextlib}{%
\subsubsection{\texorpdfstring{La bibliothèque
\texttt{contextlib}}{La bibliothèque contextlib}}\label{la-bibliothuxe8que-contextlib}}

    Je vous signale aussi
\href{https://docs.python.org/3/library/contextlib.html}{la bibliothèque
\texttt{contextlib}} qui offre quelques utilitaires pour se définir un
contextmanager.\\

Notamment, elle permet d'implémenter un context manager sous une forme
compacte à l'aide d'une fonction génératrice - et du décorateur
\texttt{contextmanager}:

    \begin{Verbatim}[commandchars=\\\{\}]
{\color{incolor}In [{\color{incolor}7}]:} \PY{k+kn}{from} \PY{n+nn}{contextlib} \PY{k}{import} \PY{n}{contextmanager}
\end{Verbatim}


    \begin{Verbatim}[commandchars=\\\{\}]
{\color{incolor}In [{\color{incolor}8}]:} \PY{c+c1}{\PYZsh{} l\PYZsq{}objet compact\PYZus{}timer est un context manager !}
        \PY{n+nd}{@contextmanager}
        \PY{k}{def} \PY{n+nf}{compact\PYZus{}timer}\PY{p}{(}\PY{n}{message}\PY{p}{)}\PY{p}{:}
            \PY{n}{start} \PY{o}{=} \PY{n}{time}\PY{o}{.}\PY{n}{time}\PY{p}{(}\PY{p}{)}
            \PY{k}{yield}
            \PY{n+nb}{print}\PY{p}{(}\PY{n}{f}\PY{l+s+s2}{\PYZdq{}}\PY{l+s+si}{\PYZob{}message\PYZcb{}}\PY{l+s+s2}{: duration = }\PY{l+s+s2}{\PYZob{}}\PY{l+s+s2}{time.time() \PYZhy{} start\PYZcb{}}\PY{l+s+s2}{\PYZdq{}}\PY{p}{)}
\end{Verbatim}


    \begin{Verbatim}[commandchars=\\\{\}]
{\color{incolor}In [{\color{incolor}9}]:} \PY{k}{with} \PY{n}{compact\PYZus{}timer}\PY{p}{(}\PY{l+s+s2}{\PYZdq{}}\PY{l+s+s2}{Squares sum}\PY{l+s+s2}{\PYZdq{}}\PY{p}{)}\PY{p}{:}
            \PY{n+nb}{print}\PY{p}{(}\PY{n+nb}{sum}\PY{p}{(}\PY{n}{x}\PY{o}{*}\PY{o}{*}\PY{l+m+mi}{2} \PY{k}{for} \PY{n}{x} \PY{o+ow}{in} \PY{n+nb}{range}\PY{p}{(}\PY{l+m+mi}{10}\PY{o}{*}\PY{o}{*}\PY{l+m+mi}{5}\PY{p}{)}\PY{p}{)}\PY{p}{)}
\end{Verbatim}


    \begin{Verbatim}[commandchars=\\\{\}]
333328333350000
Squares sum: duration = 0.031200408935546875

    \end{Verbatim}

    Un peu comme on peut implémenter un itérateur à partir d'une fonction
génératrice qui fait (n'importe quel nombre de) \texttt{yield}, ici on
implémente un context manager compact sous la forme d'une fonction
génératrice.\\

Comme vous l'avez sans doute deviné sur la base de cet exemple, il faut
que la fonction fasse \textbf{exactement un \texttt{yield}}: ce qui se
passe avant le \texttt{yield} est du ressort de \texttt{\_\_enter\_\_},
et la fin est du ressort de \texttt{\_\_exit\_\_()}.\\

Bien entendu on n'a pas la même puissance d'expression avec cette
méthode par rapport à une vraie classe, mais cela permet de créer des
context managers avec le minimum de code.