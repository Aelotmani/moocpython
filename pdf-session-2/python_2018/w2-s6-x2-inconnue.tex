    
    
    
    

    

    \hypertarget{suxe9quences}{%
\section{Séquences}\label{suxe9quences}}

    \hypertarget{exercice---niveau-basique}{%
\subsection{Exercice - niveau basique}\label{exercice---niveau-basique}}

    \hypertarget{slicing}{%
\subsubsection{Slicing}\label{slicing}}

    Commençons par créer une chaîne de caractères. Ne vous inquiétez pas si
vous ne comprenez pas encore le code d'initialisation utilisé
ci-dessous.

Pour les plus curieux, l'instruction \texttt{import} permet de charger
dans votre programme une boîte à outils que l'on appelle un module.
Python vient avec de nombreux modules qui forment la bibliothèque
standard. Le plus difficile avec les modules de la bibliothèque standard
est de savoir qu'ils existent. En effet, il y en a un grand nombre et
bien souvent il existe un module pour faire ce que vous souhaitez.

Ici en particulier nous utilisons le module \texttt{string}.

    \begin{Verbatim}[commandchars=\\\{\},frame=single,framerule=0.3mm,rulecolor=\color{cellframecolor}]
{\color{incolor}In [{\color{incolor}1}]:} \PY{c+c1}{\PYZsh{} nous allons tirer profit ici d\PYZsq{}une }
        \PY{c+c1}{\PYZsh{} constante définie dans le module string}
        \PY{k+kn}{import} \PY{n+nn}{string}
        \PY{n}{chaine} \PY{o}{=} \PY{n}{string}\PY{o}{.}\PY{n}{ascii\PYZus{}lowercase}
        
        \PY{c+c1}{\PYZsh{} et voici sa valeur}
        \PY{n+nb}{print}\PY{p}{(}\PY{n}{chaine}\PY{p}{)}
\end{Verbatim}


    \begin{Verbatim}[commandchars=\\\{\},frame=single,framerule=0.3mm,rulecolor=\color{cellframecolor}]
abcdefghijklmnopqrstuvwxyz
\end{Verbatim}

    Pour chacune des sous-chaînes ci-dessous, écrire une expression de
slicing sur \texttt{chaine} qui renvoie la sous-chaîne. La cellule de
code doit retourner \texttt{True}.

    Par exemple, pour obtenir ``def''~:

    \begin{Verbatim}[commandchars=\\\{\},frame=single,framerule=0.3mm,rulecolor=\color{cellframecolor}]
{\color{incolor}In [{\color{incolor}2}]:} \PY{n}{chaine}\PY{p}{[}\PY{l+m+mi}{3}\PY{p}{:}\PY{l+m+mi}{6}\PY{p}{]} \PY{o}{==} \PY{l+s+s2}{\PYZdq{}}\PY{l+s+s2}{def}\PY{l+s+s2}{\PYZdq{}}
\end{Verbatim}


\begin{Verbatim}[commandchars=\\\{\},frame=single,framerule=0.3mm,rulecolor=\color{cellframecolor}]
{\color{outcolor}Out[{\color{outcolor}2}]:} True
\end{Verbatim}
            
    \begin{enumerate}
\def\labelenumi{\arabic{enumi})}
\tightlist
\item
  Écrivez une slice pour obtenir ``vwx'' (n'hésitez pas à utiliser les
  indices négatifs)~:
\end{enumerate}

    \begin{Verbatim}[commandchars=\\\{\},frame=single,framerule=0.3mm,rulecolor=\color{cellframecolor}]
{\color{incolor}In [{\color{incolor} }]:} \PY{c+c1}{\PYZsh{} NOTE}
        \PY{c+c1}{\PYZsh{} auto\PYZhy{}exec\PYZhy{}for\PYZhy{}latex has skipped execution of this cell}
        
        \PY{n}{chaine}\PY{p}{[} \PY{o}{\PYZlt{}}\PY{n}{votre\PYZus{}code}\PY{o}{\PYZgt{}} \PY{p}{]} \PY{o}{==} \PY{l+s+s2}{\PYZdq{}}\PY{l+s+s2}{vwx}\PY{l+s+s2}{\PYZdq{}}
\end{Verbatim}


    \begin{enumerate}
\def\labelenumi{\arabic{enumi})}
\setcounter{enumi}{1}
\tightlist
\item
  Une slice pour obtenir ``wxyz'' (avec une seule constante)~:
\end{enumerate}

    \begin{Verbatim}[commandchars=\\\{\},frame=single,framerule=0.3mm,rulecolor=\color{cellframecolor}]
{\color{incolor}In [{\color{incolor} }]:} \PY{c+c1}{\PYZsh{} NOTE}
        \PY{c+c1}{\PYZsh{} auto\PYZhy{}exec\PYZhy{}for\PYZhy{}latex has skipped execution of this cell}
        
        \PY{n}{chaine}\PY{p}{[} \PY{o}{\PYZlt{}}\PY{n}{votre\PYZus{}code}\PY{o}{\PYZgt{}} \PY{p}{]} \PY{o}{==} \PY{l+s+s2}{\PYZdq{}}\PY{l+s+s2}{wxyz}\PY{l+s+s2}{\PYZdq{}}
\end{Verbatim}


    \begin{enumerate}
\def\labelenumi{\arabic{enumi})}
\setcounter{enumi}{2}
\tightlist
\item
  Une slice pour obtenir ``dfhjlnprtvxz'' (avec deux constantes)~:
\end{enumerate}

    \begin{Verbatim}[commandchars=\\\{\},frame=single,framerule=0.3mm,rulecolor=\color{cellframecolor}]
{\color{incolor}In [{\color{incolor} }]:} \PY{c+c1}{\PYZsh{} NOTE}
        \PY{c+c1}{\PYZsh{} auto\PYZhy{}exec\PYZhy{}for\PYZhy{}latex has skipped execution of this cell}
        
        \PY{n}{chaine}\PY{p}{[} \PY{o}{\PYZlt{}}\PY{n}{votre\PYZus{}code}\PY{o}{\PYZgt{}} \PY{p}{]} \PY{o}{==} \PY{l+s+s2}{\PYZdq{}}\PY{l+s+s2}{dfhjlnprtvxz}\PY{l+s+s2}{\PYZdq{}}
\end{Verbatim}


    \begin{enumerate}
\def\labelenumi{\arabic{enumi})}
\setcounter{enumi}{3}
\tightlist
\item
  Une slice pour obtenir ``xurolifc'' (avec deux constantes)~:
\end{enumerate}

    \begin{Verbatim}[commandchars=\\\{\},frame=single,framerule=0.3mm,rulecolor=\color{cellframecolor}]
{\color{incolor}In [{\color{incolor} }]:} \PY{c+c1}{\PYZsh{} NOTE}
        \PY{c+c1}{\PYZsh{} auto\PYZhy{}exec\PYZhy{}for\PYZhy{}latex has skipped execution of this cell}
        
        \PY{n}{chaine}\PY{p}{[} \PY{o}{\PYZlt{}}\PY{n}{votre\PYZus{}code}\PY{o}{\PYZgt{}} \PY{p}{]} \PY{o}{==} \PY{l+s+s2}{\PYZdq{}}\PY{l+s+s2}{xurolifc}\PY{l+s+s2}{\PYZdq{}}
\end{Verbatim}


    \hypertarget{exercice---niveau-intermuxe9diaire}{%
\subsection{Exercice - niveau
intermédiaire}\label{exercice---niveau-intermuxe9diaire}}

    \hypertarget{longueur}{%
\subsubsection{Longueur}\label{longueur}}

    \begin{Verbatim}[commandchars=\\\{\},frame=single,framerule=0.3mm,rulecolor=\color{cellframecolor}]
{\color{incolor}In [{\color{incolor}3}]:} \PY{c+c1}{\PYZsh{} il vous faut évaluer cette cellule magique}
        \PY{c+c1}{\PYZsh{} pour charger l\PYZsq{}exercice qui suit}
        \PY{c+c1}{\PYZsh{} et autoévaluer votre réponse}
        \PY{k+kn}{from} \PY{n+nn}{corrections}\PY{n+nn}{.}\PY{n+nn}{exo\PYZus{}inconnue} \PY{k}{import} \PY{n}{exo\PYZus{}inconnue}
\end{Verbatim}


    On vous donne une chaîne \texttt{composite} dont on sait qu'elle a été
calculée à partir de deux chaînes \texttt{inconnue} et \texttt{connue}
comme ceci~:

\begin{Shaded}
\begin{Highlighting}[frame=lines,framerule=0.6mm,rulecolor=\color{asisframecolor}]
\NormalTok{composite }\OperatorTok{=}\NormalTok{ connue }\OperatorTok{+}\NormalTok{ inconnue }\OperatorTok{+}\NormalTok{ connue}
\end{Highlighting}
\end{Shaded}

    On vous donne également la chaîne \texttt{connue}. Imaginez par exemple
que vous avez (ce ne sont pas les vraies valeurs)~:

\begin{Shaded}
\begin{Highlighting}[frame=lines,framerule=0.6mm,rulecolor=\color{asisframecolor}]
\NormalTok{connue }\OperatorTok{=} \StringTok{'0bf1'}
\NormalTok{composite }\OperatorTok{=} \StringTok{'0bf1a9730e150bf1'}
\end{Highlighting}
\end{Shaded}

alors, dans ce cas~:

\begin{Shaded}
\begin{Highlighting}[frame=lines,framerule=0.6mm,rulecolor=\color{asisframecolor}]
\NormalTok{inconnue }\OperatorTok{=} \StringTok{'a9730e15'}
\end{Highlighting}
\end{Shaded}

    L'exercice consiste à écrire une fonction qui retourne la valeur de
\texttt{inconnue} à partir de celles de \texttt{composite} et
\texttt{connue}. Vous pouvez admettre que \texttt{connue} n'est pas
vide, c'est-à-dire qu'elle contient au moins un caractère.

    Vous pouvez utiliser du \emph{slicing}, et la fonction \texttt{len()},
qui retourne la longueur d'une chaîne~:

    \begin{Verbatim}[commandchars=\\\{\},frame=single,framerule=0.3mm,rulecolor=\color{cellframecolor}]
{\color{incolor}In [{\color{incolor}4}]:} \PY{n+nb}{len}\PY{p}{(}\PY{l+s+s1}{\PYZsq{}}\PY{l+s+s1}{abcd}\PY{l+s+s1}{\PYZsq{}}\PY{p}{)}
\end{Verbatim}


\begin{Verbatim}[commandchars=\\\{\},frame=single,framerule=0.3mm,rulecolor=\color{cellframecolor}]
{\color{outcolor}Out[{\color{outcolor}4}]:} 4
\end{Verbatim}
            
    \begin{Verbatim}[commandchars=\\\{\},frame=single,framerule=0.3mm,rulecolor=\color{cellframecolor}]
{\color{incolor}In [{\color{incolor}5}]:} \PY{c+c1}{\PYZsh{} à vous de jouer}
        \PY{k}{def} \PY{n+nf}{inconnue}\PY{p}{(}\PY{n}{composite}\PY{p}{,} \PY{n}{connue}\PY{p}{)}\PY{p}{:}
            \PY{l+s+s2}{\PYZdq{}}\PY{l+s+s2}{votre code}\PY{l+s+s2}{\PYZdq{}}
\end{Verbatim}


    Une fois votre code évalué, vous pouvez évaluer la cellule suivante pour
vérifier votre résultat.

    \begin{Verbatim}[commandchars=\\\{\},frame=single,framerule=0.3mm,rulecolor=\color{cellframecolor}]
{\color{incolor}In [{\color{incolor} }]:} \PY{c+c1}{\PYZsh{} NOTE}
        \PY{c+c1}{\PYZsh{} auto\PYZhy{}exec\PYZhy{}for\PYZhy{}latex has skipped execution of this cell}
        
        \PY{c+c1}{\PYZsh{} correction}
        \PY{n}{exo\PYZus{}inconnue}\PY{o}{.}\PY{n}{correction}\PY{p}{(}\PY{n}{inconnue}\PY{p}{)}
\end{Verbatim}


    Lorsque vous évaluez cette cellule, la correction vous montre~:

\begin{itemize}
\tightlist
\item
  dans la première colonne l'appel qui est fait à votre fonction~;
\item
  dans la seconde colonne la valeur attendue pour \texttt{inconnue}~;
\item
  dans la troisième colonne ce que votre code a réellement calculé.
\end{itemize}

Si toutes les lignes sont \textbf{en vert} c'est que vous avez réussi
cet exercice.

    Vous pouvez essayer autant de fois que vous voulez, mais il vous faut
alors à chaque itération~:

\begin{itemize}
\tightlist
\item
  évaluer votre cellule-réponse (là où vous définissez la fonction
  \texttt{inconnue})~;
\item
  et ensuite évaluer la cellule correction pour la mettre à jour.
\end{itemize}


    % Add a bibliography block to the postdoc
    
    
    
