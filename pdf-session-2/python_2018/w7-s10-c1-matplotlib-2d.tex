    
    
    
    

    

    \hypertarget{matplotlib---2d}{%
\section{\texorpdfstring{\texttt{matplotlib} -
2D}{matplotlib - 2D}}\label{matplotlib---2d}}

    \hypertarget{compluxe9ment---niveau-basique}{%
\subsection{Complément - niveau
basique}\label{compluxe9ment---niveau-basique}}

    Plutôt que de récrire (encore) un tutorial sur \texttt{matplotlib}, je
préfère utiliser les ressources disponibles en ligne en anglais~:

\begin{itemize}
\tightlist
\item
  pour la dimension 2~:
  https://matplotlib.org/2.0.2/users/pyplot\_tutorial.html~;
\item
  pour la dimension 3~:
  https://matplotlib.org/mpl\_toolkits/mplot3d/tutorial.html.
\end{itemize}

Je vais essentiellement utiliser des extraits tels quels. N'hésitez pas
à consulter ces documents originaux pour davantage de précisions.

    \begin{Verbatim}[commandchars=\\\{\}]
{\color{incolor}In [{\color{incolor}1}]:} \PY{c+c1}{\PYZsh{} les imports habituels}
        \PY{k+kn}{import} \PY{n+nn}{numpy} \PY{k}{as} \PY{n+nn}{np}
        \PY{k+kn}{import} \PY{n+nn}{matplotlib}\PY{n+nn}{.}\PY{n+nn}{pyplot} \PY{k}{as} \PY{n+nn}{plt}
\end{Verbatim}


    Intentionnellement dans ce notebook, on ne va pas utiliser le mode
automatique de \texttt{matplotlib} dans les notebooks (pour rappel,
\texttt{plt.ion()}), car on veut justement apprendre à utiliser
\texttt{matplotlib} dans un contexte normal.

    \hypertarget{plt.plot}{%
\subsubsection{\texorpdfstring{\texttt{plt.plot}}{plt.plot}}\label{plt.plot}}

    Nous avons déjà vu plusieurs fois comment tracer une courbe avec
\texttt{matplotlib}, avec la fonction \texttt{plot}. Si on donne
seulement \emph{une} liste de valeurs, elles sont considérées comme les
\emph{Y}, les \emph{X} étant les entiers en nombre suffisant et en
commençant à 0.

    \begin{Verbatim}[commandchars=\\\{\}]
{\color{incolor}In [{\color{incolor}2}]:} \PY{c+c1}{\PYZsh{} si je ne donne qu\PYZsq{}une seule liste à plot}
        \PY{c+c1}{\PYZsh{} alors ce sont les Y}
        \PY{n}{plt}\PY{o}{.}\PY{n}{plot}\PY{p}{(}\PY{p}{[}\PY{l+m+mi}{10}\PY{p}{,} \PY{l+m+mi}{20}\PY{p}{,} \PY{l+m+mi}{25}\PY{p}{,} \PY{l+m+mi}{28}\PY{p}{]}\PY{p}{)}
        \PY{c+c1}{\PYZsh{} on peut aussi facilement ajouter une légende}
        \PY{c+c1}{\PYZsh{} ici sur l\PYZsq{}axe des y}
        \PY{n}{plt}\PY{o}{.}\PY{n}{ylabel}\PY{p}{(}\PY{l+s+s1}{\PYZsq{}}\PY{l+s+s1}{some numbers}\PY{l+s+s1}{\PYZsq{}}\PY{p}{)}
        
        \PY{n}{plt}\PY{o}{.}\PY{n}{show}\PY{p}{(}\PY{p}{)}
\end{Verbatim}


    \begin{center}
    \adjustimage{max size={0.9\linewidth}{0.9\paperheight}}{w7-s10-c1-matplotlib-2d_files/w7-s10-c1-matplotlib-2d_8_0.png}
    \end{center}
    { \hspace*{\fill} \\}
    
    On peut changer le style utilisé par \texttt{plot} pour tracer~; ce
style est spécifié sous la forme d'une chaîne de caractères, par défaut
\texttt{\textquotesingle{}b-\textquotesingle{}}, qui signifie une ligne
bleue (\texttt{b} pour bleu, et \texttt{-} pour ligne). Ici on va
préciser à la place \texttt{ro}, \texttt{r} qui signifie rouge et
\texttt{o} qui signifie cercle.

Voyez
\href{https://matplotlib.org/2.0.2/api/pyplot_api.html\#matplotlib.pyplot.plot}{la
documentation de référence de plot} pour une liste complète.

    \begin{Verbatim}[commandchars=\\\{\}]
{\color{incolor}In [{\color{incolor}3}]:} \PY{c+c1}{\PYZsh{} mais le plus souvent on passe à plot}
        \PY{c+c1}{\PYZsh{} une liste de X ET une liste de Y}
        \PY{n}{plt}\PY{o}{.}\PY{n}{plot}\PY{p}{(}\PY{p}{[}\PY{l+m+mi}{1}\PY{p}{,} \PY{l+m+mi}{2}\PY{p}{,} \PY{l+m+mi}{3}\PY{p}{,} \PY{l+m+mi}{4}\PY{p}{,} \PY{l+m+mi}{5}\PY{p}{]}\PY{p}{,} \PY{p}{[}\PY{l+m+mi}{1}\PY{p}{,} \PY{l+m+mi}{4}\PY{p}{,} \PY{l+m+mi}{9}\PY{p}{,} \PY{l+m+mi}{16}\PY{p}{,} \PY{l+m+mi}{25}\PY{p}{]}\PY{p}{,} \PY{l+s+s1}{\PYZsq{}}\PY{l+s+s1}{ro}\PY{l+s+s1}{\PYZsq{}}\PY{p}{)}
        
        \PY{c+c1}{\PYZsh{} ici on veut dire d\PYZsq{}utiliser}
        \PY{c+c1}{\PYZsh{} pour l\PYZsq{}axe des X : entre 0 et 5}
        \PY{c+c1}{\PYZsh{} pour l\PYZsq{}axe des Y : entre \PYZhy{}5 et 20}
        \PY{n}{plt}\PY{o}{.}\PY{n}{axis}\PY{p}{(}\PY{p}{[}\PY{l+m+mi}{0}\PY{p}{,} \PY{l+m+mi}{5}\PY{p}{,} \PY{o}{\PYZhy{}}\PY{l+m+mi}{5}\PY{p}{,} \PY{l+m+mi}{20}\PY{p}{]}\PY{p}{)}
        
        \PY{n}{plt}\PY{o}{.}\PY{n}{show}\PY{p}{(}\PY{p}{)}
\end{Verbatim}


    \begin{center}
    \adjustimage{max size={0.9\linewidth}{0.9\paperheight}}{w7-s10-c1-matplotlib-2d_files/w7-s10-c1-matplotlib-2d_10_0.png}
    \end{center}
    { \hspace*{\fill} \\}
    
    On peut très simplement dessiner plusieurs fonctions dans la même zone~:

    \begin{Verbatim}[commandchars=\\\{\}]
{\color{incolor}In [{\color{incolor}4}]:} \PY{c+c1}{\PYZsh{} échantillon de points entre 0 et 5 espacés de 0.2}
        \PY{n}{t} \PY{o}{=} \PY{n}{np}\PY{o}{.}\PY{n}{arange}\PY{p}{(}\PY{l+m+mf}{0.}\PY{p}{,} \PY{l+m+mf}{5.}\PY{p}{,} \PY{l+m+mf}{0.2}\PY{p}{)}
        
        \PY{c+c1}{\PYZsh{} plusieurs styles de ligne}
        \PY{n}{plt}\PY{o}{.}\PY{n}{plot}\PY{p}{(}\PY{n}{t}\PY{p}{,} \PY{n}{t}\PY{p}{,} \PY{l+s+s1}{\PYZsq{}}\PY{l+s+s1}{r\PYZhy{}\PYZhy{}}\PY{l+s+s1}{\PYZsq{}}\PY{p}{,} \PY{n}{t}\PY{p}{,} \PY{n}{t}\PY{o}{*}\PY{o}{*}\PY{l+m+mi}{2}\PY{p}{,} \PY{l+s+s1}{\PYZsq{}}\PY{l+s+s1}{bs}\PY{l+s+s1}{\PYZsq{}}\PY{p}{,} \PY{n}{t}\PY{p}{,} \PY{n}{t}\PY{o}{*}\PY{o}{*}\PY{l+m+mi}{3}\PY{p}{,} \PY{l+s+s1}{\PYZsq{}}\PY{l+s+s1}{g\PYZca{}}\PY{l+s+s1}{\PYZsq{}}\PY{p}{)}
        \PY{c+c1}{\PYZsh{} on pourrait ajouter d\PYZsq{}autres plot bien sûr aussi}
        \PY{n}{plt}\PY{o}{.}\PY{n}{show}\PY{p}{(}\PY{p}{)}
\end{Verbatim}


    \begin{center}
    \adjustimage{max size={0.9\linewidth}{0.9\paperheight}}{w7-s10-c1-matplotlib-2d_files/w7-s10-c1-matplotlib-2d_12_0.png}
    \end{center}
    { \hspace*{\fill} \\}
    
    \hypertarget{plusieurs-subplots}{%
\subsubsection{\texorpdfstring{Plusieurs
\emph{subplots}}{Plusieurs subplots}}\label{plusieurs-subplots}}

    \begin{Verbatim}[commandchars=\\\{\}]
{\color{incolor}In [{\color{incolor}5}]:} \PY{k}{def} \PY{n+nf}{f}\PY{p}{(}\PY{n}{t}\PY{p}{)}\PY{p}{:}
            \PY{k}{return} \PY{n}{np}\PY{o}{.}\PY{n}{exp}\PY{p}{(}\PY{o}{\PYZhy{}}\PY{n}{t}\PY{p}{)} \PY{o}{*} \PY{n}{np}\PY{o}{.}\PY{n}{cos}\PY{p}{(}\PY{l+m+mi}{2}\PY{o}{*}\PY{n}{np}\PY{o}{.}\PY{n}{pi}\PY{o}{*}\PY{n}{t}\PY{p}{)}
        
        \PY{c+c1}{\PYZsh{}\PYZsh{} deux domaines presque identiques}
        \PY{c+c1}{\PYZsh{} celui\PYZhy{}ci pour les points bleus}
        \PY{n}{t1} \PY{o}{=} \PY{n}{np}\PY{o}{.}\PY{n}{arange}\PY{p}{(}\PY{l+m+mf}{0.0}\PY{p}{,} \PY{l+m+mf}{5.0}\PY{p}{,} \PY{l+m+mf}{0.1}\PY{p}{)}
        \PY{c+c1}{\PYZsh{} celui\PYZhy{}ci pour la ligne bleue}
        \PY{n}{t2} \PY{o}{=} \PY{n}{np}\PY{o}{.}\PY{n}{arange}\PY{p}{(}\PY{l+m+mf}{0.0}\PY{p}{,} \PY{l+m+mf}{5.0}\PY{p}{,} \PY{l+m+mf}{0.02}\PY{p}{)}
        
        \PY{c+c1}{\PYZsh{} cet appel n\PYZsq{}est pas nécessaire}
        \PY{c+c1}{\PYZsh{} vous pouvez vérifier qu\PYZsq{}on pourrait l\PYZsq{}enlever}
        \PY{n}{plt}\PY{o}{.}\PY{n}{figure}\PY{p}{(}\PY{l+m+mi}{1}\PY{p}{)}
        \PY{c+c1}{\PYZsh{} on crée un \PYZsq{}subplot\PYZsq{}}
        \PY{n}{plt}\PY{o}{.}\PY{n}{subplot}\PY{p}{(}\PY{l+m+mi}{211}\PY{p}{)}
        \PY{c+c1}{\PYZsh{} le fonctionnement de matplotlib est dit \PYZsq{}stateful\PYZsq{}}
        \PY{c+c1}{\PYZsh{} par défaut on dessine dans le dernier objet créé}
        \PY{n}{plt}\PY{o}{.}\PY{n}{axis}\PY{p}{(}\PY{p}{[}\PY{l+m+mi}{0}\PY{p}{,} \PY{l+m+mi}{5}\PY{p}{,} \PY{o}{\PYZhy{}}\PY{l+m+mi}{1}\PY{p}{,} \PY{l+m+mi}{1}\PY{p}{]}\PY{p}{)}
        \PY{n}{plt}\PY{o}{.}\PY{n}{plot}\PY{p}{(}\PY{n}{t1}\PY{p}{,} \PY{n}{f}\PY{p}{(}\PY{n}{t1}\PY{p}{)}\PY{p}{,} \PY{l+s+s1}{\PYZsq{}}\PY{l+s+s1}{bo}\PY{l+s+s1}{\PYZsq{}}\PY{p}{,} \PY{n}{t2}\PY{p}{,} \PY{n}{f}\PY{p}{(}\PY{n}{t2}\PY{p}{)}\PY{p}{,} \PY{l+s+s1}{\PYZsq{}}\PY{l+s+s1}{k}\PY{l+s+s1}{\PYZsq{}}\PY{p}{)}
        
        \PY{c+c1}{\PYZsh{} une deuxième subplot}
        \PY{n}{plt}\PY{o}{.}\PY{n}{subplot}\PY{p}{(}\PY{l+m+mi}{212}\PY{p}{)}
        \PY{c+c1}{\PYZsh{} on écrit dedans}
        \PY{n}{plt}\PY{o}{.}\PY{n}{plot}\PY{p}{(}\PY{n}{t2}\PY{p}{,} \PY{n}{np}\PY{o}{.}\PY{n}{cos}\PY{p}{(}\PY{l+m+mi}{2}\PY{o}{*}\PY{n}{np}\PY{o}{.}\PY{n}{pi}\PY{o}{*}\PY{n}{t2}\PY{p}{)}\PY{p}{,} \PY{l+s+s1}{\PYZsq{}}\PY{l+s+s1}{r\PYZhy{}\PYZhy{}}\PY{l+s+s1}{\PYZsq{}}\PY{p}{)}
        \PY{n}{plt}\PY{o}{.}\PY{n}{show}\PY{p}{(}\PY{p}{)}
\end{Verbatim}


    \begin{center}
    \adjustimage{max size={0.9\linewidth}{0.9\paperheight}}{w7-s10-c1-matplotlib-2d_files/w7-s10-c1-matplotlib-2d_14_0.png}
    \end{center}
    { \hspace*{\fill} \\}
    
    C'est pour pouvoir construire de tels assemblages qu'il y a une fonction
\texttt{plt.show()}, qui indique que la figure est terminée.

    Il faut revenir un peu sur les arguments passés à \texttt{subplot}.
Lorsqu'on écrit~:

\begin{Shaded}
\begin{Highlighting}[]
\NormalTok{plt.subplot(}\DecValTok{211}\NormalTok{)}
\end{Highlighting}
\end{Shaded}

ce qui est par ailleurs juste un raccourci pour~:

\begin{Shaded}
\begin{Highlighting}[]
\NormalTok{plt.subplot(}\DecValTok{2}\NormalTok{, }\DecValTok{1}\NormalTok{, }\DecValTok{1}\NormalTok{)}
\end{Highlighting}
\end{Shaded}

on veut dire qu'on veut créer un quadrillage de 2 lignes de 1 colonne,
et que le subplot va occuper le 1er emplacement.

    \hypertarget{plusieurs-figures}{%
\subsubsection{Plusieurs figures}\label{plusieurs-figures}}

    En fait, on peut créer plusieurs figures, et plusieurs \emph{subplots}
dans chaque figure. Dans l'exemple qui suit on illustre encore mieux
cette notion de \emph{statefulness}. Je commence par vous donner
l'exemple du tutorial tel quel~:

    \begin{Verbatim}[commandchars=\\\{\}]
{\color{incolor}In [{\color{incolor}6}]:} \PY{n}{plt}\PY{o}{.}\PY{n}{figure}\PY{p}{(}\PY{l+m+mi}{1}\PY{p}{)}                \PY{c+c1}{\PYZsh{} the first figure}
        \PY{n}{plt}\PY{o}{.}\PY{n}{subplot}\PY{p}{(}\PY{l+m+mi}{211}\PY{p}{)}             \PY{c+c1}{\PYZsh{} the first subplot in the first figure}
        \PY{n}{plt}\PY{o}{.}\PY{n}{axis}\PY{p}{(}\PY{p}{[}\PY{l+m+mi}{0}\PY{p}{,} \PY{l+m+mi}{2}\PY{p}{,} \PY{l+m+mi}{1}\PY{p}{,} \PY{l+m+mi}{3}\PY{p}{]}\PY{p}{)}
        \PY{n}{plt}\PY{o}{.}\PY{n}{plot}\PY{p}{(}\PY{p}{[}\PY{l+m+mi}{1}\PY{p}{,} \PY{l+m+mi}{2}\PY{p}{,} \PY{l+m+mi}{3}\PY{p}{]}\PY{p}{)}
        \PY{n}{plt}\PY{o}{.}\PY{n}{subplot}\PY{p}{(}\PY{l+m+mi}{212}\PY{p}{)}             \PY{c+c1}{\PYZsh{} the second subplot in the first figure}
        \PY{n}{plt}\PY{o}{.}\PY{n}{axis}\PY{p}{(}\PY{p}{[}\PY{l+m+mi}{0}\PY{p}{,} \PY{l+m+mi}{2}\PY{p}{,} \PY{l+m+mi}{4}\PY{p}{,} \PY{l+m+mi}{6}\PY{p}{]}\PY{p}{)}
        \PY{n}{plt}\PY{o}{.}\PY{n}{plot}\PY{p}{(}\PY{p}{[}\PY{l+m+mi}{4}\PY{p}{,} \PY{l+m+mi}{5}\PY{p}{,} \PY{l+m+mi}{6}\PY{p}{]}\PY{p}{)}
        
        
        \PY{n}{plt}\PY{o}{.}\PY{n}{figure}\PY{p}{(}\PY{l+m+mi}{2}\PY{p}{)}                \PY{c+c1}{\PYZsh{} a second figure}
        \PY{n}{plt}\PY{o}{.}\PY{n}{axis}\PY{p}{(}\PY{p}{[}\PY{l+m+mi}{0}\PY{p}{,} \PY{l+m+mi}{2}\PY{p}{,} \PY{l+m+mi}{4}\PY{p}{,} \PY{l+m+mi}{6}\PY{p}{]}\PY{p}{)}
        \PY{n}{plt}\PY{o}{.}\PY{n}{plot}\PY{p}{(}\PY{p}{[}\PY{l+m+mi}{4}\PY{p}{,} \PY{l+m+mi}{5}\PY{p}{,} \PY{l+m+mi}{6}\PY{p}{]}\PY{p}{)}          \PY{c+c1}{\PYZsh{} creates a subplot(111) by default}
        
        \PY{n}{plt}\PY{o}{.}\PY{n}{figure}\PY{p}{(}\PY{l+m+mi}{1}\PY{p}{)}                \PY{c+c1}{\PYZsh{} figure 1 current;}
                                     \PY{c+c1}{\PYZsh{} subplot(212) still current}
        \PY{n}{plt}\PY{o}{.}\PY{n}{subplot}\PY{p}{(}\PY{l+m+mi}{211}\PY{p}{)}             \PY{c+c1}{\PYZsh{} make subplot(211) in figure1 current}
        \PY{n}{plt}\PY{o}{.}\PY{n}{title}\PY{p}{(}\PY{l+s+s1}{\PYZsq{}}\PY{l+s+s1}{Easy as 1, 2, 3}\PY{l+s+s1}{\PYZsq{}}\PY{p}{)} \PY{c+c1}{\PYZsh{} subplot 211 title}
        \PY{n}{plt}\PY{o}{.}\PY{n}{show}\PY{p}{(}\PY{p}{)}
\end{Verbatim}


    \begin{Verbatim}[commandchars=\\\{\}]
/usr/local/lib/python3.7/site-packages/matplotlib/cbook/deprecation.py:107: MatplotlibDeprecationWarning: Adding an axes using the same arguments as a previous axes currently reuses the earlier instance.  In a future version, a new instance will always be created and returned.  Meanwhile, this warning can be suppressed, and the future behavior ensured, by passing a unique label to each axes instance.
  warnings.warn(message, mplDeprecation, stacklevel=1)

    \end{Verbatim}

    \begin{center}
    \adjustimage{max size={0.9\linewidth}{0.9\paperheight}}{w7-s10-c1-matplotlib-2d_files/w7-s10-c1-matplotlib-2d_19_1.png}
    \end{center}
    { \hspace*{\fill} \\}
    
    \begin{center}
    \adjustimage{max size={0.9\linewidth}{0.9\paperheight}}{w7-s10-c1-matplotlib-2d_files/w7-s10-c1-matplotlib-2d_19_2.png}
    \end{center}
    { \hspace*{\fill} \\}
    
    Cette façon de faire est améliorable. D'abord c'est source d'erreurs, il
faut se souvenir de ce qui précède, et du coup, si on change un tout
petit peu la logique, ça risque de casser tout le reste. En outre selon
les environnements, on peut obtenir un vilain avertissement.

C'est pourquoi je vous conseille plutôt, pour faire la même chose que
ci-dessus, d'utiliser \texttt{plt.subplots} qui vous retourne la figure
avec ses \emph{subplots}, que vous pouvez ranger dans des variables
Python~:

    \begin{Verbatim}[commandchars=\\\{\}]
{\color{incolor}In [{\color{incolor}7}]:} \PY{c+c1}{\PYZsh{} c\PYZsq{}est assez déroutant au départ, mais}
        \PY{c+c1}{\PYZsh{} traditionnellement les subplots sont appelés \PYZsq{}axes\PYZsq{}}
        \PY{c+c1}{\PYZsh{} c\PYZsq{}est pourquoi ici on utilise ax1, ax2 et ax3 pour désigner}
        \PY{c+c1}{\PYZsh{} des subplots}
        
        \PY{c+c1}{\PYZsh{} ici je crée une figure et deux subplots,}
        \PY{c+c1}{\PYZsh{} sur une grille de 2 lignes * 1 colonne}
        \PY{n}{fig1}\PY{p}{,} \PY{p}{(}\PY{n}{ax1}\PY{p}{,} \PY{n}{ax2}\PY{p}{)} \PY{o}{=} \PY{n}{plt}\PY{o}{.}\PY{n}{subplots}\PY{p}{(}\PY{l+m+mi}{2}\PY{p}{,} \PY{l+m+mi}{1}\PY{p}{)}
        
        \PY{c+c1}{\PYZsh{} au lieu de faire plt.plot, vous pouvez envoyer}
        \PY{c+c1}{\PYZsh{} la méthode plot à un subplot}
        \PY{n}{ax1}\PY{o}{.}\PY{n}{plot}\PY{p}{(}\PY{p}{[}\PY{l+m+mi}{1}\PY{p}{,} \PY{l+m+mi}{2}\PY{p}{,} \PY{l+m+mi}{3}\PY{p}{]}\PY{p}{)}
        \PY{n}{ax2}\PY{o}{.}\PY{n}{plot}\PY{p}{(}\PY{p}{[}\PY{l+m+mi}{4}\PY{p}{,} \PY{l+m+mi}{5}\PY{p}{,} \PY{l+m+mi}{6}\PY{p}{]}\PY{p}{)}
        
        \PY{n}{fig2}\PY{p}{,} \PY{n}{ax3} \PY{o}{=} \PY{n}{plt}\PY{o}{.}\PY{n}{subplots}\PY{p}{(}\PY{l+m+mi}{1}\PY{p}{,} \PY{l+m+mi}{1}\PY{p}{)}
        \PY{n}{ax3}\PY{o}{.}\PY{n}{plot}\PY{p}{(}\PY{p}{[}\PY{l+m+mi}{4}\PY{p}{,} \PY{l+m+mi}{5}\PY{p}{,} \PY{l+m+mi}{6}\PY{p}{]}\PY{p}{)}
        
        \PY{c+c1}{\PYZsh{} pour revenir au premier subplot}
        \PY{c+c1}{\PYZsh{} il suffit d\PYZsq{}utiliser la variable ax1}
        \PY{c+c1}{\PYZsh{} attention on avait fait avec \PYZsq{}plt.title\PYZsq{}}
        \PY{c+c1}{\PYZsh{} ici c\PYZsq{}est la méthode \PYZsq{}set\PYZus{}title\PYZsq{}}
        \PY{n}{ax1}\PY{o}{.}\PY{n}{set\PYZus{}title}\PY{p}{(}\PY{l+s+s1}{\PYZsq{}}\PY{l+s+s1}{Easy as 1, 2, 3}\PY{l+s+s1}{\PYZsq{}}\PY{p}{)}
        
        \PY{n}{plt}\PY{o}{.}\PY{n}{show}\PY{p}{(}\PY{p}{)}
\end{Verbatim}


    \begin{center}
    \adjustimage{max size={0.9\linewidth}{0.9\paperheight}}{w7-s10-c1-matplotlib-2d_files/w7-s10-c1-matplotlib-2d_21_0.png}
    \end{center}
    { \hspace*{\fill} \\}
    
    \begin{center}
    \adjustimage{max size={0.9\linewidth}{0.9\paperheight}}{w7-s10-c1-matplotlib-2d_files/w7-s10-c1-matplotlib-2d_21_1.png}
    \end{center}
    { \hspace*{\fill} \\}
    
    \hypertarget{plt.hist}{%
\subsubsection{\texorpdfstring{\texttt{plt.hist}}{plt.hist}}\label{plt.hist}}

    S'agissant de la dimension 2, voici le dernier exemple que nous tirons
du tutoriel \texttt{matplotlib}, surtout pour illustrer
\href{https://matplotlib.org/api/_as_gen/matplotlib.pyplot.hist.html?highlight=matplotlib\%20pyplot\%20hist\#matplotlib.pyplot.hist}{\texttt{plt.hist}},
mais qui montre aussi comment ajouter du texte~:

    \begin{Verbatim}[commandchars=\\\{\}]
{\color{incolor}In [{\color{incolor}8}]:} \PY{c+c1}{\PYZsh{} pour être reproductible, on fixe la graine}
        \PY{c+c1}{\PYZsh{} du générateur aléatoire}
        \PY{n}{np}\PY{o}{.}\PY{n}{random}\PY{o}{.}\PY{n}{seed}\PY{p}{(}\PY{l+m+mi}{19680801}\PY{p}{)}
        
        \PY{n}{mu}\PY{p}{,} \PY{n}{sigma} \PY{o}{=} \PY{l+m+mi}{100}\PY{p}{,} \PY{l+m+mi}{15}
        \PY{n}{x} \PY{o}{=} \PY{n}{mu} \PY{o}{+} \PY{n}{sigma} \PY{o}{*} \PY{n}{np}\PY{o}{.}\PY{n}{random}\PY{o}{.}\PY{n}{randn}\PY{p}{(}\PY{l+m+mi}{10000}\PY{p}{)}
        
        \PY{c+c1}{\PYZsh{} dessiner un histogramme}
        \PY{c+c1}{\PYZsh{} on range les valeurs en 20 boites (bins)}
        \PY{n}{n}\PY{p}{,} \PY{n}{bins}\PY{p}{,} \PY{n}{patches} \PY{o}{=} \PY{n}{plt}\PY{o}{.}\PY{n}{hist}\PY{p}{(}\PY{n}{x}\PY{p}{,} \PY{l+m+mi}{20}\PY{p}{,} \PY{n}{normed}\PY{o}{=}\PY{l+m+mi}{1}\PY{p}{,} \PY{n}{facecolor}\PY{o}{=}\PY{l+s+s1}{\PYZsq{}}\PY{l+s+s1}{g}\PY{l+s+s1}{\PYZsq{}}\PY{p}{,} \PY{n}{alpha}\PY{o}{=}\PY{l+m+mf}{0.75}\PY{p}{)}
        
        \PY{n}{plt}\PY{o}{.}\PY{n}{xlabel}\PY{p}{(}\PY{l+s+s1}{\PYZsq{}}\PY{l+s+s1}{Smarts}\PY{l+s+s1}{\PYZsq{}}\PY{p}{)}
        \PY{n}{plt}\PY{o}{.}\PY{n}{ylabel}\PY{p}{(}\PY{l+s+s1}{\PYZsq{}}\PY{l+s+s1}{Probability}\PY{l+s+s1}{\PYZsq{}}\PY{p}{)}
        \PY{n}{plt}\PY{o}{.}\PY{n}{title}\PY{p}{(}\PY{l+s+s1}{\PYZsq{}}\PY{l+s+s1}{Histogram of IQ}\PY{l+s+s1}{\PYZsq{}}\PY{p}{)}
        \PY{n}{plt}\PY{o}{.}\PY{n}{text}\PY{p}{(}\PY{l+m+mi}{60}\PY{p}{,} \PY{o}{.}\PY{l+m+mi}{025}\PY{p}{,} \PY{l+s+sa}{r}\PY{l+s+s1}{\PYZsq{}}\PY{l+s+s1}{\PYZdl{}}\PY{l+s+s1}{\PYZbs{}}\PY{l+s+s1}{mu=100,}\PY{l+s+s1}{\PYZbs{}}\PY{l+s+s1}{ }\PY{l+s+s1}{\PYZbs{}}\PY{l+s+s1}{sigma=15\PYZdl{}}\PY{l+s+s1}{\PYZsq{}}\PY{p}{)}
        \PY{n}{plt}\PY{o}{.}\PY{n}{axis}\PY{p}{(}\PY{p}{[}\PY{l+m+mi}{40}\PY{p}{,} \PY{l+m+mi}{160}\PY{p}{,} \PY{l+m+mi}{0}\PY{p}{,} \PY{l+m+mf}{0.03}\PY{p}{]}\PY{p}{)}
        \PY{n}{plt}\PY{o}{.}\PY{n}{grid}\PY{p}{(}\PY{k+kc}{True}\PY{p}{)}
        \PY{n}{plt}\PY{o}{.}\PY{n}{show}\PY{p}{(}\PY{p}{)}
\end{Verbatim}


    \begin{Verbatim}[commandchars=\\\{\}]
/usr/local/lib/python3.7/site-packages/matplotlib/axes/\_axes.py:6462: UserWarning: The 'normed' kwarg is deprecated, and has been replaced by the 'density' kwarg.
  warnings.warn("The 'normed' kwarg is deprecated, and has been "

    \end{Verbatim}

    \begin{center}
    \adjustimage{max size={0.9\linewidth}{0.9\paperheight}}{w7-s10-c1-matplotlib-2d_files/w7-s10-c1-matplotlib-2d_24_1.png}
    \end{center}
    { \hspace*{\fill} \\}
    
    \hypertarget{plt.scatter}{%
\subsubsection{\texorpdfstring{\texttt{plt.scatter}}{plt.scatter}}\label{plt.scatter}}

    Je vous recommande aussi de regarder également la fonction
\href{https://matplotlib.org/api/_as_gen/matplotlib.pyplot.scatter.html?highlight=matplotlib\%20pyplot\%20scatter\#matplotlib.pyplot.scatter}{\texttt{plt.scatter}}
qui permet de faire par exemple des choses comme ceci~:

    \begin{Verbatim}[commandchars=\\\{\}]
{\color{incolor}In [{\color{incolor}9}]:} \PY{c+c1}{\PYZsh{} pour être reproductible, on fixe la graine}
        \PY{c+c1}{\PYZsh{} du générateur aléatoire}
        \PY{n}{np}\PY{o}{.}\PY{n}{random}\PY{o}{.}\PY{n}{seed}\PY{p}{(}\PY{l+m+mi}{19680801}\PY{p}{)}
        
        \PY{n}{N} \PY{o}{=} \PY{l+m+mi}{50}
        \PY{n}{x} \PY{o}{=} \PY{n}{np}\PY{o}{.}\PY{n}{random}\PY{o}{.}\PY{n}{rand}\PY{p}{(}\PY{n}{N}\PY{p}{)}
        \PY{n}{y} \PY{o}{=} \PY{n}{np}\PY{o}{.}\PY{n}{random}\PY{o}{.}\PY{n}{rand}\PY{p}{(}\PY{n}{N}\PY{p}{)}
        \PY{n}{colors} \PY{o}{=} \PY{n}{np}\PY{o}{.}\PY{n}{random}\PY{o}{.}\PY{n}{rand}\PY{p}{(}\PY{n}{N}\PY{p}{)}
        \PY{n}{area} \PY{o}{=} \PY{n}{np}\PY{o}{.}\PY{n}{pi} \PY{o}{*} \PY{p}{(}\PY{l+m+mi}{15} \PY{o}{*} \PY{n}{np}\PY{o}{.}\PY{n}{random}\PY{o}{.}\PY{n}{rand}\PY{p}{(}\PY{n}{N}\PY{p}{)}\PY{p}{)}\PY{o}{*}\PY{o}{*}\PY{l+m+mi}{2}  \PY{c+c1}{\PYZsh{} 0 to 15 point radii}
        
        \PY{n}{plt}\PY{o}{.}\PY{n}{scatter}\PY{p}{(}\PY{n}{x}\PY{p}{,} \PY{n}{y}\PY{p}{,} \PY{n}{s}\PY{o}{=}\PY{n}{area}\PY{p}{,} \PY{n}{c}\PY{o}{=}\PY{n}{colors}\PY{p}{,} \PY{n}{alpha}\PY{o}{=}\PY{l+m+mf}{0.5}\PY{p}{)}
        \PY{n}{plt}\PY{o}{.}\PY{n}{show}\PY{p}{(}\PY{p}{)}
\end{Verbatim}


    \begin{center}
    \adjustimage{max size={0.9\linewidth}{0.9\paperheight}}{w7-s10-c1-matplotlib-2d_files/w7-s10-c1-matplotlib-2d_27_0.png}
    \end{center}
    { \hspace*{\fill} \\}
    
    \hypertarget{plt.boxplot}{%
\subsubsection{\texorpdfstring{\texttt{plt.boxplot}}{plt.boxplot}}\label{plt.boxplot}}

    Avec
\href{https://matplotlib.org/api/_as_gen/matplotlib.pyplot.boxplot.html}{\texttt{boxplot}}
vous obtenez des boîtes à moustache~:

    \begin{Verbatim}[commandchars=\\\{\}]
{\color{incolor}In [{\color{incolor}10}]:} \PY{n}{plt}\PY{o}{.}\PY{n}{figure}\PY{p}{(}\PY{n}{figsize}\PY{o}{=}\PY{p}{(}\PY{l+m+mi}{6}\PY{p}{,} \PY{l+m+mi}{3}\PY{p}{)}\PY{p}{)}
         
         \PY{n}{plt}\PY{o}{.}\PY{n}{subplot}\PY{p}{(}\PY{l+m+mi}{121}\PY{p}{)}
         \PY{c+c1}{\PYZsh{} on peut passer à boxplot une liste de suites de nombres}
         \PY{c+c1}{\PYZsh{} chaque suite donne lieu à une boite à moustache}
         \PY{c+c1}{\PYZsh{} ici 3 suites}
         \PY{n}{plt}\PY{o}{.}\PY{n}{boxplot}\PY{p}{(}\PY{p}{[}\PY{p}{[}\PY{l+m+mi}{1}\PY{p}{,} \PY{l+m+mi}{2}\PY{p}{,} \PY{l+m+mi}{3}\PY{p}{,} \PY{l+m+mi}{4}\PY{p}{,} \PY{l+m+mi}{5}\PY{p}{,} \PY{l+m+mi}{13}\PY{p}{]}\PY{p}{,} \PY{p}{[}\PY{l+m+mi}{6}\PY{p}{,} \PY{l+m+mi}{7}\PY{p}{,} \PY{l+m+mi}{8}\PY{p}{,} \PY{l+m+mi}{10}\PY{p}{,} \PY{l+m+mi}{11}\PY{p}{,} \PY{l+m+mi}{12}\PY{p}{]}\PY{p}{,} \PY{p}{[}\PY{l+m+mi}{11}\PY{p}{,} \PY{l+m+mi}{12} \PY{p}{,}\PY{l+m+mi}{13}\PY{p}{]}\PY{p}{]}\PY{p}{)}
         \PY{n}{plt}\PY{o}{.}\PY{n}{ylim}\PY{p}{(}\PY{l+m+mi}{0}\PY{p}{,} \PY{l+m+mi}{14}\PY{p}{)}
         
         \PY{n}{plt}\PY{o}{.}\PY{n}{subplot}\PY{p}{(}\PY{l+m+mi}{122}\PY{p}{)}
         \PY{c+c1}{\PYZsh{} on peut aussi comme toujours lui passer un ndarray numpy}
         \PY{c+c1}{\PYZsh{} attention c\PYZsq{}est lu dans l\PYZsq{}autre sens, ici aussi on a 3 suites de nombres}
         \PY{n}{plt}\PY{o}{.}\PY{n}{boxplot}\PY{p}{(}\PY{n}{np}\PY{o}{.}\PY{n}{array}\PY{p}{(}\PY{p}{[}\PY{p}{[}\PY{l+m+mi}{1}\PY{p}{,} \PY{l+m+mi}{6}\PY{p}{,} \PY{l+m+mi}{11}\PY{p}{]}\PY{p}{,}
                               \PY{p}{[}\PY{l+m+mi}{2}\PY{p}{,} \PY{l+m+mi}{7}\PY{p}{,} \PY{l+m+mi}{12}\PY{p}{]}\PY{p}{,}
                               \PY{p}{[}\PY{l+m+mi}{3}\PY{p}{,} \PY{l+m+mi}{8}\PY{p}{,} \PY{l+m+mi}{13}\PY{p}{]}\PY{p}{,}
                               \PY{p}{[}\PY{l+m+mi}{4}\PY{p}{,} \PY{l+m+mi}{10}\PY{p}{,} \PY{l+m+mi}{11}\PY{p}{]}\PY{p}{,}
                               \PY{p}{[}\PY{l+m+mi}{5}\PY{p}{,} \PY{l+m+mi}{11}\PY{p}{,} \PY{l+m+mi}{12}\PY{p}{]}\PY{p}{,}
                               \PY{p}{[}\PY{l+m+mi}{13}\PY{p}{,} \PY{l+m+mi}{12}\PY{p}{,} \PY{l+m+mi}{13}\PY{p}{]}\PY{p}{]}\PY{p}{)}\PY{p}{)}
         \PY{n}{plt}\PY{o}{.}\PY{n}{show}\PY{p}{(}\PY{p}{)}
\end{Verbatim}


    \begin{center}
    \adjustimage{max size={0.9\linewidth}{0.9\paperheight}}{w7-s10-c1-matplotlib-2d_files/w7-s10-c1-matplotlib-2d_30_0.png}
    \end{center}
    { \hspace*{\fill} \\}
    

    % Add a bibliography block to the postdoc
    
    
    
