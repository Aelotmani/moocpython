    
    
    
    

    

    \hypertarget{noms-de-variables}{%
\section{Noms de variables}\label{noms-de-variables}}

    \hypertarget{compluxe9ment---niveau-basique}{%
\subsection{Complément - niveau
basique}\label{compluxe9ment---niveau-basique}}

    Revenons sur les noms de variables autorisés ou non.

    Les noms les plus simples sont constitués de lettres. Par exemple~:

    \begin{Verbatim}[commandchars=\\\{\}]
{\color{incolor}In [{\color{incolor}1}]:} \PY{n}{factoriel} \PY{o}{=} \PY{l+m+mi}{1}
\end{Verbatim}


    On peut utiliser aussi les majuscules, mais attention cela définit une
variable différente. Ainsi~:

    \begin{Verbatim}[commandchars=\\\{\}]
{\color{incolor}In [{\color{incolor}2}]:} \PY{n}{Factoriel} \PY{o}{=} \PY{l+m+mi}{100}
        \PY{n}{factoriel} \PY{o}{==} \PY{n}{Factoriel}
\end{Verbatim}


\begin{Verbatim}[commandchars=\\\{\}]
{\color{outcolor}Out[{\color{outcolor}2}]:} False
\end{Verbatim}
            
    Le signe \texttt{==} permet de tester si deux variables ont la même
valeur. Si les variables ont la même valeur, le test retournera
\texttt{True}, et \texttt{False} sinon. On y reviendra bien entendu.

    \hypertarget{conventions-habituelles}{%
\subsubsection{Conventions habituelles}\label{conventions-habituelles}}

    En règle générale, on utilise \textbf{uniquement des minuscules} pour
désigner les variables simples (ainsi d'ailleurs que pour les noms de
fonctions), les majuscules sont réservées en principe pour d'autres
sortes de variables, comme les noms de classe, que nous verrons
ultérieurement.

Notons qu'il s'agit uniquement d'une convention, ceci n'est pas imposé
par le langage lui-même.

    Pour des raisons de lisibilité, il est également possible d'utiliser le
tiret bas \texttt{\_} dans les noms de variables. On préfèrera ainsi~:

    \begin{Verbatim}[commandchars=\\\{\}]
{\color{incolor}In [{\color{incolor}3}]:} \PY{n}{age\PYZus{}moyen} \PY{o}{=} \PY{l+m+mi}{75} \PY{c+c1}{\PYZsh{} oui}
\end{Verbatim}


    plutôt que ceci (bien qu'autorisé par le langage)~:

    \begin{Verbatim}[commandchars=\\\{\}]
{\color{incolor}In [{\color{incolor}4}]:} \PY{n}{AgeMoyen} \PY{o}{=} \PY{l+m+mi}{75} \PY{c+c1}{\PYZsh{} autorisé, mais non}
\end{Verbatim}


    On peut également utiliser des chiffres dans les noms de variables comme
par exemple~:

    \begin{Verbatim}[commandchars=\\\{\}]
{\color{incolor}In [{\color{incolor}5}]:} \PY{n}{age\PYZus{}moyen\PYZus{}dept75} \PY{o}{=} \PY{l+m+mi}{80}
\end{Verbatim}


    avec la restriction toutefois que le premier caractère ne peut pas être
un chiffre, cette affectation est donc refusée~:

    \begin{Verbatim}[commandchars=\\\{\}]
{\color{incolor}In [{\color{incolor} }]:} \PY{c+c1}{\PYZsh{} NOTE}
        \PY{c+c1}{\PYZsh{} automatic execution has skipped execution of this cell}
        
        \PY{l+m+mi}{75}\PY{n}{\PYZus{}age\PYZus{}moyen} \PY{o}{=} \PY{l+m+mi}{80} \PY{c+c1}{\PYZsh{} erreur de syntaxe}
\end{Verbatim}


    \hypertarget{le-tiret-bas-comme-premier-caractuxe8re}{%
\subsubsection{Le tiret bas comme premier
caractère}\label{le-tiret-bas-comme-premier-caractuxe8re}}

    Il est par contre, possible de faire commencer un nom de variable par un
tiret bas comme premier caractère~; toutefois, à ce stade, nous vous
déconseillons d'utiliser cette pratique qui est réservée à des
conventions de nommage bien spécifiques.

    \begin{Verbatim}[commandchars=\\\{\}]
{\color{incolor}In [{\color{incolor}6}]:} \PY{n}{\PYZus{}autorise\PYZus{}mais\PYZus{}deconseille} \PY{o}{=} \PY{l+s+s1}{\PYZsq{}}\PY{l+s+s1}{Voir le PEP 008}\PY{l+s+s1}{\PYZsq{}}
\end{Verbatim}


    Et en tout cas, il est \textbf{fortement déconseillé} d'utiliser des
noms de la forme \texttt{\_\_variable\_\_} qui sont réservés au langage.
Nous reviendrons sur ce point dans le futur, mais regardez par exemple
cette variable que nous n'avons définie nulle part mais qui pourtant
existe bel et bien~:

    \begin{Verbatim}[commandchars=\\\{\}]
{\color{incolor}In [{\color{incolor}7}]:} \PY{n+nv+vm}{\PYZus{}\PYZus{}name\PYZus{}\PYZus{}}  \PY{c+c1}{\PYZsh{} ne définissez pas vous\PYZhy{}même de variables de ce genre}
\end{Verbatim}


\begin{Verbatim}[commandchars=\\\{\}]
{\color{outcolor}Out[{\color{outcolor}7}]:} '\_\_main\_\_'
\end{Verbatim}
            
    \hypertarget{ponctuation}{%
\subsubsection{Ponctuation}\label{ponctuation}}

    Dans la plage des caractères ASCII, il n'est \textbf{pas possible}
d'utiliser d'autres caractères que les caractères alphanumériques et le
tiret bas. Notamment le tiret haut \texttt{-} est interprété comme
l'opération de soustraction. Attention donc à cette erreur fréquente~:

    \begin{Verbatim}[commandchars=\\\{\}]
{\color{incolor}In [{\color{incolor} }]:} \PY{c+c1}{\PYZsh{} NOTE}
        \PY{c+c1}{\PYZsh{} automatic execution has skipped execution of this cell}
        
        \PY{n}{age}\PY{o}{\PYZhy{}}\PY{n}{moyen} \PY{o}{=} \PY{l+m+mi}{75}  \PY{c+c1}{\PYZsh{} erreur : en fait python l\PYZsq{}interprète comme \PYZsq{}age \PYZhy{} moyen = 75\PYZsq{}}
\end{Verbatim}


    \hypertarget{caractuxe8res-exotiques}{%
\subsubsection{Caractères exotiques}\label{caractuxe8res-exotiques}}

    En Python 3, il est maintenant aussi possible d'utiliser des caractères
Unicode dans les identificateurs~:

    \begin{Verbatim}[commandchars=\\\{\}]
{\color{incolor}In [{\color{incolor}8}]:} \PY{c+c1}{\PYZsh{} les caractères accentués sont permis}
        \PY{n}{nom\PYZus{}élève} \PY{o}{=} \PY{l+s+s2}{\PYZdq{}}\PY{l+s+s2}{Jules Maigret}\PY{l+s+s2}{\PYZdq{}}
\end{Verbatim}


    \begin{Verbatim}[commandchars=\\\{\}]
{\color{incolor}In [{\color{incolor}9}]:} \PY{c+c1}{\PYZsh{} ainsi que l\PYZsq{}alphabet grec}
        \PY{k+kn}{from} \PY{n+nn}{math} \PY{k}{import} \PY{n}{cos}\PY{p}{,} \PY{n}{pi} \PY{k}{as} \PY{n}{π}
        \PY{n}{θ} \PY{o}{=} \PY{n}{π} \PY{o}{/} \PY{l+m+mi}{4}
        \PY{n}{cos}\PY{p}{(}\PY{n}{θ}\PY{p}{)}
\end{Verbatim}


\begin{Verbatim}[commandchars=\\\{\}]
{\color{outcolor}Out[{\color{outcolor}9}]:} 0.7071067811865476
\end{Verbatim}
            
    Tous les caractères Unicode ne sont pas permis - heureusement car cela
serait source de confusion. Nous citons dans les références les
documents qui précisent quels sont exactement les caractères autorisés.

    \begin{Verbatim}[commandchars=\\\{\}]
{\color{incolor}In [{\color{incolor} }]:} \PY{c+c1}{\PYZsh{} NOTE}
        \PY{c+c1}{\PYZsh{} automatic execution has skipped execution of this cell}
        
        \PY{c+c1}{\PYZsh{} ce caractère n\PYZsq{}est pas autorisé, car il}
        \PY{c+c1}{\PYZsh{} est considéré comme un signe mathématique (produit)}
        \PY{err}{∏} \PY{o}{=} \PY{l+m+mi}{10}
\end{Verbatim}


    \begin{Verbatim}[commandchars=\\\{\}]
{\color{incolor}In [{\color{incolor} }]:} \PY{c+c1}{\PYZsh{} NOTE}
        \PY{c+c1}{\PYZsh{} automatic execution has skipped execution of this cell}
        
        \PY{c+c1}{\PYZsh{} ce caractère est encore différent, c\PYZsq{}est aussi}
        \PY{c+c1}{\PYZsh{} un pi grec mais pas le même, cette fois\PYZhy{}ci}
        \PY{c+c1}{\PYZsh{} c\PYZsq{}est un nom de variable acceptable mais }
        \PY{c+c1}{\PYZsh{} il n\PYZsq{}est pas défini}
        \PY{n}{𝞟}
\end{Verbatim}


    \hypertarget{conseil}{%
\paragraph{Conseil}\label{conseil}}

Il est \textbf{très vivement} recommandé~:

\begin{itemize}
\tightlist
\item
  tout d'abord de coder \textbf{en anglais}~;
\item
  ensuite de \textbf{ne pas} définir des identificateurs avec des
  caractères non ASCII, dans toute la mesure du possible~, voyez par
  exemple la confusion que peut créer le fait de nommer un
  identificateur π ou 𝞟 ou ∏~;
\item
  enfin si vous utilisez un encodage autre que UTF-8, vous
  \textbf{devez} bien \textbf{spécifier l'encodage} utilisé dans votre
  fichier source~; nous y reviendrons en deuxième semaine.
\end{itemize}

    \hypertarget{pour-en-savoir-plus}{%
\subsubsection{Pour en savoir plus}\label{pour-en-savoir-plus}}

    Pour les esprits curieux, Guido van Rossum, le fondateur de Python, est
le co-auteur d'un document qui décrit les conventions de codage à
utiliser dans la bibliothèque standard Python. Ces règles sont plus
restrictives que ce que le langage permet de faire, mais constituent une
lecture intéressante si vous projetez d'écrire beaucoup de Python.

Voir dans le PEP 008
\href{http://legacy.python.org/dev/peps/pep-0008/\#descriptive-naming-styles}{la
section consacrée aux règles de nommage - (en anglais)}

    Voir enfin, au sujet des caractères exotiques dans les identificateurs~:

\begin{itemize}
\tightlist
\item
  https://www.python.org/dev/peps/pep-3131/ qui définit les caractères
  exotiques autorisés, et qui repose à son tour sur
\item
  http://www.unicode.org/reports/tr31/ (très technique~!)
\end{itemize}


    % Add a bibliography block to the postdoc
    
    
    
