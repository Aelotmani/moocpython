    \hypertarget{un-peu-de-calcul-sur-les-types}{%
\section{Un peu de calcul sur les
types}\label{un-peu-de-calcul-sur-les-types}}

    \hypertarget{compluxe9ment---niveau-basique}{%
\subsection{Complément - niveau
basique}\label{compluxe9ment---niveau-basique}}

    \hypertarget{la-fonction-type}{%
\subsubsection{\texorpdfstring{La fonction
\texttt{type}}{La fonction type}}\label{la-fonction-type}}

    Nous avons vu dans la vidéo que chaque objet possède un type. On peut
très simplement accéder au type d'un objet en appelant une fonction
\emph{built-in}, c'est-à-dire prédéfinie dans Python, qui s'appelle, eh
bien oui, \texttt{type}.\\

    On l'utilise tout simplement comme ceci~:

    \begin{Verbatim}[commandchars=\\\{\}]
{\color{incolor}In [{\color{incolor}1}]:} \PY{n+nb}{type}\PY{p}{(}\PY{l+m+mi}{1}\PY{p}{)}
\end{Verbatim}


\begin{Verbatim}[commandchars=\\\{\}]
{\color{outcolor}Out[{\color{outcolor}1}]:} int
\end{Verbatim}
            
    \begin{Verbatim}[commandchars=\\\{\}]
{\color{incolor}In [{\color{incolor}2}]:} \PY{n+nb}{type}\PY{p}{(}\PY{l+s+s1}{\PYZsq{}}\PY{l+s+s1}{spam}\PY{l+s+s1}{\PYZsq{}}\PY{p}{)}
\end{Verbatim}


\begin{Verbatim}[commandchars=\\\{\}]
{\color{outcolor}Out[{\color{outcolor}2}]:} str
\end{Verbatim}
            
    Cette fonction est assez peu utilisée par les programmeurs expérimentés,
mais va nous être utile à bien comprendre le langage, notamment pour
manipuler les valeurs numériques.

    \hypertarget{types-variables-et-objets}{%
\subsubsection{Types, variables et
objets}\label{types-variables-et-objets}}

    On a vu également que le type est attaché \textbf{à l'objet} et non à la
variable.

    \begin{Verbatim}[commandchars=\\\{\}]
{\color{incolor}In [{\color{incolor}3}]:} \PY{n}{x} \PY{o}{=} \PY{l+m+mi}{1}
        \PY{n+nb}{type}\PY{p}{(}\PY{n}{x}\PY{p}{)}
\end{Verbatim}


\begin{Verbatim}[commandchars=\\\{\}]
{\color{outcolor}Out[{\color{outcolor}3}]:} int
\end{Verbatim}
            
    \begin{Verbatim}[commandchars=\\\{\}]
{\color{incolor}In [{\color{incolor}4}]:} \PY{c+c1}{\PYZsh{} la variable x peut référencer un objet de n\PYZsq{}importe quel type}
        
        \PY{n}{x} \PY{o}{=} \PY{p}{[}\PY{l+m+mi}{1}\PY{p}{,} \PY{l+m+mi}{2}\PY{p}{,} \PY{l+m+mi}{3}\PY{p}{]}
        \PY{n+nb}{type}\PY{p}{(}\PY{n}{x}\PY{p}{)}
\end{Verbatim}


\begin{Verbatim}[commandchars=\\\{\}]
{\color{outcolor}Out[{\color{outcolor}4}]:} list
\end{Verbatim}
            
    \hypertarget{compluxe9ment---niveau-avancuxe9}{%
\subsection{Complément - niveau
avancé}\label{compluxe9ment---niveau-avancuxe9}}

    \hypertarget{la-fonction-isinstance}{%
\subsubsection{\texorpdfstring{La fonction
\texttt{isinstance}}{La fonction isinstance}}\label{la-fonction-isinstance}}

    Une autre fonction prédéfinie, voisine de \texttt{type} mais plus utile
dans la pratique, est la fonction \texttt{isinstance} qui permet de
savoir si un objet est d'un type donné. Par exemple~:

    \begin{Verbatim}[commandchars=\\\{\}]
{\color{incolor}In [{\color{incolor}5}]:} \PY{n+nb}{isinstance}\PY{p}{(}\PY{l+m+mi}{23}\PY{p}{,} \PY{n+nb}{int}\PY{p}{)}
\end{Verbatim}


\begin{Verbatim}[commandchars=\\\{\}]
{\color{outcolor}Out[{\color{outcolor}5}]:} True
\end{Verbatim}
            
    À la vue de ce seul exemple, on pourrait penser que \texttt{isinstance}
est presque identique à \texttt{type}~; en réalité elle est un peu plus
élaborée, notamment pour la programmation objet et l'héritage, nous
aurons l'occasion d'y revenir.\\

    On remarque ici en passant que la variable \texttt{int} est connue de
Python alors que nous ne l'avons pas définie. Il s'agit d'une variable
prédéfinie, qui désigne le type des entiers, que nous étudierons très
bientôt.\\

    Pour conclure sur \texttt{isinstance}, cette fonction est utile en
pratique précisément parce que Python est à typage dynamique. Aussi il
est souvent utile de s'assurer qu'une variable passée à une fonction est
du (ou des) type(s) attendu(s), puisque contrairement à un langage typé
statiquement comme C++, on n'a aucune garantie de ce genre à
l'exécution. À nouveau, nous aurons l'occasion de revenir sur ce point.