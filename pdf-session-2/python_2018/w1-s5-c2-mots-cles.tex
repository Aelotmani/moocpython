    \hypertarget{les-mots-cluxe9s-de-python}{%
\section{Les mots-clés de Python}\label{les-mots-cluxe9s-de-python}}

    \hypertarget{mots-ruxe9servuxe9s}{%
\subsubsection{Mots réservés}\label{mots-ruxe9servuxe9s}}

    Il existe en Python certains mots spéciaux, qu'on appelle des mots-clés,
ou \emph{keywords} en anglais, qui sont réservés et \textbf{ne peuvent
pas être utilisés} comme identifiants, c'est-à-dire comme un nom de
variable.\\

    C'est le cas par exemple pour l'instruction \texttt{if}, que nous
verrons prochainement, qui permet bien entendu d'exécuter tel ou tel
code selon le résultat d'un test.

    \begin{Verbatim}[commandchars=\\\{\}]
{\color{incolor}In [{\color{incolor}1}]:} \PY{n}{variable} \PY{o}{=} \PY{l+m+mi}{15}
        \PY{k}{if} \PY{n}{variable} \PY{o}{\PYZlt{}}\PY{o}{=} \PY{l+m+mi}{10}\PY{p}{:}
            \PY{n+nb}{print}\PY{p}{(}\PY{l+s+s2}{\PYZdq{}}\PY{l+s+s2}{en dessous de la moyenne}\PY{l+s+s2}{\PYZdq{}}\PY{p}{)}
        \PY{k}{else}\PY{p}{:}
            \PY{n+nb}{print}\PY{p}{(}\PY{l+s+s2}{\PYZdq{}}\PY{l+s+s2}{au dessus}\PY{l+s+s2}{\PYZdq{}}\PY{p}{)}
\end{Verbatim}


    \begin{Verbatim}[commandchars=\\\{\}]
au dessus

    \end{Verbatim}

    À cause de la présence de cette instruction dans le langage, il n'est
pas autorisé d'appeler une variable \texttt{if}.

    \begin{Verbatim}[commandchars=\\\{\}]
{\color{incolor}In [{\color{incolor}2}]:} \PY{c+c1}{\PYZsh{} interdit, if est un mot\PYZhy{}clé}
        \PY{k}{if} \PY{o}{=} \PY{l+m+mi}{1}
\end{Verbatim}


    \begin{Verbatim}[commandchars=\\\{\}]

          File "<ipython-input-2-f16082c36546>", line 2
        if = 1
           \^{}
    SyntaxError: invalid syntax
    \end{Verbatim}

    \hypertarget{liste-compluxe8te}{%
\subsubsection{Liste complète}\label{liste-compluxe8te}}

    Voici la liste complète des mots-clés~:

    \begin{longtable}[]{@{}rrrrr@{}}
\toprule
~ & ~ & ~ & ~ & ~\tabularnewline
\midrule
\endhead
\textbf{False} & await & else & import & pass\tabularnewline
\textbf{None} & break & except & in & raise\tabularnewline
\textbf{True} & class & finally & is & return\tabularnewline
and & continue & for & lambda & try\tabularnewline
as & def & from & \textbf{nonlocal} & while\tabularnewline
assert & del & global & not & with\tabularnewline
async & elif & if & or & yield\tabularnewline
\bottomrule
\end{longtable}

    Nous avons indiqué \textbf{en gras} les nouveautés \textbf{par rapport à
Python 2} (sachant que réciproquement \texttt{exec} et \texttt{print}
ont perdu leur statut de mot-clé depuis Python 2, ce sont maintenant des
fonctions).\\

    Il vous faudra donc y prêter attention, surtout au début, mais avec un
tout petit peu d'habitude vous saurez rapidement les éviter.\\

Vous remarquerez aussi que tous les bons éditeurs de texte supportant du
code Python vont colorer les mots-clés différemment des variables. Par
exemple, IDLE colorie les mots-clés en orange, vous pouvez donc très
facilement vous rendre compte que vous allez, par erreur, en utiliser un
comme nom de variable.\\

Cette fonctionnalité, dite de \emph{coloration syntaxique}, permet
d'identifier d'un coup d'œil, grâce à un code de couleur, le rôle des
différents éléments de votre code~: variables, mots-clés, etc.~D'une
manière générale, nous vous déconseillons fortement d'utiliser un
éditeur de texte qui n'offre pas cette fonctionnalité de coloration
syntaxique.

    \hypertarget{pour-en-savoir-plus}{%
\subsubsection{Pour en savoir plus}\label{pour-en-savoir-plus}}

    On peut se reporter à cette page~:

\href{https://docs.python.org/3/reference/lexical\_analysis.html\#keywords}{https://docs.python.org/3/reference/lexical\_analysis.html\#keywords}