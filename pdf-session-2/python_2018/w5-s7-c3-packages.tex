    
    
    
    

    

    \hypertarget{la-notion-de-package}{%
\section{La notion de package}\label{la-notion-de-package}}

    \hypertarget{compluxe9ment---niveau-basique}{%
\subsection{Complément - niveau
basique}\label{compluxe9ment---niveau-basique}}

    Dans ce complément, nous approfondissons la notion de module, qui a été
introduite dans les vidéos, et nous décrivons la notion de
\emph{package} qui permet de créer des bibliothèques plus structurées
qu'avec un simple module.

    Pour ce notebook nous aurons besoin de deux utilitaires pour voir le
code correspondant aux modules et packages que nous manipulons~:

    \begin{Verbatim}[commandchars=\\\{\}]
{\color{incolor}In [{\color{incolor}1}]:} \PY{k+kn}{from} \PY{n+nn}{modtools} \PY{k}{import} \PY{n}{show\PYZus{}module}
\end{Verbatim}


    \hypertarget{rappel-sur-les-modules}{%
\subsubsection{Rappel sur les modules}\label{rappel-sur-les-modules}}

    Nous avons vu dans la vidéo qu'on peut charger une bibliothèque,
lorsqu'elle se présente sous la forme d'un seul fichier source, au
travers d'un objet python de type \textbf{module}.

    Chargeons un module ``jouet''~:

    \begin{Verbatim}[commandchars=\\\{\}]
{\color{incolor}In [{\color{incolor}2}]:} \PY{k+kn}{import} \PY{n+nn}{module\PYZus{}simple}
\end{Verbatim}


    \begin{Verbatim}[commandchars=\\\{\}]
Chargement du module module\_simple

    \end{Verbatim}

    Voyons à quoi ressemble ce module~:

    \begin{Verbatim}[commandchars=\\\{\}]
{\color{incolor}In [{\color{incolor}3}]:} \PY{n}{show\PYZus{}module}\PY{p}{(}\PY{n}{module\PYZus{}simple}\PY{p}{)}
\end{Verbatim}


    \begin{Verbatim}[commandchars=\\\{\}]
Fichier /Users/tparment/git/flotpython/modules/module\_simple.py
----------------------------------------
1|print("Chargement du module", \_\_name\_\_)
2|
3|def spam(n):
4|    "Le polynôme (n+1)*(n-3)"
5|    return n**2 - 2*n - 3

    \end{Verbatim}

    On a bien compris maintenant que le module joue le rôle d'\textbf{espace
de nom}, dans le sens où~:

    \begin{Verbatim}[commandchars=\\\{\}]
{\color{incolor}In [{\color{incolor}4}]:} \PY{c+c1}{\PYZsh{} on peut définir sans risque une variable globale \PYZsq{}spam\PYZsq{}}
        \PY{n}{spam} \PY{o}{=} \PY{l+s+s1}{\PYZsq{}}\PY{l+s+s1}{eggs}\PY{l+s+s1}{\PYZsq{}}
        \PY{n+nb}{print}\PY{p}{(}\PY{l+s+s2}{\PYZdq{}}\PY{l+s+s2}{spam globale}\PY{l+s+s2}{\PYZdq{}}\PY{p}{,} \PY{n}{spam}\PY{p}{)}
\end{Verbatim}


    \begin{Verbatim}[commandchars=\\\{\}]
spam globale eggs

    \end{Verbatim}

    \begin{Verbatim}[commandchars=\\\{\}]
{\color{incolor}In [{\color{incolor}5}]:} \PY{c+c1}{\PYZsh{} qui est indépendante de celle définie dans le module}
        \PY{n+nb}{print}\PY{p}{(}\PY{l+s+s2}{\PYZdq{}}\PY{l+s+s2}{spam du module}\PY{l+s+s2}{\PYZdq{}}\PY{p}{,} \PY{n}{module\PYZus{}simple}\PY{o}{.}\PY{n}{spam}\PY{p}{)}
\end{Verbatim}


    \begin{Verbatim}[commandchars=\\\{\}]
spam du module <function spam at 0x102904620>

    \end{Verbatim}

    Pour résumer, un module est donc un objet python qui correspond à la
fois à~:

\begin{itemize}
\tightlist
\item
  un (seul) \textbf{fichier} sur le disque~;
\item
  et un \textbf{espace de nom} pour les variables du programme.
\end{itemize}

    \hypertarget{la-notion-de-package}{%
\subsubsection{La notion de package}\label{la-notion-de-package}}

    Lorsqu'il s'agit d'implémenter une très grosse bibliothèque, il n'est
pas concevable de tout concentrer en un seul fichier. C'est là
qu'intervient la notion de \textbf{package}, qui est un peu aux
\textbf{répertoires} ce que que le \textbf{module} est aux
\textbf{fichiers}.

    Nous allons illustrer ceci en créant un package qui contient un module.
Pour cela nous créons une arborescence de fichiers comme ceci~:

\begin{verbatim}
package_jouet/
    __init__.py
    module_jouet.py
\end{verbatim}

    On importe un package exactement comme un module~:

    \begin{Verbatim}[commandchars=\\\{\}]
{\color{incolor}In [{\color{incolor}6}]:} \PY{k+kn}{import} \PY{n+nn}{package\PYZus{}jouet}
\end{Verbatim}


    \begin{Verbatim}[commandchars=\\\{\}]
chargement du package package\_jouet
Chargement du module package\_jouet.module\_jouet dans le package 'package\_jouet'

    \end{Verbatim}

    Voici le contenu de ces deux fichiers~:

    \begin{Verbatim}[commandchars=\\\{\}]
{\color{incolor}In [{\color{incolor}7}]:} \PY{n}{show\PYZus{}module}\PY{p}{(}\PY{n}{package\PYZus{}jouet}\PY{p}{)}
\end{Verbatim}


    \begin{Verbatim}[commandchars=\\\{\}]
Fichier /Users/tparment/git/flotpython/modules/package\_jouet/\_\_init\_\_.py
----------------------------------------
1|print("chargement du package", \_\_name\_\_)
2|
3|spam = ['a', 'b', 'c']
4|
5|\# on peut forcer l'import de modules
6|import package\_jouet.module\_jouet
7|
8|\# et définir des raccourcis
9|jouet = package\_jouet.module\_jouet.jouet

    \end{Verbatim}

    \begin{Verbatim}[commandchars=\\\{\}]
{\color{incolor}In [{\color{incolor}8}]:} \PY{n}{show\PYZus{}module}\PY{p}{(}\PY{n}{package\PYZus{}jouet}\PY{o}{.}\PY{n}{module\PYZus{}jouet}\PY{p}{)}
\end{Verbatim}


    \begin{Verbatim}[commandchars=\\\{\}]
Fichier /Users/tparment/git/flotpython/modules/package\_jouet/module\_jouet.py
----------------------------------------
1|print("Chargement du module", \_\_name\_\_, "dans le package 'package\_jouet'")
2|
3|jouet = 'une variable définie dans package\_jouet.module\_jouet'

    \end{Verbatim}

    Comme on le voit, le package porte \textbf{le même nom} que le
répertoire, c'est-à-dire que, de même que le module
\texttt{module\_simple} correspond au fichier
\texttt{module\_simple.py}, le package python \texttt{package\_jouet}
corrrespond au répertoire \texttt{package\_jouet}.

    Cependant, pour définir un package, il faut \textbf{obligatoirement}
créer dans le répertoire (celui, donc, que l'on veut exposer à python),
un fichier nommé \textbf{\texttt{\_\_init\_\_.py}}.

Comme on le voit, importer un package revient essentiellement à charger
le fichier \texttt{\_\_init\_\_.py} dans le répertoire correspondant.

    On a coutume de faire la différence entre package et module, mais en
termes d'implémentation les deux objets sont en fait de même nature, ce
sont des modules~:

    \begin{Verbatim}[commandchars=\\\{\}]
{\color{incolor}In [{\color{incolor}9}]:} \PY{n+nb}{type}\PY{p}{(}\PY{n}{package\PYZus{}jouet}\PY{p}{)}
\end{Verbatim}


\begin{Verbatim}[commandchars=\\\{\}]
{\color{outcolor}Out[{\color{outcolor}9}]:} module
\end{Verbatim}
            
    \begin{Verbatim}[commandchars=\\\{\}]
{\color{incolor}In [{\color{incolor}10}]:} \PY{n+nb}{type}\PY{p}{(}\PY{n}{package\PYZus{}jouet}\PY{o}{.}\PY{n}{module\PYZus{}jouet}\PY{p}{)}
\end{Verbatim}


\begin{Verbatim}[commandchars=\\\{\}]
{\color{outcolor}Out[{\color{outcolor}10}]:} module
\end{Verbatim}
            
    Ainsi, le package se présente aussi comme un espace de nom, à présent on
a une troisième variable \texttt{spam} qui est encore différente des
deux autres~:

    \begin{Verbatim}[commandchars=\\\{\}]
{\color{incolor}In [{\color{incolor}11}]:} \PY{n}{package\PYZus{}jouet}\PY{o}{.}\PY{n}{spam}
\end{Verbatim}


\begin{Verbatim}[commandchars=\\\{\}]
{\color{outcolor}Out[{\color{outcolor}11}]:} ['a', 'b', 'c']
\end{Verbatim}
            
    L'espace de noms du package permet de référencer les packages ou modules
qu'il contient, comme on l'a vu ci-dessus, le package référence le
module au travers de son attribut \texttt{module\_jouet}~:

    \begin{Verbatim}[commandchars=\\\{\}]
{\color{incolor}In [{\color{incolor}12}]:} \PY{n}{package\PYZus{}jouet}\PY{o}{.}\PY{n}{module\PYZus{}jouet}
\end{Verbatim}


\begin{Verbatim}[commandchars=\\\{\}]
{\color{outcolor}Out[{\color{outcolor}12}]:} <module 'package\_jouet.module\_jouet' from '/Users/tparment/git/flotpython/modules/package\_jouet/module\_jouet.py'>
\end{Verbatim}
            
    \hypertarget{uxe0-quoi-sert-__init__.py}{%
\subsubsection{\texorpdfstring{À quoi sert
\texttt{\_\_init\_\_.py}~?}{À quoi sert \_\_init\_\_.py~?}}\label{uxe0-quoi-sert-__init__.py}}

    Vous remarquerez que le module \texttt{module\_jouet} a été chargé au
même moment que \texttt{package\_jouet}. Ce comportement \textbf{n'est
pas implicite}. C'est nous qui avons explicitement choisi d'importer le
module dans le package (dans \texttt{\_\_init\_\_.py}).

    Cette technique correpond à un usage assez fréquent, où on veut exposer
directement dans l'espace de nom du package des symboles qui sont en
réalité définis dans un module.

Avec le code ci-dessus, après avoir importé \texttt{package\_jouet},
nous pouvons utiliser

    \begin{Verbatim}[commandchars=\\\{\}]
{\color{incolor}In [{\color{incolor}13}]:} \PY{n}{package\PYZus{}jouet}\PY{o}{.}\PY{n}{jouet}
\end{Verbatim}


\begin{Verbatim}[commandchars=\\\{\}]
{\color{outcolor}Out[{\color{outcolor}13}]:} 'une variable définie dans package\_jouet.module\_jouet'
\end{Verbatim}
            
    alors qu'en fait il faudrait écrire en toute rigueur

    \begin{Verbatim}[commandchars=\\\{\}]
{\color{incolor}In [{\color{incolor}14}]:} \PY{n}{package\PYZus{}jouet}\PY{o}{.}\PY{n}{module\PYZus{}jouet}\PY{o}{.}\PY{n}{jouet}
\end{Verbatim}


\begin{Verbatim}[commandchars=\\\{\}]
{\color{outcolor}Out[{\color{outcolor}14}]:} 'une variable définie dans package\_jouet.module\_jouet'
\end{Verbatim}
            
    Mais cela impose alors à l'utilisateur d'avoir une connaissance sur
l'organisation interne de la bibliothèque, ce qui est considéré comme
une mauvaise pratique.

D'abord, cela donne facilement des noms à rallonge et du coup nuit à la
lisibilité, ce n'est pas pratique. Mais surtout, que se passerait-il
alors si le développeur du package voulait renommer des modules à
l'intérieur de la bibliothèque~? On ne veut pas que ce genre de décision
ait un impact sur les utilisateurs.

    Au delà de cet usage permettant de définir une sorte de raccourcis, le
code placé dans \texttt{\_\_init\_\_.py} est chargé d'initialiser la
bibliothèque. Le fichier \textbf{peut être vide} mais \textbf{doit
absolument exister}. Nous vous mettons en garde car c'est une erreur
fréquente de l'oublier. Sans lui vous ne pourrez importer ni le package,
ni les modules ou sous-packages qu'il contient.

    À nouveau c'est ce fichier qui est chargé par l'interpréteur python
lorsque vous importez le package. Comme pour les modules, le fichier
n'est chargé qu'une seule fois par l'interpréteur python, s'il rencontre
plus tard à nouveau le même \texttt{import}, il l'ignore
silencieusement.

    \hypertarget{pour-en-savoir-plus}{%
\subsubsection{Pour en savoir plus}\label{pour-en-savoir-plus}}

    Voir la \href{https://docs.python.org/3/tutorial/modules.html}{section
sur les modules} dans la documentation python, et notamment la
\href{https://docs.python.org/3/tutorial/modules.html\#packages}{section
sur les packages}.


    % Add a bibliography block to the postdoc
    
    
    
