    
    
    
    

    

    \hypertarget{lopuxe9rateur-is}{%
\section{\texorpdfstring{L'opérateur
\texttt{is}}{L'opérateur is}}\label{lopuxe9rateur-is}}

    \hypertarget{compluxe9ment---niveau-basique}{%
\subsection{Complément - niveau
basique}\label{compluxe9ment---niveau-basique}}

    \begin{Verbatim}[commandchars=\\\{\}]
{\color{incolor}In [{\color{incolor}1}]:} \PY{o}{\PYZpc{}}\PY{k}{load\PYZus{}ext} ipythontutor
\end{Verbatim}


    \hypertarget{les-opuxe9rateurs-is-et}{%
\subsubsection{\texorpdfstring{Les opérateurs \texttt{is} et
\texttt{==}}{Les opérateurs is et ==}}\label{les-opuxe9rateurs-is-et}}

    \begin{itemize}
\tightlist
\item
  nous avons déjà parlé de l'opérateur \texttt{==} qui \textbf{compare
  la valeur} de deux objets~;
\item
  python fournit aussi un opérateur \texttt{is} qui permet de savoir si
  deux valeurs correspondent \textbf{au même objet} en mémoire.
\end{itemize}

Nous allons illustrer la différence entre ces deux opérateurs.

    \hypertarget{scuxe9nario-1}{%
\subparagraph{Scénario 1}\label{scuxe9nario-1}}

    \begin{Verbatim}[commandchars=\\\{\}]
{\color{incolor}In [{\color{incolor}2}]:} \PY{c+c1}{\PYZsh{} deux listes identiques}
        \PY{n}{a} \PY{o}{=} \PY{p}{[}\PY{l+m+mi}{1}\PY{p}{,} \PY{l+m+mi}{2}\PY{p}{]}
        \PY{n}{b} \PY{o}{=} \PY{p}{[}\PY{l+m+mi}{1}\PY{p}{,} \PY{l+m+mi}{2}\PY{p}{]}
        
        \PY{c+c1}{\PYZsh{} les deux objets se ressemblent}
        \PY{n+nb}{print}\PY{p}{(}\PY{l+s+s1}{\PYZsq{}}\PY{l+s+s1}{==}\PY{l+s+s1}{\PYZsq{}}\PY{p}{,} \PY{n}{a} \PY{o}{==} \PY{n}{b}\PY{p}{)}
\end{Verbatim}


    \begin{Verbatim}[commandchars=\\\{\}]
== True

    \end{Verbatim}

    \begin{Verbatim}[commandchars=\\\{\}]
{\color{incolor}In [{\color{incolor}3}]:} \PY{c+c1}{\PYZsh{} mais ce ne sont pas les mêmes objets}
        \PY{n+nb}{print}\PY{p}{(}\PY{l+s+s1}{\PYZsq{}}\PY{l+s+s1}{is}\PY{l+s+s1}{\PYZsq{}}\PY{p}{,} \PY{n}{a} \PY{o+ow}{is} \PY{n}{b}\PY{p}{)}
\end{Verbatim}


    \begin{Verbatim}[commandchars=\\\{\}]
is False

    \end{Verbatim}

    \hypertarget{scuxe9nario-2}{%
\subparagraph{Scénario 2}\label{scuxe9nario-2}}

    \begin{Verbatim}[commandchars=\\\{\}]
{\color{incolor}In [{\color{incolor}4}]:} \PY{c+c1}{\PYZsh{} par contre ici il n\PYZsq{}y a qu\PYZsq{}une liste}
        \PY{n}{a} \PY{o}{=} \PY{p}{[}\PY{l+m+mi}{1}\PY{p}{,} \PY{l+m+mi}{2}\PY{p}{]}
        
        \PY{c+c1}{\PYZsh{} et les deux variables}
        \PY{c+c1}{\PYZsh{} référencent le même objet}
        \PY{n}{b} \PY{o}{=} \PY{n}{a}
        
        \PY{c+c1}{\PYZsh{} non seulement les deux expressions se ressemblent}
        \PY{n+nb}{print}\PY{p}{(}\PY{l+s+s1}{\PYZsq{}}\PY{l+s+s1}{==}\PY{l+s+s1}{\PYZsq{}}\PY{p}{,} \PY{n}{a} \PY{o}{==} \PY{n}{b}\PY{p}{)}
\end{Verbatim}


    \begin{Verbatim}[commandchars=\\\{\}]
== True

    \end{Verbatim}

    \begin{Verbatim}[commandchars=\\\{\}]
{\color{incolor}In [{\color{incolor}5}]:} \PY{c+c1}{\PYZsh{} mais elles désignent le même objet}
        \PY{n+nb}{print}\PY{p}{(}\PY{l+s+s1}{\PYZsq{}}\PY{l+s+s1}{is}\PY{l+s+s1}{\PYZsq{}}\PY{p}{,} \PY{n}{a} \PY{o+ow}{is} \PY{n}{b}\PY{p}{)}
\end{Verbatim}


    \begin{Verbatim}[commandchars=\\\{\}]
is True

    \end{Verbatim}

    \hypertarget{la-muxeame-chose-sous-pythontutor}{%
\subsubsection{La même chose sous
pythontutor}\label{la-muxeame-chose-sous-pythontutor}}

    \hypertarget{scuxe9nario-1}{%
\subparagraph{Scénario 1}\label{scuxe9nario-1}}

    \begin{Verbatim}[commandchars=\\\{\}]
{\color{incolor}In [{\color{incolor} }]:} \PY{c+c1}{\PYZsh{} NOTE}
        \PY{c+c1}{\PYZsh{} automatic execution has skipped execution of this cell}
        
        \PY{o}{\PYZpc{}\PYZpc{}}\PY{k}{ipythontutor} curInstr=2
        a = [1, 2]
        b = [1, 2]
\end{Verbatim}


    \hypertarget{scuxe9nario-2}{%
\subparagraph{Scénario 2}\label{scuxe9nario-2}}

    \begin{Verbatim}[commandchars=\\\{\}]
{\color{incolor}In [{\color{incolor} }]:} \PY{c+c1}{\PYZsh{} NOTE}
        \PY{c+c1}{\PYZsh{} automatic execution has skipped execution of this cell}
        
        \PY{o}{\PYZpc{}\PYZpc{}}\PY{k}{ipythontutor} curInstr=1
        \PYZsh{} équivalent à la forme ci\PYZhy{}dessus
        \PYZsh{} a = [1, 2]
        \PYZsh{} b = a
        a = b = [1, 2]
\end{Verbatim}


    \hypertarget{utilisez-is-plutuxf4t-que-lorsque-cest-possible}{%
\subsubsection{\texorpdfstring{Utilisez \texttt{is} plutôt que
\texttt{==} lorsque c'est
possible}{Utilisez is plutôt que == lorsque c'est possible}}\label{utilisez-is-plutuxf4t-que-lorsque-cest-possible}}

    La pratique usuelle est d'utiliser \texttt{is} lorsqu'on compare avec un
objet qui est un singleton, comme typiquement \texttt{None}.

    Par exemple on préfèrera écrire~:

    \begin{Verbatim}[commandchars=\\\{\}]
{\color{incolor}In [{\color{incolor}6}]:} \PY{n}{undef} \PY{o}{=} \PY{k+kc}{None}
        
        \PY{k}{if} \PY{n}{undef} \PY{o+ow}{is} \PY{k+kc}{None}\PY{p}{:}
            \PY{n+nb}{print}\PY{p}{(}\PY{l+s+s1}{\PYZsq{}}\PY{l+s+s1}{indéfini}\PY{l+s+s1}{\PYZsq{}}\PY{p}{)}
\end{Verbatim}


    \begin{Verbatim}[commandchars=\\\{\}]
indéfini

    \end{Verbatim}

    Plutôt que~:

    \begin{Verbatim}[commandchars=\\\{\}]
{\color{incolor}In [{\color{incolor}7}]:} \PY{k}{if} \PY{n}{undef} \PY{o}{==} \PY{k+kc}{None}\PY{p}{:}
            \PY{n+nb}{print}\PY{p}{(}\PY{l+s+s1}{\PYZsq{}}\PY{l+s+s1}{indéfini}\PY{l+s+s1}{\PYZsq{}}\PY{p}{)}
\end{Verbatim}


    \begin{Verbatim}[commandchars=\\\{\}]
indéfini

    \end{Verbatim}

    Qui se comporte de la même manière (à nouveau, parce qu'on compare avec
\texttt{None}), mais est légèrement moins lisible, et franchement moins
pythonique. :)

    Notez aussi et surtout que \texttt{is} est \textbf{plus efficace} que
\texttt{==}. En effet \texttt{is} peut être évalué en temps constant,
puisqu'il s'agit essentiellement de comparer les deux adresses. Alors
que pour \texttt{==} il peut s'agir de parcourir toute une structure de
données possiblement très complexe.

    \hypertarget{compluxe9ment---niveau-intermuxe9diaire}{%
\subsection{Complément - niveau
intermédiaire}\label{compluxe9ment---niveau-intermuxe9diaire}}

    \hypertarget{la-fonction-id}{%
\subsubsection{\texorpdfstring{La fonction
\texttt{id}}{La fonction id}}\label{la-fonction-id}}

    Pour bien comprendre le fonctionnement de \texttt{is} nous allons voir
la fonction \texttt{id} qui retourne un identificateur unique pour
chaque objet~; un modèle mental acceptable est celui d'adresse mémoire.

    \begin{Verbatim}[commandchars=\\\{\}]
{\color{incolor}In [{\color{incolor}8}]:} \PY{n+nb}{id}\PY{p}{(}\PY{k+kc}{True}\PY{p}{)}
\end{Verbatim}


\begin{Verbatim}[commandchars=\\\{\}]
{\color{outcolor}Out[{\color{outcolor}8}]:} 4522206720
\end{Verbatim}
            
    Comme vous vous en doutez, l'opérateur \texttt{is} peut être décrit
formellement à partir de \texttt{id} comme ceci~:

(\texttt{a\ is\ b}) \(\Longleftrightarrow\) (\texttt{id(a)\ ==\ id(b)})

    \hypertarget{certains-types-de-base-sont-des-singletons}{%
\subsubsection{Certains types de base sont des
singletons}\label{certains-types-de-base-sont-des-singletons}}

    Un singleton est un objet qui n'existe qu'en un seul exemplaire dans la
mémoire. Un usage classique des singletons en Python est de minimiser le
nombre d'objets immuables en mémoire. Voyons ce que cela nous donne avec
des entiers~:

    \begin{Verbatim}[commandchars=\\\{\}]
{\color{incolor}In [{\color{incolor}9}]:} \PY{n}{a} \PY{o}{=} \PY{l+m+mi}{3}
        \PY{n}{b} \PY{o}{=} \PY{l+m+mi}{3}
        \PY{n+nb}{print}\PY{p}{(}\PY{l+s+s1}{\PYZsq{}}\PY{l+s+s1}{a}\PY{l+s+s1}{\PYZsq{}}\PY{p}{,} \PY{n+nb}{id}\PY{p}{(}\PY{n}{a}\PY{p}{)}\PY{p}{,} \PY{l+s+s1}{\PYZsq{}}\PY{l+s+s1}{b}\PY{l+s+s1}{\PYZsq{}}\PY{p}{,} \PY{n+nb}{id}\PY{p}{(}\PY{n}{b}\PY{p}{)}\PY{p}{)}
\end{Verbatim}


    \begin{Verbatim}[commandchars=\\\{\}]
a 4522610768 b 4522610768

    \end{Verbatim}

    Tiens, c'est curieux, nous avons ici deux objets, que l'on pourrait
penser différents, mais en fait ce sont les mêmes~; \texttt{a} et
\texttt{b} désignent \textbf{le même objet} python, et on a~:

    \begin{Verbatim}[commandchars=\\\{\}]
{\color{incolor}In [{\color{incolor}10}]:} \PY{n}{a} \PY{o+ow}{is} \PY{n}{b}
\end{Verbatim}


\begin{Verbatim}[commandchars=\\\{\}]
{\color{outcolor}Out[{\color{outcolor}10}]:} True
\end{Verbatim}
            
    Il se trouve que, dans le cas des petits entiers, python réalise une
optimisation de l'utilisation de la mémoire. Quel que soit le nombre de
variables dont la valeur est \texttt{3}, un seul objet correspondant à
l'entier \texttt{3} est alloué et créé, pour éviter d'engorger la
mémoire. On dit que l'entier \texttt{3} est implémenté comme un
singleton~; nous reverrons ceci en exercice.

    On trouve cette optimisation avec quelques autres objets python, comme
par exemple~:

    \begin{Verbatim}[commandchars=\\\{\}]
{\color{incolor}In [{\color{incolor}11}]:} \PY{n}{a} \PY{o}{=} \PY{l+s+s2}{\PYZdq{}}\PY{l+s+s2}{\PYZdq{}}
         \PY{n}{b} \PY{o}{=} \PY{l+s+s2}{\PYZdq{}}\PY{l+s+s2}{\PYZdq{}}
         \PY{n}{a} \PY{o+ow}{is} \PY{n}{b}
\end{Verbatim}


\begin{Verbatim}[commandchars=\\\{\}]
{\color{outcolor}Out[{\color{outcolor}11}]:} True
\end{Verbatim}
            
    Ou encore, plus surprenant~:

    \begin{Verbatim}[commandchars=\\\{\}]
{\color{incolor}In [{\color{incolor}12}]:} \PY{n}{a} \PY{o}{=} \PY{l+s+s2}{\PYZdq{}}\PY{l+s+s2}{foo}\PY{l+s+s2}{\PYZdq{}}
         \PY{n}{b} \PY{o}{=} \PY{l+s+s2}{\PYZdq{}}\PY{l+s+s2}{foo}\PY{l+s+s2}{\PYZdq{}}
         \PY{n}{a} \PY{o+ow}{is} \PY{n}{b}
\end{Verbatim}


\begin{Verbatim}[commandchars=\\\{\}]
{\color{outcolor}Out[{\color{outcolor}12}]:} True
\end{Verbatim}
            
    \textbf{Conclusion} cette optimisation ne touche aucun type mutable
(heureusement)~; pour les types immuables, il n'est pas extrêmement
important de savoir en détail quels objets sont implémentés de la sorte.

Ce qui est par contre extrêmement important est de comprendre la
différence entre \texttt{is} et \texttt{==}, et de les utiliser à bon
escient au risque d'écrire du code fragile.

    \hypertarget{pour-en-savoir-plus}{%
\subsubsection{Pour en savoir plus}\label{pour-en-savoir-plus}}

    Aux étudiants de niveau avancé, nous recommandons la lecture de la
section ``Objects, values and types'' dans la documentation Python~:

https://docs.python.org/3/reference/datamodel.html\#objects-values-and-types

qui aborde également la notion de ``garbage collection'', que nous
n'aurons pas le temps d'approfondir dans ce MOOC.


    % Add a bibliography block to the postdoc
    
    
    
