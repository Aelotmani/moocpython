    
    
    
    

    

    \hypertarget{dataframe-de-pandas}{%
\section{\texorpdfstring{\texttt{DataFrame} de
\texttt{pandas}}{DataFrame de pandas}}\label{dataframe-de-pandas}}

    \hypertarget{compluxe9ment---niveau-intermuxe9diaire}{%
\subsection{Complément - niveau
intermédiaire}\label{compluxe9ment---niveau-intermuxe9diaire}}

    \hypertarget{cruxe9ation-dune-dataframe}{%
\subsubsection{\texorpdfstring{Création d'une
\texttt{DataFrame}}{Création d'une DataFrame}}\label{cruxe9ation-dune-dataframe}}

    Une \texttt{DataFrame} est un tableau \texttt{numpy} à deux dimensions
avec un index pour les lignes et un index pour les colonnes. Il y a de
nombreuses manières de construire une \texttt{DataFrame}.

    \begin{Verbatim}[commandchars=\\\{\}]
{\color{incolor}In [{\color{incolor}1}]:} \PY{c+c1}{\PYZsh{} Regardons la construction d\PYZsq{}une DataFrame}
        \PY{k+kn}{import} \PY{n+nn}{numpy} \PY{k}{as} \PY{n+nn}{np}
        \PY{k+kn}{import} \PY{n+nn}{pandas} \PY{k}{as} \PY{n+nn}{pd}
        
        \PY{c+c1}{\PYZsh{} Créons une Series pour définir des âges}
        \PY{n}{age} \PY{o}{=} \PY{n}{pd}\PY{o}{.}\PY{n}{Series}\PY{p}{(}\PY{p}{[}\PY{l+m+mi}{30}\PY{p}{,} \PY{l+m+mi}{20}\PY{p}{,} \PY{l+m+mi}{50}\PY{p}{]}\PY{p}{,} \PY{n}{index}\PY{o}{=}\PY{p}{[}\PY{l+s+s1}{\PYZsq{}}\PY{l+s+s1}{alice}\PY{l+s+s1}{\PYZsq{}}\PY{p}{,} \PY{l+s+s1}{\PYZsq{}}\PY{l+s+s1}{bob}\PY{l+s+s1}{\PYZsq{}}\PY{p}{,} \PY{l+s+s1}{\PYZsq{}}\PY{l+s+s1}{julie}\PY{l+s+s1}{\PYZsq{}}\PY{p}{]}\PY{p}{)}
        
        \PY{c+c1}{\PYZsh{} et une Series pour définir des tailles}
        \PY{n}{height} \PY{o}{=} \PY{n}{pd}\PY{o}{.}\PY{n}{Series}\PY{p}{(}\PY{p}{[}\PY{l+m+mi}{150}\PY{p}{,} \PY{l+m+mi}{170}\PY{p}{,} \PY{l+m+mi}{168}\PY{p}{]}\PY{p}{,} \PY{n}{index}\PY{o}{=}\PY{p}{[}\PY{l+s+s1}{\PYZsq{}}\PY{l+s+s1}{alice}\PY{l+s+s1}{\PYZsq{}}\PY{p}{,} \PY{l+s+s1}{\PYZsq{}}\PY{l+s+s1}{marc}\PY{l+s+s1}{\PYZsq{}}\PY{p}{,} \PY{l+s+s1}{\PYZsq{}}\PY{l+s+s1}{julie}\PY{l+s+s1}{\PYZsq{}}\PY{p}{]}\PY{p}{)}
        
        \PY{c+c1}{\PYZsh{} On peut maintenant combiner ces deux Series en DataFrame,}
        \PY{c+c1}{\PYZsh{} chaque Series définissant une colonne, une manière de le faire est}
        \PY{c+c1}{\PYZsh{} de définir un dictionnaire qui contient pour clef le nom de la colonne}
        \PY{c+c1}{\PYZsh{} et pour valeur la Series correspondante}
        \PY{n}{stat} \PY{o}{=} \PY{n}{pd}\PY{o}{.}\PY{n}{DataFrame}\PY{p}{(}\PY{p}{\PYZob{}}\PY{l+s+s1}{\PYZsq{}}\PY{l+s+s1}{age}\PY{l+s+s1}{\PYZsq{}}\PY{p}{:} \PY{n}{age}\PY{p}{,} \PY{l+s+s1}{\PYZsq{}}\PY{l+s+s1}{height}\PY{l+s+s1}{\PYZsq{}}\PY{p}{:} \PY{n}{height}\PY{p}{\PYZcb{}}\PY{p}{)}
        \PY{n+nb}{print}\PY{p}{(}\PY{n}{stat}\PY{p}{)}
\end{Verbatim}


    \begin{Verbatim}[commandchars=\\\{\}]
        age  height
alice  30.0   150.0
bob    20.0     NaN
julie  50.0   168.0
marc    NaN   170.0

    \end{Verbatim}

    On remarque que \texttt{pandas} fait automatiquement l'alignement des
index, lorsqu'une valeur n'est pas présente, elle est automatiquement
remplacée par \texttt{NaN}. \texttt{Panda} va également broadcaster une
valeur unique définissant une colonne sur toutes les lignes. Regardons
cela~:

    \begin{Verbatim}[commandchars=\\\{\}]
{\color{incolor}In [{\color{incolor}2}]:} \PY{n}{stat} \PY{o}{=} \PY{n}{pd}\PY{o}{.}\PY{n}{DataFrame}\PY{p}{(}\PY{p}{\PYZob{}}\PY{l+s+s1}{\PYZsq{}}\PY{l+s+s1}{age}\PY{l+s+s1}{\PYZsq{}}\PY{p}{:} \PY{n}{age}\PY{p}{,} \PY{l+s+s1}{\PYZsq{}}\PY{l+s+s1}{height}\PY{l+s+s1}{\PYZsq{}}\PY{p}{:} \PY{n}{height}\PY{p}{,} \PY{l+s+s1}{\PYZsq{}}\PY{l+s+s1}{city}\PY{l+s+s1}{\PYZsq{}}\PY{p}{:} \PY{l+s+s1}{\PYZsq{}}\PY{l+s+s1}{Nice}\PY{l+s+s1}{\PYZsq{}}\PY{p}{\PYZcb{}}\PY{p}{)}
        \PY{n+nb}{print}\PY{p}{(}\PY{n}{stat}\PY{p}{)}
\end{Verbatim}


    \begin{Verbatim}[commandchars=\\\{\}]
        age  height  city
alice  30.0   150.0  Nice
bob    20.0     NaN  Nice
julie  50.0   168.0  Nice
marc    NaN   170.0  Nice

    \end{Verbatim}

    \begin{Verbatim}[commandchars=\\\{\}]
{\color{incolor}In [{\color{incolor}3}]:} \PY{c+c1}{\PYZsh{} On peut maintenant accéder aux index des lignes et des colonnes}
        
        \PY{c+c1}{\PYZsh{} l\PYZsq{}index des lignes}
        \PY{n+nb}{print}\PY{p}{(}\PY{n}{stat}\PY{o}{.}\PY{n}{index}\PY{p}{)}
\end{Verbatim}


    \begin{Verbatim}[commandchars=\\\{\}]
Index(['alice', 'bob', 'julie', 'marc'], dtype='object')

    \end{Verbatim}

    \begin{Verbatim}[commandchars=\\\{\}]
{\color{incolor}In [{\color{incolor}4}]:} \PY{c+c1}{\PYZsh{} l\PYZsq{}index des colonnes}
        \PY{n+nb}{print}\PY{p}{(}\PY{n}{stat}\PY{o}{.}\PY{n}{columns}\PY{p}{)}
\end{Verbatim}


    \begin{Verbatim}[commandchars=\\\{\}]
Index(['age', 'height', 'city'], dtype='object')

    \end{Verbatim}

    Il y a de nombreuses manières d'accéder aux éléments de la
\texttt{DataFrame}, certaines sont bonnes et d'autres à proscrire,
commençons par prendre de bonnes habitudes. Comme il s'agit d'une
structure à deux dimensions, il faut donner un indice de ligne et de
colonne~:

    \begin{Verbatim}[commandchars=\\\{\}]
{\color{incolor}In [{\color{incolor}5}]:} \PY{c+c1}{\PYZsh{} Quel est l\PYZsq{}âge de alice}
        \PY{n}{a} \PY{o}{=} \PY{n}{stat}\PY{o}{.}\PY{n}{loc}\PY{p}{[}\PY{l+s+s1}{\PYZsq{}}\PY{l+s+s1}{alice}\PY{l+s+s1}{\PYZsq{}}\PY{p}{,} \PY{l+s+s1}{\PYZsq{}}\PY{l+s+s1}{age}\PY{l+s+s1}{\PYZsq{}}\PY{p}{]}
\end{Verbatim}


    \begin{Verbatim}[commandchars=\\\{\}]
{\color{incolor}In [{\color{incolor}6}]:} \PY{c+c1}{\PYZsh{} a est un flottant}
        \PY{n+nb}{type}\PY{p}{(}\PY{n}{a}\PY{p}{)}\PY{p}{,} \PY{n}{a}
\end{Verbatim}


\begin{Verbatim}[commandchars=\\\{\}]
{\color{outcolor}Out[{\color{outcolor}6}]:} (numpy.float64, 30.0)
\end{Verbatim}
            
    \begin{Verbatim}[commandchars=\\\{\}]
{\color{incolor}In [{\color{incolor}7}]:} \PY{c+c1}{\PYZsh{} Quel est la moyenne de tous les âges}
        \PY{n}{c} \PY{o}{=} \PY{n}{stat}\PY{o}{.}\PY{n}{loc}\PY{p}{[}\PY{p}{:}\PY{p}{,} \PY{l+s+s1}{\PYZsq{}}\PY{l+s+s1}{age}\PY{l+s+s1}{\PYZsq{}}\PY{p}{]}
        \PY{n}{m} \PY{o}{=} \PY{n}{c}\PY{o}{.}\PY{n}{mean}\PY{p}{(}\PY{p}{)}
        \PY{n+nb}{print}\PY{p}{(}\PY{n}{f}\PY{l+s+s2}{\PYZdq{}}\PY{l+s+s2}{L}\PY{l+s+s2}{\PYZsq{}}\PY{l+s+s2}{âge moyen est de }\PY{l+s+si}{\PYZob{}m:.1f\PYZcb{}}\PY{l+s+s2}{ ans.}\PY{l+s+s2}{\PYZdq{}}\PY{p}{)}
\end{Verbatim}


    \begin{Verbatim}[commandchars=\\\{\}]
L'âge moyen est de 33.3 ans.

    \end{Verbatim}

    \begin{Verbatim}[commandchars=\\\{\}]
{\color{incolor}In [{\color{incolor}8}]:} \PY{c+c1}{\PYZsh{} c est une Series}
        \PY{n+nb}{type}\PY{p}{(}\PY{n}{c}\PY{p}{)}
\end{Verbatim}


\begin{Verbatim}[commandchars=\\\{\}]
{\color{outcolor}Out[{\color{outcolor}8}]:} pandas.core.series.Series
\end{Verbatim}
            
    \begin{Verbatim}[commandchars=\\\{\}]
{\color{incolor}In [{\color{incolor}9}]:} \PY{c+c1}{\PYZsh{} et m est un flottant}
        \PY{n+nb}{type}\PY{p}{(}\PY{n}{m}\PY{p}{)}
\end{Verbatim}


\begin{Verbatim}[commandchars=\\\{\}]
{\color{outcolor}Out[{\color{outcolor}9}]:} numpy.float64
\end{Verbatim}
            
    On peut déjà noter plusieurs choses intéressantes~:

\begin{itemize}
\item
  On peut utiliser \texttt{.loc{[}{]}} et \texttt{.iloc} comme pour les
  \texttt{Series}. Pour les \texttt{DataFrame} c'est encore plus
  important parce qu'il y a plus de risques d'ambiguïtés (notamment
  entre les lignes et les colonnes, on y reviendra)~;
\item
  la méthode \texttt{mean} calcule la moyenne, ça n'est pas surprenant,
  mais ignore les \texttt{NaN}. C'est en général ce que l'on veut. Si
  vous vous demandez comment savoir si la méthode que vous utilisez
  ignore ou pas les \texttt{NaN}, le mieux est de regarder l'aide de
  cette méthode. Il existe pour un certain nombre de méthodes deux
  versions~: une qui ignore les \texttt{NaN} et une autre qui les prend
  en compte~; on en reparlera.
\end{itemize}

    Une autre manière de construire une \texttt{DataFrame} est de partir
d'un \texttt{array} de \texttt{numpy}, et de spécifier les index pour
les lignes et les colonnes avec les arguments \texttt{index} et
\texttt{columns}~:

    \begin{Verbatim}[commandchars=\\\{\}]
{\color{incolor}In [{\color{incolor}10}]:} \PY{n}{a} \PY{o}{=} \PY{n}{np}\PY{o}{.}\PY{n}{random}\PY{o}{.}\PY{n}{randint}\PY{p}{(}\PY{l+m+mi}{1}\PY{p}{,} \PY{l+m+mi}{20}\PY{p}{,} \PY{l+m+mi}{9}\PY{p}{)}\PY{o}{.}\PY{n}{reshape}\PY{p}{(}\PY{l+m+mi}{3}\PY{p}{,} \PY{l+m+mi}{3}\PY{p}{)}
         \PY{n}{p} \PY{o}{=} \PY{n}{pd}\PY{o}{.}\PY{n}{DataFrame}\PY{p}{(}\PY{n}{a}\PY{p}{,} \PY{n}{index}\PY{o}{=}\PY{p}{[}\PY{l+s+s1}{\PYZsq{}}\PY{l+s+s1}{a}\PY{l+s+s1}{\PYZsq{}}\PY{p}{,} \PY{l+s+s1}{\PYZsq{}}\PY{l+s+s1}{b}\PY{l+s+s1}{\PYZsq{}}\PY{p}{,} \PY{l+s+s1}{\PYZsq{}}\PY{l+s+s1}{c}\PY{l+s+s1}{\PYZsq{}}\PY{p}{]}\PY{p}{,} \PY{n}{columns}\PY{o}{=}\PY{p}{[}\PY{l+s+s1}{\PYZsq{}}\PY{l+s+s1}{x}\PY{l+s+s1}{\PYZsq{}}\PY{p}{,} \PY{l+s+s1}{\PYZsq{}}\PY{l+s+s1}{y}\PY{l+s+s1}{\PYZsq{}}\PY{p}{,} \PY{l+s+s1}{\PYZsq{}}\PY{l+s+s1}{z}\PY{l+s+s1}{\PYZsq{}}\PY{p}{]}\PY{p}{)}
         \PY{n+nb}{print}\PY{p}{(}\PY{n}{p}\PY{p}{)}
\end{Verbatim}


    \begin{Verbatim}[commandchars=\\\{\}]
    x   y   z
a   5  13   9
b   4   8  12
c  19  12   1

    \end{Verbatim}

    \hypertarget{importation-et-exportation-de-donnuxe9es}{%
\subsubsection{Importation et exportation de
données}\label{importation-et-exportation-de-donnuxe9es}}

    En pratique, il est très fréquent que les données qu'on manipule soient
stockées dans un fichier ou une base de données. Il existe en
\texttt{pandas} de nombreux utilitaires pour importer et exporter des
données et les convertir automatiquement en \texttt{DataFrame}. Vous
pouvez importer ou exporter du CSV, JSON, HTML, Excel, HDF5, SQL, Python
pickle, etc.

    À titre d'illustration écrivons la \texttt{DataFrame} \texttt{p} dans
différents formats.

    \begin{Verbatim}[commandchars=\\\{\}]
{\color{incolor}In [{\color{incolor}11}]:} \PY{c+c1}{\PYZsh{} écrivons notre DataFrame dans un fichier CSV}
         \PY{n}{p}\PY{o}{.}\PY{n}{to\PYZus{}csv}\PY{p}{(}\PY{l+s+s1}{\PYZsq{}}\PY{l+s+s1}{my\PYZus{}data.csv}\PY{l+s+s1}{\PYZsq{}}\PY{p}{)}
         \PY{o}{!}cat my\PYZus{}data.csv
\end{Verbatim}


    \begin{Verbatim}[commandchars=\\\{\}]
,x,y,z
a,5,13,9
b,4,8,12
c,19,12,1

    \end{Verbatim}

    \begin{Verbatim}[commandchars=\\\{\}]
{\color{incolor}In [{\color{incolor}12}]:} \PY{c+c1}{\PYZsh{} et dans un fichier JSON}
         \PY{n}{p}\PY{o}{.}\PY{n}{to\PYZus{}json}\PY{p}{(}\PY{l+s+s1}{\PYZsq{}}\PY{l+s+s1}{my\PYZus{}data.json}\PY{l+s+s1}{\PYZsq{}}\PY{p}{)}
         \PY{o}{!}cat my\PYZus{}data.json
\end{Verbatim}


    \begin{Verbatim}[commandchars=\\\{\}]
\{"x":\{"a":5,"b":4,"c":19\},"y":\{"a":13,"b":8,"c":12\},"z":\{"a":9,"b":12,"c":1\}\}
    \end{Verbatim}

    \begin{Verbatim}[commandchars=\\\{\}]
{\color{incolor}In [{\color{incolor}13}]:} \PY{c+c1}{\PYZsh{} on peut maintenant recharger notre fichier, la conversion en DataFrame est automatique}
         \PY{n}{new\PYZus{}p} \PY{o}{=} \PY{n}{pd}\PY{o}{.}\PY{n}{read\PYZus{}json}\PY{p}{(}\PY{l+s+s1}{\PYZsq{}}\PY{l+s+s1}{my\PYZus{}data.json}\PY{l+s+s1}{\PYZsq{}}\PY{p}{)}
         \PY{n+nb}{print}\PY{p}{(}\PY{n}{new\PYZus{}p}\PY{p}{)}
\end{Verbatim}


    \begin{Verbatim}[commandchars=\\\{\}]
    x   y   z
a   5  13   9
b   4   8  12
c  19  12   1

    \end{Verbatim}

    Pour la gestion des autres formats, comme il s'agit de quelque chose de
très spécifique et sans difficulté particulière, je vous renvoie
simplement à la documentation~:

http://pandas.pydata.org/pandas-docs/stable/io.html

    \hypertarget{manipulation-dune-dataframe}{%
\subsubsection{\texorpdfstring{Manipulation d'une
\texttt{DataFrame}}{Manipulation d'une DataFrame}}\label{manipulation-dune-dataframe}}

    \begin{Verbatim}[commandchars=\\\{\}]
{\color{incolor}In [{\color{incolor}14}]:} \PY{c+c1}{\PYZsh{} construisons maintenant une DataFrame jouet}
         
         \PY{c+c1}{\PYZsh{} voici une liste de prénoms}
         \PY{n}{names} \PY{o}{=} \PY{p}{[}\PY{l+s+s1}{\PYZsq{}}\PY{l+s+s1}{alice}\PY{l+s+s1}{\PYZsq{}}\PY{p}{,} \PY{l+s+s1}{\PYZsq{}}\PY{l+s+s1}{bob}\PY{l+s+s1}{\PYZsq{}}\PY{p}{,} \PY{l+s+s1}{\PYZsq{}}\PY{l+s+s1}{marc}\PY{l+s+s1}{\PYZsq{}}\PY{p}{,} \PY{l+s+s1}{\PYZsq{}}\PY{l+s+s1}{bill}\PY{l+s+s1}{\PYZsq{}}\PY{p}{,} \PY{l+s+s1}{\PYZsq{}}\PY{l+s+s1}{sonia}\PY{l+s+s1}{\PYZsq{}}\PY{p}{]}
         
         \PY{c+c1}{\PYZsh{} créons trois Series qui formeront les trois colonnes}
         \PY{n}{age} \PY{o}{=} \PY{n}{pd}\PY{o}{.}\PY{n}{Series}\PY{p}{(}\PY{p}{[}\PY{l+m+mi}{12}\PY{p}{,} \PY{l+m+mi}{13}\PY{p}{,} \PY{l+m+mi}{16}\PY{p}{,} \PY{l+m+mi}{11}\PY{p}{,} \PY{l+m+mi}{16}\PY{p}{]}\PY{p}{,} \PY{n}{index}\PY{o}{=}\PY{n}{names}\PY{p}{)}
         \PY{n}{height} \PY{o}{=} \PY{n}{pd}\PY{o}{.}\PY{n}{Series}\PY{p}{(}\PY{p}{[}\PY{l+m+mi}{130}\PY{p}{,} \PY{l+m+mi}{140}\PY{p}{,} \PY{l+m+mi}{176}\PY{p}{,} \PY{l+m+mi}{120}\PY{p}{,} \PY{l+m+mi}{165}\PY{p}{]}\PY{p}{,} \PY{n}{index}\PY{o}{=}\PY{n}{names}\PY{p}{)}
         \PY{n}{sex} \PY{o}{=} \PY{n}{pd}\PY{o}{.}\PY{n}{Series}\PY{p}{(}\PY{n+nb}{list}\PY{p}{(}\PY{l+s+s1}{\PYZsq{}}\PY{l+s+s1}{fmmmf}\PY{l+s+s1}{\PYZsq{}}\PY{p}{)}\PY{p}{,} \PY{n}{index}\PY{o}{=}\PY{n}{names}\PY{p}{)}
         
         \PY{c+c1}{\PYZsh{} créons maintenant la DataFrame}
         \PY{n}{p} \PY{o}{=} \PY{n}{pd}\PY{o}{.}\PY{n}{DataFrame}\PY{p}{(}\PY{p}{\PYZob{}}\PY{l+s+s1}{\PYZsq{}}\PY{l+s+s1}{age}\PY{l+s+s1}{\PYZsq{}}\PY{p}{:} \PY{n}{age}\PY{p}{,} \PY{l+s+s1}{\PYZsq{}}\PY{l+s+s1}{height}\PY{l+s+s1}{\PYZsq{}}\PY{p}{:} \PY{n}{height}\PY{p}{,} \PY{l+s+s1}{\PYZsq{}}\PY{l+s+s1}{sex}\PY{l+s+s1}{\PYZsq{}}\PY{p}{:} \PY{n}{sex}\PY{p}{\PYZcb{}}\PY{p}{)}
         \PY{n+nb}{print}\PY{p}{(}\PY{n}{p}\PY{p}{)}
\end{Verbatim}


    \begin{Verbatim}[commandchars=\\\{\}]
       age  height sex
alice   12     130   f
bob     13     140   m
marc    16     176   m
bill    11     120   m
sonia   16     165   f

    \end{Verbatim}

    \begin{Verbatim}[commandchars=\\\{\}]
{\color{incolor}In [{\color{incolor}15}]:} \PY{c+c1}{\PYZsh{} et chargeons le jeux de données sur les pourboires de seaborn}
         \PY{k+kn}{import} \PY{n+nn}{seaborn} \PY{k}{as} \PY{n+nn}{sns}
         \PY{n}{tips} \PY{o}{=} \PY{n}{sns}\PY{o}{.}\PY{n}{load\PYZus{}dataset}\PY{p}{(}\PY{l+s+s1}{\PYZsq{}}\PY{l+s+s1}{tips}\PY{l+s+s1}{\PYZsq{}}\PY{p}{)}
\end{Verbatim}


    \texttt{pandas} offre de nombreuses possibilités d'explorer les données.
Attention, dans mes exemples je vais alterner entre le
\texttt{DataFrame} \texttt{p} et le \texttt{DataFrame} \texttt{tips}
suivant les besoins de l'explication.

    \begin{Verbatim}[commandchars=\\\{\}]
{\color{incolor}In [{\color{incolor}16}]:} \PY{c+c1}{\PYZsh{} afficher les premières lignes}
         \PY{n}{tips}\PY{o}{.}\PY{n}{head}\PY{p}{(}\PY{p}{)}
\end{Verbatim}


\begin{Verbatim}[commandchars=\\\{\}]
{\color{outcolor}Out[{\color{outcolor}16}]:}    total\_bill   tip     sex smoker  day    time  size
         0       16.99  1.01  Female     No  Sun  Dinner     2
         1       10.34  1.66    Male     No  Sun  Dinner     3
         2       21.01  3.50    Male     No  Sun  Dinner     3
         3       23.68  3.31    Male     No  Sun  Dinner     2
         4       24.59  3.61  Female     No  Sun  Dinner     4
\end{Verbatim}
            
    \begin{Verbatim}[commandchars=\\\{\}]
{\color{incolor}In [{\color{incolor}17}]:} \PY{c+c1}{\PYZsh{} et les dernière lignes}
         \PY{n}{tips}\PY{o}{.}\PY{n}{tail}\PY{p}{(}\PY{p}{)}
\end{Verbatim}


\begin{Verbatim}[commandchars=\\\{\}]
{\color{outcolor}Out[{\color{outcolor}17}]:}      total\_bill   tip     sex smoker   day    time  size
         239       29.03  5.92    Male     No   Sat  Dinner     3
         240       27.18  2.00  Female    Yes   Sat  Dinner     2
         241       22.67  2.00    Male    Yes   Sat  Dinner     2
         242       17.82  1.75    Male     No   Sat  Dinner     2
         243       18.78  3.00  Female     No  Thur  Dinner     2
\end{Verbatim}
            
    \begin{Verbatim}[commandchars=\\\{\}]
{\color{incolor}In [{\color{incolor}18}]:} \PY{c+c1}{\PYZsh{} l\PYZsq{}index des lignes}
         \PY{n}{p}\PY{o}{.}\PY{n}{index}
\end{Verbatim}


\begin{Verbatim}[commandchars=\\\{\}]
{\color{outcolor}Out[{\color{outcolor}18}]:} Index(['alice', 'bob', 'marc', 'bill', 'sonia'], dtype='object')
\end{Verbatim}
            
    \begin{Verbatim}[commandchars=\\\{\}]
{\color{incolor}In [{\color{incolor}19}]:} \PY{c+c1}{\PYZsh{} et l\PYZsq{}index des colonnes}
         \PY{n}{p}\PY{o}{.}\PY{n}{columns}
\end{Verbatim}


\begin{Verbatim}[commandchars=\\\{\}]
{\color{outcolor}Out[{\color{outcolor}19}]:} Index(['age', 'height', 'sex'], dtype='object')
\end{Verbatim}
            
    \begin{Verbatim}[commandchars=\\\{\}]
{\color{incolor}In [{\color{incolor}20}]:} \PY{c+c1}{\PYZsh{} et afficher uniquement les valeurs}
         \PY{n}{p}\PY{o}{.}\PY{n}{values}
\end{Verbatim}


\begin{Verbatim}[commandchars=\\\{\}]
{\color{outcolor}Out[{\color{outcolor}20}]:} array([[12, 130, 'f'],
                [13, 140, 'm'],
                [16, 176, 'm'],
                [11, 120, 'm'],
                [16, 165, 'f']], dtype=object)
\end{Verbatim}
            
    \begin{Verbatim}[commandchars=\\\{\}]
{\color{incolor}In [{\color{incolor}21}]:} \PY{c+c1}{\PYZsh{} échanger lignes et colonnes}
         \PY{c+c1}{\PYZsh{} cf. la transposition de matrices}
         \PY{n}{p}\PY{o}{.}\PY{n}{T}
\end{Verbatim}


\begin{Verbatim}[commandchars=\\\{\}]
{\color{outcolor}Out[{\color{outcolor}21}]:}        alice  bob marc bill sonia
         age       12   13   16   11    16
         height   130  140  176  120   165
         sex        f    m    m    m     f
\end{Verbatim}
            
    Pour finir, il y a la méthodes \texttt{describe} qui permet d'obtenir
des premières statistiques sur un \texttt{DataFrame}. \texttt{describe}
permet de calculer des statistiques sur des type numériques, mais aussi
sur des types chaînes de caractères.

    \begin{Verbatim}[commandchars=\\\{\}]
{\color{incolor}In [{\color{incolor}22}]:} \PY{c+c1}{\PYZsh{} par défaut describe ne prend en compte que les colonnes numériques}
         \PY{n}{p}\PY{o}{.}\PY{n}{describe}\PY{p}{(}\PY{p}{)}
\end{Verbatim}


\begin{Verbatim}[commandchars=\\\{\}]
{\color{outcolor}Out[{\color{outcolor}22}]:}              age      height
         count   5.000000    5.000000
         mean   13.600000  146.200000
         std     2.302173   23.605084
         min    11.000000  120.000000
         25\%    12.000000  130.000000
         50\%    13.000000  140.000000
         75\%    16.000000  165.000000
         max    16.000000  176.000000
\end{Verbatim}
            
    \begin{Verbatim}[commandchars=\\\{\}]
{\color{incolor}In [{\color{incolor}23}]:} \PY{c+c1}{\PYZsh{} mais on peut le forcer à prendre en compte toutes les colonnes}
         \PY{n}{p}\PY{o}{.}\PY{n}{describe}\PY{p}{(}\PY{n}{include}\PY{o}{=}\PY{l+s+s1}{\PYZsq{}}\PY{l+s+s1}{all}\PY{l+s+s1}{\PYZsq{}}\PY{p}{)}
\end{Verbatim}


\begin{Verbatim}[commandchars=\\\{\}]
{\color{outcolor}Out[{\color{outcolor}23}]:}               age      height  sex
         count    5.000000    5.000000    5
         unique        NaN         NaN    2
         top           NaN         NaN    m
         freq          NaN         NaN    3
         mean    13.600000  146.200000  NaN
         std      2.302173   23.605084  NaN
         min     11.000000  120.000000  NaN
         25\%     12.000000  130.000000  NaN
         50\%     13.000000  140.000000  NaN
         75\%     16.000000  165.000000  NaN
         max     16.000000  176.000000  NaN
\end{Verbatim}
            
    \hypertarget{requuxeates-sur-une-dataframe}{%
\subsubsection{\texorpdfstring{Requêtes sur une
\texttt{DataFrame}}{Requêtes sur une DataFrame}}\label{requuxeates-sur-une-dataframe}}

    On peut maintenant commencer à faire des requêtes sur les
\texttt{DataFrames}. Les \texttt{DataFrame} supportent la notion de
masque que l'on a vue pour les \texttt{ndarray} de \texttt{numpy} et
pour les \texttt{Series}.

    \begin{Verbatim}[commandchars=\\\{\}]
{\color{incolor}In [{\color{incolor}24}]:} \PY{c+c1}{\PYZsh{} p.loc prend soit un label de ligne}
         \PY{n+nb}{print}\PY{p}{(}\PY{n}{p}\PY{o}{.}\PY{n}{loc}\PY{p}{[}\PY{l+s+s1}{\PYZsq{}}\PY{l+s+s1}{sonia}\PY{l+s+s1}{\PYZsq{}}\PY{p}{]}\PY{p}{)}
\end{Verbatim}


    \begin{Verbatim}[commandchars=\\\{\}]
age        16
height    165
sex         f
Name: sonia, dtype: object

    \end{Verbatim}

    \begin{Verbatim}[commandchars=\\\{\}]
{\color{incolor}In [{\color{incolor}25}]:} \PY{c+c1}{\PYZsh{} ou alors un label de ligne ET de colonne}
         \PY{n+nb}{print}\PY{p}{(}\PY{n}{p}\PY{o}{.}\PY{n}{loc}\PY{p}{[}\PY{l+s+s1}{\PYZsq{}}\PY{l+s+s1}{sonia}\PY{l+s+s1}{\PYZsq{}}\PY{p}{,} \PY{l+s+s1}{\PYZsq{}}\PY{l+s+s1}{age}\PY{l+s+s1}{\PYZsq{}}\PY{p}{]}\PY{p}{)}
\end{Verbatim}


    \begin{Verbatim}[commandchars=\\\{\}]
16

    \end{Verbatim}

    On peut mettre à la place d'une label~:

\begin{itemize}
\tightlist
\item
  une liste de labels~;
\item
  un slice sur les labels~;
\item
  un masque (c'est-à-dire un tableau de booléens)~;
\item
  un callable qui retourne une des trois premières possibilités.
\end{itemize}

Noter que l'on peut également utiliser la notation \texttt{.iloc{[}{]}}
avec les mêmes règles, mais elle est moins utile.

Je recommande de toujours utiliser la notation
\texttt{.loc{[}lignes,\ colonnes{]}} pour éviter toute ambiguïté. Nous
verrons que les notations \texttt{.loc{[}lignes{]}} ou pire seulement
\texttt{{[}label{]}} sont sources d'erreurs.

Regardons maintenant d'autres exemples plus sophistiqués~:

    \begin{Verbatim}[commandchars=\\\{\}]
{\color{incolor}In [{\color{incolor}26}]:} \PY{c+c1}{\PYZsh{} un masque sur les femmes}
         \PY{n}{p}\PY{o}{.}\PY{n}{loc}\PY{p}{[}\PY{p}{:}\PY{p}{,} \PY{l+s+s1}{\PYZsq{}}\PY{l+s+s1}{sex}\PY{l+s+s1}{\PYZsq{}}\PY{p}{]} \PY{o}{==} \PY{l+s+s1}{\PYZsq{}}\PY{l+s+s1}{f}\PY{l+s+s1}{\PYZsq{}}
\end{Verbatim}


\begin{Verbatim}[commandchars=\\\{\}]
{\color{outcolor}Out[{\color{outcolor}26}]:} alice     True
         bob      False
         marc     False
         bill     False
         sonia     True
         Name: sex, dtype: bool
\end{Verbatim}
            
    \begin{Verbatim}[commandchars=\\\{\}]
{\color{incolor}In [{\color{incolor}27}]:} \PY{c+c1}{\PYZsh{} si bien que pour construire un tableau}
         \PY{c+c1}{\PYZsh{} avec uniquement les femmes}
         \PY{n}{p}\PY{o}{.}\PY{n}{loc}\PY{p}{[}\PY{n}{p}\PY{o}{.}\PY{n}{loc}\PY{p}{[}\PY{p}{:}\PY{p}{,} \PY{l+s+s1}{\PYZsq{}}\PY{l+s+s1}{sex}\PY{l+s+s1}{\PYZsq{}}\PY{p}{]} \PY{o}{==} \PY{l+s+s1}{\PYZsq{}}\PY{l+s+s1}{f}\PY{l+s+s1}{\PYZsq{}}\PY{p}{,} \PY{p}{:}\PY{p}{]}
\end{Verbatim}


\begin{Verbatim}[commandchars=\\\{\}]
{\color{outcolor}Out[{\color{outcolor}27}]:}        age  height sex
         alice   12     130   f
         sonia   16     165   f
\end{Verbatim}
            
    \begin{Verbatim}[commandchars=\\\{\}]
{\color{incolor}In [{\color{incolor}28}]:} \PY{c+c1}{\PYZsh{} si on veut ne garder uniquement}
         \PY{c+c1}{\PYZsh{} que les femmes de plus de 14 ans}
         \PY{n}{p}\PY{o}{.}\PY{n}{loc}\PY{p}{[}\PY{p}{(}\PY{n}{p}\PY{o}{.}\PY{n}{loc}\PY{p}{[}\PY{p}{:}\PY{p}{,} \PY{l+s+s1}{\PYZsq{}}\PY{l+s+s1}{sex}\PY{l+s+s1}{\PYZsq{}}\PY{p}{]} \PY{o}{==} \PY{l+s+s1}{\PYZsq{}}\PY{l+s+s1}{f}\PY{l+s+s1}{\PYZsq{}}\PY{p}{)} \PY{o}{\PYZam{}} \PY{p}{(}\PY{n}{p}\PY{o}{.}\PY{n}{loc}\PY{p}{[}\PY{p}{:}\PY{p}{,} \PY{l+s+s1}{\PYZsq{}}\PY{l+s+s1}{age}\PY{l+s+s1}{\PYZsq{}}\PY{p}{]} \PY{o}{\PYZgt{}} \PY{l+m+mi}{14}\PY{p}{)}\PY{p}{,} \PY{p}{:}\PY{p}{]}
\end{Verbatim}


\begin{Verbatim}[commandchars=\\\{\}]
{\color{outcolor}Out[{\color{outcolor}28}]:}        age  height sex
         sonia   16     165   f
\end{Verbatim}
            
    \begin{Verbatim}[commandchars=\\\{\}]
{\color{incolor}In [{\color{incolor}29}]:} \PY{c+c1}{\PYZsh{} quelle est la moyenne de \PYZsq{}total\PYZus{}bill\PYZsq{} pour les femmes}
         \PY{n}{addition\PYZus{}f} \PY{o}{=} \PY{n}{tips}\PY{o}{.}\PY{n}{loc}\PY{p}{[}\PY{n}{tips}\PY{o}{.}\PY{n}{loc}\PY{p}{[}\PY{p}{:}\PY{p}{,} \PY{l+s+s1}{\PYZsq{}}\PY{l+s+s1}{sex}\PY{l+s+s1}{\PYZsq{}}\PY{p}{]} \PY{o}{==} \PY{l+s+s1}{\PYZsq{}}\PY{l+s+s1}{Female}\PY{l+s+s1}{\PYZsq{}}\PY{p}{,} \PY{l+s+s1}{\PYZsq{}}\PY{l+s+s1}{total\PYZus{}bill}\PY{l+s+s1}{\PYZsq{}}\PY{p}{]}\PY{o}{.}\PY{n}{mean}\PY{p}{(}\PY{p}{)}
         \PY{n+nb}{print}\PY{p}{(}\PY{n}{f}\PY{l+s+s2}{\PYZdq{}}\PY{l+s+s2}{addition moyenne des femmes : }\PY{l+s+si}{\PYZob{}addition\PYZus{}f:.2f\PYZcb{}}\PY{l+s+s2}{\PYZdq{}}\PY{p}{)}
\end{Verbatim}


    \begin{Verbatim}[commandchars=\\\{\}]
addition moyenne des femmes : 18.06

    \end{Verbatim}

    \begin{Verbatim}[commandchars=\\\{\}]
{\color{incolor}In [{\color{incolor}30}]:} \PY{c+c1}{\PYZsh{} quelle est la note moyenne des hommes}
         \PY{n}{addition\PYZus{}h} \PY{o}{=} \PY{n}{tips}\PY{o}{.}\PY{n}{loc}\PY{p}{[}\PY{n}{tips}\PY{o}{.}\PY{n}{loc}\PY{p}{[}\PY{p}{:}\PY{p}{,} \PY{l+s+s1}{\PYZsq{}}\PY{l+s+s1}{sex}\PY{l+s+s1}{\PYZsq{}}\PY{p}{]} \PY{o}{==} \PY{l+s+s1}{\PYZsq{}}\PY{l+s+s1}{Male}\PY{l+s+s1}{\PYZsq{}}\PY{p}{,} \PY{l+s+s1}{\PYZsq{}}\PY{l+s+s1}{total\PYZus{}bill}\PY{l+s+s1}{\PYZsq{}}\PY{p}{]}\PY{o}{.}\PY{n}{mean}\PY{p}{(}\PY{p}{)}
         \PY{n+nb}{print}\PY{p}{(}\PY{n}{f}\PY{l+s+s2}{\PYZdq{}}\PY{l+s+s2}{addition moyenne des hommes : }\PY{l+s+si}{\PYZob{}addition\PYZus{}h:.2f\PYZcb{}}\PY{l+s+s2}{\PYZdq{}}\PY{p}{)}
\end{Verbatim}


    \begin{Verbatim}[commandchars=\\\{\}]
addition moyenne des hommes : 20.74

    \end{Verbatim}

    \begin{Verbatim}[commandchars=\\\{\}]
{\color{incolor}In [{\color{incolor}31}]:} \PY{c+c1}{\PYZsh{} qui laisse le plus grand pourcentage de pourboire :}
         \PY{c+c1}{\PYZsh{} les hommes ou les femmes ?}
         
         \PY{n}{pourboire\PYZus{}f} \PY{o}{=} \PY{n}{tips}\PY{o}{.}\PY{n}{loc}\PY{p}{[}\PY{n}{tips}\PY{o}{.}\PY{n}{loc}\PY{p}{[}\PY{p}{:}\PY{p}{,} \PY{l+s+s1}{\PYZsq{}}\PY{l+s+s1}{sex}\PY{l+s+s1}{\PYZsq{}}\PY{p}{]} \PY{o}{==} \PY{l+s+s1}{\PYZsq{}}\PY{l+s+s1}{Female}\PY{l+s+s1}{\PYZsq{}}\PY{p}{,} \PY{l+s+s1}{\PYZsq{}}\PY{l+s+s1}{tip}\PY{l+s+s1}{\PYZsq{}}\PY{p}{]}\PY{o}{.}\PY{n}{mean}\PY{p}{(}\PY{p}{)}
         \PY{n}{pourboire\PYZus{}h} \PY{o}{=} \PY{n}{tips}\PY{o}{.}\PY{n}{loc}\PY{p}{[}\PY{n}{tips}\PY{o}{.}\PY{n}{loc}\PY{p}{[}\PY{p}{:}\PY{p}{,} \PY{l+s+s1}{\PYZsq{}}\PY{l+s+s1}{sex}\PY{l+s+s1}{\PYZsq{}}\PY{p}{]} \PY{o}{==} \PY{l+s+s1}{\PYZsq{}}\PY{l+s+s1}{Male}\PY{l+s+s1}{\PYZsq{}}\PY{p}{,} \PY{l+s+s1}{\PYZsq{}}\PY{l+s+s1}{tip}\PY{l+s+s1}{\PYZsq{}}\PY{p}{]}\PY{o}{.}\PY{n}{mean}\PY{p}{(}\PY{p}{)}
         
         \PY{n+nb}{print}\PY{p}{(}\PY{n}{f}\PY{l+s+s2}{\PYZdq{}}\PY{l+s+s2}{Les femmes laissent }\PY{l+s+s2}{\PYZob{}}\PY{l+s+s2}{pourboire\PYZus{}f/addition\PYZus{}f:.2}\PY{l+s+s2}{\PYZpc{}}\PY{l+s+s2}{\PYZcb{} de pourboire}\PY{l+s+s2}{\PYZdq{}}\PY{p}{)}
         \PY{n+nb}{print}\PY{p}{(}\PY{n}{f}\PY{l+s+s2}{\PYZdq{}}\PY{l+s+s2}{Les hommes laissent }\PY{l+s+s2}{\PYZob{}}\PY{l+s+s2}{pourboire\PYZus{}h/addition\PYZus{}h:.2}\PY{l+s+s2}{\PYZpc{}}\PY{l+s+s2}{\PYZcb{} de pourboire}\PY{l+s+s2}{\PYZdq{}}\PY{p}{)}
\end{Verbatim}


    \begin{Verbatim}[commandchars=\\\{\}]
Les femmes laissent 15.69\% de pourboire
Les hommes laissent 14.89\% de pourboire

    \end{Verbatim}

    \hypertarget{erreurs-fruxe9quentes-et-ambiguuxeftuxe9s-sur-les-requuxeates}{%
\subsubsection{Erreurs fréquentes et ambiguïtés sur les
requêtes}\label{erreurs-fruxe9quentes-et-ambiguuxeftuxe9s-sur-les-requuxeates}}

    Nous avons vu une manière simple et non ambiguë de faire des requêtes
sur les \texttt{DataFrame}. Nous allons voir qu'il existe d'autres
manières qui ont pour seul avantage d'être plus concises, mais sources
de nombreuses erreurs.

\textbf{Souvenez-vous, utilisez toujours la notation
\texttt{.loc{[}lignes,\ colonnes{]}} sinon, soyez sûr de savoir ce qui
est réellement calculé}.

    \begin{Verbatim}[commandchars=\\\{\}]
{\color{incolor}In [{\color{incolor}32}]:} \PY{c+c1}{\PYZsh{} commençons par la notation la plus classique}
         \PY{n}{p}\PY{p}{[}\PY{l+s+s1}{\PYZsq{}}\PY{l+s+s1}{sex}\PY{l+s+s1}{\PYZsq{}}\PY{p}{]}  \PY{c+c1}{\PYZsh{} prend forcément un label de colonne}
\end{Verbatim}


\begin{Verbatim}[commandchars=\\\{\}]
{\color{outcolor}Out[{\color{outcolor}32}]:} alice    f
         bob      m
         marc     m
         bill     m
         sonia    f
         Name: sex, dtype: object
\end{Verbatim}
            
    \begin{Verbatim}[commandchars=\\\{\}]
{\color{incolor}In [{\color{incolor}33}]:} \PY{c+c1}{\PYZsh{} mais par contre, si on passe un slice, c\PYZsq{}est forcément des lignes,}
         \PY{c+c1}{\PYZsh{} assez perturbant et source de confusion.}
         \PY{n}{p}\PY{p}{[}\PY{l+s+s1}{\PYZsq{}}\PY{l+s+s1}{alice}\PY{l+s+s1}{\PYZsq{}}\PY{p}{:} \PY{l+s+s1}{\PYZsq{}}\PY{l+s+s1}{marc}\PY{l+s+s1}{\PYZsq{}}\PY{p}{]}
\end{Verbatim}


\begin{Verbatim}[commandchars=\\\{\}]
{\color{outcolor}Out[{\color{outcolor}33}]:}        age  height sex
         alice   12     130   f
         bob     13     140   m
         marc    16     176   m
\end{Verbatim}
            
    \begin{Verbatim}[commandchars=\\\{\}]
{\color{incolor}In [{\color{incolor}34}]:} \PY{c+c1}{\PYZsh{} on peut même directement accéder à une colonne par son nom}
         \PY{n}{p}\PY{o}{.}\PY{n}{age}
\end{Verbatim}


\begin{Verbatim}[commandchars=\\\{\}]
{\color{outcolor}Out[{\color{outcolor}34}]:} alice    12
         bob      13
         marc     16
         bill     11
         sonia    16
         Name: age, dtype: int64
\end{Verbatim}
            
    Mais c'est \textbf{fortement déconseillé} parce que si un attribut de
même nom existe sur une \texttt{DataFrame}, alors la priorité est donnée
à l'attribut, et non à la colonne~:

    \begin{Verbatim}[commandchars=\\\{\}]
{\color{incolor}In [{\color{incolor}35}]:} \PY{c+c1}{\PYZsh{} ajoutons une colonne qui a pour nom une méthode qui existe sur}
         \PY{c+c1}{\PYZsh{} les DataFrame}
         \PY{n}{p}\PY{p}{[}\PY{l+s+s1}{\PYZsq{}}\PY{l+s+s1}{mean}\PY{l+s+s1}{\PYZsq{}}\PY{p}{]} \PY{o}{=} \PY{l+m+mi}{1}
         \PY{n+nb}{print}\PY{p}{(}\PY{n}{p}\PY{p}{)}
\end{Verbatim}


    \begin{Verbatim}[commandchars=\\\{\}]
       age  height sex  mean
alice   12     130   f     1
bob     13     140   m     1
marc    16     176   m     1
bill    11     120   m     1
sonia   16     165   f     1

    \end{Verbatim}

    \begin{Verbatim}[commandchars=\\\{\}]
{\color{incolor}In [{\color{incolor}36}]:} \PY{c+c1}{\PYZsh{} je peux bien accéder}
         \PY{c+c1}{\PYZsh{} à la colonne sex}
         \PY{n}{p}\PY{o}{.}\PY{n}{sex}
\end{Verbatim}


\begin{Verbatim}[commandchars=\\\{\}]
{\color{outcolor}Out[{\color{outcolor}36}]:} alice    f
         bob      m
         marc     m
         bill     m
         sonia    f
         Name: sex, dtype: object
\end{Verbatim}
            
    \begin{Verbatim}[commandchars=\\\{\}]
{\color{incolor}In [{\color{incolor}37}]:} \PY{c+c1}{\PYZsh{} mais pas à la colonne mean}
         \PY{n}{p}\PY{o}{.}\PY{n}{mean}
\end{Verbatim}


\begin{Verbatim}[commandchars=\\\{\}]
{\color{outcolor}Out[{\color{outcolor}37}]:} <bound method DataFrame.mean of        age  height sex  mean
         alice   12     130   f     1
         bob     13     140   m     1
         marc    16     176   m     1
         bill    11     120   m     1
         sonia   16     165   f     1>
\end{Verbatim}
            
    \begin{Verbatim}[commandchars=\\\{\}]
{\color{incolor}In [{\color{incolor}38}]:} \PY{c+c1}{\PYZsh{} à nouveau, la seule méthode non ambiguë est d\PYZsq{}utiliser .loc}
         \PY{n}{p}\PY{o}{.}\PY{n}{loc}\PY{p}{[}\PY{p}{:}\PY{p}{,} \PY{l+s+s1}{\PYZsq{}}\PY{l+s+s1}{mean}\PY{l+s+s1}{\PYZsq{}}\PY{p}{]}
\end{Verbatim}


\begin{Verbatim}[commandchars=\\\{\}]
{\color{outcolor}Out[{\color{outcolor}38}]:} alice    1
         bob      1
         marc     1
         bill     1
         sonia    1
         Name: mean, dtype: int64
\end{Verbatim}
            
    \begin{Verbatim}[commandchars=\\\{\}]
{\color{incolor}In [{\color{incolor}39}]:} \PY{c+c1}{\PYZsh{} supprimons maintenant la colonne mean *en place* (par défaut,}
         \PY{c+c1}{\PYZsh{} drop retourne une nouvelle DataFrame)}
         \PY{n}{p}\PY{o}{.}\PY{n}{drop}\PY{p}{(}\PY{n}{columns}\PY{o}{=}\PY{l+s+s1}{\PYZsq{}}\PY{l+s+s1}{mean}\PY{l+s+s1}{\PYZsq{}}\PY{p}{,} \PY{n}{inplace}\PY{o}{=}\PY{k+kc}{True}\PY{p}{)}
         \PY{n+nb}{print}\PY{p}{(}\PY{n}{p}\PY{p}{)}
\end{Verbatim}


    \begin{Verbatim}[commandchars=\\\{\}]
       age  height sex
alice   12     130   f
bob     13     140   m
marc    16     176   m
bill    11     120   m
sonia   16     165   f

    \end{Verbatim}

    Pour aller plus loin, vous pouvez lire la documentation officielle~:

http://pandas.pydata.org/pandas-docs/stable/indexing.html

    \hypertarget{universal-functions-et-pandas}{%
\subsubsection{\texorpdfstring{\emph{Universal functions} et
\texttt{pandas}}{Universal functions et pandas}}\label{universal-functions-et-pandas}}

    Ça n'est pas une surprise, les \texttt{Series} et \texttt{DataFrame} de
\texttt{pandas} supportent les \texttt{ufunc} de \texttt{numpy}. Mais il
y a une subtilité. Il est parfaitement légitime et correct d'appliquer
une \texttt{ufunc} de \texttt{numpy} sur les éléments d'une
\texttt{DataFrame}~:

    \begin{Verbatim}[commandchars=\\\{\}]
{\color{incolor}In [{\color{incolor}40}]:} \PY{n}{d} \PY{o}{=} \PY{n}{pd}\PY{o}{.}\PY{n}{DataFrame}\PY{p}{(}\PY{n}{np}\PY{o}{.}\PY{n}{random}\PY{o}{.}\PY{n}{randint}\PY{p}{(}
             \PY{l+m+mi}{1}\PY{p}{,} \PY{l+m+mi}{10}\PY{p}{,} \PY{l+m+mi}{9}\PY{p}{)}\PY{o}{.}\PY{n}{reshape}\PY{p}{(}\PY{l+m+mi}{3}\PY{p}{,} \PY{l+m+mi}{3}\PY{p}{)}\PY{p}{,} \PY{n}{columns}\PY{o}{=}\PY{n+nb}{list}\PY{p}{(}\PY{l+s+s1}{\PYZsq{}}\PY{l+s+s1}{abc}\PY{l+s+s1}{\PYZsq{}}\PY{p}{)}\PY{p}{)}
         \PY{n+nb}{print}\PY{p}{(}\PY{n}{d}\PY{p}{)}
\end{Verbatim}


    \begin{Verbatim}[commandchars=\\\{\}]
   a  b  c
0  2  1  3
1  4  5  8
2  4  4  7

    \end{Verbatim}

    \begin{Verbatim}[commandchars=\\\{\}]
{\color{incolor}In [{\color{incolor}41}]:} \PY{n}{np}\PY{o}{.}\PY{n}{log}\PY{p}{(}\PY{n}{d}\PY{p}{)}
\end{Verbatim}


\begin{Verbatim}[commandchars=\\\{\}]
{\color{outcolor}Out[{\color{outcolor}41}]:}           a         b         c
         0  0.693147  0.000000  1.098612
         1  1.386294  1.609438  2.079442
         2  1.386294  1.386294  1.945910
\end{Verbatim}
            
    Nous remarquons que comme on s'y attend, la \texttt{ufunc} a été
appliquée à chaque élément de la \texttt{DataFrame} et que les labels
des lignes et colonnes ont été préservés.

    Par contre, si l'on a besoin d'alignement de labels, c'est le cas avec
toutes les opérations qui s'appliquent sur deux objets comme une
addition, alors les \texttt{ufunc} de \texttt{numpy} ne \textbf{vont pas
faire} ce à quoi on s'attend. Elles vont faire les opérations sur les
tableaux \texttt{numpy} sans prendre en compte les labels.

Pour avoir un alignement des labels, il faut utiliser les \texttt{ufunc}
de \texttt{pandas}.

    \begin{Verbatim}[commandchars=\\\{\}]
{\color{incolor}In [{\color{incolor}42}]:} \PY{c+c1}{\PYZsh{} prenons deux Series}
         \PY{n}{s1} \PY{o}{=} \PY{n}{pd}\PY{o}{.}\PY{n}{Series}\PY{p}{(}\PY{p}{[}\PY{l+m+mi}{10}\PY{p}{,} \PY{l+m+mi}{20}\PY{p}{,} \PY{l+m+mi}{30}\PY{p}{]}\PY{p}{,}
                        \PY{n}{index}\PY{o}{=}\PY{n+nb}{list}\PY{p}{(}\PY{l+s+s1}{\PYZsq{}}\PY{l+s+s1}{abc}\PY{l+s+s1}{\PYZsq{}}\PY{p}{)}\PY{p}{)}
         \PY{n+nb}{print}\PY{p}{(}\PY{n}{s1}\PY{p}{)}
\end{Verbatim}


    \begin{Verbatim}[commandchars=\\\{\}]
a    10
b    20
c    30
dtype: int64

    \end{Verbatim}

    \begin{Verbatim}[commandchars=\\\{\}]
{\color{incolor}In [{\color{incolor}43}]:} \PY{c+c1}{\PYZsh{}}
         \PY{n}{s2} \PY{o}{=} \PY{n}{pd}\PY{o}{.}\PY{n}{Series}\PY{p}{(}\PY{p}{[}\PY{l+m+mi}{12}\PY{p}{,} \PY{l+m+mi}{22}\PY{p}{,} \PY{l+m+mi}{32}\PY{p}{]}\PY{p}{,}
                        \PY{n}{index}\PY{o}{=}\PY{n+nb}{list}\PY{p}{(}\PY{l+s+s1}{\PYZsq{}}\PY{l+s+s1}{acd}\PY{l+s+s1}{\PYZsq{}}\PY{p}{)}\PY{p}{)}
         \PY{n+nb}{print}\PY{p}{(}\PY{n}{s2}\PY{p}{)}
\end{Verbatim}


    \begin{Verbatim}[commandchars=\\\{\}]
a    12
c    22
d    32
dtype: int64

    \end{Verbatim}

    \begin{Verbatim}[commandchars=\\\{\}]
{\color{incolor}In [{\color{incolor}44}]:} \PY{c+c1}{\PYZsh{} la ufunc numpy fait la somme}
         \PY{c+c1}{\PYZsh{} des arrays sans prendre en compte}
         \PY{c+c1}{\PYZsh{} les labels, donc sans alignement}
         \PY{n}{np}\PY{o}{.}\PY{n}{add}\PY{p}{(}\PY{n}{s1}\PY{p}{,} \PY{n}{s2}\PY{p}{)}
\end{Verbatim}


\begin{Verbatim}[commandchars=\\\{\}]
{\color{outcolor}Out[{\color{outcolor}44}]:} a    22
         b    42
         c    62
         dtype: int64
\end{Verbatim}
            
    \begin{Verbatim}[commandchars=\\\{\}]
{\color{incolor}In [{\color{incolor}45}]:} \PY{c+c1}{\PYZsh{} la ufunc pandas va faire}
         \PY{c+c1}{\PYZsh{} un alignement des labels}
         \PY{c+c1}{\PYZsh{} cet appel est équivalent à s1 + s2}
         \PY{n}{s1}\PY{o}{.}\PY{n}{add}\PY{p}{(}\PY{n}{s2}\PY{p}{)}
\end{Verbatim}


\begin{Verbatim}[commandchars=\\\{\}]
{\color{outcolor}Out[{\color{outcolor}45}]:} a    22.0
         b     NaN
         c    52.0
         d     NaN
         dtype: float64
\end{Verbatim}
            
    \begin{Verbatim}[commandchars=\\\{\}]
{\color{incolor}In [{\color{incolor}46}]:} \PY{c+c1}{\PYZsh{} comme on l\PYZsq{}a vu sur le complément précédent, les valeurs absentes sont}
         \PY{c+c1}{\PYZsh{} remplacées par NaN, mais on peut changer ce comportement lors de}
         \PY{c+c1}{\PYZsh{} l\PYZsq{}appel de .add}
         \PY{n}{s1}\PY{o}{.}\PY{n}{add}\PY{p}{(}\PY{n}{s2}\PY{p}{,} \PY{n}{fill\PYZus{}value}\PY{o}{=}\PY{l+m+mi}{0}\PY{p}{)}
\end{Verbatim}


\begin{Verbatim}[commandchars=\\\{\}]
{\color{outcolor}Out[{\color{outcolor}46}]:} a    22.0
         b    20.0
         c    52.0
         d    32.0
         dtype: float64
\end{Verbatim}
            
    \begin{Verbatim}[commandchars=\\\{\}]
{\color{incolor}In [{\color{incolor}47}]:} \PY{c+c1}{\PYZsh{} regardons un autre exemple sur des DataFrame}
         \PY{c+c1}{\PYZsh{} on affiche tout ça dans les cellules suivantes}
         \PY{n}{names} \PY{o}{=} \PY{p}{[}\PY{l+s+s1}{\PYZsq{}}\PY{l+s+s1}{alice}\PY{l+s+s1}{\PYZsq{}}\PY{p}{,} \PY{l+s+s1}{\PYZsq{}}\PY{l+s+s1}{bob}\PY{l+s+s1}{\PYZsq{}}\PY{p}{,} \PY{l+s+s1}{\PYZsq{}}\PY{l+s+s1}{charle}\PY{l+s+s1}{\PYZsq{}}\PY{p}{]}
         
         \PY{n}{bananas} \PY{o}{=} \PY{n}{pd}\PY{o}{.}\PY{n}{Series}\PY{p}{(}\PY{p}{[}\PY{l+m+mi}{10}\PY{p}{,} \PY{l+m+mi}{3}\PY{p}{,} \PY{l+m+mi}{9}\PY{p}{]}\PY{p}{,} \PY{n}{index}\PY{o}{=}\PY{n}{names}\PY{p}{)}
         \PY{n}{oranges} \PY{o}{=} \PY{n}{pd}\PY{o}{.}\PY{n}{Series}\PY{p}{(}\PY{p}{[}\PY{l+m+mi}{3}\PY{p}{,} \PY{l+m+mi}{11}\PY{p}{,} \PY{l+m+mi}{6}\PY{p}{]}\PY{p}{,} \PY{n}{index}\PY{o}{=}\PY{n}{names}\PY{p}{)}
         \PY{n}{fruits\PYZus{}jan} \PY{o}{=} \PY{n}{pd}\PY{o}{.}\PY{n}{DataFrame}\PY{p}{(}\PY{p}{\PYZob{}}\PY{l+s+s1}{\PYZsq{}}\PY{l+s+s1}{bananas}\PY{l+s+s1}{\PYZsq{}}\PY{p}{:} \PY{n}{bananas}\PY{p}{,} \PY{l+s+s1}{\PYZsq{}}\PY{l+s+s1}{orange}\PY{l+s+s1}{\PYZsq{}}\PY{p}{:} \PY{n}{oranges}\PY{p}{\PYZcb{}}\PY{p}{)}
         
         \PY{n}{bananas} \PY{o}{=} \PY{n}{pd}\PY{o}{.}\PY{n}{Series}\PY{p}{(}\PY{p}{[}\PY{l+m+mi}{6}\PY{p}{,} \PY{l+m+mi}{1}\PY{p}{]}\PY{p}{,} \PY{n}{index}\PY{o}{=}\PY{n}{names}\PY{p}{[}\PY{p}{:}\PY{o}{\PYZhy{}}\PY{l+m+mi}{1}\PY{p}{]}\PY{p}{)}
         \PY{n}{apples} \PY{o}{=} \PY{n}{pd}\PY{o}{.}\PY{n}{Series}\PY{p}{(}\PY{p}{[}\PY{l+m+mi}{8}\PY{p}{,} \PY{l+m+mi}{5}\PY{p}{]}\PY{p}{,} \PY{n}{index}\PY{o}{=}\PY{n}{names}\PY{p}{[}\PY{l+m+mi}{1}\PY{p}{:}\PY{p}{]}\PY{p}{)}
         \PY{n}{fruits\PYZus{}feb} \PY{o}{=} \PY{n}{pd}\PY{o}{.}\PY{n}{DataFrame}\PY{p}{(}\PY{p}{\PYZob{}}\PY{l+s+s1}{\PYZsq{}}\PY{l+s+s1}{bananas}\PY{l+s+s1}{\PYZsq{}}\PY{p}{:} \PY{n}{bananas}\PY{p}{,} \PY{l+s+s1}{\PYZsq{}}\PY{l+s+s1}{apples}\PY{l+s+s1}{\PYZsq{}}\PY{p}{:} \PY{n}{apples}\PY{p}{\PYZcb{}}\PY{p}{)}
\end{Verbatim}


    \begin{Verbatim}[commandchars=\\\{\}]
{\color{incolor}In [{\color{incolor}48}]:} \PY{c+c1}{\PYZsh{} ce qui donne}
         \PY{n}{fruits\PYZus{}jan}
\end{Verbatim}


\begin{Verbatim}[commandchars=\\\{\}]
{\color{outcolor}Out[{\color{outcolor}48}]:}         bananas  orange
         alice        10       3
         bob           3      11
         charle        9       6
\end{Verbatim}
            
    \begin{Verbatim}[commandchars=\\\{\}]
{\color{incolor}In [{\color{incolor}49}]:} \PY{c+c1}{\PYZsh{} et}
         \PY{n}{fruits\PYZus{}feb}
\end{Verbatim}


\begin{Verbatim}[commandchars=\\\{\}]
{\color{outcolor}Out[{\color{outcolor}49}]:}         bananas  apples
         alice       6.0     NaN
         bob         1.0     8.0
         charle      NaN     5.0
\end{Verbatim}
            
    \begin{Verbatim}[commandchars=\\\{\}]
{\color{incolor}In [{\color{incolor}50}]:} \PY{c+c1}{\PYZsh{} regardons maintenant la somme des fruits mangés}
         \PY{n}{eaten\PYZus{}fruits} \PY{o}{=} \PY{n}{fruits\PYZus{}jan} \PY{o}{+} \PY{n}{fruits\PYZus{}feb}
         \PY{n+nb}{print}\PY{p}{(}\PY{n}{eaten\PYZus{}fruits}\PY{p}{)}
\end{Verbatim}


    \begin{Verbatim}[commandchars=\\\{\}]
        apples  bananas  orange
alice      NaN     16.0     NaN
bob        NaN      4.0     NaN
charle     NaN      NaN     NaN

    \end{Verbatim}

    \begin{Verbatim}[commandchars=\\\{\}]
{\color{incolor}In [{\color{incolor}51}]:} \PY{c+c1}{\PYZsh{} On a bien un alignement des labels, mais il y a beaucoup de valeurs}
         \PY{c+c1}{\PYZsh{} manquantes. Corrigeons cela on remplaçant les valeurs manquantes par 0}
         \PY{n}{eaten\PYZus{}fruits} \PY{o}{=} \PY{n}{fruits\PYZus{}jan}\PY{o}{.}\PY{n}{add}\PY{p}{(}\PY{n}{fruits\PYZus{}feb}\PY{p}{,} \PY{n}{fill\PYZus{}value}\PY{o}{=}\PY{l+m+mi}{0}\PY{p}{)}
         \PY{n+nb}{print}\PY{p}{(}\PY{n}{eaten\PYZus{}fruits}\PY{p}{)}
\end{Verbatim}


    \begin{Verbatim}[commandchars=\\\{\}]
        apples  bananas  orange
alice      NaN     16.0     3.0
bob        8.0      4.0    11.0
charle     5.0      9.0     6.0

    \end{Verbatim}

    Notons que lorsqu'une valeur est absente dans toutes les
\texttt{DataFrame}, \texttt{NaN} est conservé.

    Un dernière subtilité à connaître lors de l'alignement des labels
intervient lorsque vous faites une opération sur une \texttt{DataFrame}
et une \texttt{Series}. \texttt{pandas} va considérer la \texttt{Series}
comme une ligne et va la broadcaster sur les autres lignes. Par
conséquent, l'index de la \texttt{Series} va être considéré comme des
colonnes et aligné avec les colonnes de la \texttt{DataFrame}.

    \begin{Verbatim}[commandchars=\\\{\}]
{\color{incolor}In [{\color{incolor}52}]:} \PY{n}{dataframe} \PY{o}{=} \PY{n}{pd}\PY{o}{.}\PY{n}{DataFrame}\PY{p}{(}
             \PY{n}{np}\PY{o}{.}\PY{n}{random}\PY{o}{.}\PY{n}{randint}\PY{p}{(}\PY{l+m+mi}{1}\PY{p}{,} \PY{l+m+mi}{10}\PY{p}{,} \PY{n}{size}\PY{o}{=}\PY{p}{(}\PY{l+m+mi}{3}\PY{p}{,} \PY{l+m+mi}{3}\PY{p}{)}\PY{p}{)}\PY{p}{,}
             \PY{n}{columns}\PY{o}{=}\PY{n+nb}{list}\PY{p}{(}\PY{l+s+s1}{\PYZsq{}}\PY{l+s+s1}{abc}\PY{l+s+s1}{\PYZsq{}}\PY{p}{)}\PY{p}{,} \PY{n}{index}\PY{o}{=}\PY{n+nb}{list}\PY{p}{(}\PY{l+s+s1}{\PYZsq{}}\PY{l+s+s1}{xyz}\PY{l+s+s1}{\PYZsq{}}\PY{p}{)}\PY{p}{)}
         \PY{n}{dataframe}
\end{Verbatim}


\begin{Verbatim}[commandchars=\\\{\}]
{\color{outcolor}Out[{\color{outcolor}52}]:}    a  b  c
         x  9  6  7
         y  1  7  5
         z  3  2  9
\end{Verbatim}
            
    \begin{Verbatim}[commandchars=\\\{\}]
{\color{incolor}In [{\color{incolor}53}]:} \PY{n}{series\PYZus{}row} \PY{o}{=} \PY{n}{pd}\PY{o}{.}\PY{n}{Series}\PY{p}{(}
             \PY{p}{[}\PY{l+m+mi}{100}\PY{p}{,} \PY{l+m+mi}{100}\PY{p}{,} \PY{l+m+mi}{100}\PY{p}{]}\PY{p}{,}
             \PY{n}{index}\PY{o}{=}\PY{n+nb}{list}\PY{p}{(}\PY{l+s+s1}{\PYZsq{}}\PY{l+s+s1}{abc}\PY{l+s+s1}{\PYZsq{}}\PY{p}{)}\PY{p}{)}
         \PY{n}{series\PYZus{}row}
\end{Verbatim}


\begin{Verbatim}[commandchars=\\\{\}]
{\color{outcolor}Out[{\color{outcolor}53}]:} a    100
         b    100
         c    100
         dtype: int64
\end{Verbatim}
            
    \begin{Verbatim}[commandchars=\\\{\}]
{\color{incolor}In [{\color{incolor}54}]:} \PY{n}{series\PYZus{}col} \PY{o}{=} \PY{n}{pd}\PY{o}{.}\PY{n}{Series}\PY{p}{(}
             \PY{p}{[}\PY{l+m+mi}{200}\PY{p}{,} \PY{l+m+mi}{200}\PY{p}{,} \PY{l+m+mi}{200}\PY{p}{]}\PY{p}{,}
             \PY{n}{index}\PY{o}{=}\PY{n+nb}{list}\PY{p}{(}\PY{l+s+s1}{\PYZsq{}}\PY{l+s+s1}{xyz}\PY{l+s+s1}{\PYZsq{}}\PY{p}{)}\PY{p}{)}
         \PY{n}{series\PYZus{}col}
\end{Verbatim}


\begin{Verbatim}[commandchars=\\\{\}]
{\color{outcolor}Out[{\color{outcolor}54}]:} x    200
         y    200
         z    200
         dtype: int64
\end{Verbatim}
            
    \begin{Verbatim}[commandchars=\\\{\}]
{\color{incolor}In [{\color{incolor}55}]:} \PY{c+c1}{\PYZsh{} la Series est considérée comme une ligne et son index}
         \PY{c+c1}{\PYZsh{} s\PYZsq{}aligne sur les colonnes de la DataFrame}
         \PY{c+c1}{\PYZsh{} la Series va être broadcastée}
         \PY{c+c1}{\PYZsh{} sur les autres lignes de la DataFrame}
         
         \PY{n}{dataframe} \PY{o}{+} \PY{n}{series\PYZus{}row}
\end{Verbatim}


\begin{Verbatim}[commandchars=\\\{\}]
{\color{outcolor}Out[{\color{outcolor}55}]:}      a    b    c
         x  109  106  107
         y  101  107  105
         z  103  102  109
\end{Verbatim}
            
    \begin{Verbatim}[commandchars=\\\{\}]
{\color{incolor}In [{\color{incolor}56}]:} \PY{c+c1}{\PYZsh{} du coup si les labels ne correspondent pas,}
         \PY{c+c1}{\PYZsh{} le résultat sera le suivant}
         
         \PY{n}{dataframe} \PY{o}{+} \PY{n}{series\PYZus{}col}
\end{Verbatim}


\begin{Verbatim}[commandchars=\\\{\}]
{\color{outcolor}Out[{\color{outcolor}56}]:}     a   b   c   x   y   z
         x NaN NaN NaN NaN NaN NaN
         y NaN NaN NaN NaN NaN NaN
         z NaN NaN NaN NaN NaN NaN
\end{Verbatim}
            
    \begin{Verbatim}[commandchars=\\\{\}]
{\color{incolor}In [{\color{incolor}57}]:} \PY{c+c1}{\PYZsh{} on peut dans ce cas, changer le comportement par défaut en forçant}
         \PY{c+c1}{\PYZsh{} l\PYZsq{}alignement de la Series suivant un autre axe avec l\PYZsq{}argument axis}
         
         \PY{n}{dataframe}\PY{o}{.}\PY{n}{add}\PY{p}{(}\PY{n}{series\PYZus{}col}\PY{p}{,} \PY{n}{axis}\PY{o}{=}\PY{l+m+mi}{0}\PY{p}{)}
\end{Verbatim}


\begin{Verbatim}[commandchars=\\\{\}]
{\color{outcolor}Out[{\color{outcolor}57}]:}      a    b    c
         x  209  206  207
         y  201  207  205
         z  203  202  209
\end{Verbatim}
            
    Ici, \texttt{axis=0} signifie que la \texttt{Series} est considérée
comme une colonne est qu'elle va être broadcastée sur les autres
colonnes (le long de l'axe de ligne).

    \hypertarget{opuxe9rations-sur-les-chauxeenes-de-caractuxe8res}{%
\subsubsection{Opérations sur les chaînes de
caractères}\label{opuxe9rations-sur-les-chauxeenes-de-caractuxe8res}}

    Nous allons maintenant parler de la vectorisation des opérations sur les
chaînes de caractères. Il y a plusieurs choses importantes à savoir~:

\begin{itemize}
\tightlist
\item
  les méthodes sur les chaînes de caractères ne sont disponibles que
  pour les \texttt{Series} et les \texttt{Index}, mais pas pour les
  \texttt{DataFrame}~;
\item
  ces méthodes ignorent les \texttt{NaN} et remplacent les valeurs qui
  ne sont pas des chaînes de caractères par \texttt{NaN}~;
\item
  ces méthodes retournent une copie de l'objet (\texttt{Series} ou
  \texttt{Index}), il n'y a pas de modification en place~;
\item
  la plupart des méthodes Python sur le type \texttt{str} existe sous
  forme vectorisée~;
\item
  on accède à ces méthodes avec la syntaxe~:

  \begin{itemize}
  \tightlist
  \item
    \texttt{Series.str.\textless{}vectorized\ method\ name\textgreater{}}
  \item
    \texttt{Index.str.\textless{}vectorized\ method\ name\textgreater{}}
  \end{itemize}
\end{itemize}

Regardons quelques exemples~:

    \begin{Verbatim}[commandchars=\\\{\}]
{\color{incolor}In [{\color{incolor}58}]:} \PY{c+c1}{\PYZsh{} Créons une Series avec des noms ayant une capitalisation inconsistante}
         \PY{c+c1}{\PYZsh{} et une mauvaise gestion des espaces}
         \PY{n}{names} \PY{o}{=} \PY{p}{[}\PY{l+s+s1}{\PYZsq{}}\PY{l+s+s1}{alice }\PY{l+s+s1}{\PYZsq{}}\PY{p}{,} \PY{l+s+s1}{\PYZsq{}}\PY{l+s+s1}{  bOB}\PY{l+s+s1}{\PYZsq{}}\PY{p}{,} \PY{l+s+s1}{\PYZsq{}}\PY{l+s+s1}{Marc}\PY{l+s+s1}{\PYZsq{}}\PY{p}{,} \PY{l+s+s1}{\PYZsq{}}\PY{l+s+s1}{bill}\PY{l+s+s1}{\PYZsq{}}\PY{p}{,} \PY{l+m+mi}{3}\PY{p}{,} \PY{l+s+s1}{\PYZsq{}}\PY{l+s+s1}{ JULIE }\PY{l+s+s1}{\PYZsq{}}\PY{p}{,} \PY{n}{np}\PY{o}{.}\PY{n}{NaN}\PY{p}{]}
         \PY{n}{age} \PY{o}{=} \PY{n}{pd}\PY{o}{.}\PY{n}{Series}\PY{p}{(}\PY{n}{names}\PY{p}{)}
\end{Verbatim}


    \begin{Verbatim}[commandchars=\\\{\}]
{\color{incolor}In [{\color{incolor}59}]:} \PY{c+c1}{\PYZsh{} nettoyons maintenant ces données}
         
         \PY{c+c1}{\PYZsh{} on met en minuscule}
         \PY{n}{a} \PY{o}{=} \PY{n}{age}\PY{o}{.}\PY{n}{str}\PY{o}{.}\PY{n}{lower}\PY{p}{(}\PY{p}{)}
         
         \PY{c+c1}{\PYZsh{} on enlève les espaces}
         \PY{n}{a} \PY{o}{=} \PY{n}{a}\PY{o}{.}\PY{n}{str}\PY{o}{.}\PY{n}{strip}\PY{p}{(}\PY{p}{)}
         \PY{n}{a}
\end{Verbatim}


\begin{Verbatim}[commandchars=\\\{\}]
{\color{outcolor}Out[{\color{outcolor}59}]:} 0    alice
         1      bob
         2     marc
         3     bill
         4      NaN
         5    julie
         6      NaN
         dtype: object
\end{Verbatim}
            
    \begin{Verbatim}[commandchars=\\\{\}]
{\color{incolor}In [{\color{incolor}60}]:} \PY{c+c1}{\PYZsh{} comme les méthodes vectorisées retournent un objet de même type, on}
         \PY{c+c1}{\PYZsh{} peut les chaîner comme ceci}
         
         \PY{p}{[}\PY{n}{x} \PY{k}{for} \PY{n}{x} \PY{o+ow}{in} \PY{n}{age}\PY{o}{.}\PY{n}{str}\PY{o}{.}\PY{n}{lower}\PY{p}{(}\PY{p}{)}\PY{o}{.}\PY{n}{str}\PY{o}{.}\PY{n}{strip}\PY{p}{(}\PY{p}{)}\PY{p}{]}
\end{Verbatim}


\begin{Verbatim}[commandchars=\\\{\}]
{\color{outcolor}Out[{\color{outcolor}60}]:} ['alice', 'bob', 'marc', 'bill', nan, 'julie', nan]
\end{Verbatim}
            
    On peut également utiliser l'indexation des \texttt{str} de manière
vectorisée~:

    \begin{Verbatim}[commandchars=\\\{\}]
{\color{incolor}In [{\color{incolor}61}]:} \PY{n+nb}{print}\PY{p}{(}\PY{n}{a}\PY{p}{)}
\end{Verbatim}


    \begin{Verbatim}[commandchars=\\\{\}]
0    alice
1      bob
2     marc
3     bill
4      NaN
5    julie
6      NaN
dtype: object

    \end{Verbatim}

    \begin{Verbatim}[commandchars=\\\{\}]
{\color{incolor}In [{\color{incolor}62}]:} \PY{n+nb}{print}\PY{p}{(}\PY{n}{a}\PY{o}{.}\PY{n}{str}\PY{p}{[}\PY{o}{\PYZhy{}}\PY{l+m+mi}{1}\PY{p}{]}\PY{p}{)}
\end{Verbatim}


    \begin{Verbatim}[commandchars=\\\{\}]
0      e
1      b
2      c
3      l
4    NaN
5      e
6    NaN
dtype: object

    \end{Verbatim}

    Pour aller plus loin vous pouvez lire la documentation officielle~:

http://pandas.pydata.org/pandas-docs/stable/text.html

    \hypertarget{gestion-des-valeurs-manquantes}{%
\subsubsection{Gestion des valeurs
manquantes}\label{gestion-des-valeurs-manquantes}}

    Nous avons vu que des opérations sur les \texttt{DataFrame} pouvaient
générer des valeurs \texttt{NaN} lors de l'alignement. Il est également
possible d'avoir de telles valeurs \emph{manquantes} dans votre jeu de
données original. \texttt{pandas} offre plusieurs possibilités pour
gérer correctement ces valeurs manquantes.

    Avant de voir ces différentes possibilités, définissons cette notion de
valeur manquante.

Une valeur manquante peut-être représentée avec \texttt{pandas} soit par
\texttt{np.NaN} soit par l'objet Python \texttt{None}.

\begin{itemize}
\tightlist
\item
  \texttt{np.NaN} est un objet de type \texttt{float}, par conséquent il
  ne peut apparaître que dans un array de \texttt{float} ou un array
  d'\texttt{object}. Notons que \texttt{np.NaN} apparaît avec
  \texttt{pandas} comme simplement \texttt{NaN} et que dans la suite on
  utilise de manière indifférente les deux notations, par contre, dans
  du code, il faut obligatoirement utiliser \texttt{np.NaN}~;

  \begin{itemize}
  \tightlist
  \item
    si on ajoute un \texttt{NaN} dans un array d'entier, ils seront
    convertis en \texttt{float64}~;
  \item
    si on ajoute un \texttt{NaN} dans un array de booléens, ils seront
    convertis en \texttt{object}~;
  \end{itemize}
\item
  \texttt{NaN} est contaminant, toute opération avec un \texttt{NaN} a
  pour résultat \texttt{NaN}~;
\item
  lorsque l'on utilise \texttt{None}, il est automatiquement converti en
  \texttt{NaN} lorsque le type de l'array est numérique.
\end{itemize}

Illustrons ces propriétés~:

    \begin{Verbatim}[commandchars=\\\{\}]
{\color{incolor}In [{\color{incolor}63}]:} \PY{c+c1}{\PYZsh{} une Series d\PYZsq{}entiers}
         \PY{n}{s} \PY{o}{=} \PY{n}{pd}\PY{o}{.}\PY{n}{Series}\PY{p}{(}\PY{p}{[}\PY{l+m+mi}{1}\PY{p}{,} \PY{l+m+mi}{2}\PY{p}{]}\PY{p}{)}
         \PY{n}{s}
\end{Verbatim}


\begin{Verbatim}[commandchars=\\\{\}]
{\color{outcolor}Out[{\color{outcolor}63}]:} 0    1
         1    2
         dtype: int64
\end{Verbatim}
            
    \begin{Verbatim}[commandchars=\\\{\}]
{\color{incolor}In [{\color{incolor}64}]:} \PY{c+c1}{\PYZsh{} on insère un NaN, la Series est alors convertie en float64}
         \PY{n}{s}\PY{p}{[}\PY{l+m+mi}{0}\PY{p}{]} \PY{o}{=} \PY{n}{np}\PY{o}{.}\PY{n}{NaN}
         \PY{n}{s}
\end{Verbatim}


\begin{Verbatim}[commandchars=\\\{\}]
{\color{outcolor}Out[{\color{outcolor}64}]:} 0    NaN
         1    2.0
         dtype: float64
\end{Verbatim}
            
    \begin{Verbatim}[commandchars=\\\{\}]
{\color{incolor}In [{\color{incolor}65}]:} \PY{c+c1}{\PYZsh{} on réinitialise}
         \PY{n}{s} \PY{o}{=} \PY{n}{pd}\PY{o}{.}\PY{n}{Series}\PY{p}{(}\PY{p}{[}\PY{l+m+mi}{1}\PY{p}{,} \PY{l+m+mi}{2}\PY{p}{]}\PY{p}{)}
         \PY{n}{s}
\end{Verbatim}


\begin{Verbatim}[commandchars=\\\{\}]
{\color{outcolor}Out[{\color{outcolor}65}]:} 0    1
         1    2
         dtype: int64
\end{Verbatim}
            
    \begin{Verbatim}[commandchars=\\\{\}]
{\color{incolor}In [{\color{incolor}66}]:} \PY{c+c1}{\PYZsh{} et on insère None}
         \PY{n}{s}\PY{p}{[}\PY{l+m+mi}{0}\PY{p}{]} \PY{o}{=} \PY{k+kc}{None}
         
         \PY{c+c1}{\PYZsh{} Le résultat est le même}
         \PY{c+c1}{\PYZsh{} None est converti en NaN}
         \PY{n}{s}
\end{Verbatim}


\begin{Verbatim}[commandchars=\\\{\}]
{\color{outcolor}Out[{\color{outcolor}66}]:} 0    NaN
         1    2.0
         dtype: float64
\end{Verbatim}
            
    Regardons maintenant, les méthodes de \texttt{pandas} pour gérer les
valeurs manquantes (donc \texttt{NaN} ou \texttt{None})~:

\begin{itemize}
\tightlist
\item
  \texttt{isna()} retourne un masque mettant à \texttt{True} les valeurs
  manquantes (il y a un alias \texttt{isnull()})~;
\item
  \texttt{notna()} retourne un masque mettant à \texttt{False} les
  valeurs manquantes (il y a un alias \texttt{notnull()})~;
\item
  \texttt{dropna()} retourne un nouvel objet sans les valeurs
  manquantes~;
\item
  \texttt{fillna()} retourne un nouvel objet avec les valeurs manquantes
  remplacées.
\end{itemize}

On remarque que l'ajout d'alias pour les méthodes est de nouveau une
source de confusion avec laquelle il faut vivre.

On remarque également qu'alors que \texttt{isnull()} et
\texttt{notnull()} sont des méthodes simples, \texttt{dropna()} et
\texttt{fillna()} impliquent l'utilisation de stratégies. Regardons
cela~:

    \begin{Verbatim}[commandchars=\\\{\}]
{\color{incolor}In [{\color{incolor}67}]:} \PY{c+c1}{\PYZsh{} créons une DataFrame avec quelques valeurs manquantes}
         \PY{n}{names} \PY{o}{=} \PY{p}{[}\PY{l+s+s1}{\PYZsq{}}\PY{l+s+s1}{alice}\PY{l+s+s1}{\PYZsq{}}\PY{p}{,} \PY{l+s+s1}{\PYZsq{}}\PY{l+s+s1}{bob}\PY{l+s+s1}{\PYZsq{}}\PY{p}{,} \PY{l+s+s1}{\PYZsq{}}\PY{l+s+s1}{charles}\PY{l+s+s1}{\PYZsq{}}\PY{p}{]}
         \PY{n}{bananas} \PY{o}{=} \PY{n}{pd}\PY{o}{.}\PY{n}{Series}\PY{p}{(}\PY{p}{[}\PY{l+m+mi}{6}\PY{p}{,} \PY{l+m+mi}{1}\PY{p}{]}\PY{p}{,} \PY{n}{index}\PY{o}{=}\PY{n}{names}\PY{p}{[}\PY{p}{:}\PY{o}{\PYZhy{}}\PY{l+m+mi}{1}\PY{p}{]}\PY{p}{)}
         \PY{n}{apples} \PY{o}{=} \PY{n}{pd}\PY{o}{.}\PY{n}{Series}\PY{p}{(}\PY{p}{[}\PY{l+m+mi}{8}\PY{p}{,} \PY{l+m+mi}{5}\PY{p}{]}\PY{p}{,} \PY{n}{index}\PY{o}{=}\PY{n}{names}\PY{p}{[}\PY{l+m+mi}{1}\PY{p}{:}\PY{p}{]}\PY{p}{)}
         \PY{n}{fruits\PYZus{}feb} \PY{o}{=} \PY{n}{pd}\PY{o}{.}\PY{n}{DataFrame}\PY{p}{(}\PY{p}{\PYZob{}}\PY{l+s+s1}{\PYZsq{}}\PY{l+s+s1}{bananas}\PY{l+s+s1}{\PYZsq{}}\PY{p}{:} \PY{n}{bananas}\PY{p}{,} \PY{l+s+s1}{\PYZsq{}}\PY{l+s+s1}{apples}\PY{l+s+s1}{\PYZsq{}}\PY{p}{:} \PY{n}{apples}\PY{p}{\PYZcb{}}\PY{p}{)}
         \PY{n+nb}{print}\PY{p}{(}\PY{n}{fruits\PYZus{}feb}\PY{p}{)}
\end{Verbatim}


    \begin{Verbatim}[commandchars=\\\{\}]
         bananas  apples
alice        6.0     NaN
bob          1.0     8.0
charles      NaN     5.0

    \end{Verbatim}

    \begin{Verbatim}[commandchars=\\\{\}]
{\color{incolor}In [{\color{incolor}68}]:} \PY{n}{fruits\PYZus{}feb}\PY{o}{.}\PY{n}{isna}\PY{p}{(}\PY{p}{)}
\end{Verbatim}


\begin{Verbatim}[commandchars=\\\{\}]
{\color{outcolor}Out[{\color{outcolor}68}]:}          bananas  apples
         alice      False    True
         bob        False   False
         charles     True   False
\end{Verbatim}
            
    \begin{Verbatim}[commandchars=\\\{\}]
{\color{incolor}In [{\color{incolor}69}]:} \PY{n}{fruits\PYZus{}feb}\PY{o}{.}\PY{n}{notna}\PY{p}{(}\PY{p}{)}
\end{Verbatim}


\begin{Verbatim}[commandchars=\\\{\}]
{\color{outcolor}Out[{\color{outcolor}69}]:}          bananas  apples
         alice       True   False
         bob         True    True
         charles    False    True
\end{Verbatim}
            
    Par défaut, \texttt{dropna()} va enlever toutes les lignes qui
contiennent au moins une valeur manquante. Mais on peut changer ce
comportement avec des arguments~:

    \begin{Verbatim}[commandchars=\\\{\}]
{\color{incolor}In [{\color{incolor}70}]:} \PY{n}{p} \PY{o}{=} \PY{n}{pd}\PY{o}{.}\PY{n}{DataFrame}\PY{p}{(}\PY{p}{[}\PY{p}{[}\PY{l+m+mi}{1}\PY{p}{,} \PY{l+m+mi}{2}\PY{p}{,} \PY{n}{np}\PY{o}{.}\PY{n}{NaN}\PY{p}{]}\PY{p}{,} \PY{p}{[}\PY{l+m+mi}{3}\PY{p}{,} \PY{n}{np}\PY{o}{.}\PY{n}{NaN}\PY{p}{,} \PY{n}{np}\PY{o}{.}\PY{n}{NaN}\PY{p}{]}\PY{p}{,} \PY{p}{[}\PY{l+m+mi}{7}\PY{p}{,} \PY{l+m+mi}{5}\PY{p}{,} \PY{n}{np}\PY{o}{.}\PY{n}{NaN}\PY{p}{]}\PY{p}{]}\PY{p}{)}
         \PY{n+nb}{print}\PY{p}{(}\PY{n}{p}\PY{p}{)}
\end{Verbatim}


    \begin{Verbatim}[commandchars=\\\{\}]
   0    1   2
0  1  2.0 NaN
1  3  NaN NaN
2  7  5.0 NaN

    \end{Verbatim}

    \begin{Verbatim}[commandchars=\\\{\}]
{\color{incolor}In [{\color{incolor}71}]:} \PY{c+c1}{\PYZsh{} comportement par défaut, j\PYZsq{}enlève toutes les lignes avec au moins}
         \PY{c+c1}{\PYZsh{} une valeur manquante; il ne reste rien !}
         \PY{n}{p}\PY{o}{.}\PY{n}{dropna}\PY{p}{(}\PY{p}{)}
\end{Verbatim}


\begin{Verbatim}[commandchars=\\\{\}]
{\color{outcolor}Out[{\color{outcolor}71}]:} Empty DataFrame
         Columns: [0, 1, 2]
         Index: []
\end{Verbatim}
            
    \begin{Verbatim}[commandchars=\\\{\}]
{\color{incolor}In [{\color{incolor}72}]:} \PY{c+c1}{\PYZsh{} maintenant, je fais l\PYZsq{}opération par colonne}
         \PY{n}{p}\PY{o}{.}\PY{n}{dropna}\PY{p}{(}\PY{n}{axis}\PY{o}{=}\PY{l+m+mi}{1}\PY{p}{)}
\end{Verbatim}


\begin{Verbatim}[commandchars=\\\{\}]
{\color{outcolor}Out[{\color{outcolor}72}]:}    0
         0  1
         1  3
         2  7
\end{Verbatim}
            
    \begin{Verbatim}[commandchars=\\\{\}]
{\color{incolor}In [{\color{incolor}73}]:} \PY{c+c1}{\PYZsh{} je fais l\PYZsq{}opération par colonne si toute la colonne est manquante}
         \PY{n}{p}\PY{o}{.}\PY{n}{dropna}\PY{p}{(}\PY{n}{axis}\PY{o}{=}\PY{l+m+mi}{1}\PY{p}{,} \PY{n}{how}\PY{o}{=}\PY{l+s+s1}{\PYZsq{}}\PY{l+s+s1}{all}\PY{l+s+s1}{\PYZsq{}}\PY{p}{)}
\end{Verbatim}


\begin{Verbatim}[commandchars=\\\{\}]
{\color{outcolor}Out[{\color{outcolor}73}]:}    0    1
         0  1  2.0
         1  3  NaN
         2  7  5.0
\end{Verbatim}
            
    \begin{Verbatim}[commandchars=\\\{\}]
{\color{incolor}In [{\color{incolor}74}]:} \PY{c+c1}{\PYZsh{} je fais l\PYZsq{}opération par ligne si au moins 2 valeurs sont manquantes}
         \PY{n}{p}\PY{o}{.}\PY{n}{dropna}\PY{p}{(}\PY{n}{thresh}\PY{o}{=}\PY{l+m+mi}{2}\PY{p}{)}
\end{Verbatim}


\begin{Verbatim}[commandchars=\\\{\}]
{\color{outcolor}Out[{\color{outcolor}74}]:}    0    1   2
         0  1  2.0 NaN
         2  7  5.0 NaN
\end{Verbatim}
            
    Par défaut, \texttt{fillna()} remplace les valeurs manquantes avec un
argument pas défaut. Mais on peut ici aussi changer ce comportement.
Regardons cela~:

    \begin{Verbatim}[commandchars=\\\{\}]
{\color{incolor}In [{\color{incolor}75}]:} \PY{n+nb}{print}\PY{p}{(}\PY{n}{p}\PY{p}{)}
\end{Verbatim}


    \begin{Verbatim}[commandchars=\\\{\}]
   0    1   2
0  1  2.0 NaN
1  3  NaN NaN
2  7  5.0 NaN

    \end{Verbatim}

    \begin{Verbatim}[commandchars=\\\{\}]
{\color{incolor}In [{\color{incolor}76}]:} \PY{c+c1}{\PYZsh{} je remplace les valeurs manquantes par \PYZhy{}1}
         \PY{n}{p}\PY{o}{.}\PY{n}{fillna}\PY{p}{(}\PY{o}{\PYZhy{}}\PY{l+m+mi}{1}\PY{p}{)}
\end{Verbatim}


\begin{Verbatim}[commandchars=\\\{\}]
{\color{outcolor}Out[{\color{outcolor}76}]:}    0    1    2
         0  1  2.0 -1.0
         1  3 -1.0 -1.0
         2  7  5.0 -1.0
\end{Verbatim}
            
    \begin{Verbatim}[commandchars=\\\{\}]
{\color{incolor}In [{\color{incolor}77}]:} \PY{c+c1}{\PYZsh{} je remplace les valeurs manquantes avec la valeur suivante sur la colonne}
         \PY{c+c1}{\PYZsh{} bfill est pour back fill, c\PYZsq{}est\PYZhy{}à\PYZhy{}dire remplace en arrière à partir des}
         \PY{c+c1}{\PYZsh{} valeurs existantes}
         \PY{n}{p}\PY{o}{.}\PY{n}{fillna}\PY{p}{(}\PY{n}{method}\PY{o}{=}\PY{l+s+s1}{\PYZsq{}}\PY{l+s+s1}{bfill}\PY{l+s+s1}{\PYZsq{}}\PY{p}{)}
\end{Verbatim}


\begin{Verbatim}[commandchars=\\\{\}]
{\color{outcolor}Out[{\color{outcolor}77}]:}    0    1   2
         0  1  2.0 NaN
         1  3  5.0 NaN
         2  7  5.0 NaN
\end{Verbatim}
            
    \begin{Verbatim}[commandchars=\\\{\}]
{\color{incolor}In [{\color{incolor}78}]:} \PY{c+c1}{\PYZsh{} je remplace les valeurs manquantes avec la valeur précédente sur la ligne}
         \PY{c+c1}{\PYZsh{} ffill est pour forward fill, remplace en avant à partir des valeurs}
         \PY{c+c1}{\PYZsh{} existantes}
         \PY{n}{p}\PY{o}{.}\PY{n}{fillna}\PY{p}{(}\PY{n}{method}\PY{o}{=}\PY{l+s+s1}{\PYZsq{}}\PY{l+s+s1}{ffill}\PY{l+s+s1}{\PYZsq{}}\PY{p}{,} \PY{n}{axis}\PY{o}{=}\PY{l+m+mi}{1}\PY{p}{)}
\end{Verbatim}


\begin{Verbatim}[commandchars=\\\{\}]
{\color{outcolor}Out[{\color{outcolor}78}]:}      0    1    2
         0  1.0  2.0  2.0
         1  3.0  3.0  3.0
         2  7.0  5.0  5.0
\end{Verbatim}
            
    Regardez l'aide de ces méthodes pour aller plus loin.

    \begin{Verbatim}[commandchars=\\\{\}]
{\color{incolor}In [{\color{incolor}79}]:} p.dropna\PY{o}{?}
\end{Verbatim}


    \begin{Verbatim}[commandchars=\\\{\}]
{\color{incolor}In [{\color{incolor}80}]:} p.fillna\PY{o}{?}
\end{Verbatim}


    \hypertarget{analyse-statistique-des-donnuxe9es}{%
\subsubsection{Analyse statistique des
données}\label{analyse-statistique-des-donnuxe9es}}

    Nous n'avons pas le temps de couvrir les possibilités d'analyse
statistique de la suite data science de Python. \texttt{pandas} offre
quelques possibilités basiques avec des calculs de moyennes, d'écarts
types ou de covariances que l'on peut éventuellement appliquer par
fenêtres à un jeux de données. Pour avoir plus de détails dessus vous
pouvez consulter cette documentation~:

http://pandas.pydata.org/pandas-docs/stable/computation.html

Dans la suite data science de Python, il a aussi des modules spécialisés
dans l'analyse statistique comme~:

\begin{itemize}
\tightlist
\item
  \href{http://www.statsmodels.org/stable/index.html}{StatsModels}
\item
  \href{http://scikit-learn.org/stable/}{ScikitLearn}
\end{itemize}

ou des outils de calculs scientifiques plus génériques comme
\href{https://www.scipy.org/}{SciPy}.

De nouveau, il s'agit d'outils appliqués à des domaines spécifiques et
ils se basent tous sur le couple \texttt{numpy}/\texttt{pandas}.

    \hypertarget{compluxe9ment---niveau-avancuxe9}{%
\subsection{Complément - niveau
avancé}\label{compluxe9ment---niveau-avancuxe9}}

    \hypertarget{les-multiindex}{%
\subsubsection{Les MultiIndex}\label{les-multiindex}}

    \texttt{pandas} avait historiquement d'autres structures de données en
plus des \texttt{Series} et des \texttt{DataFrame} permettant d'exprimer
des dimensionnalités supérieures à 2, comme par exemple les
\texttt{Panel}. Mais pour des raisons de maintenance du code et
d'optimisation, les développeurs ont décidé de ne garder que les
\texttt{Series} et les \texttt{DataFrame}. Alors, comment exprimer des
données avec plus de deux dimensions~?

On utilise pour cela des \texttt{MultiIndex}. Un \texttt{MultiIndex} est
un index qui peut être utilisé partout où l'on utilise un index (dans
une \texttt{Series}, ou comme ligne ou colonne d'une \texttt{DataFrame})
et qui a pour caractéristique d'avoir plusieurs niveaux.

Comme tous types d'index, et parce qu'un \texttt{MultiIndex} est une
sous classe d'\texttt{Index}, \texttt{pandas} va correctement aligner
les \texttt{Series} et les \texttt{DataFrame} avec des
\texttt{MultiIndex}.

Regardons tout de suite un exemple~:

    \begin{Verbatim}[commandchars=\\\{\}]
{\color{incolor}In [{\color{incolor}81}]:} \PY{c+c1}{\PYZsh{} construisons une DataFrame jouet}
         
         \PY{c+c1}{\PYZsh{} voici une liste de prénoms}
         \PY{n}{names} \PY{o}{=} \PY{p}{[}\PY{l+s+s1}{\PYZsq{}}\PY{l+s+s1}{alice}\PY{l+s+s1}{\PYZsq{}}\PY{p}{,} \PY{l+s+s1}{\PYZsq{}}\PY{l+s+s1}{bob}\PY{l+s+s1}{\PYZsq{}}\PY{p}{,} \PY{l+s+s1}{\PYZsq{}}\PY{l+s+s1}{sonia}\PY{l+s+s1}{\PYZsq{}}\PY{p}{]}
         
         \PY{c+c1}{\PYZsh{} créons trois Series qui formeront trois colonnes}
         \PY{n}{age} \PY{o}{=} \PY{n}{pd}\PY{o}{.}\PY{n}{Series}\PY{p}{(}\PY{p}{[}\PY{l+m+mi}{12}\PY{p}{,} \PY{l+m+mi}{13}\PY{p}{,} \PY{l+m+mi}{16}\PY{p}{]}\PY{p}{,} \PY{n}{index}\PY{o}{=}\PY{n}{names}\PY{p}{)}
         \PY{n}{height} \PY{o}{=} \PY{n}{pd}\PY{o}{.}\PY{n}{Series}\PY{p}{(}\PY{p}{[}\PY{l+m+mi}{130}\PY{p}{,} \PY{l+m+mi}{140}\PY{p}{,} \PY{l+m+mi}{165}\PY{p}{]}\PY{p}{,} \PY{n}{index}\PY{o}{=}\PY{n}{names}\PY{p}{)}
         \PY{n}{sex} \PY{o}{=} \PY{n}{pd}\PY{o}{.}\PY{n}{Series}\PY{p}{(}\PY{n+nb}{list}\PY{p}{(}\PY{l+s+s1}{\PYZsq{}}\PY{l+s+s1}{fmf}\PY{l+s+s1}{\PYZsq{}}\PY{p}{)}\PY{p}{,} \PY{n}{index}\PY{o}{=}\PY{n}{names}\PY{p}{)}
         
         \PY{c+c1}{\PYZsh{} créons maintenant la DataFrame}
         \PY{n}{p} \PY{o}{=} \PY{n}{pd}\PY{o}{.}\PY{n}{DataFrame}\PY{p}{(}\PY{p}{\PYZob{}}\PY{l+s+s1}{\PYZsq{}}\PY{l+s+s1}{age}\PY{l+s+s1}{\PYZsq{}}\PY{p}{:} \PY{n}{age}\PY{p}{,} \PY{l+s+s1}{\PYZsq{}}\PY{l+s+s1}{height}\PY{l+s+s1}{\PYZsq{}}\PY{p}{:} \PY{n}{height}\PY{p}{,} \PY{l+s+s1}{\PYZsq{}}\PY{l+s+s1}{sex}\PY{l+s+s1}{\PYZsq{}}\PY{p}{:} \PY{n}{sex}\PY{p}{\PYZcb{}}\PY{p}{)}
         \PY{n+nb}{print}\PY{p}{(}\PY{n}{p}\PY{p}{)}
\end{Verbatim}


    \begin{Verbatim}[commandchars=\\\{\}]
       age  height sex
alice   12     130   f
bob     13     140   m
sonia   16     165   f

    \end{Verbatim}

    \begin{Verbatim}[commandchars=\\\{\}]
{\color{incolor}In [{\color{incolor}82}]:} \PY{c+c1}{\PYZsh{} unstack, en première approximation, permet de passer d\PYZsq{}une DataFrame à}
         \PY{c+c1}{\PYZsh{} une Series avec un MultiIndex}
         \PY{n}{s} \PY{o}{=} \PY{n}{p}\PY{o}{.}\PY{n}{unstack}\PY{p}{(}\PY{p}{)}
         \PY{n+nb}{print}\PY{p}{(}\PY{n}{s}\PY{p}{)}
\end{Verbatim}


    \begin{Verbatim}[commandchars=\\\{\}]
age     alice     12
        bob       13
        sonia     16
height  alice    130
        bob      140
        sonia    165
sex     alice      f
        bob        m
        sonia      f
dtype: object

    \end{Verbatim}

    \begin{Verbatim}[commandchars=\\\{\}]
{\color{incolor}In [{\color{incolor}83}]:} \PY{c+c1}{\PYZsh{} et voici donc l\PYZsq{}index de cette Series}
         \PY{n}{s}\PY{o}{.}\PY{n}{index}
\end{Verbatim}


\begin{Verbatim}[commandchars=\\\{\}]
{\color{outcolor}Out[{\color{outcolor}83}]:} MultiIndex(levels=[['age', 'height', 'sex'], ['alice', 'bob', 'sonia']],
                    labels=[[0, 0, 0, 1, 1, 1, 2, 2, 2], [0, 1, 2, 0, 1, 2, 0, 1, 2]])
\end{Verbatim}
            
    Il existe évidemment des moyens de créer directement un
\texttt{MultiIndex} et ensuite de le définir comme index d'une
\texttt{Series} ou comme index de ligne ou colonne d'une
\texttt{DataFrame}~:

    \begin{Verbatim}[commandchars=\\\{\}]
{\color{incolor}In [{\color{incolor}84}]:} \PY{c+c1}{\PYZsh{} on peut créer un MultiIndex à partir d\PYZsq{}une liste de liste}
         \PY{n}{names} \PY{o}{=} \PY{p}{[}\PY{l+s+s1}{\PYZsq{}}\PY{l+s+s1}{alice}\PY{l+s+s1}{\PYZsq{}}\PY{p}{,} \PY{l+s+s1}{\PYZsq{}}\PY{l+s+s1}{alice}\PY{l+s+s1}{\PYZsq{}}\PY{p}{,} \PY{l+s+s1}{\PYZsq{}}\PY{l+s+s1}{alice}\PY{l+s+s1}{\PYZsq{}}\PY{p}{,} \PY{l+s+s1}{\PYZsq{}}\PY{l+s+s1}{bob}\PY{l+s+s1}{\PYZsq{}}\PY{p}{,} \PY{l+s+s1}{\PYZsq{}}\PY{l+s+s1}{bob}\PY{l+s+s1}{\PYZsq{}}\PY{p}{,} \PY{l+s+s1}{\PYZsq{}}\PY{l+s+s1}{bob}\PY{l+s+s1}{\PYZsq{}}\PY{p}{]}
         \PY{n}{age} \PY{o}{=} \PY{p}{[}\PY{l+m+mi}{2014}\PY{p}{,} \PY{l+m+mi}{2015}\PY{p}{,} \PY{l+m+mi}{2016}\PY{p}{,} \PY{l+m+mi}{2014}\PY{p}{,} \PY{l+m+mi}{2015}\PY{p}{,} \PY{l+m+mi}{2016}\PY{p}{]}
         \PY{n}{s\PYZus{}list} \PY{o}{=} \PY{n}{pd}\PY{o}{.}\PY{n}{Series}\PY{p}{(}\PY{p}{[}\PY{l+m+mi}{40}\PY{p}{,} \PY{l+m+mi}{42}\PY{p}{,} \PY{l+m+mi}{45}\PY{p}{,} \PY{l+m+mi}{38}\PY{p}{,} \PY{l+m+mi}{40}\PY{p}{,} \PY{l+m+mi}{40}\PY{p}{]}\PY{p}{,} \PY{n}{index}\PY{o}{=}\PY{p}{[}\PY{n}{names}\PY{p}{,} \PY{n}{age}\PY{p}{]}\PY{p}{)}
         \PY{n+nb}{print}\PY{p}{(}\PY{n}{s\PYZus{}list}\PY{p}{)}
\end{Verbatim}


    \begin{Verbatim}[commandchars=\\\{\}]
alice  2014    40
       2015    42
       2016    45
bob    2014    38
       2015    40
       2016    40
dtype: int64

    \end{Verbatim}

    \begin{Verbatim}[commandchars=\\\{\}]
{\color{incolor}In [{\color{incolor}85}]:} \PY{c+c1}{\PYZsh{} ou à partir d\PYZsq{}un dictionnaire de tuples}
         \PY{n}{s\PYZus{}tuple} \PY{o}{=} \PY{n}{pd}\PY{o}{.}\PY{n}{Series}\PY{p}{(}\PY{p}{\PYZob{}}\PY{p}{(}\PY{l+s+s1}{\PYZsq{}}\PY{l+s+s1}{alice}\PY{l+s+s1}{\PYZsq{}}\PY{p}{,} \PY{l+m+mi}{2014}\PY{p}{)}\PY{p}{:} \PY{l+m+mi}{40}\PY{p}{,}
                              \PY{p}{(}\PY{l+s+s1}{\PYZsq{}}\PY{l+s+s1}{alice}\PY{l+s+s1}{\PYZsq{}}\PY{p}{,} \PY{l+m+mi}{2015}\PY{p}{)}\PY{p}{:} \PY{l+m+mi}{42}\PY{p}{,}
                              \PY{p}{(}\PY{l+s+s1}{\PYZsq{}}\PY{l+s+s1}{alice}\PY{l+s+s1}{\PYZsq{}}\PY{p}{,} \PY{l+m+mi}{2016}\PY{p}{)}\PY{p}{:} \PY{l+m+mi}{45}\PY{p}{,}
                              \PY{p}{(}\PY{l+s+s1}{\PYZsq{}}\PY{l+s+s1}{bob}\PY{l+s+s1}{\PYZsq{}}\PY{p}{,} \PY{l+m+mi}{2014}\PY{p}{)}\PY{p}{:} \PY{l+m+mi}{38}\PY{p}{,}
                              \PY{p}{(}\PY{l+s+s1}{\PYZsq{}}\PY{l+s+s1}{bob}\PY{l+s+s1}{\PYZsq{}}\PY{p}{,} \PY{l+m+mi}{2015}\PY{p}{)}\PY{p}{:} \PY{l+m+mi}{40}\PY{p}{,}
                              \PY{p}{(}\PY{l+s+s1}{\PYZsq{}}\PY{l+s+s1}{bob}\PY{l+s+s1}{\PYZsq{}}\PY{p}{,} \PY{l+m+mi}{2016}\PY{p}{)}\PY{p}{:} \PY{l+m+mi}{40}\PY{p}{\PYZcb{}}\PY{p}{)}
         
         \PY{n+nb}{print}\PY{p}{(}\PY{n}{s\PYZus{}tuple}\PY{p}{)}
\end{Verbatim}


    \begin{Verbatim}[commandchars=\\\{\}]
alice  2014    40
       2015    42
       2016    45
bob    2014    38
       2015    40
       2016    40
dtype: int64

    \end{Verbatim}

    \begin{Verbatim}[commandchars=\\\{\}]
{\color{incolor}In [{\color{incolor}86}]:} \PY{c+c1}{\PYZsh{} ou avec la méthode from\PYZus{}product()}
         \PY{n}{name} \PY{o}{=} \PY{p}{[}\PY{l+s+s1}{\PYZsq{}}\PY{l+s+s1}{alice}\PY{l+s+s1}{\PYZsq{}}\PY{p}{,} \PY{l+s+s1}{\PYZsq{}}\PY{l+s+s1}{bob}\PY{l+s+s1}{\PYZsq{}}\PY{p}{]}
         \PY{n}{year} \PY{o}{=} \PY{p}{[}\PY{l+m+mi}{2014}\PY{p}{,} \PY{l+m+mi}{2015}\PY{p}{,} \PY{l+m+mi}{2016}\PY{p}{]}
         \PY{n}{i} \PY{o}{=} \PY{n}{pd}\PY{o}{.}\PY{n}{MultiIndex}\PY{o}{.}\PY{n}{from\PYZus{}product}\PY{p}{(}\PY{p}{[}\PY{n}{name}\PY{p}{,} \PY{n}{year}\PY{p}{]}\PY{p}{)}
         \PY{n}{s} \PY{o}{=} \PY{n}{pd}\PY{o}{.}\PY{n}{Series}\PY{p}{(}\PY{p}{[}\PY{l+m+mi}{40}\PY{p}{,} \PY{l+m+mi}{42}\PY{p}{,} \PY{l+m+mi}{45}\PY{p}{,} \PY{l+m+mi}{38}\PY{p}{,} \PY{l+m+mi}{40}\PY{p}{,} \PY{l+m+mi}{40}\PY{p}{]}\PY{p}{,} \PY{n}{index}\PY{o}{=}\PY{n}{i}\PY{p}{)}
         \PY{n+nb}{print}\PY{p}{(}\PY{n}{s}\PY{p}{)}
\end{Verbatim}


    \begin{Verbatim}[commandchars=\\\{\}]
alice  2014    40
       2015    42
       2016    45
bob    2014    38
       2015    40
       2016    40
dtype: int64

    \end{Verbatim}

    On peut même nommer les niveaux d'un \texttt{MultiIndex}.

    \begin{Verbatim}[commandchars=\\\{\}]
{\color{incolor}In [{\color{incolor}87}]:} \PY{n}{name} \PY{o}{=} \PY{p}{[}\PY{l+s+s1}{\PYZsq{}}\PY{l+s+s1}{alice}\PY{l+s+s1}{\PYZsq{}}\PY{p}{,} \PY{l+s+s1}{\PYZsq{}}\PY{l+s+s1}{bob}\PY{l+s+s1}{\PYZsq{}}\PY{p}{]}
         \PY{n}{year} \PY{o}{=} \PY{p}{[}\PY{l+m+mi}{2014}\PY{p}{,} \PY{l+m+mi}{2015}\PY{p}{,} \PY{l+m+mi}{2016}\PY{p}{]}
         \PY{n}{i} \PY{o}{=} \PY{n}{pd}\PY{o}{.}\PY{n}{MultiIndex}\PY{o}{.}\PY{n}{from\PYZus{}product}\PY{p}{(}\PY{p}{[}\PY{n}{name}\PY{p}{,} \PY{n}{year}\PY{p}{]}\PY{p}{,} \PY{n}{names}\PY{o}{=}\PY{p}{[}\PY{l+s+s1}{\PYZsq{}}\PY{l+s+s1}{name}\PY{l+s+s1}{\PYZsq{}}\PY{p}{,} \PY{l+s+s1}{\PYZsq{}}\PY{l+s+s1}{year}\PY{l+s+s1}{\PYZsq{}}\PY{p}{]}\PY{p}{)}
         \PY{n}{s} \PY{o}{=} \PY{n}{pd}\PY{o}{.}\PY{n}{Series}\PY{p}{(}\PY{p}{[}\PY{l+m+mi}{40}\PY{p}{,} \PY{l+m+mi}{42}\PY{p}{,} \PY{l+m+mi}{45}\PY{p}{,} \PY{l+m+mi}{38}\PY{p}{,} \PY{l+m+mi}{40}\PY{p}{,} \PY{l+m+mi}{40}\PY{p}{]}\PY{p}{,} \PY{n}{index}\PY{o}{=}\PY{n}{i}\PY{p}{)}
         \PY{n+nb}{print}\PY{p}{(}\PY{n}{s}\PY{p}{)}
\end{Verbatim}


    \begin{Verbatim}[commandchars=\\\{\}]
name   year
alice  2014    40
       2015    42
       2016    45
bob    2014    38
       2015    40
       2016    40
dtype: int64

    \end{Verbatim}

    \begin{Verbatim}[commandchars=\\\{\}]
{\color{incolor}In [{\color{incolor}88}]:} \PY{c+c1}{\PYZsh{} on peut changer le nom des niveaux du MultiIndex}
         \PY{n}{s}\PY{o}{.}\PY{n}{index}\PY{o}{.}\PY{n}{names} \PY{o}{=} \PY{p}{[}\PY{l+s+s1}{\PYZsq{}}\PY{l+s+s1}{NAMES}\PY{l+s+s1}{\PYZsq{}}\PY{p}{,} \PY{l+s+s1}{\PYZsq{}}\PY{l+s+s1}{YEARS}\PY{l+s+s1}{\PYZsq{}}\PY{p}{]}
         \PY{n+nb}{print}\PY{p}{(}\PY{n}{s}\PY{p}{)}
\end{Verbatim}


    \begin{Verbatim}[commandchars=\\\{\}]
NAMES  YEARS
alice  2014     40
       2015     42
       2016     45
bob    2014     38
       2015     40
       2016     40
dtype: int64

    \end{Verbatim}

    Créons maintenant une \texttt{DataFrame} jouet avec des
\texttt{MultiIndex} pour étudier comment accéder aux éléments de la
\texttt{DataFrame}.

    \begin{Verbatim}[commandchars=\\\{\}]
{\color{incolor}In [{\color{incolor}89}]:} \PY{n}{index} \PY{o}{=} \PY{n}{pd}\PY{o}{.}\PY{n}{MultiIndex}\PY{o}{.}\PY{n}{from\PYZus{}product}\PY{p}{(}\PY{p}{[}\PY{p}{[}\PY{l+m+mi}{2013}\PY{p}{,} \PY{l+m+mi}{2014}\PY{p}{]}\PY{p}{,}
                                             \PY{p}{[}\PY{l+m+mi}{1}\PY{p}{,} \PY{l+m+mi}{2}\PY{p}{,} \PY{l+m+mi}{3}\PY{p}{]}\PY{p}{]}\PY{p}{,}
                                            \PY{n}{names}\PY{o}{=}\PY{p}{[}\PY{l+s+s1}{\PYZsq{}}\PY{l+s+s1}{year}\PY{l+s+s1}{\PYZsq{}}\PY{p}{,}
                                                   \PY{l+s+s1}{\PYZsq{}}\PY{l+s+s1}{visit}\PY{l+s+s1}{\PYZsq{}}\PY{p}{]}\PY{p}{)}
         
         \PY{n}{columns} \PY{o}{=} \PY{n}{pd}\PY{o}{.}\PY{n}{MultiIndex}\PY{o}{.}\PY{n}{from\PYZus{}product}\PY{p}{(}\PY{p}{[}\PY{p}{[}\PY{l+s+s1}{\PYZsq{}}\PY{l+s+s1}{Bob}\PY{l+s+s1}{\PYZsq{}}\PY{p}{,} \PY{l+s+s1}{\PYZsq{}}\PY{l+s+s1}{Sue}\PY{l+s+s1}{\PYZsq{}}\PY{p}{]}\PY{p}{,}
                                               \PY{p}{[}\PY{l+s+s1}{\PYZsq{}}\PY{l+s+s1}{avant}\PY{l+s+s1}{\PYZsq{}}\PY{p}{,} \PY{l+s+s1}{\PYZsq{}}\PY{l+s+s1}{arrière}\PY{l+s+s1}{\PYZsq{}}\PY{p}{]}\PY{p}{]}\PY{p}{,}
                                              \PY{n}{names}\PY{o}{=}\PY{p}{[}\PY{l+s+s1}{\PYZsq{}}\PY{l+s+s1}{client}\PY{l+s+s1}{\PYZsq{}}\PY{p}{,}
                                                     \PY{l+s+s1}{\PYZsq{}}\PY{l+s+s1}{pression}\PY{l+s+s1}{\PYZsq{}}\PY{p}{]}\PY{p}{)}
         
         \PY{c+c1}{\PYZsh{} on crée des pressions de pneus factices}
         \PY{n}{data} \PY{o}{=} \PY{l+m+mi}{2} \PY{o}{+} \PY{n}{np}\PY{o}{.}\PY{n}{random}\PY{o}{.}\PY{n}{rand}\PY{p}{(}\PY{l+m+mi}{6}\PY{p}{,} \PY{l+m+mi}{4}\PY{p}{)}
         
         \PY{c+c1}{\PYZsh{} on crée la DataFrame}
         \PY{n}{mecanics\PYZus{}data} \PY{o}{=} \PY{n}{pd}\PY{o}{.}\PY{n}{DataFrame}\PY{p}{(}\PY{n}{data}\PY{p}{,} \PY{n}{index}\PY{o}{=}\PY{n}{index}\PY{p}{,} \PY{n}{columns}\PY{o}{=}\PY{n}{columns}\PY{p}{)}
         \PY{n+nb}{print}\PY{p}{(}\PY{n}{mecanics\PYZus{}data}\PY{p}{)}
\end{Verbatim}


    \begin{Verbatim}[commandchars=\\\{\}]
client           Bob                 Sue          
pression       avant   arrière     avant   arrière
year visit                                        
2013 1      2.550825  2.403969  2.861040  2.734954
     2      2.663264  2.062700  2.438937  2.955536
     3      2.580402  2.668350  2.786914  2.033058
2014 1      2.037373  2.083491  2.930867  2.818915
     2      2.844628  2.930757  2.925297  2.666986
     3      2.865377  2.076819  2.721758  2.260890

    \end{Verbatim}

    Il y a plusieurs manières d'accéder aux éléments, mais une seule que
l'on recommande~:

\textbf{utilisez la notation
\texttt{.loc{[}ligne,\ colonne{]},\ .iloc{[}ligne,\ colonne{]}}}.

    \begin{Verbatim}[commandchars=\\\{\}]
{\color{incolor}In [{\color{incolor}90}]:} \PY{c+c1}{\PYZsh{} pression en 2013 pour Bob}
         \PY{n}{mecanics\PYZus{}data}\PY{o}{.}\PY{n}{loc}\PY{p}{[}\PY{l+m+mi}{2013}\PY{p}{,} \PY{l+s+s1}{\PYZsq{}}\PY{l+s+s1}{Bob}\PY{l+s+s1}{\PYZsq{}}\PY{p}{]}
\end{Verbatim}


\begin{Verbatim}[commandchars=\\\{\}]
{\color{outcolor}Out[{\color{outcolor}90}]:} pression     avant   arrière
         visit                       
         1         2.550825  2.403969
         2         2.663264  2.062700
         3         2.580402  2.668350
\end{Verbatim}
            
    \begin{Verbatim}[commandchars=\\\{\}]
{\color{incolor}In [{\color{incolor}91}]:} \PY{c+c1}{\PYZsh{} pour accéder aux sous niveaux du MultiIndex, on utilise des tuples}
         \PY{n}{mecanics\PYZus{}data}\PY{o}{.}\PY{n}{loc}\PY{p}{[}\PY{p}{(}\PY{l+m+mi}{2013}\PY{p}{,} \PY{l+m+mi}{2}\PY{p}{)}\PY{p}{,} \PY{p}{(}\PY{l+s+s1}{\PYZsq{}}\PY{l+s+s1}{Bob}\PY{l+s+s1}{\PYZsq{}}\PY{p}{,} \PY{l+s+s1}{\PYZsq{}}\PY{l+s+s1}{avant}\PY{l+s+s1}{\PYZsq{}}\PY{p}{)}\PY{p}{]}
\end{Verbatim}


\begin{Verbatim}[commandchars=\\\{\}]
{\color{outcolor}Out[{\color{outcolor}91}]:} 2.6632638263356445
\end{Verbatim}
            
    Le slice sur le \texttt{MultiIndex} est un peu délicat. On peut utiliser
la notation \texttt{:} si on veut slicer sur tous les éléments d'un
\texttt{MultiIndex}, sans prendre en compte un niveau. Si on spécifie
les niveaux, il faut utiliser un objet \texttt{slice} ou
\texttt{pd.IndexSlice}~:

    \begin{Verbatim}[commandchars=\\\{\}]
{\color{incolor}In [{\color{incolor}92}]:} \PY{c+c1}{\PYZsh{} slice(None) signifie tous les éléments du niveau}
         \PY{n+nb}{print}\PY{p}{(}\PY{n}{mecanics\PYZus{}data}\PY{o}{.}\PY{n}{loc}\PY{p}{[}\PY{n+nb}{slice}\PY{p}{(}\PY{p}{(}\PY{l+m+mi}{2013}\PY{p}{,} \PY{l+m+mi}{2}\PY{p}{)}\PY{p}{,} \PY{p}{(}\PY{l+m+mi}{2014}\PY{p}{,} \PY{l+m+mi}{1}\PY{p}{)}\PY{p}{)}\PY{p}{,} \PY{p}{(}\PY{l+s+s1}{\PYZsq{}}\PY{l+s+s1}{Sue}\PY{l+s+s1}{\PYZsq{}}\PY{p}{,} \PY{n+nb}{slice}\PY{p}{(}\PY{k+kc}{None}\PY{p}{)}\PY{p}{)}\PY{p}{]}\PY{p}{)}
\end{Verbatim}


    \begin{Verbatim}[commandchars=\\\{\}]
client           Sue          
pression       avant   arrière
year visit                    
2013 2      2.438937  2.955536
     3      2.786914  2.033058
2014 1      2.930867  2.818915

    \end{Verbatim}

    \begin{Verbatim}[commandchars=\\\{\}]
{\color{incolor}In [{\color{incolor}93}]:} \PY{c+c1}{\PYZsh{} on peut utiliser la notation : si on ne distingue par les niveaux}
         \PY{n+nb}{print}\PY{p}{(}\PY{n}{mecanics\PYZus{}data}\PY{o}{.}\PY{n}{loc}\PY{p}{[}\PY{p}{(}\PY{n+nb}{slice}\PY{p}{(}\PY{k+kc}{None}\PY{p}{)}\PY{p}{,} \PY{n+nb}{slice}\PY{p}{(}\PY{l+m+mi}{1}\PY{p}{,} \PY{l+m+mi}{2}\PY{p}{)}\PY{p}{)}\PY{p}{,} \PY{p}{:}\PY{p}{]}\PY{p}{)}
\end{Verbatim}


    \begin{Verbatim}[commandchars=\\\{\}]
client           Bob                 Sue          
pression       avant   arrière     avant   arrière
year visit                                        
2013 1      2.550825  2.403969  2.861040  2.734954
     2      2.663264  2.062700  2.438937  2.955536
2014 1      2.037373  2.083491  2.930867  2.818915
     2      2.844628  2.930757  2.925297  2.666986

    \end{Verbatim}

    \begin{Verbatim}[commandchars=\\\{\}]
{\color{incolor}In [{\color{incolor}94}]:} \PY{c+c1}{\PYZsh{} on peut aussi utiliser pd.IndexSlice pour slicer avec une notation}
         \PY{c+c1}{\PYZsh{} un peu plus concise}
         \PY{n}{idx} \PY{o}{=} \PY{n}{pd}\PY{o}{.}\PY{n}{IndexSlice}
         \PY{n+nb}{print}\PY{p}{(}\PY{n}{mecanics\PYZus{}data}\PY{o}{.}\PY{n}{loc}\PY{p}{[}\PY{n}{idx}\PY{p}{[}\PY{p}{:}\PY{p}{,} \PY{l+m+mi}{1}\PY{p}{:}\PY{l+m+mi}{2}\PY{p}{]}\PY{p}{,} \PY{n}{idx}\PY{p}{[}\PY{l+s+s1}{\PYZsq{}}\PY{l+s+s1}{Sue}\PY{l+s+s1}{\PYZsq{}}\PY{p}{,} \PY{p}{:}\PY{p}{]}\PY{p}{]}\PY{p}{)}
\end{Verbatim}


    \begin{Verbatim}[commandchars=\\\{\}]
client           Sue          
pression       avant   arrière
year visit                    
2013 1      2.861040  2.734954
     2      2.438937  2.955536
2014 1      2.930867  2.818915
     2      2.925297  2.666986

    \end{Verbatim}

    Pour aller plus loin, regardez la documentation des
\texttt{MultiIndex}~:

http://pandas.pydata.org/pandas-docs/stable/advanced.html

    \hypertarget{conclusion}{%
\subsection{Conclusion}\label{conclusion}}

    La \texttt{DataFrame} est la structure de données la plus souple et la
plus puissante de \texttt{pandas}. Nous avons vu comment créer des
\texttt{DataFrame} et comment accéder aux éléments. Nous verrons dans le
prochain complément les techniques permettant de faire des opérations
complexes (et proches dans l'esprit de ce que l'on peut faire avec une
base de données) comme les opérations de \texttt{merge} ou de
\texttt{groupby}.


    % Add a bibliography block to the postdoc
    
    
    
