    \hypertarget{isinstance}{%
\section{\texorpdfstring{\texttt{isinstance}}{isinstance}}\label{isinstance}}

    \hypertarget{compluxe9ment---niveau-basique}{%
\subsection{Complément - niveau
basique}\label{compluxe9ment---niveau-basique}}

    \hypertarget{typage-dynamique}{%
\subsubsection{Typage dynamique}\label{typage-dynamique}}

    En première semaine, nous avons rapidement mentionné les concepts de
typage statique et dynamique.\\

Avec la fonction prédéfinie \texttt{isinstance} - qui peut être par
ailleurs utile dans d'autres contextes - vous pouvez facilement~:

\begin{itemize}
	\item 
	vérifier qu'un argument d'une fonction a bien le type attendu,
	\item
	traiter différemment les entrées selon leur type.
\end{itemize}

    Voyons tout de suite sur un exemple simple comment on pourrait définir
une fonction qui travaille sur un entier, mais qui par commodité peut
aussi accepter un entier passé comme une chaîne de caractères, ou même
une liste d'entiers (auquel cas on renvoie la liste des factorielles)~:

    \begin{Verbatim}[commandchars=\\\{\}]
{\color{incolor}In [{\color{incolor}1}]:} \PY{k}{def} \PY{n+nf}{factoriel}\PY{p}{(}\PY{n}{argument}\PY{p}{)}\PY{p}{:}
            \PY{c+c1}{\PYZsh{} si on reçoit un entier}
            \PY{k}{if} \PY{n+nb}{isinstance}\PY{p}{(}\PY{n}{argument}\PY{p}{,} \PY{n+nb}{int}\PY{p}{)}\PY{p}{:}              \PY{c+c1}{\PYZsh{} (*)}
                \PY{k}{return} \PY{l+m+mi}{1} \PY{k}{if} \PY{n}{argument} \PY{o}{\PYZlt{}}\PY{o}{=} \PY{l+m+mi}{1} \PY{k}{else} \PY{n}{argument} \PY{o}{*} \PY{n}{factoriel}\PY{p}{(}\PY{n}{argument} \PY{o}{\PYZhy{}} \PY{l+m+mi}{1}\PY{p}{)}
            \PY{c+c1}{\PYZsh{} convertir en entier si on reçoit une chaîne}
            \PY{k}{elif} \PY{n+nb}{isinstance}\PY{p}{(}\PY{n}{argument}\PY{p}{,} \PY{n+nb}{str}\PY{p}{)}\PY{p}{:}
                \PY{k}{return} \PY{n}{factoriel}\PY{p}{(}\PY{n+nb}{int}\PY{p}{(}\PY{n}{argument}\PY{p}{)}\PY{p}{)}
            \PY{c+c1}{\PYZsh{} la liste des résultats si on reçoit un tuple ou une liste }
            \PY{k}{elif} \PY{n+nb}{isinstance}\PY{p}{(}\PY{n}{argument}\PY{p}{,} \PY{p}{(}\PY{n+nb}{tuple}\PY{p}{,} \PY{n+nb}{list}\PY{p}{)}\PY{p}{)}\PY{p}{:}  \PY{c+c1}{\PYZsh{} (**)}
                \PY{k}{return} \PY{p}{[}\PY{n}{factoriel}\PY{p}{(}\PY{n}{i}\PY{p}{)} \PY{k}{for} \PY{n}{i} \PY{o+ow}{in} \PY{n}{argument}\PY{p}{]}
            \PY{c+c1}{\PYZsh{} sinon on lève une exception}
            \PY{k}{else}\PY{p}{:}
                \PY{k}{raise} \PY{n+ne}{TypeError}\PY{p}{(}\PY{n}{argument}\PY{p}{)}
\end{Verbatim}


    \begin{Verbatim}[commandchars=\\\{\}]
{\color{incolor}In [{\color{incolor}2}]:} \PY{n+nb}{print}\PY{p}{(}\PY{l+s+s2}{\PYZdq{}}\PY{l+s+s2}{entier}\PY{l+s+s2}{\PYZdq{}}\PY{p}{,} \PY{n}{factoriel}\PY{p}{(}\PY{l+m+mi}{4}\PY{p}{)}\PY{p}{)}
        \PY{n+nb}{print}\PY{p}{(}\PY{l+s+s2}{\PYZdq{}}\PY{l+s+s2}{chaine}\PY{l+s+s2}{\PYZdq{}}\PY{p}{,} \PY{n}{factoriel}\PY{p}{(}\PY{l+s+s2}{\PYZdq{}}\PY{l+s+s2}{8}\PY{l+s+s2}{\PYZdq{}}\PY{p}{)}\PY{p}{)}
        \PY{n+nb}{print}\PY{p}{(}\PY{l+s+s2}{\PYZdq{}}\PY{l+s+s2}{tuple}\PY{l+s+s2}{\PYZdq{}}\PY{p}{,} \PY{n}{factoriel}\PY{p}{(}\PY{p}{(}\PY{l+m+mi}{4}\PY{p}{,} \PY{l+m+mi}{8}\PY{p}{)}\PY{p}{)}\PY{p}{)}
\end{Verbatim}


    \begin{Verbatim}[commandchars=\\\{\}]
entier 24
chaine 40320
tuple [24, 40320]

    \end{Verbatim}

    Remarquez que la fonction \texttt{isinstance} \textbf{possède elle-même}
une logique de ce genre, puisqu'en ligne 3 \texttt{(*)} nous lui avons
passé en deuxième argument un type (\texttt{int}), alors qu'en ligne 11
\texttt{(**)} on lui a passé un tuple de deux types. Dans ce second cas
naturellement, elle vérifie si l'objet (le premier argument) est
\textbf{de l'un des types} mentionnés dans le tuple.

    \hypertarget{compluxe9ment---niveau-intermuxe9diaire}{%
\subsection{Complément - niveau
intermédiaire}\label{compluxe9ment---niveau-intermuxe9diaire}}

    \hypertarget{le-module-types}{%
\subsubsection{\texorpdfstring{Le module
\texttt{types}}{Le module types}}\label{le-module-types}}

    Le module \texttt{types} définit un certain nombre de constantes qui
peuvent être utiles dans ce contexte - vous trouverez une liste
exhaustive à la fin de ce notebook. Par exemple~:

    \begin{Verbatim}[commandchars=\\\{\}]
{\color{incolor}In [{\color{incolor}3}]:} \PY{k+kn}{from} \PY{n+nn}{types} \PY{k}{import} \PY{n}{FunctionType}
        \PY{n+nb}{isinstance}\PY{p}{(}\PY{n}{factoriel}\PY{p}{,} \PY{n}{FunctionType}\PY{p}{)}
\end{Verbatim}


\begin{Verbatim}[commandchars=\\\{\}]
{\color{outcolor}Out[{\color{outcolor}3}]:} True
\end{Verbatim}
            
    Mais méfiez-vous toutefois des fonctions \emph{built-in}, qui sont de
type \texttt{BuiltinFunctionType}

    \begin{Verbatim}[commandchars=\\\{\}]
{\color{incolor}In [{\color{incolor}4}]:} \PY{k+kn}{from} \PY{n+nn}{types} \PY{k}{import} \PY{n}{BuiltinFunctionType}
        \PY{n+nb}{isinstance}\PY{p}{(}\PY{n+nb}{len}\PY{p}{,} \PY{n}{BuiltinFunctionType}\PY{p}{)}
\end{Verbatim}


\begin{Verbatim}[commandchars=\\\{\}]
{\color{outcolor}Out[{\color{outcolor}4}]:} True
\end{Verbatim}
            
    \begin{Verbatim}[commandchars=\\\{\}]
{\color{incolor}In [{\color{incolor}5}]:} \PY{c+c1}{\PYZsh{} alors qu\PYZsq{}on pourrait penser que}
        \PY{n+nb}{isinstance}\PY{p}{(}\PY{n+nb}{len}\PY{p}{,} \PY{n}{FunctionType}\PY{p}{)}
\end{Verbatim}


\begin{Verbatim}[commandchars=\\\{\}]
{\color{outcolor}Out[{\color{outcolor}5}]:} False
\end{Verbatim}
            
    \hypertarget{isinstance-vs-type}{%
\subsubsection{\texorpdfstring{\texttt{isinstance} \emph{vs}
\texttt{type}}{isinstance vs type}}\label{isinstance-vs-type}}

    Il est recommandé d'utiliser \texttt{isinstance} par rapport à la
fonction \texttt{type}. Tout d'abord, cela permet, on vient de le voir,
de prendre en compte plusieurs types.\\

Mais aussi et surtout \texttt{isinstance} supporte la notion d'héritage
qui est centrale dans le cadre de la programmation orientée objet, sur
laquelle nous allons anticiper un tout petit peu par rapport aux
présentations de la semaine prochaine.\\

Avec la programmation objet, vous pouvez définir vos propres types. On
peut par exemple définir une classe \texttt{Animal} qui convient pour
tous les animaux, puis définir une sous-classe \texttt{Mammifere}. On
dit que la classe \texttt{Mammifere} \emph{hérite} de la classe
\texttt{Animal}, et on l'appelle sous-classe parce qu'elle représente
une partie des animaux ; et donc tout ce qu'on peut faire sur les
animaux peut être fait sur les mammifères.\\

En voici une implémentation très rudimentaire, uniquement pour illustrer
le principe de l'héritage. Si ce qui suit vous semble difficile à
comprendre, pas d'inquiétude, nous reviendrons sur ce sujet lorsque nous
parlerons des classes.

    \begin{Verbatim}[commandchars=\\\{\}]
{\color{incolor}In [{\color{incolor}6}]:} \PY{k}{class} \PY{n+nc}{Animal}\PY{p}{:}
            \PY{k}{def} \PY{n+nf}{\PYZus{}\PYZus{}init\PYZus{}\PYZus{}}\PY{p}{(}\PY{n+nb+bp}{self}\PY{p}{,} \PY{n}{name}\PY{p}{)}\PY{p}{:}
                \PY{n+nb+bp}{self}\PY{o}{.}\PY{n}{name} \PY{o}{=} \PY{n}{name}
        
        \PY{k}{class} \PY{n+nc}{Mammifere}\PY{p}{(}\PY{n}{Animal}\PY{p}{)}\PY{p}{:}
            \PY{k}{def} \PY{n+nf}{\PYZus{}\PYZus{}init\PYZus{}\PYZus{}}\PY{p}{(}\PY{n+nb+bp}{self}\PY{p}{,} \PY{n}{name}\PY{p}{)}\PY{p}{:}
                \PY{n}{Animal}\PY{o}{.}\PY{n+nf+fm}{\PYZus{}\PYZus{}init\PYZus{}\PYZus{}}\PY{p}{(}\PY{n+nb+bp}{self}\PY{p}{,} \PY{n}{name}\PY{p}{)}
\end{Verbatim}


    Ce qui nous intéresse dans l'immédiat c'est que \texttt{isinstance}
permet dans ce contexte de faire des choses qu'on ne peut pas faire
directement avec la fonction \texttt{type}, comme ceci~:

    \begin{Verbatim}[commandchars=\\\{\}]
{\color{incolor}In [{\color{incolor}7}]:} \PY{c+c1}{\PYZsh{} c\PYZsq{}est comme ceci qu\PYZsq{}on peut créer un objet de type `Animal` (méthode \PYZus{}\PYZus{}init\PYZus{}\PYZus{})}
        \PY{n}{requin} \PY{o}{=} \PY{n}{Animal}\PY{p}{(}\PY{l+s+s1}{\PYZsq{}}\PY{l+s+s1}{requin}\PY{l+s+s1}{\PYZsq{}}\PY{p}{)}
        \PY{c+c1}{\PYZsh{} idem pour un Mammifere}
        \PY{n}{baleine} \PY{o}{=} \PY{n}{Mammifere}\PY{p}{(}\PY{l+s+s1}{\PYZsq{}}\PY{l+s+s1}{baleine}\PY{l+s+s1}{\PYZsq{}}\PY{p}{)}
        
        \PY{c+c1}{\PYZsh{} bien sûr ici la réponse est \PYZsq{}True\PYZsq{}}
        \PY{n+nb}{print}\PY{p}{(}\PY{l+s+s2}{\PYZdq{}}\PY{l+s+s2}{l}\PY{l+s+s2}{\PYZsq{}}\PY{l+s+s2}{objet baleine est\PYZhy{}il un mammifere ?}\PY{l+s+s2}{\PYZdq{}}\PY{p}{,} \PY{n+nb}{isinstance}\PY{p}{(}\PY{n}{baleine}\PY{p}{,} \PY{n}{Mammifere}\PY{p}{)}\PY{p}{)}
\end{Verbatim}


    \begin{Verbatim}[commandchars=\\\{\}]
l'objet baleine est-il un mammifere ? True

    \end{Verbatim}

    \begin{Verbatim}[commandchars=\\\{\}]
{\color{incolor}In [{\color{incolor}8}]:} \PY{c+c1}{\PYZsh{} ici c\PYZsq{}est moins évident, mais la réponse est \PYZsq{}True\PYZsq{} aussi}
        \PY{n+nb}{print}\PY{p}{(}\PY{l+s+s2}{\PYZdq{}}\PY{l+s+s2}{l}\PY{l+s+s2}{\PYZsq{}}\PY{l+s+s2}{objet baleine est\PYZhy{}il un animal ?}\PY{l+s+s2}{\PYZdq{}}\PY{p}{,} \PY{n+nb}{isinstance}\PY{p}{(}\PY{n}{baleine}\PY{p}{,} \PY{n}{Animal}\PY{p}{)}\PY{p}{)}
\end{Verbatim}


    \begin{Verbatim}[commandchars=\\\{\}]
l'objet baleine est-il un animal ? True

    \end{Verbatim}

    Vous voyez qu'ici, bien que l'objet baleine soit de type
\texttt{Mammifere}, on peut le considérer comme étant \textbf{aussi} de
type \texttt{Animal}.\\

Ceci est motivé de la façon suivante~: comme on l'a dit plus haut, tout
ce qu'on peut faire (en matière notamment d'envoi de méthodes) sur un
objet de type \texttt{Animal}, on peut le faire sur un objet de type
\texttt{Mammifere}. Dit en termes ensemblistes, l'ensemble des
mammifères est inclus dans l'ensemble des animaux.

    \hypertarget{annexe---les-symboles-du-module-types}{%
\subsubsection{\texorpdfstring{Annexe - Les symboles du module
\texttt{types}}{Annexe - Les symboles du module types}}\label{annexe---les-symboles-du-module-types}}

    Vous pouvez consulter
\href{https://docs.python.org/3/library/types.html}{la documentation du
module \texttt{types}}.

    \begin{Verbatim}[commandchars=\\\{\}]
{\color{incolor}In [{\color{incolor}9}]:} \PY{c+c1}{\PYZsh{} voici par ailleurs la liste de ses attributs}
        \PY{k+kn}{import} \PY{n+nn}{types} 
        \PY{n+nb}{dir}\PY{p}{(}\PY{n}{types}\PY{p}{)}
\end{Verbatim}


\begin{Verbatim}[commandchars=\\\{\}]
{\color{outcolor}Out[{\color{outcolor}9}]:} ['AsyncGeneratorType',
         'BuiltinFunctionType',
         'BuiltinMethodType',
         'CodeType',
         'CoroutineType',
         'DynamicClassAttribute',
         'FrameType',
         'FunctionType',
         'GeneratorType',
         'GetSetDescriptorType',
         'LambdaType',
         'MappingProxyType',
         'MemberDescriptorType',
         'MethodType',
         'ModuleType',
         'SimpleNamespace',
         'TracebackType',
         '\_GeneratorWrapper',
         '\_\_all\_\_',
         '\_\_builtins\_\_',
         '\_\_cached\_\_',
         '\_\_doc\_\_',
         '\_\_file\_\_',
         '\_\_loader\_\_',
         '\_\_name\_\_',
         '\_\_package\_\_',
         '\_\_spec\_\_',
         '\_ag',
         '\_calculate\_meta',
         '\_collections\_abc',
         '\_functools',
         'coroutine',
         'new\_class',
         'prepare\_class']
\end{Verbatim}