    
    
    
    

    

    \hypertarget{suxe9ries-temporelles-en-pandas}{%
\section{\texorpdfstring{Séries temporelles en
\texttt{pandas}}{Séries temporelles en pandas}}\label{suxe9ries-temporelles-en-pandas}}

    \hypertarget{compluxe9ment---niveau-intermuxe9diaire}{%
\subsection{Complément - niveau
intermédiaire}\label{compluxe9ment---niveau-intermuxe9diaire}}

    \hypertarget{parsing-des-dates-et-gestion-des-erreurs}{%
\subsubsection{Parsing des dates et gestion des
erreurs}\label{parsing-des-dates-et-gestion-des-erreurs}}

    Lorsqu'il y a des erreurs de parsing des dates, pandas offre la
possibilité de lancer une exception ou de produire un objet \texttt{NaT}
pour \emph{Not a Number} qui se manipule ensuite comme un \texttt{NaN}.

    \begin{Verbatim}[commandchars=\\\{\},frame=single,framerule=0.3mm,rulecolor=\color{cellframecolor}]
{\color{incolor}In [{\color{incolor}1}]:} \PY{k+kn}{import} \PY{n+nn}{pandas} \PY{k}{as} \PY{n+nn}{pd}
        \PY{n}{date} \PY{o}{=} \PY{l+s+s1}{\PYZsq{}}\PY{l+s+s1}{100/06/2018}\PY{l+s+s1}{\PYZsq{}} \PY{c+c1}{\PYZsh{} cette date ne peut pas être parsée}
        
        \PY{k}{try}\PY{p}{:}
            \PY{n}{pd}\PY{o}{.}\PY{n}{to\PYZus{}datetime}\PY{p}{(}\PY{n}{date}\PY{p}{)} \PY{c+c1}{\PYZsh{} comportement pas défaut qui lance une exception}
        \PY{k}{except} \PY{n+ne}{ValueError} \PY{k}{as} \PY{n}{e}\PY{p}{:}
            \PY{n+nb}{print}\PY{p}{(}\PY{n}{e}\PY{p}{)}
\end{Verbatim}


    \begin{Verbatim}[commandchars=\\\{\},frame=single,framerule=0.3mm,rulecolor=\color{cellframecolor}]
('Unknown string format:', '100/06/2018')
\end{Verbatim}

    \begin{Verbatim}[commandchars=\\\{\},frame=single,framerule=0.3mm,rulecolor=\color{cellframecolor}]
{\color{incolor}In [{\color{incolor}2}]:} \PY{c+c1}{\PYZsh{} retourne l\PYZsq{}input en cas d\PYZsq{}erreur}
        \PY{n}{pd}\PY{o}{.}\PY{n}{to\PYZus{}datetime}\PY{p}{(}\PY{n}{date}\PY{p}{,} \PY{n}{errors}\PY{o}{=}\PY{l+s+s1}{\PYZsq{}}\PY{l+s+s1}{ignore}\PY{l+s+s1}{\PYZsq{}}\PY{p}{)}
\end{Verbatim}


\begin{Verbatim}[commandchars=\\\{\},frame=single,framerule=0.3mm,rulecolor=\color{cellframecolor}]
{\color{outcolor}Out[{\color{outcolor}2}]:} '100/06/2018'
\end{Verbatim}
            
    \begin{Verbatim}[commandchars=\\\{\},frame=single,framerule=0.3mm,rulecolor=\color{cellframecolor}]
{\color{incolor}In [{\color{incolor}3}]:} \PY{c+c1}{\PYZsh{} retourne NaT en cas d\PYZsq{}erreur}
        \PY{n}{pd}\PY{o}{.}\PY{n}{to\PYZus{}datetime}\PY{p}{(}\PY{n}{date}\PY{p}{,} \PY{n}{errors}\PY{o}{=}\PY{l+s+s1}{\PYZsq{}}\PY{l+s+s1}{coerce}\PY{l+s+s1}{\PYZsq{}}\PY{p}{)}
\end{Verbatim}


\begin{Verbatim}[commandchars=\\\{\},frame=single,framerule=0.3mm,rulecolor=\color{cellframecolor}]
{\color{outcolor}Out[{\color{outcolor}3}]:} NaT
\end{Verbatim}
            
    \begin{Verbatim}[commandchars=\\\{\},frame=single,framerule=0.3mm,rulecolor=\color{cellframecolor}]
{\color{incolor}In [{\color{incolor}4}]:} \PY{c+c1}{\PYZsh{} la dernière date n\PYZsq{}est pas valide}
        \PY{n}{d} \PY{o}{=} \PY{n}{pd}\PY{o}{.}\PY{n}{to\PYZus{}datetime}\PY{p}{(}\PY{p}{[}\PY{l+s+s1}{\PYZsq{}}\PY{l+s+s1}{jun 2018}\PY{l+s+s1}{\PYZsq{}}\PY{p}{,} \PY{l+s+s1}{\PYZsq{}}\PY{l+s+s1}{10/12/1980}\PY{l+s+s1}{\PYZsq{}}\PY{p}{,}
                            \PY{l+s+s1}{\PYZsq{}}\PY{l+s+s1}{25 january 2000}\PY{l+s+s1}{\PYZsq{}}\PY{p}{,} \PY{l+s+s1}{\PYZsq{}}\PY{l+s+s1}{100 june 1900}\PY{l+s+s1}{\PYZsq{}}\PY{p}{]}\PY{p}{,} 
                           \PY{n}{errors}\PY{o}{=}\PY{l+s+s1}{\PYZsq{}}\PY{l+s+s1}{coerce}\PY{l+s+s1}{\PYZsq{}}\PY{p}{)}
        \PY{n+nb}{print}\PY{p}{(}\PY{n}{d}\PY{p}{)}
\end{Verbatim}


    \begin{Verbatim}[commandchars=\\\{\},frame=single,framerule=0.3mm,rulecolor=\color{cellframecolor}]
DatetimeIndex(['2018-06-01', '1980-10-12', '2000-01-25', 'NaT'], dtype='datetime64[ns]', freq=None)
\end{Verbatim}

    \begin{Verbatim}[commandchars=\\\{\},frame=single,framerule=0.3mm,rulecolor=\color{cellframecolor}]
{\color{incolor}In [{\color{incolor}5}]:} \PY{c+c1}{\PYZsh{} on peut utiliser les méthodes pour les NaN directement sur un NaT}
        \PY{n}{d}\PY{o}{.}\PY{n}{fillna}\PY{p}{(}\PY{n}{pd}\PY{o}{.}\PY{n}{to\PYZus{}datetime}\PY{p}{(}\PY{l+s+s1}{\PYZsq{}}\PY{l+s+s1}{10 june 1980}\PY{l+s+s1}{\PYZsq{}}\PY{p}{)}\PY{p}{)}
\end{Verbatim}


\begin{Verbatim}[commandchars=\\\{\},frame=single,framerule=0.3mm,rulecolor=\color{cellframecolor}]
{\color{outcolor}Out[{\color{outcolor}5}]:} DatetimeIndex(['2018-06-01', '1980-10-12', '2000-01-25', '1980-06-10'], dtype='datetime64[ns]', freq=None)
\end{Verbatim}
            
    \hypertarget{pour-aller-plus-loin}{%
\subsubsection{Pour aller plus loin}\label{pour-aller-plus-loin}}

    Vous trouverez de nombreux exemples dans la documentation officielle de
pandas :

\begin{itemize}
\tightlist
\item
  https://pandas.pydata.org/pandas-docs/stable/timeseries.html
\end{itemize}

    \hypertarget{conclusion}{%
\subsection{Conclusion}\label{conclusion}}

    Ce notebook clôt notre survol de \texttt{numpy} et \texttt{pandas}.
C'est un sujet vaste que nous avons déjà largement dégrossi. Pour aller
plus loin vous avez évidemment la documentation officielle de
\texttt{numpy} et \texttt{pandas}~:

\begin{itemize}
\tightlist
\item
  https://docs.scipy.org/doc/numpy-1.13.0/reference/
\item
  http://pandas.pydata.org/pandas-docs/stable/index.html
\end{itemize}

    Mais vous avez aussi l'excellent livre de Jake VanderPlas ``Python Data
Science Handbook'' qui est entièrement disponible sous forme de
notebooks en ligne~:

https://github.com/jakevdp/PythonDataScienceHandbook

Il s'agit d'un très beau travail (c'est rare) utilisant les dernières
versions de Python, \texttt{pandas} and \texttt{numpy} (c'est encore
plus rare), fait par un physicien qui fait de la data science et qui a
contribué au développement de nombreux modules de data science en
Python.

Je vous conseille par ailleurs, pour ceux qui sont à l'aise en anglais,
\href{https://www.youtube.com/watch?v=_ZEWDGpM-vM}{une série de 10
vidéos sur YouTube} publiées par le même Jake VanderPlas, où il étudie
un jeu de données du début (chargement des données) à la fin
(classification).

    Pour finir, si vous voulez faire de la data science, il y a un livre
incontournable~: ``An Introduction de Statistical Learning'' de G.
James, D. Witten, T. Hastie, R. Tibshirani. Ce livre utilise R, mais
vous pouvez facilement l'appliquer en utilisant \texttt{pandas}.

Les auteurs mettent à disposition gratuitement le PDF du livre ici~:

http://www-bcf.usc.edu/\textasciitilde{}gareth/ISL/

    N'oubliez pas, si ces ressources vous sont utiles, d'acheter ces livres
pour supporter ces auteurs. Les ressources de grande qualité sont rares,
elles demandent un travail énorme à produire, elles doivent être
encouragées et recompensées.


    % Add a bibliography block to the postdoc
    
    
    
