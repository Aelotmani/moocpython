    
    
    
    

    

    \hypertarget{linstruction-del}{%
\section{\texorpdfstring{L'instruction
\texttt{del}}{L'instruction del}}\label{linstruction-del}}

    \hypertarget{compluxe9ment---niveau-basique}{%
\subsection{Complément - niveau
basique}\label{compluxe9ment---niveau-basique}}

    Voici un récapitulatif sur l'instruction \texttt{del} selon le contexte
dans lequel elle est utilisée.

    \hypertarget{sur-une-variable}{%
\subsubsection{Sur une variable}\label{sur-une-variable}}

    On peut annuler la définition d'une variable, avec \texttt{del}.

Pour l'illustrer, nous utilisons un bloc
\texttt{try\ \ldots{}\ except\ \ldots{}} pour attraper le cas échéant
l'exception \texttt{NameError}, qui est produite lorsqu'on référence une
variable qui n'est pas définie.

    \begin{Verbatim}[commandchars=\\\{\},frame=single,framerule=0.3mm,rulecolor=\color{cellframecolor}]
{\color{incolor}In [{\color{incolor}1}]:} \PY{c+c1}{\PYZsh{} la variable a n\PYZsq{}est pas définie}
        \PY{k}{try}\PY{p}{:}
            \PY{n+nb}{print}\PY{p}{(}\PY{l+s+s1}{\PYZsq{}}\PY{l+s+s1}{a=}\PY{l+s+s1}{\PYZsq{}}\PY{p}{,} \PY{n}{a}\PY{p}{)}
        \PY{k}{except} \PY{n+ne}{NameError} \PY{k}{as} \PY{n}{e}\PY{p}{:}
            \PY{n+nb}{print}\PY{p}{(}\PY{l+s+s2}{\PYZdq{}}\PY{l+s+s2}{a n}\PY{l+s+s2}{\PYZsq{}}\PY{l+s+s2}{est pas définie}\PY{l+s+s2}{\PYZdq{}}\PY{p}{)}
\end{Verbatim}


    \begin{Verbatim}[commandchars=\\\{\},frame=single,framerule=0.3mm,rulecolor=\color{cellframecolor}]
a n'est pas définie
\end{Verbatim}

    \begin{Verbatim}[commandchars=\\\{\},frame=single,framerule=0.3mm,rulecolor=\color{cellframecolor}]
{\color{incolor}In [{\color{incolor}2}]:} \PY{c+c1}{\PYZsh{} on la définit}
        \PY{n}{a} \PY{o}{=} \PY{l+m+mi}{10}
        
        \PY{c+c1}{\PYZsh{} aucun souci ici, l\PYZsq{}exception n\PYZsq{}est pas levée}
        \PY{k}{try}\PY{p}{:}
            \PY{n+nb}{print}\PY{p}{(}\PY{l+s+s1}{\PYZsq{}}\PY{l+s+s1}{a=}\PY{l+s+s1}{\PYZsq{}}\PY{p}{,} \PY{n}{a}\PY{p}{)}
        \PY{k}{except} \PY{n+ne}{NameError} \PY{k}{as} \PY{n}{e}\PY{p}{:}
            \PY{n+nb}{print}\PY{p}{(}\PY{l+s+s2}{\PYZdq{}}\PY{l+s+s2}{a n}\PY{l+s+s2}{\PYZsq{}}\PY{l+s+s2}{est pas définie}\PY{l+s+s2}{\PYZdq{}}\PY{p}{)}
\end{Verbatim}


    \begin{Verbatim}[commandchars=\\\{\},frame=single,framerule=0.3mm,rulecolor=\color{cellframecolor}]
a= 10
\end{Verbatim}

    \begin{Verbatim}[commandchars=\\\{\},frame=single,framerule=0.3mm,rulecolor=\color{cellframecolor}]
{\color{incolor}In [{\color{incolor}3}]:} \PY{c+c1}{\PYZsh{} maintenant on peut effacer la variable}
        \PY{k}{del} \PY{n}{a}
        
        \PY{c+c1}{\PYZsh{} c\PYZsq{}est comme si on ne l\PYZsq{}avait pas définie}
        \PY{c+c1}{\PYZsh{} dans la cellule précédente}
        \PY{k}{try}\PY{p}{:}
            \PY{n+nb}{print}\PY{p}{(}\PY{l+s+s1}{\PYZsq{}}\PY{l+s+s1}{a=}\PY{l+s+s1}{\PYZsq{}}\PY{p}{,} \PY{n}{a}\PY{p}{)}
        \PY{k}{except} \PY{n+ne}{NameError} \PY{k}{as} \PY{n}{e}\PY{p}{:}
            \PY{n+nb}{print}\PY{p}{(}\PY{l+s+s2}{\PYZdq{}}\PY{l+s+s2}{a n}\PY{l+s+s2}{\PYZsq{}}\PY{l+s+s2}{est pas définie}\PY{l+s+s2}{\PYZdq{}}\PY{p}{)}
\end{Verbatim}


    \begin{Verbatim}[commandchars=\\\{\},frame=single,framerule=0.3mm,rulecolor=\color{cellframecolor}]
a n'est pas définie
\end{Verbatim}

    \hypertarget{sur-une-liste}{%
\subsubsection{Sur une liste}\label{sur-une-liste}}

    On peut enlever d'une liste les éléments qui correspondent à une
\emph{slice}~:

    \begin{Verbatim}[commandchars=\\\{\},frame=single,framerule=0.3mm,rulecolor=\color{cellframecolor}]
{\color{incolor}In [{\color{incolor}4}]:} \PY{c+c1}{\PYZsh{} on se donne une liste}
        \PY{n}{l} \PY{o}{=} \PY{n+nb}{list}\PY{p}{(}\PY{n+nb}{range}\PY{p}{(}\PY{l+m+mi}{12}\PY{p}{)}\PY{p}{)}
        \PY{n+nb}{print}\PY{p}{(}\PY{n}{l}\PY{p}{)}
\end{Verbatim}


    \begin{Verbatim}[commandchars=\\\{\},frame=single,framerule=0.3mm,rulecolor=\color{cellframecolor}]
[0, 1, 2, 3, 4, 5, 6, 7, 8, 9, 10, 11]
\end{Verbatim}

    \begin{Verbatim}[commandchars=\\\{\},frame=single,framerule=0.3mm,rulecolor=\color{cellframecolor}]
{\color{incolor}In [{\color{incolor}5}]:} \PY{c+c1}{\PYZsh{} on considère une slice dans cette liste}
        \PY{n+nb}{print}\PY{p}{(}\PY{l+s+s1}{\PYZsq{}}\PY{l+s+s1}{slice=}\PY{l+s+s1}{\PYZsq{}}\PY{p}{,} \PY{n}{l}\PY{p}{[}\PY{l+m+mi}{2}\PY{p}{:}\PY{l+m+mi}{10}\PY{p}{:}\PY{l+m+mi}{3}\PY{p}{]}\PY{p}{)}
        
        \PY{c+c1}{\PYZsh{} voyons ce que ça donne si on efface cette slice}
        \PY{k}{del} \PY{n}{l}\PY{p}{[}\PY{l+m+mi}{2}\PY{p}{:}\PY{l+m+mi}{10}\PY{p}{:}\PY{l+m+mi}{3}\PY{p}{]}
        \PY{n+nb}{print}\PY{p}{(}\PY{l+s+s2}{\PYZdq{}}\PY{l+s+s2}{après del}\PY{l+s+s2}{\PYZdq{}}\PY{p}{,} \PY{n}{l}\PY{p}{)}
\end{Verbatim}


    \begin{Verbatim}[commandchars=\\\{\},frame=single,framerule=0.3mm,rulecolor=\color{cellframecolor}]
slice= [2, 5, 8]
après del [0, 1, 3, 4, 6, 7, 9, 10, 11]
\end{Verbatim}

    \hypertarget{sur-un-dictionnaire}{%
\subsubsection{Sur un dictionnaire}\label{sur-un-dictionnaire}}

    Avec \texttt{del} on peut enlever une clé, et donc la valeur
correspondante, d'un dictionnaire~:

    \begin{Verbatim}[commandchars=\\\{\},frame=single,framerule=0.3mm,rulecolor=\color{cellframecolor}]
{\color{incolor}In [{\color{incolor}6}]:} \PY{c+c1}{\PYZsh{} partons d\PYZsq{}un dictionaire simple}
        \PY{n}{d} \PY{o}{=} \PY{n+nb}{dict}\PY{p}{(}\PY{n}{foo}\PY{o}{=}\PY{l+s+s1}{\PYZsq{}}\PY{l+s+s1}{bar}\PY{l+s+s1}{\PYZsq{}}\PY{p}{,} \PY{n}{spam}\PY{o}{=}\PY{l+s+s1}{\PYZsq{}}\PY{l+s+s1}{eggs}\PY{l+s+s1}{\PYZsq{}}\PY{p}{,} \PY{n}{a}\PY{o}{=}\PY{l+s+s1}{\PYZsq{}}\PY{l+s+s1}{b}\PY{l+s+s1}{\PYZsq{}}\PY{p}{)}
        \PY{n+nb}{print}\PY{p}{(}\PY{n}{d}\PY{p}{)}
\end{Verbatim}


    \begin{Verbatim}[commandchars=\\\{\},frame=single,framerule=0.3mm,rulecolor=\color{cellframecolor}]
\{'foo': 'bar', 'spam': 'eggs', 'a': 'b'\}
\end{Verbatim}

    \begin{Verbatim}[commandchars=\\\{\},frame=single,framerule=0.3mm,rulecolor=\color{cellframecolor}]
{\color{incolor}In [{\color{incolor}7}]:} \PY{c+c1}{\PYZsh{} on peut enlever une clé avec del}
        \PY{k}{del} \PY{n}{d}\PY{p}{[}\PY{l+s+s1}{\PYZsq{}}\PY{l+s+s1}{a}\PY{l+s+s1}{\PYZsq{}}\PY{p}{]}
        \PY{n+nb}{print}\PY{p}{(}\PY{n}{d}\PY{p}{)}
\end{Verbatim}


    \begin{Verbatim}[commandchars=\\\{\},frame=single,framerule=0.3mm,rulecolor=\color{cellframecolor}]
\{'foo': 'bar', 'spam': 'eggs'\}
\end{Verbatim}

    \hypertarget{on-peut-passer-plusieurs-arguments-uxe0-del}{%
\subsubsection{\texorpdfstring{On peut passer plusieurs arguments à
\texttt{del}}{On peut passer plusieurs arguments à del}}\label{on-peut-passer-plusieurs-arguments-uxe0-del}}

    \begin{Verbatim}[commandchars=\\\{\},frame=single,framerule=0.3mm,rulecolor=\color{cellframecolor}]
{\color{incolor}In [{\color{incolor}8}]:} \PY{c+c1}{\PYZsh{} Voyons où en sont nos données}
        \PY{n+nb}{print}\PY{p}{(}\PY{l+s+s1}{\PYZsq{}}\PY{l+s+s1}{l}\PY{l+s+s1}{\PYZsq{}}\PY{p}{,} \PY{n}{l}\PY{p}{)}
        \PY{n+nb}{print}\PY{p}{(}\PY{l+s+s1}{\PYZsq{}}\PY{l+s+s1}{d}\PY{l+s+s1}{\PYZsq{}}\PY{p}{,} \PY{n}{d}\PY{p}{)}
\end{Verbatim}


    \begin{Verbatim}[commandchars=\\\{\},frame=single,framerule=0.3mm,rulecolor=\color{cellframecolor}]
l [0, 1, 3, 4, 6, 7, 9, 10, 11]
d \{'foo': 'bar', 'spam': 'eggs'\}
\end{Verbatim}

    \begin{Verbatim}[commandchars=\\\{\},frame=single,framerule=0.3mm,rulecolor=\color{cellframecolor}]
{\color{incolor}In [{\color{incolor}9}]:} \PY{c+c1}{\PYZsh{} on peut invoquer \PYZsq{}del\PYZsq{} avec plusieurs expressions}
        \PY{c+c1}{\PYZsh{} séparées par une virgule}
        \PY{k}{del} \PY{n}{l}\PY{p}{[}\PY{l+m+mi}{3}\PY{p}{:}\PY{p}{]}\PY{p}{,} \PY{n}{d}\PY{p}{[}\PY{l+s+s1}{\PYZsq{}}\PY{l+s+s1}{spam}\PY{l+s+s1}{\PYZsq{}}\PY{p}{]}
        
        \PY{n+nb}{print}\PY{p}{(}\PY{l+s+s1}{\PYZsq{}}\PY{l+s+s1}{l}\PY{l+s+s1}{\PYZsq{}}\PY{p}{,} \PY{n}{l}\PY{p}{)}
        \PY{n+nb}{print}\PY{p}{(}\PY{l+s+s1}{\PYZsq{}}\PY{l+s+s1}{d}\PY{l+s+s1}{\PYZsq{}}\PY{p}{,} \PY{n}{d}\PY{p}{)}
\end{Verbatim}


    \begin{Verbatim}[commandchars=\\\{\},frame=single,framerule=0.3mm,rulecolor=\color{cellframecolor}]
l [0, 1, 3]
d \{'foo': 'bar'\}
\end{Verbatim}

    \hypertarget{pour-en-savoir-plus}{%
\subsubsection{Pour en savoir plus}\label{pour-en-savoir-plus}}

    La page sur
\href{https://docs.python.org/3/reference/simple_stmts.html\#the-del-statement}{l'instruction
\texttt{del}} dans la documentation Python.


    % Add a bibliography block to the postdoc
    
    
    
