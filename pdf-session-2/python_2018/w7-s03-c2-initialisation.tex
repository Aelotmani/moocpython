    
    
    
    

    

    \hypertarget{cruxe9ation-de-tableaux}{%
\section{Création de tableaux}\label{cruxe9ation-de-tableaux}}

    \hypertarget{compluxe9ment---niveau-basique}{%
\subsection{Complément - niveau
basique}\label{compluxe9ment---niveau-basique}}

    Passons rapidement en revue quelques méthodes pour créer des tableaux
\texttt{numpy}.

    \begin{Verbatim}[commandchars=\\\{\},frame=single,framerule=0.3mm,rulecolor=\color{cellframecolor}]
{\color{incolor}In [{\color{incolor}1}]:} \PY{k+kn}{import} \PY{n+nn}{numpy} \PY{k}{as} \PY{n+nn}{np}
\end{Verbatim}


    \hypertarget{non-initialisuxe9-np.empty}{%
\subsubsection{\texorpdfstring{Non initialisé~:
\texttt{np.empty}}{Non initialisé~: np.empty}}\label{non-initialisuxe9-np.empty}}

    La méthode la plus efficace pour créer un tableau \texttt{numpy}
consiste à faire l'allocation de la mémoire mais sans l'initialiser~:

    \begin{Verbatim}[commandchars=\\\{\},frame=single,framerule=0.3mm,rulecolor=\color{cellframecolor}]
{\color{incolor}In [{\color{incolor}2}]:} \PY{n}{memory} \PY{o}{=} \PY{n}{np}\PY{o}{.}\PY{n}{empty}\PY{p}{(}\PY{n}{dtype}\PY{o}{=}\PY{n}{np}\PY{o}{.}\PY{n}{int8}\PY{p}{,}
                          \PY{n}{shape}\PY{o}{=}\PY{p}{(}\PY{l+m+mi}{1}\PY{n}{\PYZus{}000}\PY{p}{,} \PY{l+m+mi}{1}\PY{n}{\PYZus{}000}\PY{p}{)}\PY{p}{)}
\end{Verbatim}


    J'en profite pour attirer votre attention sur l'impression des gros
tableaux où l'on s'efforce de vous montrer les coins~:

    \begin{Verbatim}[commandchars=\\\{\},frame=single,framerule=0.3mm,rulecolor=\color{cellframecolor}]
{\color{incolor}In [{\color{incolor}3}]:} \PY{n+nb}{print}\PY{p}{(}\PY{n}{memory}\PY{p}{)}
\end{Verbatim}


    \begin{Verbatim}[commandchars=\\\{\},frame=single,framerule=0.3mm,rulecolor=\color{cellframecolor}]
[[0 0 0 {\ldots} 0 0 0]
 [0 0 0 {\ldots} 0 0 0]
 [0 0 0 {\ldots} 0 0 0]
 {\ldots}
 [0 0 0 {\ldots} 0 0 0]
 [0 0 0 {\ldots} 0 0 0]
 [0 0 0 {\ldots} 0 0 0]]
\end{Verbatim}

    Il se \emph{peut} que vous voyiez ici des valeurs particulières~; selon
votre OS, il y a une probabilité non nulle que vous ne voyiez ici que
des zéros. C'est un peu comme avec les dictionnaires qui, depuis la
version 3.6, peuvent donner l'impression de conserver l'ordre dans
lequel les clés ont été créées. Ici c'est un peu la même chose, vous ne
devez pas écrire un programme qui repose sur le fait que
\texttt{np.empty} retourne un tableau garni de zéros (utilisez alors
\texttt{np.zeros}, que l'on va voir tout de suite).

    \hypertarget{tableaux-constants}{%
\subsubsection{Tableaux constants}\label{tableaux-constants}}

    On peut aussi créer et initialiser un tableau avec \texttt{np.zeros} et
\texttt{np.ones}~:

    \begin{Verbatim}[commandchars=\\\{\},frame=single,framerule=0.3mm,rulecolor=\color{cellframecolor}]
{\color{incolor}In [{\color{incolor}4}]:} \PY{n}{zeros} \PY{o}{=} \PY{n}{np}\PY{o}{.}\PY{n}{zeros}\PY{p}{(}\PY{n}{dtype}\PY{o}{=}\PY{n}{np}\PY{o}{.}\PY{n}{complex128}\PY{p}{,} \PY{n}{shape}\PY{o}{=}\PY{p}{(}\PY{l+m+mi}{1}\PY{n}{\PYZus{}000}\PY{p}{,} \PY{l+m+mi}{100}\PY{p}{)}\PY{p}{)}
        \PY{n+nb}{print}\PY{p}{(}\PY{n}{zeros}\PY{p}{)}
\end{Verbatim}


    \begin{Verbatim}[commandchars=\\\{\},frame=single,framerule=0.3mm,rulecolor=\color{cellframecolor}]
[[0.+0.j 0.+0.j 0.+0.j {\ldots} 0.+0.j 0.+0.j 0.+0.j]
 [0.+0.j 0.+0.j 0.+0.j {\ldots} 0.+0.j 0.+0.j 0.+0.j]
 [0.+0.j 0.+0.j 0.+0.j {\ldots} 0.+0.j 0.+0.j 0.+0.j]
 {\ldots}
 [0.+0.j 0.+0.j 0.+0.j {\ldots} 0.+0.j 0.+0.j 0.+0.j]
 [0.+0.j 0.+0.j 0.+0.j {\ldots} 0.+0.j 0.+0.j 0.+0.j]
 [0.+0.j 0.+0.j 0.+0.j {\ldots} 0.+0.j 0.+0.j 0.+0.j]]
\end{Verbatim}

    \begin{Verbatim}[commandchars=\\\{\},frame=single,framerule=0.3mm,rulecolor=\color{cellframecolor}]
{\color{incolor}In [{\color{incolor}5}]:} \PY{n}{fours} \PY{o}{=} \PY{l+m+mi}{4} \PY{o}{*} \PY{n}{np}\PY{o}{.}\PY{n}{ones}\PY{p}{(}\PY{n}{dtype}\PY{o}{=}\PY{n+nb}{float}\PY{p}{,} \PY{n}{shape}\PY{o}{=}\PY{p}{(}\PY{l+m+mi}{8}\PY{p}{,} \PY{l+m+mi}{8}\PY{p}{)}\PY{p}{)}
        \PY{n}{fours}
\end{Verbatim}


\begin{Verbatim}[commandchars=\\\{\},frame=single,framerule=0.3mm,rulecolor=\color{cellframecolor}]
{\color{outcolor}Out[{\color{outcolor}5}]:} array([[4., 4., 4., 4., 4., 4., 4., 4.],
               [4., 4., 4., 4., 4., 4., 4., 4.],
               [4., 4., 4., 4., 4., 4., 4., 4.],
               [4., 4., 4., 4., 4., 4., 4., 4.],
               [4., 4., 4., 4., 4., 4., 4., 4.],
               [4., 4., 4., 4., 4., 4., 4., 4.],
               [4., 4., 4., 4., 4., 4., 4., 4.],
               [4., 4., 4., 4., 4., 4., 4., 4.]])
\end{Verbatim}
            
    \hypertarget{progression-arithmuxe9tique-arange}{%
\subsubsection{\texorpdfstring{Progression arithmétique~:
\texttt{arange}}{Progression arithmétique~: arange}}\label{progression-arithmuxe9tique-arange}}

    En guise de rappel, avec \texttt{arange} on peut créer des tableaux de
valeurs espacées d'une valeur constante. Ça ressemble donc un peu au
\texttt{range} de Python natif~:

    \begin{Verbatim}[commandchars=\\\{\},frame=single,framerule=0.3mm,rulecolor=\color{cellframecolor}]
{\color{incolor}In [{\color{incolor}6}]:} \PY{n}{np}\PY{o}{.}\PY{n}{arange}\PY{p}{(}\PY{l+m+mi}{4}\PY{p}{)}
\end{Verbatim}


\begin{Verbatim}[commandchars=\\\{\},frame=single,framerule=0.3mm,rulecolor=\color{cellframecolor}]
{\color{outcolor}Out[{\color{outcolor}6}]:} array([0, 1, 2, 3])
\end{Verbatim}
            
    \begin{Verbatim}[commandchars=\\\{\},frame=single,framerule=0.3mm,rulecolor=\color{cellframecolor}]
{\color{incolor}In [{\color{incolor}7}]:} \PY{n}{np}\PY{o}{.}\PY{n}{arange}\PY{p}{(}\PY{l+m+mi}{1}\PY{p}{,} \PY{l+m+mi}{5}\PY{p}{)}
\end{Verbatim}


\begin{Verbatim}[commandchars=\\\{\},frame=single,framerule=0.3mm,rulecolor=\color{cellframecolor}]
{\color{outcolor}Out[{\color{outcolor}7}]:} array([1, 2, 3, 4])
\end{Verbatim}
            
    Sauf qu'on peut y passer un pas qui n'est pas entier~:

    \begin{Verbatim}[commandchars=\\\{\},frame=single,framerule=0.3mm,rulecolor=\color{cellframecolor}]
{\color{incolor}In [{\color{incolor}8}]:} \PY{n}{np}\PY{o}{.}\PY{n}{arange}\PY{p}{(}\PY{l+m+mi}{5}\PY{p}{,} \PY{l+m+mi}{7}\PY{p}{,} \PY{o}{.}\PY{l+m+mi}{5}\PY{p}{)}
\end{Verbatim}


\begin{Verbatim}[commandchars=\\\{\},frame=single,framerule=0.3mm,rulecolor=\color{cellframecolor}]
{\color{outcolor}Out[{\color{outcolor}8}]:} array([5. , 5.5, 6. , 6.5])
\end{Verbatim}
            
    \hypertarget{progression-arithmuxe9tique-linspace}{%
\subsubsection{\texorpdfstring{Progression arithmétique~:
\texttt{linspace}}{Progression arithmétique~: linspace}}\label{progression-arithmuxe9tique-linspace}}

    Mais bien souvent, plutôt que de préciser \emph{le pas} entre deux
valeurs, on préfère préciser \emph{le nombre} de points~; et aussi
inclure la deuxième borne. C'est ce que fait \texttt{linspace}, c'est
très utile pour modéliser une fonction sur un intervalle~; on a déjà vu
des exemples de ce genre~:

    \begin{Verbatim}[commandchars=\\\{\},frame=single,framerule=0.3mm,rulecolor=\color{cellframecolor}]
{\color{incolor}In [{\color{incolor}9}]:} \PY{o}{\PYZpc{}}\PY{k}{matplotlib} inline
        \PY{k+kn}{import} \PY{n+nn}{matplotlib}\PY{n+nn}{.}\PY{n+nn}{pyplot} \PY{k}{as} \PY{n+nn}{plt}
        \PY{n}{plt}\PY{o}{.}\PY{n}{ion}\PY{p}{(}\PY{p}{)}
\end{Verbatim}


    \begin{Verbatim}[commandchars=\\\{\},frame=single,framerule=0.3mm,rulecolor=\color{cellframecolor}]
{\color{incolor}In [{\color{incolor}10}]:} \PY{n}{X} \PY{o}{=} \PY{n}{np}\PY{o}{.}\PY{n}{linspace}\PY{p}{(}\PY{o}{\PYZhy{}}\PY{l+m+mf}{3.}\PY{p}{,} \PY{o}{+}\PY{l+m+mf}{3.}\PY{p}{)}
         \PY{n}{Y} \PY{o}{=} \PY{n}{np}\PY{o}{.}\PY{n}{exp}\PY{p}{(}\PY{n}{X}\PY{p}{)}
         
         \PY{n}{plt}\PY{o}{.}\PY{n}{plot}\PY{p}{(}\PY{n}{X}\PY{p}{,} \PY{n}{Y}\PY{p}{)}\PY{p}{;}
\end{Verbatim}


    \begin{center}
    \adjustimage{max size={0.9\linewidth}{0.9\paperheight}}{w7-s03-c2-initialisation_files/w7-s03-c2-initialisation_24_0.png}
    \end{center}
    { \hspace*{\fill} \\}
    
    \begin{Verbatim}[commandchars=\\\{\},frame=single,framerule=0.3mm,rulecolor=\color{cellframecolor}]
{\color{incolor}In [{\color{incolor}11}]:} \PY{c+c1}{\PYZsh{} par défaut linspace crée 50 points}
         \PY{c+c1}{\PYZsh{} avec moins de points}
         
         \PY{n}{X} \PY{o}{=} \PY{n}{np}\PY{o}{.}\PY{n}{linspace}\PY{p}{(}\PY{l+m+mi}{1}\PY{o}{/}\PY{l+m+mi}{10}\PY{p}{,} \PY{l+m+mi}{10}\PY{p}{,} \PY{n}{num} \PY{o}{=} \PY{l+m+mi}{5}\PY{p}{)}
         \PY{n}{plt}\PY{o}{.}\PY{n}{plot}\PY{p}{(}\PY{n}{X}\PY{p}{,} \PY{n}{np}\PY{o}{.}\PY{n}{log}\PY{p}{(}\PY{n}{X}\PY{p}{)}\PY{p}{)}\PY{p}{;}
\end{Verbatim}


    \begin{center}
    \adjustimage{max size={0.9\linewidth}{0.9\paperheight}}{w7-s03-c2-initialisation_files/w7-s03-c2-initialisation_25_0.png}
    \end{center}
    { \hspace*{\fill} \\}
    
    Pour des intervalles en progression géométrique, voyez
\texttt{np.geomspace}.

    \hypertarget{multi-dimensions-indices}{%
\subsubsection{\texorpdfstring{Multi-dimensions~:
\texttt{indices}}{Multi-dimensions~: indices}}\label{multi-dimensions-indices}}

    La méthode \texttt{np.indices} se comporte un peu comme \texttt{arange}
mais pour plusieurs directions~; voyons ça sur un exemple~:

    \begin{Verbatim}[commandchars=\\\{\},frame=single,framerule=0.3mm,rulecolor=\color{cellframecolor}]
{\color{incolor}In [{\color{incolor}12}]:} \PY{n}{ix}\PY{p}{,} \PY{n}{iy} \PY{o}{=} \PY{n}{np}\PY{o}{.}\PY{n}{indices}\PY{p}{(}\PY{p}{(}\PY{l+m+mi}{3}\PY{p}{,} \PY{l+m+mi}{5}\PY{p}{)}\PY{p}{)}
\end{Verbatim}


    \begin{Verbatim}[commandchars=\\\{\},frame=single,framerule=0.3mm,rulecolor=\color{cellframecolor}]
{\color{incolor}In [{\color{incolor}13}]:} \PY{n}{ix}
\end{Verbatim}


\begin{Verbatim}[commandchars=\\\{\},frame=single,framerule=0.3mm,rulecolor=\color{cellframecolor}]
{\color{outcolor}Out[{\color{outcolor}13}]:} array([[0, 0, 0, 0, 0],
                [1, 1, 1, 1, 1],
                [2, 2, 2, 2, 2]])
\end{Verbatim}
            
    \begin{Verbatim}[commandchars=\\\{\},frame=single,framerule=0.3mm,rulecolor=\color{cellframecolor}]
{\color{incolor}In [{\color{incolor}14}]:} \PY{n}{iy}
\end{Verbatim}


\begin{Verbatim}[commandchars=\\\{\},frame=single,framerule=0.3mm,rulecolor=\color{cellframecolor}]
{\color{outcolor}Out[{\color{outcolor}14}]:} array([[0, 1, 2, 3, 4],
                [0, 1, 2, 3, 4],
                [0, 1, 2, 3, 4]])
\end{Verbatim}
            
    Cette fonction s'appelle \texttt{indices} parce qu'elle produit des
tableaux (ici 2 car on lui a passé une \texttt{shape} à deux dimensions)
qui contiennent, à la case \emph{(i,~j)}, \(i\) (pour le premier
tableau) ou \(j\) pour le second.

    Ainsi, si vous voulez construire un tableau de taille (2, 4) dans
lequel, par exemple~:

\begin{verbatim}
tab[i, j] = 200*i + 2*j + 50
\end{verbatim}

Vous n'avez qu'à faire~:

    \begin{Verbatim}[commandchars=\\\{\},frame=single,framerule=0.3mm,rulecolor=\color{cellframecolor}]
{\color{incolor}In [{\color{incolor}15}]:} \PY{n}{ix}\PY{p}{,} \PY{n}{iy} \PY{o}{=} \PY{n}{np}\PY{o}{.}\PY{n}{indices}\PY{p}{(}\PY{p}{(}\PY{l+m+mi}{2}\PY{p}{,} \PY{l+m+mi}{4}\PY{p}{)}\PY{p}{)}
         \PY{n}{tab} \PY{o}{=} \PY{l+m+mi}{200}\PY{o}{*}\PY{n}{ix} \PY{o}{+} \PY{l+m+mi}{2}\PY{o}{*}\PY{n}{iy} \PY{o}{+} \PY{l+m+mi}{50}
         \PY{n}{tab}
\end{Verbatim}


\begin{Verbatim}[commandchars=\\\{\},frame=single,framerule=0.3mm,rulecolor=\color{cellframecolor}]
{\color{outcolor}Out[{\color{outcolor}15}]:} array([[ 50,  52,  54,  56],
                [250, 252, 254, 256]])
\end{Verbatim}
            
    \hypertarget{multi-dimensions-meshgrid}{%
\subsubsection{\texorpdfstring{Multi-dimensions~:
\texttt{meshgrid}}{Multi-dimensions~: meshgrid}}\label{multi-dimensions-meshgrid}}

    Si vous voulez créer un tableau un peu comme avec \texttt{linspace},
mais en plusieurs dimensions~: imaginez par exemple que vous voulez
tracer une fonction à deux entrées~:

\(f : (x, y) \longrightarrow cos(x) + cos^2(y)\)

Sur un pavé délimité par~:

\(x \in [-\pi, +\pi], y \in [3\pi, 5\pi]\)

    Il vous faut donc créer un tableau, disons de 50 x 50 points, qui
réalise un maillage uniforme de ce pavé, et pour ça vous pouvez utiliser
\texttt{meshgrid}. Pour commencer~:

    \begin{Verbatim}[commandchars=\\\{\},frame=single,framerule=0.3mm,rulecolor=\color{cellframecolor}]
{\color{incolor}In [{\color{incolor}16}]:} \PY{c+c1}{\PYZsh{} on fabrique deux tableaux qui échantillonnent}
         \PY{c+c1}{\PYZsh{} de manière uniforme les intervalles en X et en Y}
         \PY{c+c1}{\PYZsh{} on prend un pas de 10 dans les deux sens, ça nous donnera}
         \PY{c+c1}{\PYZsh{} 100 points pour couvrir l\PYZsq{}espace carré qui nous intéresse}
         
         \PY{n}{Xticks}\PY{p}{,} \PY{n}{Yticks} \PY{o}{=} \PY{p}{(}\PY{n}{np}\PY{o}{.}\PY{n}{linspace}\PY{p}{(}\PY{o}{\PYZhy{}}\PY{n}{np}\PY{o}{.}\PY{n}{pi}\PY{p}{,} \PY{n}{np}\PY{o}{.}\PY{n}{pi}\PY{p}{,} \PY{n}{num}\PY{o}{=}\PY{l+m+mi}{10}\PY{p}{)}\PY{p}{,}
                           \PY{n}{np}\PY{o}{.}\PY{n}{linspace}\PY{p}{(}\PY{l+m+mi}{3}\PY{o}{*}\PY{n}{np}\PY{o}{.}\PY{n}{pi}\PY{p}{,} \PY{l+m+mi}{5}\PY{o}{*}\PY{n}{np}\PY{o}{.}\PY{n}{pi}\PY{p}{,} \PY{n}{num}\PY{o}{=}\PY{l+m+mi}{10}\PY{p}{)}\PY{p}{)}
\end{Verbatim}


    Avec meshgrid, on va créer deux tableaux, qui sont respectivement les
(100) X et les (100) Y de notre maillage~:

    \begin{Verbatim}[commandchars=\\\{\},frame=single,framerule=0.3mm,rulecolor=\color{cellframecolor}]
{\color{incolor}In [{\color{incolor}17}]:} \PY{c+c1}{\PYZsh{} avec meshgrid on les croise}
         \PY{c+c1}{\PYZsh{} ça fait comme un produit cartésien, }
         \PY{c+c1}{\PYZsh{} en extrayant les X et les Y du résultat}
         
         \PY{n}{X}\PY{p}{,} \PY{n}{Y} \PY{o}{=} \PY{n}{np}\PY{o}{.}\PY{n}{meshgrid}\PY{p}{(}\PY{n}{Xticks}\PY{p}{,} \PY{n}{Yticks}\PY{p}{)}
         
         \PY{c+c1}{\PYZsh{} chacun des deux est donc de taille 10 x 10}
         \PY{n}{X}\PY{o}{.}\PY{n}{shape}\PY{p}{,} \PY{n}{Y}\PY{o}{.}\PY{n}{shape}
\end{Verbatim}


\begin{Verbatim}[commandchars=\\\{\},frame=single,framerule=0.3mm,rulecolor=\color{cellframecolor}]
{\color{outcolor}Out[{\color{outcolor}17}]:} ((10, 10), (10, 10))
\end{Verbatim}
            
    Que peut-on faire avec ça~? Eh bien, en fait, on a tout ce qu'il nous
faut pour afficher notre fonction~:

    \begin{Verbatim}[commandchars=\\\{\},frame=single,framerule=0.3mm,rulecolor=\color{cellframecolor}]
{\color{incolor}In [{\color{incolor}18}]:} \PY{c+c1}{\PYZsh{} un tableau 10 x 10 qui contient les images de f()}
         \PY{c+c1}{\PYZsh{} sur les points de la grille}
         \PY{n}{Z} \PY{o}{=} \PY{n}{np}\PY{o}{.}\PY{n}{cos}\PY{p}{(}\PY{n}{X}\PY{p}{)} \PY{o}{+} \PY{n}{np}\PY{o}{.}\PY{n}{cos}\PY{p}{(}\PY{n}{Y}\PY{p}{)}\PY{o}{*}\PY{o}{*}\PY{l+m+mi}{2}
\end{Verbatim}


    \begin{Verbatim}[commandchars=\\\{\},frame=single,framerule=0.3mm,rulecolor=\color{cellframecolor}]
{\color{incolor}In [{\color{incolor}19}]:} \PY{k+kn}{from} \PY{n+nn}{mpl\PYZus{}toolkits}\PY{n+nn}{.}\PY{n+nn}{mplot3d} \PY{k}{import} \PY{n}{Axes3D}
         \PY{n}{fig} \PY{o}{=} \PY{n}{plt}\PY{o}{.}\PY{n}{figure}\PY{p}{(}\PY{p}{)}
         \PY{n}{ax} \PY{o}{=} \PY{n}{fig}\PY{o}{.}\PY{n}{add\PYZus{}subplot}\PY{p}{(}\PY{l+m+mi}{111}\PY{p}{,} \PY{n}{projection}\PY{o}{=}\PY{l+s+s1}{\PYZsq{}}\PY{l+s+s1}{3d}\PY{l+s+s1}{\PYZsq{}}\PY{p}{)}
         \PY{n}{ax}\PY{o}{.}\PY{n}{plot\PYZus{}wireframe}\PY{p}{(}\PY{n}{X}\PY{p}{,} \PY{n}{Y}\PY{p}{,} \PY{n}{Z}\PY{p}{)}\PY{p}{;}
\end{Verbatim}


    \begin{center}
    \adjustimage{max size={0.9\linewidth}{0.9\paperheight}}{w7-s03-c2-initialisation_files/w7-s03-c2-initialisation_43_0.png}
    \end{center}
    { \hspace*{\fill} \\}
    
    Je vous laisse vous convaincre qu'il est facile d'écrire
\texttt{np.indices} à partir de \texttt{np.meshgrid} et
\texttt{np.arange}.


    % Add a bibliography block to the postdoc
    
    
    
