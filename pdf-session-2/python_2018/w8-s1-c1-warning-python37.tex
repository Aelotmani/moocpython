    
    
    
    

    

    \hypertarget{avertissement-relatif-uxe0-asyncio-et-python-3.7}{%
\section{\texorpdfstring{Avertissement relatif à \texttt{asyncio} et
Python-3.7}{Avertissement relatif à asyncio et Python-3.7}}\label{avertissement-relatif-uxe0-asyncio-et-python-3.7}}

    \hypertarget{compluxe9ment---niveau-intermuxe9diaire}{%
\subsection{Complément - niveau
intermédiaire}\label{compluxe9ment---niveau-intermuxe9diaire}}

    \hypertarget{le-cours-se-concentre-sur-python-3.6}{%
\subsubsection{Le cours se concentre sur
Python-3.6}\label{le-cours-se-concentre-sur-python-3.6}}

    Comme on l'a dit en préambule du cours, notre version de
\textbf{référence} est \textbf{Python-3.6}. C'est la version utilisée à
la fois dans les vidéos, et dans les notebooks.

Puisque cette semaine est consacrée à \textbf{\texttt{asyncio}}, il faut
savoir que cette brique technologique est \textbf{relativement récente},
et qu'elle est du coup, plus que d'autres aspects de Python,
\textbf{sujette à des évolutions}.

    \hypertarget{un-ruxe9sumuxe9-des-nouveautuxe9s}{%
\subsubsection{Un résumé des
nouveautés}\label{un-ruxe9sumuxe9-des-nouveautuxe9s}}

    Vous trouverez à la fin de la semaine, dans la séquence consacrée aux
bonnes pratiques, un résumé des améliorations apportées depuis la
version 3.6.

    \hypertarget{lessentiel-est-toujours-dactualituxe9}{%
\subsubsection{L'essentiel est toujours
d'actualité}\label{lessentiel-est-toujours-dactualituxe9}}

    Cela étant dit, nos buts ici étaient principalement:

\begin{itemize}
\tightlist
\item
  de vous faire découvrir ce nouveau paradigme,
\item
  de vous faire sentir dans quelles applications cela peut avoir un
  apport très précieux,
\item
  de bien vous faire comprendre ce qui se passe à l'exécution,
\item
  et de vous donner un aperçu de la façon dont tout cela est implémenté.
\end{itemize}

    Les plus grosses différences concernent la prise en main. Comme nous
allons bientôt le voir, le \emph{``hello world''} en \texttt{asyncio}
était en Python-3.6 un peu \emph{awkward}, c'est-à-dire que pour faire
fonctionner une coroutine, il fallait faire faire quelque chose comme~:

; impact beaucoup plus limité sur du code de production. \textbf{XXX}

donner un exemple rapide - en anticipant


    % Add a bibliography block to the postdoc
    
    
    
