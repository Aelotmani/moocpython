    \hypertarget{expressions-ruxe9guliuxe8res}{%
\section{Expressions régulières}\label{expressions-ruxe9guliuxe8res}}

    Nous vous proposons dans ce notebook quelques exercices sur les
expressions régulières. Faisons quelques remarques avant de commencer~:

\begin{itemize}
\tightlist
\item
  nous nous concentrons sur l'écriture de l'expression régulière en
  elle-même, et pas sur l'utilisation de la bibliothèque~;
\item
  en particulier, tous les exercices font appel à \texttt{re.match}
  entre votre \emph{regexp} et une liste de chaînes d'entrée qui servent
  de jeux de test.
\end{itemize}

    \hypertarget{liens-utiles}{%
\subparagraph{Liens utiles}\label{liens-utiles}}

    Pour travailler sur ces exercices, il pourra être profitable d'avoir
sous la main~:

\begin{itemize}
\tightlist
\item
  la
  \href{https://docs.python.org/3/library/re.html\#regular-expression-syntax}{documentation
  officielle}~;
\item
  et \href{https://pythex.org/}{cet outil interactif sur
  https://pythex.org/} qui permet d'avoir un retour presque immédiat, et
  donc d'accélérer la mise au point.
\end{itemize}

    \hypertarget{exercice---niveau-intermuxe9diaire-1}{%
\subsection{Exercice - niveau intermédiaire
(1)}\label{exercice---niveau-intermuxe9diaire-1}}

    \hypertarget{identificateurs-python}{%
\subparagraph{Identificateurs Python}\label{identificateurs-python}}

    \begin{Verbatim}[commandchars=\\\{\}]
{\color{incolor}In [{\color{incolor} }]:} \PY{c+c1}{\PYZsh{} évaluez cette cellule pour charger l\PYZsq{}exercice}
        \PY{k+kn}{from} \PY{n+nn}{regexp\PYZus{}pythonid} \PY{k}{import} \PY{n}{exo\PYZus{}pythonid}
\end{Verbatim}


    On vous demande d'écrire une expression régulière qui décrit les noms de
variable en Python. Pour cet exercice on se concentre sur les caractères
ASCII. On exclut donc les noms de variables qui pourraient contenir des
caractères exotiques comme les caractères accentués ou autres lettres
grecques.\\

Il s'agit donc de reconnaître toutes les chaînes qui commencent par une
lettre ou un \texttt{\_}, suivi de lettres, chiffres ou \texttt{\_}.

    \begin{Verbatim}[commandchars=\\\{\}]
{\color{incolor}In [{\color{incolor} }]:} \PY{c+c1}{\PYZsh{} quelques exemples de résultat attendus}
        \PY{n}{exo\PYZus{}pythonid}\PY{o}{.}\PY{n}{example}\PY{p}{(}\PY{p}{)}
\end{Verbatim}


    \begin{Verbatim}[commandchars=\\\{\}]
{\color{incolor}In [{\color{incolor} }]:} \PY{c+c1}{\PYZsh{} à vous de jouer: écrivez ici}
        \PY{c+c1}{\PYZsh{} sous forme de chaîne votre expression régulière}
        
        \PY{n}{regexp\PYZus{}pythonid} \PY{o}{=} \PY{l+s+sa}{r}\PY{l+s+s2}{\PYZdq{}}\PY{l+s+s2}{\PYZlt{}votre\PYZus{}regexp\PYZgt{}}\PY{l+s+s2}{\PYZdq{}}
\end{Verbatim}


    \begin{Verbatim}[commandchars=\\\{\}]
{\color{incolor}In [{\color{incolor} }]:} \PY{c+c1}{\PYZsh{} évaluez cette cellule pour valider votre code}
        \PY{n}{exo\PYZus{}pythonid}\PY{o}{.}\PY{n}{correction}\PY{p}{(}\PY{n}{regexp\PYZus{}pythonid}\PY{p}{)}
\end{Verbatim}


    \hypertarget{exercice---niveau-intermuxe9diaire-2}{%
\subsection{Exercice - niveau intermédiaire
(2)}\label{exercice---niveau-intermuxe9diaire-2}}

    \hypertarget{lignes-avec-nom-et-pruxe9nom}{%
\subparagraph{Lignes avec nom et
prénom}\label{lignes-avec-nom-et-pruxe9nom}}

    \begin{Verbatim}[commandchars=\\\{\}]
{\color{incolor}In [{\color{incolor} }]:} \PY{c+c1}{\PYZsh{} pour charger l\PYZsq{}exercice}
        \PY{k+kn}{from} \PY{n+nn}{corrections}\PY{n+nn}{.}\PY{n+nn}{regexp\PYZus{}agenda} \PY{k}{import} \PY{n}{exo\PYZus{}agenda}
\end{Verbatim}


    On veut reconnaître dans un fichier toutes les lignes qui contiennent un
nom et un prénom.

    \begin{Verbatim}[commandchars=\\\{\}]
{\color{incolor}In [{\color{incolor} }]:} \PY{n}{exo\PYZus{}agenda}\PY{o}{.}\PY{n}{example}\PY{p}{(}\PY{p}{)}
\end{Verbatim}


    Plus précisément, on cherche les chaînes qui~:

\begin{itemize}
\tightlist
\item
  commencent par une suite - possiblement vide - de caractères
  alphanumériques (vous pouvez utiliser \texttt{\textbackslash{}w}) ou
  tiret haut (\texttt{-}) qui constitue le prénom~;
\item
  contiennent ensuite comme séparateur le caractère `deux-points'
  \texttt{:}~;
\item
  contiennent ensuite une suite - cette fois jamais vide - de caractères
  alphanumériques, qui constitue le nom~;
\item
  et enfin contiennent un deuxième caractère \texttt{:} mais
  optionnellement seulement.
\end{itemize}

    On vous demande de construire une expression régulière qui définit les
deux groupes \texttt{nom} et \texttt{prenom}, et qui rejette les lignes
qui ne satisfont pas ces critères.

    \begin{Verbatim}[commandchars=\\\{\}]
{\color{incolor}In [{\color{incolor} }]:} \PY{c+c1}{\PYZsh{} entrez votre regexp ici}
        \PY{c+c1}{\PYZsh{} il faudra la faire terminer par \PYZbs{}Z}
        \PY{c+c1}{\PYZsh{} regardez ce qui se passe si vous ne le faites pas}
        
        \PY{n}{regexp\PYZus{}agenda} \PY{o}{=} \PY{l+s+sa}{r}\PY{l+s+s2}{\PYZdq{}}\PY{l+s+s2}{\PYZlt{}votre regexp\PYZgt{}}\PY{l+s+s2}{\PYZbs{}}\PY{l+s+s2}{Z}\PY{l+s+s2}{\PYZdq{}}
\end{Verbatim}


    \begin{Verbatim}[commandchars=\\\{\}]
{\color{incolor}In [{\color{incolor} }]:} \PY{c+c1}{\PYZsh{} évaluez cette cellule pour valider votre code}
        \PY{n}{exo\PYZus{}agenda}\PY{o}{.}\PY{n}{correction}\PY{p}{(}\PY{n}{regexp\PYZus{}agenda}\PY{p}{)}
\end{Verbatim}


    \hypertarget{exercice---niveau-intermuxe9diaire-3}{%
\subsection{Exercice - niveau intermédiaire
(3)}\label{exercice---niveau-intermuxe9diaire-3}}

    \hypertarget{numuxe9ros-de-tuxe9luxe9phone}{%
\subparagraph{Numéros de
téléphone}\label{numuxe9ros-de-tuxe9luxe9phone}}

    \begin{Verbatim}[commandchars=\\\{\}]
{\color{incolor}In [{\color{incolor} }]:} \PY{c+c1}{\PYZsh{} pour charger l\PYZsq{}exercice}
        \PY{k+kn}{from} \PY{n+nn}{corrections}\PY{n+nn}{.}\PY{n+nn}{regexp\PYZus{}phone} \PY{k}{import} \PY{n}{exo\PYZus{}phone}
\end{Verbatim}


    Cette fois on veut reconnaître des numéros de téléphone français, qui
peuvent être~:

\begin{itemize}
\tightlist
\item
  soit au format contenant 10 chiffres dont le premier est un
  \texttt{0}~;
\item
  soit un format international commençant par \texttt{+33} suivie de 9
  chiffres.
\end{itemize}

Dans tous les cas on veut trouver dans le groupe `number' les 9 chiffres
vraiment significatifs, comme ceci~:

    \begin{Verbatim}[commandchars=\\\{\}]
{\color{incolor}In [{\color{incolor} }]:} \PY{n}{exo\PYZus{}phone}\PY{o}{.}\PY{n}{example}\PY{p}{(}\PY{p}{)}
\end{Verbatim}


    \begin{Verbatim}[commandchars=\\\{\}]
{\color{incolor}In [{\color{incolor} }]:} \PY{c+c1}{\PYZsh{} votre regexp}
        \PY{c+c1}{\PYZsh{} à nouveau il faut terminer la regexp par \PYZbs{}Z}
        \PY{n}{regexp\PYZus{}phone} \PY{o}{=} \PY{l+s+sa}{r}\PY{l+s+s2}{\PYZdq{}}\PY{l+s+s2}{\PYZlt{}votre regexp\PYZgt{}}\PY{l+s+s2}{\PYZbs{}}\PY{l+s+s2}{Z}\PY{l+s+s2}{\PYZdq{}}
\end{Verbatim}


    \begin{Verbatim}[commandchars=\\\{\}]
{\color{incolor}In [{\color{incolor} }]:} \PY{c+c1}{\PYZsh{} évaluez cette cellule pour valider votre code}
        \PY{n}{exo\PYZus{}phone}\PY{o}{.}\PY{n}{correction}\PY{p}{(}\PY{n}{regexp\PYZus{}phone}\PY{p}{)}
\end{Verbatim}


    \hypertarget{exercice---niveau-avancuxe9}{%
\subsection{Exercice - niveau
avancé}\label{exercice---niveau-avancuxe9}}

    Vu comment sont conçus les exercices, vous ne pouvez pas passer à
\texttt{re.compile} un drapeau comme \texttt{re.IGNORECASE} ou autre~;
sachez cependant que vous pouvez \textbf{\emph{embarquer} ces drapeaux
dans la \emph{regexp}} elle-même~; par exemple pour rendre la regexp
insensible à la casse de caractères, au lieu d'appeler
\texttt{re.compile} avec le flag \texttt{re.I}, vous pouvez utiliser
\texttt{(?i)} comme ceci~:

    \begin{Verbatim}[commandchars=\\\{\}]
{\color{incolor}In [{\color{incolor} }]:} \PY{k+kn}{import} \PY{n+nn}{re}
\end{Verbatim}


    \begin{Verbatim}[commandchars=\\\{\}]
{\color{incolor}In [{\color{incolor} }]:} \PY{c+c1}{\PYZsh{} on peut embarquer les flags comme IGNORECASE}
        \PY{c+c1}{\PYZsh{} directement dans la regexp}
        \PY{c+c1}{\PYZsh{} c\PYZsq{}est équivalent de faire ceci}
        
        \PY{n}{re\PYZus{}obj} \PY{o}{=} \PY{n}{re}\PY{o}{.}\PY{n}{compile}\PY{p}{(}\PY{l+s+s2}{\PYZdq{}}\PY{l+s+s2}{abc}\PY{l+s+s2}{\PYZdq{}}\PY{p}{,} \PY{n}{flags}\PY{o}{=}\PY{n}{re}\PY{o}{.}\PY{n}{IGNORECASE}\PY{p}{)}
        \PY{n}{re\PYZus{}obj}\PY{o}{.}\PY{n}{match}\PY{p}{(}\PY{l+s+s2}{\PYZdq{}}\PY{l+s+s2}{ABC}\PY{l+s+s2}{\PYZdq{}}\PY{p}{)}\PY{o}{.}\PY{n}{group}\PY{p}{(}\PY{l+m+mi}{0}\PY{p}{)}
\end{Verbatim}


    \begin{Verbatim}[commandchars=\\\{\}]
{\color{incolor}In [{\color{incolor} }]:} \PY{c+c1}{\PYZsh{} ou cela}
        
        \PY{n}{re}\PY{o}{.}\PY{n}{match}\PY{p}{(}\PY{l+s+s2}{\PYZdq{}}\PY{l+s+s2}{(?i)abc}\PY{l+s+s2}{\PYZdq{}}\PY{p}{,}\PY{l+s+s2}{\PYZdq{}}\PY{l+s+s2}{ABC}\PY{l+s+s2}{\PYZdq{}}\PY{p}{)}\PY{o}{.}\PY{n}{group}\PY{p}{(}\PY{l+m+mi}{0}\PY{p}{)}
\end{Verbatim}


    \begin{Verbatim}[commandchars=\\\{\}]
{\color{incolor}In [{\color{incolor} }]:} \PY{c+c1}{\PYZsh{} les flags comme (?i) doivent apparaître}
        \PY{c+c1}{\PYZsh{} en premier dans la regexp}
        \PY{n}{re}\PY{o}{.}\PY{n}{match}\PY{p}{(}\PY{l+s+s2}{\PYZdq{}}\PY{l+s+s2}{abc(?i)}\PY{l+s+s2}{\PYZdq{}}\PY{p}{,}\PY{l+s+s2}{\PYZdq{}}\PY{l+s+s2}{ABC}\PY{l+s+s2}{\PYZdq{}}\PY{p}{)}\PY{o}{.}\PY{n}{group}\PY{p}{(}\PY{l+m+mi}{0}\PY{p}{)}
\end{Verbatim}


    Pour plus de précisions sur ce trait, que nous avons laissé de côté dans
le complément pour ne pas trop l'alourdir, voyez
\href{https://docs.python.org/3/library/re.html\#regular-expression-syntax}{la
documentation sur les expressions régulières} et cherchez la première
occurrence de \texttt{iLmsux}.

    \hypertarget{duxe9cortiquer-une-url}{%
\subsubsection{Décortiquer une URL}\label{duxe9cortiquer-une-url}}

    On vous demande d'écrire une expression régulière qui permette
d'analyser des URLs.

Voici les conventions que nous avons adoptées pour l'exercice~:

\begin{itemize}
\tightlist
\item
  la chaîne contient les parties suivantes~:

  \begin{itemize}
  \tightlist
  \item
    \texttt{\textless{}protocol\textgreater{}://\textless{}location\textgreater{}/\textless{}path\textgreater{}}~;
  \end{itemize}
\item
  l'URL commence par le nom d'un protocole qui doit être parmi
  \texttt{http}, \texttt{https}, \texttt{ftp}, \texttt{ssh}~;
\item
  le nom du protocole peut contenir de manière indifférente des
  minuscules ou des majuscules~;
\item
  ensuite doit venir la séquence \texttt{://}~;
\item
  ensuite on va trouver une chaîne
  \texttt{\textless{}location\textgreater{}} qui contient~:

  \begin{itemize}
  \tightlist
  \item
    potentiellement un nom d'utilisateur, et s'il est présent,
    potentiellement un mot de passe~;
  \item
    obligatoirement un nom de \texttt{hostname}~;
  \item
    potentiellement un numéro de port~;
  \end{itemize}
\item
  lorsque les 4 parties sont présentes dans
  \texttt{\textless{}location\textgreater{}}, cela se présente comme
  ceci~:

  \begin{itemize}
  \tightlist
  \item
    \texttt{\textless{}location\textgreater{}\ =\ \textless{}user\textgreater{}:\textless{}password\textgreater{}@\textless{}hostname\textgreater{}:\textless{}port\textgreater{}}~;
  \end{itemize}
\item
  si l'on note entre crochets les parties optionnelles, cela donne~:

  \begin{itemize}
  \tightlist
  \item
    \texttt{\textless{}location\textgreater{}\ =\ {[}\textless{}user\textgreater{}{[}:\textless{}password\textgreater{}{]}@{]}\textless{}hostname\textgreater{}{[}:\textless{}port\textgreater{}{]}}~;
  \end{itemize}
\item
  le champ \texttt{\textless{}user\textgreater{}} ne peut contenir que
  des caractères alphanumériques~; si le \texttt{@} est présent le champ
  \texttt{\textless{}user\textgreater{}} ne peut pas être vide~;
\item
  le champ \texttt{\textless{}password\textgreater{}} peut contenir tout
  sauf un \texttt{:} et de même, si le \texttt{:} est présent le champ
  \texttt{\textless{}password\textgreater{}} ne peut pas être vide~;
\item
  le champ \texttt{\textless{}hostname\textgreater{}} peut contenir un
  suite non-vide de caractères alphanumériques, underscores, ou
  \texttt{.}~;
\item
  le champ \texttt{\textless{}port\textgreater{}} ne contient que des
  chiffres, et il est non vide si le \texttt{:} est spécifié~;
\item
  le champ \texttt{\textless{}path\textgreater{}} peut être vide.
\end{itemize}

Enfin, vous devez définir les groupes \texttt{proto}, \texttt{user},
\texttt{password}, \texttt{hostname}, \texttt{port} et \texttt{path} qui
sont utilisés pour vérifier votre résultat. Dans la case
\texttt{Résultat\ attendu}, vous trouverez soit \texttt{None} si la
regexp ne filtre pas l'intégralité de l'entrée, ou bien une liste
ordonnée de tuples qui donnent la valeur de ces groupes~; vous n'avez
rien à faire pour construire ces tuples, c'est l'exercice qui s'en
occupe.

    \begin{Verbatim}[commandchars=\\\{\}]
{\color{incolor}In [{\color{incolor} }]:} \PY{c+c1}{\PYZsh{} pour charger l\PYZsq{}exercice}
        \PY{k+kn}{from} \PY{n+nn}{corrections}\PY{n+nn}{.}\PY{n+nn}{regexp\PYZus{}url} \PY{k}{import} \PY{n}{exo\PYZus{}url}
\end{Verbatim}


    \begin{Verbatim}[commandchars=\\\{\}]
{\color{incolor}In [{\color{incolor} }]:} \PY{c+c1}{\PYZsh{} exemples du résultat attendu}
        \PY{n}{exo\PYZus{}url}\PY{o}{.}\PY{n}{example}\PY{p}{(}\PY{p}{)}
\end{Verbatim}


    \begin{Verbatim}[commandchars=\\\{\}]
{\color{incolor}In [{\color{incolor} }]:} \PY{c+c1}{\PYZsh{} n\PYZsq{}hésitez pas à construire votre regexp petit à petit}
        
        \PY{n}{regexp\PYZus{}url} \PY{o}{=} \PY{l+s+s2}{\PYZdq{}}\PY{l+s+s2}{\PYZlt{}votre\PYZus{}regexp\PYZgt{}}\PY{l+s+s2}{\PYZdq{}}
\end{Verbatim}


    \begin{Verbatim}[commandchars=\\\{\}]
{\color{incolor}In [{\color{incolor} }]:} \PY{n}{exo\PYZus{}url}\PY{o}{.}\PY{n}{correction}\PY{p}{(}\PY{n}{regexp\PYZus{}url}\PY{p}{)}
\end{Verbatim}