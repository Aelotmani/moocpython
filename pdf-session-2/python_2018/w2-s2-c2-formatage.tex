    
    
    
    

    

    \hypertarget{formatage-de-chauxeenes-de-caractuxe8res}{%
\section{Formatage de chaînes de
caractères}\label{formatage-de-chauxeenes-de-caractuxe8res}}

    \hypertarget{compluxe9ment---niveau-basique}{%
\subsection{Complément - niveau
basique}\label{compluxe9ment---niveau-basique}}

    On désigne par formatage les outils qui permettent d'obtenir une
présentation fine des résultats, que ce soit pour améliorer la
lisibilité lorsqu'on s'adresse à des humains, ou pour respecter la
syntaxe d'un outil auquel on veut passer les données pour un traitement
ultérieur.

    \hypertarget{la-fonction-print}{%
\subsubsection{\texorpdfstring{La fonction
\texttt{print}}{La fonction print}}\label{la-fonction-print}}

    Nous avons jusqu'à maintenant presque toujours utilisé la fonction
\texttt{print} pour afficher nos résultats. Comme on l'a vu, celle-ci
réalise un formatage sommaire~: elle insère un espace entre les valeurs
qui lui sont passées.

    \begin{Verbatim}[commandchars=\\\{\},frame=single,framerule=0.3mm,rulecolor=\color{cellframecolor}]
{\color{incolor}In [{\color{incolor}1}]:} \PY{n+nb}{print}\PY{p}{(}\PY{l+m+mi}{1}\PY{p}{,} \PY{l+s+s1}{\PYZsq{}}\PY{l+s+s1}{a}\PY{l+s+s1}{\PYZsq{}}\PY{p}{,} \PY{l+m+mi}{12} \PY{o}{+} \PY{l+m+mi}{4}\PY{n}{j}\PY{p}{)}
\end{Verbatim}


    \begin{Verbatim}[commandchars=\\\{\},frame=single,framerule=0.3mm,rulecolor=\color{cellframecolor}]
1 a (12+4j)
\end{Verbatim}

    La seule subtilité notable concernant \texttt{print} est que, par
défaut, elle ajoute un saut de ligne à la fin. Pour éviter ce
comportement, on peut passer à la fonction un argument \texttt{end}, qui
sera inséré \emph{au lieu} du saut de ligne. Ainsi par exemple~:

    \begin{Verbatim}[commandchars=\\\{\},frame=single,framerule=0.3mm,rulecolor=\color{cellframecolor}]
{\color{incolor}In [{\color{incolor}2}]:} \PY{c+c1}{\PYZsh{} une première ligne}
        \PY{n+nb}{print}\PY{p}{(}\PY{l+s+s2}{\PYZdq{}}\PY{l+s+s2}{une}\PY{l+s+s2}{\PYZdq{}}\PY{p}{,} \PY{l+s+s2}{\PYZdq{}}\PY{l+s+s2}{seule}\PY{l+s+s2}{\PYZdq{}}\PY{p}{,} \PY{l+s+s2}{\PYZdq{}}\PY{l+s+s2}{ligne}\PY{l+s+s2}{\PYZdq{}}\PY{p}{)}
\end{Verbatim}


    \begin{Verbatim}[commandchars=\\\{\},frame=single,framerule=0.3mm,rulecolor=\color{cellframecolor}]
une seule ligne
\end{Verbatim}

    \begin{Verbatim}[commandchars=\\\{\},frame=single,framerule=0.3mm,rulecolor=\color{cellframecolor}]
{\color{incolor}In [{\color{incolor}3}]:} \PY{c+c1}{\PYZsh{} une deuxième ligne en deux appels à print}
        \PY{n+nb}{print}\PY{p}{(}\PY{l+s+s2}{\PYZdq{}}\PY{l+s+s2}{une}\PY{l+s+s2}{\PYZdq{}}\PY{p}{,} \PY{l+s+s2}{\PYZdq{}}\PY{l+s+s2}{autre}\PY{l+s+s2}{\PYZdq{}}\PY{p}{,} \PY{n}{end}\PY{o}{=}\PY{l+s+s1}{\PYZsq{}}\PY{l+s+s1}{ }\PY{l+s+s1}{\PYZsq{}}\PY{p}{)}
        \PY{n+nb}{print}\PY{p}{(}\PY{l+s+s2}{\PYZdq{}}\PY{l+s+s2}{ligne}\PY{l+s+s2}{\PYZdq{}}\PY{p}{)}
\end{Verbatim}


    \begin{Verbatim}[commandchars=\\\{\},frame=single,framerule=0.3mm,rulecolor=\color{cellframecolor}]
une autre ligne
\end{Verbatim}

    Il faut remarquer aussi que \texttt{print} est capable d'imprimer
\textbf{n'importe quel objet}. Nous l'avons déjà fait avec les listes et
les tuples, voici par exemple un module~:

    \begin{Verbatim}[commandchars=\\\{\},frame=single,framerule=0.3mm,rulecolor=\color{cellframecolor}]
{\color{incolor}In [{\color{incolor}4}]:} \PY{c+c1}{\PYZsh{} on peut imprimer par exemple un objet \PYZsq{}module\PYZsq{}}
        \PY{k+kn}{import} \PY{n+nn}{math}
        
        \PY{n+nb}{print}\PY{p}{(}\PY{l+s+s1}{\PYZsq{}}\PY{l+s+s1}{le module math est}\PY{l+s+s1}{\PYZsq{}}\PY{p}{,} \PY{n}{math}\PY{p}{)}
\end{Verbatim}


    \begin{Verbatim}[commandchars=\\\{\},frame=single,framerule=0.3mm,rulecolor=\color{cellframecolor}]
le module math est <module 'math' from '/usr/local/Cellar/python/3.7.0/Frameworks/Python.framework/Versions/3.7/lib/python3.7/lib-dynload/math.cpython-37m-darwin.so'>
\end{Verbatim}

    En anticipant un peu, voici comment \texttt{print} présente les
instances de classe (ne vous inquiétez pas, nous apprendrons dans une
semaine ultérieure ce que sont les classes et les instances).

    \begin{Verbatim}[commandchars=\\\{\},frame=single,framerule=0.3mm,rulecolor=\color{cellframecolor}]
{\color{incolor}In [{\color{incolor}5}]:} \PY{c+c1}{\PYZsh{} pour définir la classe Personne}
        \PY{k}{class} \PY{n+nc}{Personne}\PY{p}{:}
            \PY{k}{pass}
        
        \PY{c+c1}{\PYZsh{} et pour créer une instance de cette classe}
        \PY{n}{personne} \PY{o}{=} \PY{n}{Personne}\PY{p}{(}\PY{p}{)}
\end{Verbatim}


    \begin{Verbatim}[commandchars=\\\{\},frame=single,framerule=0.3mm,rulecolor=\color{cellframecolor}]
{\color{incolor}In [{\color{incolor}6}]:} \PY{c+c1}{\PYZsh{} voilà comment s\PYZsq{}affiche une instance de classe}
        \PY{n+nb}{print}\PY{p}{(}\PY{n}{personne}\PY{p}{)}
\end{Verbatim}


    \begin{Verbatim}[commandchars=\\\{\},frame=single,framerule=0.3mm,rulecolor=\color{cellframecolor}]
<\_\_main\_\_.Personne object at 0x10b40a8d0>
\end{Verbatim}

    On rencontre assez vite les limites de \texttt{print}~:

\begin{itemize}
\tightlist
\item
  d'une part, il peut être nécessaire de formater une chaîne de
  caractères sans nécessairement vouloir l'imprimer, ou en tout cas pas
  immédiatement~;
\item
  d'autre part, les espaces ajoutées peuvent être plus néfastes
  qu'utiles~;
\item
  enfin, on peut avoir besoin de préciser un nombre de chiffres
  significatifs, ou de choisir comment présenter une date.
\end{itemize}

C'est pourquoi il est plus courant de \textbf{formater} les chaînes -
c'est-à-dire de calculer des chaînes en mémoire, sans nécessairement les
imprimer de suite, et c'est ce que nous allons étudier dans ce
complément.

    \hypertarget{les-f-strings}, qui sont encore massivement utilisées
dans le code existant (surtout \texttt{\%} d'ailleurs, bien que
essentiellement obsolète).

    Mais définissons d'abord quelques données à afficher~:

    \begin{Verbatim}[commandchars=\\\{\},frame=single,framerule=0.3mm,rulecolor=\color{cellframecolor}]
{\color{incolor}In [{\color{incolor}7}]:} \PY{c+c1}{\PYZsh{} donnons\PYZhy{}nous quelques variables}
        \PY{n}{prenom}\PY{p}{,} \PY{n}{nom}\PY{p}{,} \PY{n}{age} \PY{o}{=} \PY{l+s+s1}{\PYZsq{}}\PY{l+s+s1}{Jean}\PY{l+s+s1}{\PYZsq{}}\PY{p}{,} \PY{l+s+s1}{\PYZsq{}}\PY{l+s+s1}{Dupont}\PY{l+s+s1}{\PYZsq{}}\PY{p}{,} \PY{l+m+mi}{35}
\end{Verbatim}


    \begin{Verbatim}[commandchars=\\\{\},frame=single,framerule=0.3mm,rulecolor=\color{cellframecolor}]
{\color{incolor}In [{\color{incolor}8}]:} \PY{c+c1}{\PYZsh{} mon premier f\PYZhy{}string}
        \PY{n}{f}\PY{l+s+s2}{\PYZdq{}}\PY{l+s+si}{\PYZob{}prenom\PYZcb{}}\PY{l+s+s2}{ }\PY{l+s+si}{\PYZob{}nom\PYZcb{}}\PY{l+s+s2}{ a }\PY{l+s+si}{\PYZob{}age\PYZcb{}}\PY{l+s+s2}{ ans}\PY{l+s+s2}{\PYZdq{}}
\end{Verbatim}


\begin{Verbatim}[commandchars=\\\{\},frame=single,framerule=0.3mm,rulecolor=\color{cellframecolor}]
{\color{outcolor}Out[{\color{outcolor}8}]:} 'Jean Dupont a 35 ans'
\end{Verbatim}
            
    Vous remarquez d'abord que la chaine commence par \texttt{f"}, c'est
bien sûr pour cela qu'on l'appelle un \emph{f-string}.

On peut bien entendu ajouter le \texttt{f} devant toutes les formes de
strings, qu'ils commencent par \texttt{\textquotesingle{}} ou \texttt{"}
ou \texttt{\textquotesingle{}\textquotesingle{}\textquotesingle{}} ou
\texttt{"""}.

    Ensuite vous remarquez que les zones délimitées entre \texttt{\{\}} sont
remplacées. La logique d'un \emph{f-string}, c'est tout simplement de
considérer l'intérieur d'un \texttt{\{\}} comme du code Python (une
expression pour être précis), de l'évaluer, et d'utiliser le résultat
pour remplir le \texttt{\{\}}.

    Ça veut dire, en clair, que je peux faire des calculs à l'intérieur des
\texttt{\{\}}.

    \begin{Verbatim}[commandchars=\\\{\},frame=single,framerule=0.3mm,rulecolor=\color{cellframecolor}]
{\color{incolor}In [{\color{incolor}9}]:} \PY{c+c1}{\PYZsh{} toutes les expressions sont autorisées à l\PYZsq{}intérieur d\PYZsq{}un \PYZob{}\PYZcb{}}
        \PY{n}{f}\PY{l+s+s2}{\PYZdq{}}\PY{l+s+s2}{dans 10 ans }\PY{l+s+si}{\PYZob{}prenom\PYZcb{}}\PY{l+s+s2}{ aura }\PY{l+s+s2}{\PYZob{}}\PY{l+s+s2}{age + 10\PYZcb{} ans}\PY{l+s+s2}{\PYZdq{}}
\end{Verbatim}


\begin{Verbatim}[commandchars=\\\{\},frame=single,framerule=0.3mm,rulecolor=\color{cellframecolor}]
{\color{outcolor}Out[{\color{outcolor}9}]:} 'dans 10 ans Jean aura 45 ans'
\end{Verbatim}
            
    \begin{Verbatim}[commandchars=\\\{\},frame=single,framerule=0.3mm,rulecolor=\color{cellframecolor}]
{\color{incolor}In [{\color{incolor}10}]:} \PY{c+c1}{\PYZsh{} on peut donc aussi mettre des appels de fonction}
         \PY{n}{notes} \PY{o}{=} \PY{p}{[}\PY{l+m+mi}{12}\PY{p}{,} \PY{l+m+mi}{15}\PY{p}{,} \PY{l+m+mi}{19}\PY{p}{]}
         \PY{n}{f}\PY{l+s+s2}{\PYZdq{}}\PY{l+s+s2}{nous avons pour l}\PY{l+s+s2}{\PYZsq{}}\PY{l+s+s2}{instant }\PY{l+s+s2}{\PYZob{}}\PY{l+s+s2}{len(notes)\PYZcb{} notes}\PY{l+s+s2}{\PYZdq{}}
\end{Verbatim}


\begin{Verbatim}[commandchars=\\\{\},frame=single,framerule=0.3mm,rulecolor=\color{cellframecolor}]
{\color{outcolor}Out[{\color{outcolor}10}]:} "nous avons pour l'instant 3 notes"
\end{Verbatim}
            
    Nous allons en rester là pour la partie en niveau basique. Il nous reste
à étudier comment chaque \texttt{\{\}} est formaté (par exemple comment
choisir le nombre de chiffres significatifs sur un flottant), ce point
est expliqué plus bas.

Comme vous le voyez, les \emph{f-strings} fournissent une méthode très
simple et expressive pour formater des données dans des chaînes de
caractère. Redisons-le pour être bien clair~: un \emph{f-string}
\textbf{ne réalise pas d'impression}, il faut donc le passer à
\texttt{print} si l'impression est souhaitée.

    \hypertarget{la-muxe9thode-format}{%
\subsubsection{\texorpdfstring{La méthode
\texttt{format}}{La méthode format}}\label{la-muxe9thode-format}}

    Avant l'introduction des \emph{f-strings}, la technique recommandée pour
faire du formatage était d'utiliser la méthode \texttt{format} qui est
définie sur les objets \texttt{str} et qui s'utilise comme ceci~:

    \begin{Verbatim}[commandchars=\\\{\},frame=single,framerule=0.3mm,rulecolor=\color{cellframecolor}]
{\color{incolor}In [{\color{incolor}11}]:} \PY{l+s+s2}{\PYZdq{}}\PY{l+s+si}{\PYZob{}\PYZcb{}}\PY{l+s+s2}{ }\PY{l+s+si}{\PYZob{}\PYZcb{}}\PY{l+s+s2}{ a }\PY{l+s+si}{\PYZob{}\PYZcb{}}\PY{l+s+s2}{ ans}\PY{l+s+s2}{\PYZdq{}}\PY{o}{.}\PY{n}{format}\PY{p}{(}\PY{n}{prenom}\PY{p}{,} \PY{n}{nom}\PY{p}{,} \PY{n}{age}\PY{p}{)}
\end{Verbatim}


\begin{Verbatim}[commandchars=\\\{\},frame=single,framerule=0.3mm,rulecolor=\color{cellframecolor}]
{\color{outcolor}Out[{\color{outcolor}11}]:} 'Jean Dupont a 35 ans'
\end{Verbatim}
            
    Dans cet exemple le plus simple, les données sont affichées en lieu et
place des \texttt{\{\}}, dans l'ordre où elles sont fournies.

    Cela convient bien lorsqu'on a peu de données. Si par la suite on veut
changer l'ordre par exemple des nom et prénom, on peut bien sûr échanger
l'ordre des arguments passés à format, ou encore utiliser la
\textbf{liaison par position}, comme ceci~:

    \begin{Verbatim}[commandchars=\\\{\},frame=single,framerule=0.3mm,rulecolor=\color{cellframecolor}]
{\color{incolor}In [{\color{incolor}12}]:} \PY{l+s+s2}{\PYZdq{}}\PY{l+s+si}{\PYZob{}1\PYZcb{}}\PY{l+s+s2}{ }\PY{l+s+si}{\PYZob{}0\PYZcb{}}\PY{l+s+s2}{ a }\PY{l+s+si}{\PYZob{}2\PYZcb{}}\PY{l+s+s2}{ ans}\PY{l+s+s2}{\PYZdq{}}\PY{o}{.}\PY{n}{format}\PY{p}{(}\PY{n}{prenom}\PY{p}{,} \PY{n}{nom}\PY{p}{,} \PY{n}{age}\PY{p}{)}
\end{Verbatim}


\begin{Verbatim}[commandchars=\\\{\},frame=single,framerule=0.3mm,rulecolor=\color{cellframecolor}]
{\color{outcolor}Out[{\color{outcolor}12}]:} 'Dupont Jean a 35 ans'
\end{Verbatim}
            
    Dans la pratique toutefois, cette forme est assez peu utile, on lui
préfère souvent la \textbf{liaison par nom} qui se présente comme ceci~:

    \begin{Verbatim}[commandchars=\\\{\},frame=single,framerule=0.3mm,rulecolor=\color{cellframecolor}]
{\color{incolor}In [{\color{incolor}13}]:} \PY{p}{(}\PY{l+s+s2}{\PYZdq{}}\PY{l+s+si}{\PYZob{}le\PYZus{}prenom\PYZcb{}}\PY{l+s+s2}{ }\PY{l+s+si}{\PYZob{}le\PYZus{}nom\PYZcb{}}\PY{l+s+s2}{ a }\PY{l+s+si}{\PYZob{}l\PYZus{}age\PYZcb{}}\PY{l+s+s2}{ ans}\PY{l+s+s2}{\PYZdq{}}
            \PY{o}{.}\PY{n}{format}\PY{p}{(}\PY{n}{le\PYZus{}nom}\PY{o}{=}\PY{n}{nom}\PY{p}{,} \PY{n}{le\PYZus{}prenom}\PY{o}{=}\PY{n}{prenom}\PY{p}{,} \PY{n}{l\PYZus{}age}\PY{o}{=}\PY{n}{age}\PY{p}{)}\PY{p}{)}
\end{Verbatim}


\begin{Verbatim}[commandchars=\\\{\},frame=single,framerule=0.3mm,rulecolor=\color{cellframecolor}]
{\color{outcolor}Out[{\color{outcolor}13}]:} 'Jean Dupont a 35 ans'
\end{Verbatim}
            
    \emph{Petite digression}: remarquez l'usage des parenthèses, qui me
permettent de couper ma ligne en deux, car sinon ce code serait trop
long pour la PEP8; on s'efforce toujours de ne pas dépasser 80
caractères de large, dans notre cas c'est utile notamment pour l'édition
du cours au format PDF.

    Reprenons: dans ce premier exemple de liaison par nom, nous avons
délibérément utilisé des noms différents pour les données externes et
pour les noms apparaissant dans le format, pour bien illustrer comment
la liaison est résolue, mais on peut aussi bien faire tout simplement~:

    \begin{Verbatim}[commandchars=\\\{\},frame=single,framerule=0.3mm,rulecolor=\color{cellframecolor}]
{\color{incolor}In [{\color{incolor}14}]:} \PY{l+s+s2}{\PYZdq{}}\PY{l+s+si}{\PYZob{}prenom\PYZcb{}}\PY{l+s+s2}{ }\PY{l+s+si}{\PYZob{}nom\PYZcb{}}\PY{l+s+s2}{ a }\PY{l+s+si}{\PYZob{}age\PYZcb{}}\PY{l+s+s2}{ ans}\PY{l+s+s2}{\PYZdq{}}\PY{o}{.}\PY{n}{format}\PY{p}{(}\PY{n}{nom}\PY{o}{=}\PY{n}{nom}\PY{p}{,} \PY{n}{prenom}\PY{o}{=}\PY{n}{prenom}\PY{p}{,} \PY{n}{age}\PY{o}{=}\PY{n}{age}\PY{p}{)}
\end{Verbatim}


\begin{Verbatim}[commandchars=\\\{\},frame=single,framerule=0.3mm,rulecolor=\color{cellframecolor}]
{\color{outcolor}Out[{\color{outcolor}14}]:} 'Jean Dupont a 35 ans'
\end{Verbatim}
            
    Voici qui conclut notre courte introduction à la méthode
\texttt{format}.

    \hypertarget{compluxe9ment---niveau-intermuxe9diaire}{%
\subsection{Complément - niveau
intermédiaire}\label{compluxe9ment---niveau-intermuxe9diaire}}

    \hypertarget{la-toute-premiuxe8re-version-du-formatage-lopuxe9rateur}}{La toute première version du formatage~: l'opérateur \%}}\label{la-toute-premiuxe8re-version-du-formatage-lopuxe9rateur}}

    \texttt{format} a été en fait introduite assez tard dans Python, pour
remplacer la technique que nous allons présenter maintenant.

Étant donné le volume de code qui a été écrit avec l'opérateur
\texttt{\%}, il nous a semblé important d'introduire brièvement cette
construction ici. Vous ne devez cependant pas utiliser cet opérateur
dans du code moderne, la manière pythonique de formater les chaînes de
caractères est le f-string.

    Le principe de l'opérateur \texttt{\%} est le suivant. On élabore comme
ci-dessus un ``format'' c'est-à-dire le patron de ce qui doit être
rendu, auquel on passe des arguments pour ``remplir'' les trous. Voyons
les exemples de tout à l'heure avec l'opérateur \texttt{\%}~:

    \begin{Verbatim}[commandchars=\\\{\},frame=single,framerule=0.3mm,rulecolor=\color{cellframecolor}]
{\color{incolor}In [{\color{incolor}15}]:} \PY{c+c1}{\PYZsh{} l\PYZsq{}ancienne façon de formater les chaînes avec \PYZpc{}}
         \PY{c+c1}{\PYZsh{} est souvent moins lisible}
         \PY{l+s+s2}{\PYZdq{}}\PY{l+s+si}{\PYZpc{}s}\PY{l+s+s2}{ }\PY{l+s+si}{\PYZpc{}s}\PY{l+s+s2}{ a }\PY{l+s+si}{\PYZpc{}s}\PY{l+s+s2}{ ans}\PY{l+s+s2}{\PYZdq{}} \PY{o}{\PYZpc{}} \PY{p}{(}\PY{n}{prenom}\PY{p}{,} \PY{n}{nom}\PY{p}{,} \PY{n}{age}\PY{p}{)}
\end{Verbatim}


\begin{Verbatim}[commandchars=\\\{\},frame=single,framerule=0.3mm,rulecolor=\color{cellframecolor}]
{\color{outcolor}Out[{\color{outcolor}15}]:} 'Jean Dupont a 35 ans'
\end{Verbatim}
            
    On pouvait également avec cet opérateur recourir à un mécanisme de
liaison par nommage, en passant par un dictionnaire. Pour anticiper un
tout petit peu sur cette notion que nous verrons très bientôt, voici
comment

    \begin{Verbatim}[commandchars=\\\{\},frame=single,framerule=0.3mm,rulecolor=\color{cellframecolor}]
{\color{incolor}In [{\color{incolor}16}]:} \PY{n}{variables} \PY{o}{=} \PY{p}{\PYZob{}}\PY{l+s+s1}{\PYZsq{}}\PY{l+s+s1}{le\PYZus{}nom}\PY{l+s+s1}{\PYZsq{}}\PY{p}{:} \PY{n}{nom}\PY{p}{,} \PY{l+s+s1}{\PYZsq{}}\PY{l+s+s1}{le\PYZus{}prenom}\PY{l+s+s1}{\PYZsq{}}\PY{p}{:} \PY{n}{prenom}\PY{p}{,} \PY{l+s+s1}{\PYZsq{}}\PY{l+s+s1}{l\PYZus{}age}\PY{l+s+s1}{\PYZsq{}}\PY{p}{:} \PY{n}{age}\PY{p}{\PYZcb{}}
         \PY{l+s+s2}{\PYZdq{}}\PY{l+s+si}{\PYZpc{}(le\PYZus{}nom)s}\PY{l+s+s2}{, }\PY{l+s+si}{\PYZpc{}(le\PYZus{}prenom)s}\PY{l+s+s2}{, }\PY{l+s+si}{\PYZpc{}(l\PYZus{}age)s}\PY{l+s+s2}{ ans}\PY{l+s+s2}{\PYZdq{}} \PY{o}{\PYZpc{}} \PY{n}{variables}
\end{Verbatim}


\begin{Verbatim}[commandchars=\\\{\},frame=single,framerule=0.3mm,rulecolor=\color{cellframecolor}]
{\color{outcolor}Out[{\color{outcolor}16}]:} 'Dupont, Jean, 35 ans'
\end{Verbatim}
            
    \hypertarget{compluxe9ment---niveau-avancuxe9}{%
\subsection{Complément - niveau
avancé}\label{compluxe9ment---niveau-avancuxe9}}

    De retour aux \emph{f-strings} et à la fonction \texttt{format}, il
arrive qu'on ait besoin de spécifier plus finement la façon dont une
valeur doit être affichée.

    \hypertarget{pruxe9cision-des-arrondis}{%
\subsubsection{Précision des arrondis}\label{pruxe9cision-des-arrondis}}

    C'est typiquement le cas avec les valeurs flottantes pour lesquelles la
précision de l'affichage vient au détriment de la lisibilité. Voici deux
formes équivalentes pour obtenir une valeur de pi arrondie~:

    \begin{Verbatim}[commandchars=\\\{\},frame=single,framerule=0.3mm,rulecolor=\color{cellframecolor}]
{\color{incolor}In [{\color{incolor}17}]:} \PY{k+kn}{from} \PY{n+nn}{math} \PY{k}{import} \PY{n}{pi}
\end{Verbatim}


    \begin{Verbatim}[commandchars=\\\{\},frame=single,framerule=0.3mm,rulecolor=\color{cellframecolor}]
{\color{incolor}In [{\color{incolor}18}]:} \PY{c+c1}{\PYZsh{} un f\PYZhy{}string}
         \PY{n}{f}\PY{l+s+s2}{\PYZdq{}}\PY{l+s+s2}{pi avec seulement 2 chiffres apres la virgule }\PY{l+s+si}{\PYZob{}pi:.2f\PYZcb{}}\PY{l+s+s2}{\PYZdq{}}
\end{Verbatim}


\begin{Verbatim}[commandchars=\\\{\},frame=single,framerule=0.3mm,rulecolor=\color{cellframecolor}]
{\color{outcolor}Out[{\color{outcolor}18}]:} 'pi avec seulement 2 chiffres apres la virgule 3.14'
\end{Verbatim}
            
    \begin{Verbatim}[commandchars=\\\{\},frame=single,framerule=0.3mm,rulecolor=\color{cellframecolor}]
{\color{incolor}In [{\color{incolor}19}]:} \PY{c+c1}{\PYZsh{} avec format() et liaison par nom}
         \PY{p}{(}\PY{l+s+s2}{\PYZdq{}}\PY{l+s+s2}{pi avec seulement 2 chiffres apres la virgule }\PY{l+s+si}{\PYZob{}flottant:.2f\PYZcb{}}\PY{l+s+s2}{\PYZdq{}}
              \PY{o}{.}\PY{n}{format}\PY{p}{(}\PY{n}{flottant}\PY{o}{=}\PY{n}{pi}\PY{p}{)}\PY{p}{)}
\end{Verbatim}


\begin{Verbatim}[commandchars=\\\{\},frame=single,framerule=0.3mm,rulecolor=\color{cellframecolor}]
{\color{outcolor}Out[{\color{outcolor}19}]:} 'pi avec seulement 2 chiffres apres la virgule 3.14'
\end{Verbatim}
            
    Dans ces deux exemples, la partie à l'intérieur des \texttt{\{\}} et à
droite du \texttt{:} s'appelle le format, ici \texttt{.2f}~; vous
remarquez que c'est le même pour les \emph{f-strings} et pour
\texttt{format}, et c'est toujours le cas. C'est pourquoi on ne verra
plus à partir d'ici que des exemples avec les \emph{f-strings}.

    \hypertarget{en-duxe9but-de-nombre}{%
\subsubsection{\texorpdfstring{\texttt{0} en début de
nombre}{0 en début de nombre}}\label{en-duxe9but-de-nombre}}

    Pour forcer un petit entier à s'afficher sur 4 caractères, avec des
\texttt{0} ajoutés au début si nécessaire~:

    \begin{Verbatim}[commandchars=\\\{\},frame=single,framerule=0.3mm,rulecolor=\color{cellframecolor}]
{\color{incolor}In [{\color{incolor}20}]:} \PY{n}{x} \PY{o}{=} \PY{l+m+mi}{15}
         
         \PY{n}{f}\PY{l+s+s2}{\PYZdq{}}\PY{l+s+si}{\PYZob{}x:04d\PYZcb{}}\PY{l+s+s2}{\PYZdq{}}
\end{Verbatim}


\begin{Verbatim}[commandchars=\\\{\},frame=single,framerule=0.3mm,rulecolor=\color{cellframecolor}]
{\color{outcolor}Out[{\color{outcolor}20}]:} '0015'
\end{Verbatim}
            
    Ici on utilise le format \texttt{d} (toutes ces lettres \texttt{d},
\texttt{f}, \texttt{g} viennent des formats ancestraux de la libc comme
\texttt{printf}). Ici avec \texttt{04d} on précise qu'on veut une sortie
sur 4 caractères et qu'il faut remplir à gauche si nécessaire avec des
\texttt{0}.

    \hypertarget{largeur-fixe}{%
\subsubsection{Largeur fixe}\label{largeur-fixe}}

    Dans certains cas, on a besoin d'afficher des données en colonnes de
largeur fixe, on utilise pour cela les formats \texttt{\textless{}}
\texttt{\^{}} et \texttt{\textgreater{}} pour afficher à gauche, au
centre, ou à droite d'une zone de largeur fixe~:

    \begin{Verbatim}[commandchars=\\\{\},frame=single,framerule=0.3mm,rulecolor=\color{cellframecolor}]
{\color{incolor}In [{\color{incolor}21}]:} \PY{c+c1}{\PYZsh{} les données à afficher}
         \PY{n}{comptes} \PY{o}{=} \PY{p}{[}
          \PY{p}{(}\PY{l+s+s1}{\PYZsq{}}\PY{l+s+s1}{Apollin}\PY{l+s+s1}{\PYZsq{}}\PY{p}{,} \PY{l+s+s1}{\PYZsq{}}\PY{l+s+s1}{Dupont}\PY{l+s+s1}{\PYZsq{}}\PY{p}{,} \PY{l+m+mi}{127}\PY{p}{)}\PY{p}{,}
          \PY{p}{(}\PY{l+s+s1}{\PYZsq{}}\PY{l+s+s1}{Myrtille}\PY{l+s+s1}{\PYZsq{}}\PY{p}{,} \PY{l+s+s1}{\PYZsq{}}\PY{l+s+s1}{Lamartine}\PY{l+s+s1}{\PYZsq{}}\PY{p}{,} \PY{l+m+mi}{25432}\PY{p}{)}\PY{p}{,}
          \PY{p}{(}\PY{l+s+s1}{\PYZsq{}}\PY{l+s+s1}{Prune}\PY{l+s+s1}{\PYZsq{}}\PY{p}{,} \PY{l+s+s1}{\PYZsq{}}\PY{l+s+s1}{Soc}\PY{l+s+s1}{\PYZsq{}}\PY{p}{,} \PY{l+m+mi}{827465}\PY{p}{)}\PY{p}{,}
         \PY{p}{]}
         
         \PY{k}{for} \PY{n}{prenom}\PY{p}{,} \PY{n}{nom}\PY{p}{,} \PY{n}{solde} \PY{o+ow}{in} \PY{n}{comptes}\PY{p}{:}
             \PY{n+nb}{print}\PY{p}{(}\PY{n}{f}\PY{l+s+s2}{\PYZdq{}}\PY{l+s+si}{\PYZob{}prenom:\PYZlt{}10\PYZcb{}}\PY{l+s+s2}{ \PYZhy{}\PYZhy{} }\PY{l+s+si}{\PYZob{}nom:\PYZca{}12\PYZcb{}}\PY{l+s+s2}{ \PYZhy{}\PYZhy{} }\PY{l+s+si}{\PYZob{}solde:\PYZgt{}8\PYZcb{}}\PY{l+s+s2}{ €}\PY{l+s+s2}{\PYZdq{}}\PY{p}{)}
\end{Verbatim}


    \begin{Verbatim}[commandchars=\\\{\},frame=single,framerule=0.3mm,rulecolor=\color{cellframecolor}]
Apollin    --    Dupont    --      127 €
Myrtille   --  Lamartine   --    25432 €
Prune      --     Soc      --   827465 €
\end{Verbatim}

    \hypertarget{voir-aussi}{%
\subsubsection{Voir aussi}\label{voir-aussi}}

    Nous vous invitons à vous reporter à la documentation de \texttt{format}
pour plus de détails
\href{https://docs.python.org/3/library/string.html\#formatstrings}{sur
les formats disponibles}, et notamment aux
\href{https://docs.python.org/3/library/string.html\#format-examples}{nombreux
exemples} qui y figurent.


    % Add a bibliography block to the postdoc
    
    
    
