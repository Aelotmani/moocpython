    \hypertarget{passage-darguments}{%
\section{Passage d'arguments}\label{passage-darguments}}

    \hypertarget{exercice---niveau-basique}{%
\subsection{Exercice - niveau basique}\label{exercice---niveau-basique}}

    \begin{Verbatim}[commandchars=\\\{\}]
{\color{incolor}In [{\color{incolor} }]:} \PY{c+c1}{\PYZsh{} pour charger l\PYZsq{}exercice}
        \PY{k+kn}{from} \PY{n+nn}{corrections}\PY{n+nn}{.}\PY{n+nn}{exo\PYZus{}distance} \PY{k}{import} \PY{n}{exo\PYZus{}distance}
\end{Verbatim}


    Vous devez écrire une fonction \texttt{distance} qui prend un nombre
quelconque d'arguments numériques non complexes, et qui retourne la
racine carrée de la somme des carrés des arguments.\\

Plus précisément~: \(distance\) (\(x_1\), \ldots{}, \(x_n\)) =
\(\sqrt{\sum x_i^2}\)\\

Par convention on fixe que \(distance\) () = 0

    \begin{Verbatim}[commandchars=\\\{\}]
{\color{incolor}In [{\color{incolor} }]:} \PY{c+c1}{\PYZsh{} des exemples}
        \PY{n}{exo\PYZus{}distance}\PY{o}{.}\PY{n}{example}\PY{p}{(}\PY{p}{)}
\end{Verbatim}


    \begin{Verbatim}[commandchars=\\\{\}]
{\color{incolor}In [{\color{incolor} }]:} \PY{c+c1}{\PYZsh{} ATTENTION vous devez aussi définir les arguments de la fonction}
        \PY{k}{def} \PY{n+nf}{distance}\PY{p}{(}\PY{n}{votre}\PY{p}{,} \PY{n}{signature}\PY{p}{)}\PY{p}{:}
            \PY{k}{return} \PY{l+s+s2}{\PYZdq{}}\PY{l+s+s2}{votre code}\PY{l+s+s2}{\PYZdq{}}
\end{Verbatim}


    \begin{Verbatim}[commandchars=\\\{\}]
{\color{incolor}In [{\color{incolor} }]:} \PY{c+c1}{\PYZsh{} la correction}
        \PY{n}{exo\PYZus{}distance}\PY{o}{.}\PY{n}{correction}\PY{p}{(}\PY{n}{distance}\PY{p}{)}
\end{Verbatim}


    \hypertarget{exercice---niveau-intermuxe9diaire}{%
\subsection{Exercice - niveau
intermédiaire}\label{exercice---niveau-intermuxe9diaire}}

    \begin{Verbatim}[commandchars=\\\{\}]
{\color{incolor}In [{\color{incolor} }]:} \PY{c+c1}{\PYZsh{} Pour charger l\PYZsq{}exercice}
        \PY{k+kn}{from} \PY{n+nn}{corrections}\PY{n+nn}{.}\PY{n+nn}{exo\PYZus{}numbers} \PY{k}{import} \PY{n}{exo\PYZus{}numbers}
\end{Verbatim}


    On vous demande d'écrire une fonction \texttt{numbers}
    
\begin{itemize}
	\item 
	qui prend en argument un nombre quelconque d'entiers,
	\item
	et qui retourne un tuple contenant
	\begin{itemize}
		\item 
		la somme
		\item
		le minimum
		\item
		le maximum de ses arguments.
	\end{itemize}
\end{itemize}

    Si aucun argument n'est passé, \texttt{numbers} doit renvoyer un tuple
contenant 3 entiers \texttt{0}.

    \begin{Verbatim}[commandchars=\\\{\}]
{\color{incolor}In [{\color{incolor} }]:} \PY{c+c1}{\PYZsh{} par exemple}
        \PY{n}{exo\PYZus{}numbers}\PY{o}{.}\PY{n}{example}\PY{p}{(}\PY{p}{)}
\end{Verbatim}


    En guise d'indice, je vous invite à regarder les fonctions
\emph{built-in}
\href{https://docs.python.org/3/library/functions.html\#sum}{\texttt{sum}},
\href{https://docs.python.org/3/library/functions.html\#min}{\texttt{min}}
et
\href{https://docs.python.org/3/library/functions.html\#max}{\texttt{max}}.

    \begin{Verbatim}[commandchars=\\\{\}]
{\color{incolor}In [{\color{incolor} }]:} \PY{c+c1}{\PYZsh{} vous devez définir votre propre signature}
        \PY{k}{def} \PY{n+nf}{numbers}\PY{p}{(}\PY{n}{votre}\PY{p}{,} \PY{n}{signature}\PY{p}{)}\PY{p}{:}
            \PY{l+s+s2}{\PYZdq{}}\PY{l+s+s2}{\PYZlt{}votre\PYZus{}code\PYZgt{}}\PY{l+s+s2}{\PYZdq{}}
\end{Verbatim}


    \begin{Verbatim}[commandchars=\\\{\}]
{\color{incolor}In [{\color{incolor} }]:} \PY{c+c1}{\PYZsh{} pour vérifier votre code}
        \PY{n}{exo\PYZus{}numbers}\PY{o}{.}\PY{n}{correction}\PY{p}{(}\PY{n}{numbers}\PY{p}{)}
\end{Verbatim}