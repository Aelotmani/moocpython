    
    
    
    

    

    \hypertarget{un-piuxe8ge-courant}{%
\section{Un piège courant}\label{un-piuxe8ge-courant}}

    \hypertarget{compluxe9ment---niveau-basique}{%
\subsection{Complément - niveau
basique}\label{compluxe9ment---niveau-basique}}

    \hypertarget{nutilisez-pas-dobjet-mutable-pour-les-valeurs-par-duxe9faut}{%
\subsubsection{N'utilisez pas d'objet mutable pour les valeurs par
défaut}\label{nutilisez-pas-dobjet-mutable-pour-les-valeurs-par-duxe9faut}}

    En Python il existe un piège dans lequel il est très facile de tomber.
Aussi si vous voulez aller à l'essentiel~: \textbf{n'utilisez pas
d'objet mutable pour les valeurs par défaut} lors de la définition d'une
fonction.

Si vous avez besoin d'écrire une fonction qui prend en argument par
défaut une liste ou un dictionnaire vide, voici comment faire

    \begin{Verbatim}[commandchars=\\\{\}]
{\color{incolor}In [{\color{incolor}1}]:} \PY{c+c1}{\PYZsh{} ne faites SURTOUT PAS ça}
        \PY{k}{def} \PY{n+nf}{ne\PYZus{}faites\PYZus{}pas\PYZus{}ca}\PY{p}{(}\PY{n}{options}\PY{o}{=}\PY{p}{\PYZob{}}\PY{p}{\PYZcb{}}\PY{p}{)}\PY{p}{:}
            \PY{l+s+s2}{\PYZdq{}}\PY{l+s+s2}{faire quelque chose}\PY{l+s+s2}{\PYZdq{}}
\end{Verbatim}


    \begin{Verbatim}[commandchars=\\\{\}]
{\color{incolor}In [{\color{incolor}2}]:} \PY{c+c1}{\PYZsh{} mais plutôt comme ceci}
        \PY{k}{def} \PY{n+nf}{mais\PYZus{}plutot\PYZus{}ceci}\PY{p}{(}\PY{n}{options}\PY{o}{=}\PY{k+kc}{None}\PY{p}{)}\PY{p}{:}
            \PY{k}{if} \PY{n}{options} \PY{o+ow}{is} \PY{k+kc}{None}\PY{p}{:} 
                \PY{n}{options} \PY{o}{=} \PY{p}{\PYZob{}}\PY{p}{\PYZcb{}}
            \PY{l+s+s2}{\PYZdq{}}\PY{l+s+s2}{faire quelque chose}\PY{l+s+s2}{\PYZdq{}}
\end{Verbatim}


    \hypertarget{compluxe9ment---niveau-intermuxe9diaire}{%
\subsection{Complément - niveau
intermédiaire}\label{compluxe9ment---niveau-intermuxe9diaire}}

    \hypertarget{que-se-passe-t-il-si-on-le-fait}{%
\subsubsection{Que se passe-t-il si on le
fait~?}\label{que-se-passe-t-il-si-on-le-fait}}

    Considérons le cas relativement simple d'une fonction qui calcule une
valeur - ici un entier aléatoire entre 0 et 10 -, et l'ajoute à une
liste passée par l'appelant.

    Et pour rendre la vie de l'appelant plus facile, on se dit qu'il peut
être utile de faire en sorte que si l'appelant n'a pas de liste sous la
main, on va créer pour lui une liste vide. Et pour ça on fait~:

    \begin{Verbatim}[commandchars=\\\{\}]
{\color{incolor}In [{\color{incolor}3}]:} \PY{k+kn}{import} \PY{n+nn}{random}
        
        \PY{c+c1}{\PYZsh{} l\PYZsq{}intention ici est que si l\PYZsq{}appelant ne fournit pas }
        \PY{c+c1}{\PYZsh{} la liste en entrée, on crée pour lui une liste vide}
        \PY{k}{def} \PY{n+nf}{ajouter\PYZus{}un\PYZus{}aleatoire}\PY{p}{(}\PY{n}{resultats}\PY{o}{=}\PY{p}{[}\PY{p}{]}\PY{p}{)}\PY{p}{:}
            \PY{n}{resultats}\PY{o}{.}\PY{n}{append}\PY{p}{(}\PY{n}{random}\PY{o}{.}\PY{n}{randint}\PY{p}{(}\PY{l+m+mi}{0}\PY{p}{,} \PY{l+m+mi}{10}\PY{p}{)}\PY{p}{)}
            \PY{k}{return} \PY{n}{resultats}
\end{Verbatim}


    Si on appelle cette fonction une première fois, tout semble bien aller

    \begin{Verbatim}[commandchars=\\\{\}]
{\color{incolor}In [{\color{incolor}4}]:} \PY{n}{ajouter\PYZus{}un\PYZus{}aleatoire}\PY{p}{(}\PY{p}{)}
\end{Verbatim}


\begin{Verbatim}[commandchars=\\\{\}]
{\color{outcolor}Out[{\color{outcolor}4}]:} [10]
\end{Verbatim}
            
    Sauf que, si on appelle la fonction une deuxième fois, on a une
surprise~!

    \begin{Verbatim}[commandchars=\\\{\}]
{\color{incolor}In [{\color{incolor}5}]:} \PY{n}{ajouter\PYZus{}un\PYZus{}aleatoire}\PY{p}{(}\PY{p}{)}
\end{Verbatim}


\begin{Verbatim}[commandchars=\\\{\}]
{\color{outcolor}Out[{\color{outcolor}5}]:} [10, 9]
\end{Verbatim}
            
    \hypertarget{pourquoi}{%
\subsubsection{Pourquoi~?}\label{pourquoi}}

    Le problème ici est qu'une valeur par défaut - ici l'expression
\texttt{{[}{]}} - est évaluée \textbf{une fois} au moment de la
\textbf{définition} de la fonction.

Toutes les fois où la fonction est appelée avec cet argument manquant,
on va utiliser comme valeur par défaut \textbf{le même objet}, qui la
première fois est bien une liste vide, mais qui se fait modifier par le
premier appel.

Si bien que la deuxième fois on réutilise la même liste \textbf{qui
n'est plus vide}. Pour aller plus loin, vous pouvez regarder la
documentation Python sur
\href{https://docs.python.org/3/faq/programming.html\#why-are-default-values-shared-between-objects}{ce
problème}.


    % Add a bibliography block to the postdoc
    
    
    
