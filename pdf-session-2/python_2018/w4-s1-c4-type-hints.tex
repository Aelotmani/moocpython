    
    
    
    

    

    \hypertarget{type-hints}{%
\section{\texorpdfstring{\emph{Type
hints}}{Type hints}}\label{type-hints}}

    \hypertarget{compluxe9ment---niveau-intermuxe9diaire}{%
\subsection{Complément - niveau
intermédiaire}\label{compluxe9ment---niveau-intermuxe9diaire}}

    \hypertarget{langages-compiluxe9s}{%
\subsubsection{Langages compilés}\label{langages-compiluxe9s}}

    Nous avons évoqué en première semaine le typage, lorsque nous avons
comparé Python avec les langages compilés. Dans un langage compilé avec
typage statique, on \textbf{doit fournir du typage}, ce qui fait qu'on
écrit typiquement une fonction comme ceci~:

\begin{verbatim}
int factoriel(int n) {
  return (n<=1) ? 1 : n * factoriel(n-1);
}
\end{verbatim}

ce qui signifie que la fonction factoriel prend un premier argument qui
est un entier, et qu'elle retourne également un entier.

    Nous avons vu également que, par contraste, pour écrire une fonction en
Python, on n'a \textbf{pas besoin} de préciser \textbf{le type} des
arguments ni du retour de la fonction.

    \hypertarget{vous-pouvez-aussi-typer-votre-code-python}{%
\subsubsection{Vous pouvez aussi typer votre code
python}\label{vous-pouvez-aussi-typer-votre-code-python}}

    Cependant depuis la version 3.5, python supporte un mécanisme
\textbf{totalement optionnel} qui vous permet d'annoter les arguments
des fonctions avec des informations de typage, ce mécanisme est connu
sous le nom de \emph{type hints}, et ça se présente comme ceci~:

    \hypertarget{typer-une-variable}{%
\subparagraph{typer une variable}\label{typer-une-variable}}

    \begin{Verbatim}[commandchars=\\\{\}]
{\color{incolor}In [{\color{incolor}1}]:} \PY{c+c1}{\PYZsh{} pour typer une variable avec les type hints}
        \PY{n}{nb\PYZus{}items} \PY{p}{:} \PY{n+nb}{int} \PY{o}{=} \PY{l+m+mi}{0}
\end{Verbatim}


    \begin{Verbatim}[commandchars=\\\{\}]
{\color{incolor}In [{\color{incolor}2}]:} \PY{n}{nb\PYZus{}items}
\end{Verbatim}


\begin{Verbatim}[commandchars=\\\{\}]
{\color{outcolor}Out[{\color{outcolor}2}]:} 0
\end{Verbatim}
            
    \hypertarget{typer-les-paramuxe8tres-et-le-retour-dune-fonction}{%
\subparagraph{typer les paramètres et le retour d'une
fonction}\label{typer-les-paramuxe8tres-et-le-retour-dune-fonction}}

    \begin{Verbatim}[commandchars=\\\{\}]
{\color{incolor}In [{\color{incolor}3}]:} \PY{c+c1}{\PYZsh{} une fonction factorielle avec des type hints}
        \PY{k}{def} \PY{n+nf}{fact}\PY{p}{(}\PY{n}{n} \PY{p}{:} \PY{n+nb}{int}\PY{p}{)} \PY{o}{\PYZhy{}}\PY{o}{\PYZgt{}} \PY{n+nb}{int}\PY{p}{:}
            \PY{k}{return} \PY{l+m+mi}{1} \PY{k}{if} \PY{n}{n} \PY{o}{\PYZlt{}}\PY{o}{=} \PY{l+m+mi}{1} \PY{k}{else} \PY{n}{n} \PY{o}{*} \PY{n}{fact}\PY{p}{(}\PY{n}{n}\PY{o}{\PYZhy{}}\PY{l+m+mi}{1}\PY{p}{)}
\end{Verbatim}


    \begin{Verbatim}[commandchars=\\\{\}]
{\color{incolor}In [{\color{incolor}4}]:} \PY{n}{fact}\PY{p}{(}\PY{l+m+mi}{12}\PY{p}{)}
\end{Verbatim}


\begin{Verbatim}[commandchars=\\\{\}]
{\color{outcolor}Out[{\color{outcolor}4}]:} 479001600
\end{Verbatim}
            
    \hypertarget{usages}{%
\subsubsection{Usages}\label{usages}}

    À ce stade, on peut entrevoir les usages suivants à ce type
d'annotation~:

\begin{itemize}
\tightlist
\item
  tout d'abord, et évidemment, cela peut permettre de mieux documenter
  le code~;
\item
  les environnements de développement sont susceptibles de vous aider de
  manière plus effective~; si quelque part vous écrivez
  \texttt{z\ =\ fact(12)}, le fait de savoir que \texttt{z} est entier
  permet de fournir une complétion plus pertinente lorsque vous
  commencez à écrire \texttt{z.{[}TAB{]}}~;
\item
  on peut espérer trouver des erreurs dans les passages d'arguments à un
  stade plus précoce du développement.
\end{itemize}

    Par contre ce qui est très très clairement annoncé également, c'est que
ces informations de typage sont \textbf{totalement facultatives}, et que
le langage les \textbf{ignore totalement}.

    \begin{Verbatim}[commandchars=\\\{\}]
{\color{incolor}In [{\color{incolor}5}]:} \PY{c+c1}{\PYZsh{} l\PYZsq{}interpréteur ignore totalement ces informations}
        \PY{k}{def} \PY{n+nf}{fake\PYZus{}fact}\PY{p}{(}\PY{n}{n} \PY{p}{:} \PY{n+nb}{str}\PY{p}{)} \PY{o}{\PYZhy{}}\PY{o}{\PYZgt{}} \PY{n+nb}{str}\PY{p}{:}
            \PY{k}{return} \PY{l+m+mi}{1} \PY{k}{if} \PY{n}{n} \PY{o}{\PYZlt{}}\PY{o}{=} \PY{l+m+mi}{1} \PY{k}{else} \PY{n}{n} \PY{o}{*} \PY{n}{fake\PYZus{}fact}\PY{p}{(}\PY{n}{n}\PY{o}{\PYZhy{}}\PY{l+m+mi}{1}\PY{p}{)}
        
        \PY{c+c1}{\PYZsh{} on peut appeler fake\PYZus{}fact avec un int alors }
        \PY{c+c1}{\PYZsh{} que c\PYZsq{}est déclaré pour des str}
        \PY{n}{fake\PYZus{}fact}\PY{p}{(}\PY{l+m+mi}{12}\PY{p}{)}
\end{Verbatim}


\begin{Verbatim}[commandchars=\\\{\}]
{\color{outcolor}Out[{\color{outcolor}5}]:} 479001600
\end{Verbatim}
            
    Le modèle préconisé est d'utiliser des \textbf{outils extérieurs}, qui
peuvent faire une analyse statique du code pour exploiter ces
informations à des fins de validation. Dans cette catégorie, le plus
célèbre \href{http://mypy-lang.org/}{est sans doute \texttt{mypy}}.
Notez aussi que les IDE comme PyCharm sont également capables de tirer
parti de ces annotations.

    \hypertarget{est-ce-ruxe9pandu}{%
\subsubsection{Est-ce répandu ?}\label{est-ce-ruxe9pandu}}

    Parce qu'ils ont été introduits pour la première fois avec python-3.5,
en 2015 donc, puis améliorés dans la 3.6 pour le typage des variables,
l'usage des \emph{type hints} n'est pour l'instant pas très répandu, en
proportion de code en tous cas. En outre, il aura fallu un temps de
latence avant que tous les outils (IDE's, producteurs de documentation,
outils de test, validateurs\ldots{}) ne soient améliorés pour en tirer
un profit maximal.

On peut penser que cet usage va se répandre avec le temps, peut-être /
sans doute pas de manière systématique, mais \emph{a minima} pour lever
certaines ambiguïtés.

    \hypertarget{comment-annoter-son-code}{%
\subsubsection{Comment annoter son
code}\label{comment-annoter-son-code}}

    Maintenant que nous en avons bien vu la finalité, voyons un très bref
aperçu des possibilités offertes pour la construction des types dans ce
contexte de \emph{type hints}. N'hésitez pas à vous reporter à la
documentation officielle
\href{https://docs.python.org/3/library/typing.html}{du module
\texttt{typing}} pour un exposé plus exhaustif.

    \hypertarget{le-module-typing}{%
\subparagraph{\texorpdfstring{le module
\texttt{typing}}{le module typing}}\label{le-module-typing}}

    L'ensemble des symboles que nous allons utiliser dans la suite de ce
complément provient du module \texttt{typing}

    \hypertarget{exemples-simples}{%
\subparagraph{exemples simples}\label{exemples-simples}}

    \begin{Verbatim}[commandchars=\\\{\}]
{\color{incolor}In [{\color{incolor}6}]:} \PY{k+kn}{from} \PY{n+nn}{typing} \PY{k}{import} \PY{n}{List}
\end{Verbatim}


    \begin{Verbatim}[commandchars=\\\{\}]
{\color{incolor}In [{\color{incolor}7}]:} \PY{c+c1}{\PYZsh{} une fonction qui }
        \PY{c+c1}{\PYZsh{} attend un paramètre qui soit une liste d\PYZsq{}entiers,}
        \PY{c+c1}{\PYZsh{} et qui retourne une liste de chaînes}
        \PY{k}{def} \PY{n+nf}{foo}\PY{p}{(}\PY{n}{x}\PY{p}{:} \PY{n}{List}\PY{p}{[}\PY{n+nb}{int}\PY{p}{]}\PY{p}{)} \PY{o}{\PYZhy{}}\PY{o}{\PYZgt{}} \PY{n}{List}\PY{p}{[}\PY{n+nb}{str}\PY{p}{]}\PY{p}{:}
            \PY{k}{pass}    
\end{Verbatim}


    \hypertarget{avertissement-list-vs-list}{%
\subparagraph{\texorpdfstring{avertissement~: \texttt{list} vs
\texttt{List}}{avertissement~: list vs List}}\label{avertissement-list-vs-list}}

    Remarquez bien dans l'exemple ci-dessus que nous avons utilisé
\texttt{typing.List} plutôt que le type \emph{built-in} \texttt{list},
alors que l'on a pu par contre utiliser \texttt{int} et \texttt{str}.

Les raisons pour cela sont de deux ordres~:

\begin{itemize}
\item
  tout d'abord, si je devais utiliser \texttt{list} pour construire un
  type comme \emph{liste d'entiers}, il me faudrait écrire quelque chose
  comme \texttt{list(int)} ou encore \texttt{list{[}int{]}}, et cela
  serait source de confusion car ceci a déjà une signification dans le
  langage~;
\item
  de manière plus profonde, il faut distinguer entre \texttt{list} qui
  est un type concret (un objet qui sert à construire des instances), de
  \texttt{List} qui dans ce contexte doit plus être vu comme un type
  abstrait.
\end{itemize}

    Pour bien voir cela, considérez l'exemple suivant~:

    \begin{Verbatim}[commandchars=\\\{\}]
{\color{incolor}In [{\color{incolor}8}]:} \PY{k+kn}{from} \PY{n+nn}{typing} \PY{k}{import} \PY{n}{Iterable}
\end{Verbatim}


    \begin{Verbatim}[commandchars=\\\{\}]
{\color{incolor}In [{\color{incolor}9}]:} \PY{k}{def} \PY{n+nf}{lower\PYZus{}split}\PY{p}{(}\PY{n}{sep}\PY{p}{:} \PY{n+nb}{str}\PY{p}{,} \PY{n}{inputs} \PY{p}{:} \PY{n}{Iterable}\PY{p}{[}\PY{n+nb}{str}\PY{p}{]}\PY{p}{)} \PY{o}{\PYZhy{}}\PY{o}{\PYZgt{}} \PY{n+nb}{str}\PY{p}{:}
            \PY{k}{return} \PY{n}{sep}\PY{o}{.}\PY{n}{join}\PY{p}{(}\PY{p}{[}\PY{n}{x}\PY{o}{.}\PY{n}{lower}\PY{p}{(}\PY{p}{)} \PY{k}{for} \PY{n}{x} \PY{o+ow}{in} \PY{n}{inputs}\PY{p}{]}\PY{p}{)}
\end{Verbatim}


    \begin{Verbatim}[commandchars=\\\{\}]
{\color{incolor}In [{\color{incolor}10}]:} \PY{n}{lower\PYZus{}split}\PY{p}{(}\PY{l+s+s1}{\PYZsq{}}\PY{l+s+s1}{\PYZhy{}\PYZhy{}}\PY{l+s+s1}{\PYZsq{}}\PY{p}{,} \PY{p}{(}\PY{l+s+s1}{\PYZsq{}}\PY{l+s+s1}{AB}\PY{l+s+s1}{\PYZsq{}}\PY{p}{,} \PY{l+s+s1}{\PYZsq{}}\PY{l+s+s1}{CD}\PY{l+s+s1}{\PYZsq{}}\PY{p}{,} \PY{l+s+s1}{\PYZsq{}}\PY{l+s+s1}{EF}\PY{l+s+s1}{\PYZsq{}}\PY{p}{)}\PY{p}{)}
\end{Verbatim}


\begin{Verbatim}[commandchars=\\\{\}]
{\color{outcolor}Out[{\color{outcolor}10}]:} 'ab--cd--ef'
\end{Verbatim}
            
    On voit bien dans cet exemple que \texttt{Iterable} ne correspond pas à
un type concret particulier, c'est un type abstrait dans le sens du
\emph{duck typing}.

    \hypertarget{un-exemple-plus-complet}{%
\subparagraph{un exemple plus complet}\label{un-exemple-plus-complet}}

    Voici un exemple tiré de la documentation du module \texttt{typing} qui
illustre davantage de types construits à partir des types \emph{builtin}
du langage~:

    \begin{Verbatim}[commandchars=\\\{\}]
{\color{incolor}In [{\color{incolor}11}]:} \PY{k+kn}{from} \PY{n+nn}{typing} \PY{k}{import} \PY{n}{Dict}\PY{p}{,} \PY{n}{Tuple}\PY{p}{,} \PY{n}{List}
         
         \PY{n}{ConnectionOptions} \PY{o}{=} \PY{n}{Dict}\PY{p}{[}\PY{n+nb}{str}\PY{p}{,} \PY{n+nb}{str}\PY{p}{]}
         \PY{n}{Address} \PY{o}{=} \PY{n}{Tuple}\PY{p}{[}\PY{n+nb}{str}\PY{p}{,} \PY{n+nb}{int}\PY{p}{]}
         \PY{n}{Server} \PY{o}{=} \PY{n}{Tuple}\PY{p}{[}\PY{n}{Address}\PY{p}{,} \PY{n}{ConnectionOptions}\PY{p}{]}
         
         \PY{k}{def} \PY{n+nf}{broadcast\PYZus{}message}\PY{p}{(}\PY{n}{message}\PY{p}{:} \PY{n+nb}{str}\PY{p}{,} \PY{n}{servers}\PY{p}{:} \PY{n}{List}\PY{p}{[}\PY{n}{Server}\PY{p}{]}\PY{p}{)} \PY{o}{\PYZhy{}}\PY{o}{\PYZgt{}} \PY{k+kc}{None}\PY{p}{:}
             \PY{o}{.}\PY{o}{.}\PY{o}{.}
         
         \PY{c+c1}{\PYZsh{} The static type checker will treat the previous type signature as}
         \PY{c+c1}{\PYZsh{} being exactly equivalent to this one.}
         \PY{k}{def} \PY{n+nf}{broadcast\PYZus{}message}\PY{p}{(}
                 \PY{n}{message}\PY{p}{:} \PY{n+nb}{str}\PY{p}{,}
                 \PY{n}{servers}\PY{p}{:} \PY{n}{List}\PY{p}{[}\PY{n}{Tuple}\PY{p}{[}\PY{n}{Tuple}\PY{p}{[}\PY{n+nb}{str}\PY{p}{,} \PY{n+nb}{int}\PY{p}{]}\PY{p}{,} \PY{n}{Dict}\PY{p}{[}\PY{n+nb}{str}\PY{p}{,} \PY{n+nb}{str}\PY{p}{]}\PY{p}{]}\PY{p}{]}\PY{p}{)} \PY{o}{\PYZhy{}}\PY{o}{\PYZgt{}} \PY{k+kc}{None}\PY{p}{:}
             \PY{o}{.}\PY{o}{.}\PY{o}{.}
\end{Verbatim}


    J'en profite d'ailleurs (ça n'a rien a voir, mais\ldots{}) pour vous
signaler un objet python assez étrange~:

    \begin{Verbatim}[commandchars=\\\{\}]
{\color{incolor}In [{\color{incolor}12}]:} \PY{c+c1}{\PYZsh{} L\PYZsq{}objet ... existe bel et bien en Python}
         \PY{n}{el} \PY{o}{=} \PY{o}{.}\PY{o}{.}\PY{o}{.}
         \PY{n}{el}
\end{Verbatim}


\begin{Verbatim}[commandchars=\\\{\}]
{\color{outcolor}Out[{\color{outcolor}12}]:} Ellipsis
\end{Verbatim}
            
    qui sert principalement pour le slicing multidimensionnel de numpy. Mais
ne nous égarons pas\ldots{}

    \hypertarget{typage-partiel}{%
\subparagraph{typage partiel}\label{typage-partiel}}

    Puisque c'est un mécanisme optionnel, vous pouvez tout à fait ne typer
qu'une partie de vos variables et paramètres~:

    \begin{Verbatim}[commandchars=\\\{\}]
{\color{incolor}In [{\color{incolor}13}]:} \PY{c+c1}{\PYZsh{} imaginez que vous ne typez pas n2, ni la valeur de retour}
         
         \PY{c+c1}{\PYZsh{} c\PYZsq{}est équivalent de dire ceci}
         \PY{k}{def} \PY{n+nf}{partially\PYZus{}typed}\PY{p}{(}\PY{n}{n1}\PY{p}{:} \PY{n+nb}{int}\PY{p}{,} \PY{n}{n2}\PY{p}{)}\PY{p}{:}
             \PY{k}{return} \PY{k+kc}{None}
\end{Verbatim}


    \begin{Verbatim}[commandchars=\\\{\}]
{\color{incolor}In [{\color{incolor}14}]:} \PY{c+c1}{\PYZsh{} ou cela}
         \PY{k+kn}{from} \PY{n+nn}{typing} \PY{k}{import} \PY{n}{Any}
         
         \PY{k}{def} \PY{n+nf}{partially\PYZus{}typed}\PY{p}{(}\PY{n}{n1}\PY{p}{:} \PY{n+nb}{int}\PY{p}{,} \PY{n}{n2}\PY{p}{:} \PY{n}{Any}\PY{p}{)} \PY{o}{\PYZhy{}}\PY{o}{\PYZgt{}} \PY{n}{Any}\PY{p}{:}
             \PY{k}{return} \PY{k+kc}{None}
\end{Verbatim}


    \hypertarget{alias}{%
\subparagraph{alias}\label{alias}}

    On peut facilement se définir des alias~; lorsque vous avez implémenté
un système d'identifiants basé sur le type \texttt{int}, il est
préférable de faire~:

    \begin{Verbatim}[commandchars=\\\{\}]
{\color{incolor}In [{\color{incolor}15}]:} \PY{k+kn}{from} \PY{n+nn}{typing} \PY{k}{import} \PY{n}{NewType}
         
         \PY{n}{UserId} \PY{o}{=} \PY{n}{NewType}\PY{p}{(}\PY{l+s+s1}{\PYZsq{}}\PY{l+s+s1}{UserId}\PY{l+s+s1}{\PYZsq{}}\PY{p}{,} \PY{n+nb}{int}\PY{p}{)}
         
         \PY{n}{user1\PYZus{}id} \PY{p}{:} \PY{n}{UserId} \PY{o}{=} \PY{l+m+mi}{0}
\end{Verbatim}


    plutôt que ceci, qui est beaucoup moins parlant~:

    \begin{Verbatim}[commandchars=\\\{\}]
{\color{incolor}In [{\color{incolor}16}]:} \PY{n}{user1\PYZus{}id} \PY{p}{:} \PY{n+nb}{int} \PY{o}{=} \PY{l+m+mi}{0}
\end{Verbatim}


    \hypertarget{compluxe9ment---niveau-avancuxe9}{%
\subsection{Complément - niveau
avancé}\label{compluxe9ment---niveau-avancuxe9}}

    \hypertarget{generic}{%
\subparagraph{\texorpdfstring{\texttt{Generic}}{Generic}}\label{generic}}

    Pour ceux qui connaissent déjà la notion de classe (les autres peuvent
ignorer la fin de ce complément)~:

    Grâce aux constructions \texttt{TypeVar} et \texttt{Generic}, il est
possible de manipuler une notion de \emph{variable de type}, que je vous
montre sur un exemple tiré à nouveau de la documentation du module
\texttt{typing}~:

    \begin{Verbatim}[commandchars=\\\{\}]
{\color{incolor}In [{\color{incolor}17}]:} \PY{k+kn}{from} \PY{n+nn}{typing} \PY{k}{import} \PY{n}{TypeVar}\PY{p}{,} \PY{n}{Generic}
         \PY{k+kn}{from} \PY{n+nn}{logging} \PY{k}{import} \PY{n}{Logger}
         
         \PY{n}{T} \PY{o}{=} \PY{n}{TypeVar}\PY{p}{(}\PY{l+s+s1}{\PYZsq{}}\PY{l+s+s1}{T}\PY{l+s+s1}{\PYZsq{}}\PY{p}{)}
         
         \PY{k}{class} \PY{n+nc}{LoggedVar}\PY{p}{(}\PY{n}{Generic}\PY{p}{[}\PY{n}{T}\PY{p}{]}\PY{p}{)}\PY{p}{:}
             \PY{k}{def} \PY{n+nf}{\PYZus{}\PYZus{}init\PYZus{}\PYZus{}}\PY{p}{(}\PY{n+nb+bp}{self}\PY{p}{,} \PY{n}{value}\PY{p}{:} \PY{n}{T}\PY{p}{,} \PY{n}{name}\PY{p}{:} \PY{n+nb}{str}\PY{p}{,} \PY{n}{logger}\PY{p}{:} \PY{n}{Logger}\PY{p}{)} \PY{o}{\PYZhy{}}\PY{o}{\PYZgt{}} \PY{k+kc}{None}\PY{p}{:}
                 \PY{n+nb+bp}{self}\PY{o}{.}\PY{n}{name} \PY{o}{=} \PY{n}{name}
                 \PY{n+nb+bp}{self}\PY{o}{.}\PY{n}{logger} \PY{o}{=} \PY{n}{logger}
                 \PY{n+nb+bp}{self}\PY{o}{.}\PY{n}{value} \PY{o}{=} \PY{n}{value}
         
             \PY{k}{def} \PY{n+nf}{set}\PY{p}{(}\PY{n+nb+bp}{self}\PY{p}{,} \PY{n}{new}\PY{p}{:} \PY{n}{T}\PY{p}{)} \PY{o}{\PYZhy{}}\PY{o}{\PYZgt{}} \PY{k+kc}{None}\PY{p}{:}
                 \PY{n+nb+bp}{self}\PY{o}{.}\PY{n}{log}\PY{p}{(}\PY{l+s+s1}{\PYZsq{}}\PY{l+s+s1}{Set }\PY{l+s+s1}{\PYZsq{}} \PY{o}{+} \PY{n+nb}{repr}\PY{p}{(}\PY{n+nb+bp}{self}\PY{o}{.}\PY{n}{value}\PY{p}{)}\PY{p}{)}
                 \PY{n+nb+bp}{self}\PY{o}{.}\PY{n}{value} \PY{o}{=} \PY{n}{new}
         
             \PY{k}{def} \PY{n+nf}{get}\PY{p}{(}\PY{n+nb+bp}{self}\PY{p}{)} \PY{o}{\PYZhy{}}\PY{o}{\PYZgt{}} \PY{n}{T}\PY{p}{:}
                 \PY{n+nb+bp}{self}\PY{o}{.}\PY{n}{log}\PY{p}{(}\PY{l+s+s1}{\PYZsq{}}\PY{l+s+s1}{Get }\PY{l+s+s1}{\PYZsq{}} \PY{o}{+} \PY{n+nb}{repr}\PY{p}{(}\PY{n+nb+bp}{self}\PY{o}{.}\PY{n}{value}\PY{p}{)}\PY{p}{)}
                 \PY{k}{return} \PY{n+nb+bp}{self}\PY{o}{.}\PY{n}{value}
         
             \PY{k}{def} \PY{n+nf}{log}\PY{p}{(}\PY{n+nb+bp}{self}\PY{p}{,} \PY{n}{message}\PY{p}{:} \PY{n+nb}{str}\PY{p}{)} \PY{o}{\PYZhy{}}\PY{o}{\PYZgt{}} \PY{k+kc}{None}\PY{p}{:}
                 \PY{n+nb+bp}{self}\PY{o}{.}\PY{n}{logger}\PY{o}{.}\PY{n}{info}\PY{p}{(}\PY{l+s+s1}{\PYZsq{}}\PY{l+s+si}{\PYZpc{}s}\PY{l+s+s1}{: }\PY{l+s+si}{\PYZpc{}s}\PY{l+s+s1}{\PYZsq{}}\PY{p}{,} \PY{n+nb+bp}{self}\PY{o}{.}\PY{n}{name}\PY{p}{,} \PY{n}{message}\PY{p}{)}
\end{Verbatim}


    qui vous donne je l'espère une idée de ce qu'il est possible de faire,
et jusqu'où on peut aller avec les \emph{type hints}. Si vous êtes
intéressé par cette fonctionnalité, je vous invite
\href{https://docs.python.org/3/library/typing.html\#user-defined-generic-types}{à
poursuivre la lecture ici}.

    \hypertarget{pour-en-savoir-plus}{%
\subsubsection{Pour en savoir plus}\label{pour-en-savoir-plus}}

\begin{itemize}
\item
  la documentation officielle sur
  \href{https://docs.python.org/3/library/typing.html}{le module
  typing}~;
\item
  la page d'accueil \href{http://mypy-lang.org/}{de l'outil mypy}.
\item
  le \href{https://www.python.org/dev/peps/pep-0484/}{PEP-525} sur le
  typage des paramètres et retours de fonctions, implémenté dans
  python-3.5~;
\item
  le \href{https://www.python.org/dev/peps/pep-0526/}{PEP-526} sur le
  typage des variables, implémenté dans 3.6.
\end{itemize}


    % Add a bibliography block to the postdoc
    
    
    
