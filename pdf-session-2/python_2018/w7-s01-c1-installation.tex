    
    
    
    

    

    \hypertarget{installations-suppluxe9mentaires}{%
\section{Installations
supplémentaires}\label{installations-suppluxe9mentaires}}

    \hypertarget{compluxe9ment---niveau-basique}{%
\subsection{Complément - niveau
basique}\label{compluxe9ment---niveau-basique}}

    Les outils que nous voyons cette semaine, bien que jouant un rôle majeur
dans le succès de l'écosystème Python, \textbf{ne font pas} partie de la
\textbf{distribution standard}. Cela signifie qu'il vous faut
éventuellement procéder à des installations complémentaires sur votre
ordinateur (évidemment vous pouvez utiliser les notebooks sans
installation de votre part).

    \hypertarget{comment-savoir}{%
\subsubsection{Comment savoir~?}\label{comment-savoir}}

    Pour savoir si votre installation est idoine, vous devez pouvoir faire
ceci~:

    \begin{Verbatim}[commandchars=\\\{\}]
{\color{incolor}In [{\color{incolor}1}]:} \PY{k+kn}{import} \PY{n+nn}{numpy} \PY{k}{as} \PY{n+nn}{np}
        \PY{k+kn}{import} \PY{n+nn}{matplotlib}\PY{n+nn}{.}\PY{n+nn}{pyplot} \PY{k}{as} \PY{n+nn}{plt}
\end{Verbatim}


    \begin{Verbatim}[commandchars=\\\{\}]
{\color{incolor}In [{\color{incolor}2}]:} \PY{k+kn}{import} \PY{n+nn}{pandas} \PY{k}{as} \PY{n+nn}{pd}
\end{Verbatim}


    \hypertarget{avec-anaconda}{%
\subsubsection{Avec (ana)conda}\label{avec-anaconda}}

    Si vous avez installé votre Python avec conda, selon toute probabilité
toutes ces bibliothèques sont déjà accessibles pour vous, vous n'avez
rien à faire de particulier pour pouvoir faire tourner les exemples du
cours sur votre ordinateur.

    \hypertarget{distribution-standard}{%
\subsubsection{Distribution standard}\label{distribution-standard}}

    Si vous avez installé Python à partir d'une distribution standard, vous
pouvez utiliser \texttt{pip} comme ceci ; naturellement ceci doit être
fait \textbf{dans un terminal} et non pas dans l'interpréteur python, ni
dans IDLE~:

    \begin{verbatim}
$ pip3 install numpy matplotlib pandas
\end{verbatim}

    \hypertarget{debianubuntu}{%
\subsubsection{Debian/Ubuntu}\label{debianubuntu}}

    Si vous utilisez Debian ou Ubuntu, et que vous avez déjà installé Python
avec \texttt{apt-get}, la méthode préconisée sera~:

    \begin{verbatim}
$ apt-get install python3-numpy python3-matplotlib python3-pandas
\end{verbatim}

    \hypertarget{fedora}{%
\subsubsection{Fedora}\label{fedora}}

    De manière similaire sur Fedora ou RHEL~:

    \begin{verbatim}
$ dnf install python3-numpy python3-matplotlib python3-pandas
\end{verbatim}


    % Add a bibliography block to the postdoc
    
    
    
