    
    
    
    

    

    \hypertarget{arguments-keyword-only}{%
\section{\texorpdfstring{Arguments
\emph{keyword-only}}{Arguments keyword-only}}\label{arguments-keyword-only}}

    \hypertarget{compluxe9ment---niveau-intermuxe9diaire}{%
\subsection{Complément - niveau
intermédiaire}\label{compluxe9ment---niveau-intermuxe9diaire}}

    \hypertarget{rappel}{%
\subsubsection{Rappel}\label{rappel}}

    Nous avons vu dans un précédent complément les 4 familles de paramètres
qu'on peut déclarer dans une fonction~:

\begin{enumerate}
\def\labelenumi{\arabic{enumi}.}
\tightlist
\item
  paramètres positionnels (usuels)
\item
  paramètres nommés (forme \emph{name=default})
\item
  paramètres **args* qui attrape dans un tuple le reliquat des arguments
  positionnels
\item
  paramètres **\emph{kwds} qui attrape dans un dictionnaire le reliquat
  des arguments nommés
\end{enumerate}

Pour rappel~:

    \begin{Verbatim}[commandchars=\\\{\},frame=single,framerule=0.3mm,rulecolor=\color{cellframecolor}]
{\color{incolor}In [{\color{incolor}1}]:} \PY{c+c1}{\PYZsh{} une fonction qui combine les différents }
        \PY{c+c1}{\PYZsh{} types de paramètres}
        \PY{k}{def} \PY{n+nf}{foo}\PY{p}{(}\PY{n}{a}\PY{p}{,} \PY{n}{b}\PY{o}{=}\PY{l+m+mi}{100}\PY{p}{,} \PY{o}{*}\PY{n}{args}\PY{p}{,} \PY{o}{*}\PY{o}{*}\PY{n}{kwds}\PY{p}{)}\PY{p}{:}
            \PY{n+nb}{print}\PY{p}{(}\PY{n}{f}\PY{l+s+s2}{\PYZdq{}}\PY{l+s+s2}{a=}\PY{l+s+si}{\PYZob{}a\PYZcb{}}\PY{l+s+s2}{, b=}\PY{l+s+si}{\PYZob{}b\PYZcb{}}\PY{l+s+s2}{, args=}\PY{l+s+si}{\PYZob{}args\PYZcb{}}\PY{l+s+s2}{, kwds=}\PY{l+s+si}{\PYZob{}kwds\PYZcb{}}\PY{l+s+s2}{\PYZdq{}}\PY{p}{)}
\end{Verbatim}


    \begin{Verbatim}[commandchars=\\\{\},frame=single,framerule=0.3mm,rulecolor=\color{cellframecolor}]
{\color{incolor}In [{\color{incolor}2}]:} \PY{n}{foo}\PY{p}{(}\PY{l+m+mi}{1}\PY{p}{)}
\end{Verbatim}


    \begin{Verbatim}[commandchars=\\\{\},frame=single,framerule=0.3mm,rulecolor=\color{cellframecolor}]
a=1, b=100, args=(), kwds=\{\}
\end{Verbatim}

    \begin{Verbatim}[commandchars=\\\{\},frame=single,framerule=0.3mm,rulecolor=\color{cellframecolor}]
{\color{incolor}In [{\color{incolor}3}]:} \PY{n}{foo}\PY{p}{(}\PY{l+m+mi}{1}\PY{p}{,} \PY{l+m+mi}{2}\PY{p}{)}
\end{Verbatim}


    \begin{Verbatim}[commandchars=\\\{\},frame=single,framerule=0.3mm,rulecolor=\color{cellframecolor}]
a=1, b=2, args=(), kwds=\{\}
\end{Verbatim}

    \begin{Verbatim}[commandchars=\\\{\},frame=single,framerule=0.3mm,rulecolor=\color{cellframecolor}]
{\color{incolor}In [{\color{incolor}4}]:} \PY{n}{foo}\PY{p}{(}\PY{l+m+mi}{1}\PY{p}{,} \PY{l+m+mi}{2}\PY{p}{,} \PY{l+m+mi}{3}\PY{p}{)}
\end{Verbatim}


    \begin{Verbatim}[commandchars=\\\{\},frame=single,framerule=0.3mm,rulecolor=\color{cellframecolor}]
a=1, b=2, args=(3,), kwds=\{\}
\end{Verbatim}

    \begin{Verbatim}[commandchars=\\\{\},frame=single,framerule=0.3mm,rulecolor=\color{cellframecolor}]
{\color{incolor}In [{\color{incolor}5}]:} \PY{n}{foo}\PY{p}{(}\PY{l+m+mi}{1}\PY{p}{,} \PY{l+m+mi}{2}\PY{p}{,} \PY{l+m+mi}{3}\PY{p}{,} \PY{n}{bar}\PY{o}{=}\PY{l+m+mi}{1000}\PY{p}{)}
\end{Verbatim}


    \begin{Verbatim}[commandchars=\\\{\},frame=single,framerule=0.3mm,rulecolor=\color{cellframecolor}]
a=1, b=2, args=(3,), kwds=\{'bar': 1000\}
\end{Verbatim}

    \hypertarget{un-seul-paramuxe8tre-attrape-tout}{%
\subsubsection{Un seul paramètre
attrape-tout}\label{un-seul-paramuxe8tre-attrape-tout}}

    Notez également que, de bon sens, on ne peut déclarer qu'un seul
paramètre de chacune des formes d'attrape-tout~; on ne peut pas par
exemple déclarer

\begin{Shaded}
\begin{Highlighting}[frame=lines,framerule=0.6mm,rulecolor=\color{asisframecolor}]
\CommentTok{# c'est illégal de faire ceci}
\KeywordTok{def}\NormalTok{ foo(}\OperatorTok{*}\NormalTok{args1, }\OperatorTok{*}\NormalTok{args2):}
    \ControlFlowTok{pass}
\end{Highlighting}
\end{Shaded}

car évidemment on ne saurait pas décider de ce qui va dans
\texttt{args1} et ce qui va dans \texttt{args2}.

    \hypertarget{ordre-des-duxe9clarations}{%
\subsubsection{Ordre des déclarations}\label{ordre-des-duxe9clarations}}

    L'ordre dans lequel sont déclarés les différents types de paramètres
d'une fonction est imposé par le langage. Ce que vous avez peut-être en
tête si vous avez appris \textbf{Python 2}, c'est qu'à l'époque on
devait impérativement les déclarer dans cet ordre~:

\begin{quote}
positionnels, nommés, forme \texttt{*}, forme \texttt{**}
\end{quote}

comme dans notre fonction \texttt{foo}.

    Ça reste une bonne approximation, mais depuis Python-3, les choses ont
un petit peu changé suite à
\href{https://www.python.org/dev/peps/pep-3102/}{l'adoption du PEP
3102}, qui vise à introduire la notion de paramètre qu'il faut
impérativement nommer lors de l'appel (en anglais~: \emph{keyword-only}
argument)

    Pour résumer, il est maintenant possible de déclarer des
\textbf{paramètres nommés après la forme \texttt{*}}

Voyons cela sur un exemple

    \begin{Verbatim}[commandchars=\\\{\},frame=single,framerule=0.3mm,rulecolor=\color{cellframecolor}]
{\color{incolor}In [{\color{incolor}6}]:} \PY{c+c1}{\PYZsh{} on peut déclarer un paramètre nommé **après** l\PYZsq{}attrape\PYZhy{}tout *args}
        \PY{k}{def} \PY{n+nf}{bar}\PY{p}{(}\PY{n}{a}\PY{p}{,} \PY{o}{*}\PY{n}{args}\PY{p}{,} \PY{n}{b}\PY{o}{=}\PY{l+m+mi}{100}\PY{p}{,} \PY{o}{*}\PY{o}{*}\PY{n}{kwds}\PY{p}{)}\PY{p}{:}
                \PY{n+nb}{print}\PY{p}{(}\PY{n}{f}\PY{l+s+s2}{\PYZdq{}}\PY{l+s+s2}{a=}\PY{l+s+si}{\PYZob{}a\PYZcb{}}\PY{l+s+s2}{, b=}\PY{l+s+si}{\PYZob{}b\PYZcb{}}\PY{l+s+s2}{, args=}\PY{l+s+si}{\PYZob{}args\PYZcb{}}\PY{l+s+s2}{, kwds=}\PY{l+s+si}{\PYZob{}kwds\PYZcb{}}\PY{l+s+s2}{\PYZdq{}}\PY{p}{)}
\end{Verbatim}


    L'effet de cette déclaration est que, si je veux passer un argument au
paramètre \texttt{b}, \textbf{je dois le nommer}

    \begin{Verbatim}[commandchars=\\\{\},frame=single,framerule=0.3mm,rulecolor=\color{cellframecolor}]
{\color{incolor}In [{\color{incolor}7}]:} \PY{c+c1}{\PYZsh{} je peux toujours faire ceci}
        \PY{n}{bar}\PY{p}{(}\PY{l+m+mi}{1}\PY{p}{)}
\end{Verbatim}


    \begin{Verbatim}[commandchars=\\\{\},frame=single,framerule=0.3mm,rulecolor=\color{cellframecolor}]
a=1, b=100, args=(), kwds=\{\}
\end{Verbatim}

    \begin{Verbatim}[commandchars=\\\{\},frame=single,framerule=0.3mm,rulecolor=\color{cellframecolor}]
{\color{incolor}In [{\color{incolor}8}]:} \PY{c+c1}{\PYZsh{} mais si je fais ceci l\PYZsq{}argument 2 va aller dans args}
        \PY{n}{bar}\PY{p}{(}\PY{l+m+mi}{1}\PY{p}{,} \PY{l+m+mi}{2}\PY{p}{)}
\end{Verbatim}


    \begin{Verbatim}[commandchars=\\\{\},frame=single,framerule=0.3mm,rulecolor=\color{cellframecolor}]
a=1, b=100, args=(2,), kwds=\{\}
\end{Verbatim}

    \begin{Verbatim}[commandchars=\\\{\},frame=single,framerule=0.3mm,rulecolor=\color{cellframecolor}]
{\color{incolor}In [{\color{incolor}9}]:} \PY{c+c1}{\PYZsh{} pour passer b=2, je **dois** nommer mon argument}
        \PY{n}{bar}\PY{p}{(}\PY{l+m+mi}{1}\PY{p}{,} \PY{n}{b}\PY{o}{=}\PY{l+m+mi}{2}\PY{p}{)}
\end{Verbatim}


    \begin{Verbatim}[commandchars=\\\{\},frame=single,framerule=0.3mm,rulecolor=\color{cellframecolor}]
a=1, b=2, args=(), kwds=\{\}
\end{Verbatim}

    Ce trait n'est objectivement pas utilisé massivement en Python, mais
cela peut être utile de le savoir~:

\begin{itemize}
\tightlist
\item
  en tant qu'utilisateur d'une bibliothèque, car cela vous impose une
  certaine façon d'appeler une fonction~;
\item
  en tant que concepteur d'une fonction, car cela vous permet de
  manifester qu'un paramètre optionnel joue un rôle particulier.
\end{itemize}


    % Add a bibliography block to the postdoc
    
    
    
