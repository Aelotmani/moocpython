    
    
    
    

    

    \hypertarget{expressions-guxe9nuxe9ratrices}{%
\section{Expressions
génératrices}\label{expressions-guxe9nuxe9ratrices}}

    \hypertarget{compluxe9ment---niveau-basique}{%
\subsection{Complément - niveau
basique}\label{compluxe9ment---niveau-basique}}

    \hypertarget{comment-transformer-une-compruxe9hension-de-liste-en-ituxe9rateur}{%
\subsubsection{Comment transformer une compréhension de liste en
itérateur~?}\label{comment-transformer-une-compruxe9hension-de-liste-en-ituxe9rateur}}

    Nous venons de voir les fonctions génératrices qui sont un puissant
outil pour créer facilement des itérateurs. Nous verrons prochainement
comment utiliser ces fonctions génératrices pour transformer en quelques
lignes de code vos propres objets en itérateurs.

Vous savez maintenant qu'en Python on favorise la notion d'itérateurs
puisqu'ils se manipulent comme des objets itérables et qu'ils sont en
général beaucoup plus compacts en mémoire que l'itérable correspondant.

Comme les compréhensions de listes sont fréquemment utilisées en Python,
mais qu'elles sont des itérables potentiellement gourmands en ressources
mémoire, on souhaiterait pouvoir créer un itérateur directement à partir
d'une compréhension de liste. C'est possible et très facile en Python.
Il suffit de remplacer les crochets par des parenthèses, regardons cela.

    \begin{Verbatim}[commandchars=\\\{\},frame=single,framerule=0.3mm,rulecolor=\color{cellframecolor}]
{\color{incolor}In [{\color{incolor}1}]:} \PY{c+c1}{\PYZsh{} c\PYZsq{}est une compréhension de liste}
        \PY{n}{comprehension} \PY{o}{=} \PY{p}{[}\PY{n}{x}\PY{o}{*}\PY{o}{*}\PY{l+m+mi}{2} \PY{k}{for} \PY{n}{x} \PY{o+ow}{in} \PY{n+nb}{range}\PY{p}{(}\PY{l+m+mi}{100}\PY{p}{)} \PY{k}{if} \PY{n}{x}\PY{o}{\PYZpc{}}\PY{k}{17} == 0] 
        \PY{n+nb}{print}\PY{p}{(}\PY{n}{comprehension}\PY{p}{)}
\end{Verbatim}


    \begin{Verbatim}[commandchars=\\\{\},frame=single,framerule=0.3mm,rulecolor=\color{cellframecolor}]
[0, 289, 1156, 2601, 4624, 7225]
\end{Verbatim}

    \begin{Verbatim}[commandchars=\\\{\},frame=single,framerule=0.3mm,rulecolor=\color{cellframecolor}]
{\color{incolor}In [{\color{incolor}2}]:} \PY{c+c1}{\PYZsh{} c\PYZsq{}est une expression génératrice}
        \PY{n}{generator} \PY{o}{=} \PY{p}{(}\PY{n}{x}\PY{o}{*}\PY{o}{*}\PY{l+m+mi}{2} \PY{k}{for} \PY{n}{x} \PY{o+ow}{in} \PY{n+nb}{range}\PY{p}{(}\PY{l+m+mi}{100}\PY{p}{)} \PY{k}{if} \PY{n}{x}\PY{o}{\PYZpc{}}\PY{k}{17} == 0) 
        \PY{n+nb}{print}\PY{p}{(}\PY{n}{generator}\PY{p}{)}
\end{Verbatim}


    \begin{Verbatim}[commandchars=\\\{\},frame=single,framerule=0.3mm,rulecolor=\color{cellframecolor}]
<generator object <genexpr> at 0x10e385ed0>
\end{Verbatim}

    Ensuite pour utiliser une expression génératrice, c'est très simple, on
l'utilise comme n'importe quel itérateur.

    \begin{Verbatim}[commandchars=\\\{\},frame=single,framerule=0.3mm,rulecolor=\color{cellframecolor}]
{\color{incolor}In [{\color{incolor}3}]:} \PY{n}{generator} \PY{o+ow}{is} \PY{n+nb}{iter}\PY{p}{(}\PY{n}{generator}\PY{p}{)} \PY{c+c1}{\PYZsh{} generator est bien un itérateur}
\end{Verbatim}


\begin{Verbatim}[commandchars=\\\{\},frame=single,framerule=0.3mm,rulecolor=\color{cellframecolor}]
{\color{outcolor}Out[{\color{outcolor}3}]:} True
\end{Verbatim}
            
    \begin{Verbatim}[commandchars=\\\{\},frame=single,framerule=0.3mm,rulecolor=\color{cellframecolor}]
{\color{incolor}In [{\color{incolor}4}]:} \PY{c+c1}{\PYZsh{} affiche les premiers carrés des multiples de 17}
        \PY{k}{for} \PY{n}{count}\PY{p}{,} \PY{n}{carre} \PY{o+ow}{in} \PY{n+nb}{enumerate}\PY{p}{(}\PY{n}{generator}\PY{p}{,} \PY{l+m+mi}{1}\PY{p}{)}\PY{p}{:}
            \PY{n+nb}{print}\PY{p}{(}\PY{n}{f}\PY{l+s+s1}{\PYZsq{}}\PY{l+s+s1}{Contenu de generator après }\PY{l+s+si}{\PYZob{}count\PYZcb{}}\PY{l+s+s1}{ itérations : }\PY{l+s+si}{\PYZob{}carre\PYZcb{}}\PY{l+s+s1}{\PYZsq{}}\PY{p}{)}
\end{Verbatim}


    \begin{Verbatim}[commandchars=\\\{\},frame=single,framerule=0.3mm,rulecolor=\color{cellframecolor}]
Contenu de generator après 1 itérations : 0
Contenu de generator après 2 itérations : 289
Contenu de generator après 3 itérations : 1156
Contenu de generator après 4 itérations : 2601
Contenu de generator après 5 itérations : 4624
Contenu de generator après 6 itérations : 7225
\end{Verbatim}

    Avec une expression génératrice on n'est plus limité comme avec les
compréhensions par le nombre d'éléments~:

    \begin{Verbatim}[commandchars=\\\{\},frame=single,framerule=0.3mm,rulecolor=\color{cellframecolor}]
{\color{incolor}In [{\color{incolor}5}]:} \PY{c+c1}{\PYZsh{} trop grand pour une compréhension,}
        \PY{c+c1}{\PYZsh{} mais on peut créer le générateur sans souci}
        \PY{n}{generator} \PY{o}{=} \PY{p}{(}\PY{n}{x}\PY{o}{*}\PY{o}{*}\PY{l+m+mi}{2} \PY{k}{for} \PY{n}{x} \PY{o+ow}{in} \PY{n+nb}{range}\PY{p}{(}\PY{l+m+mi}{10}\PY{o}{*}\PY{o}{*}\PY{l+m+mi}{18}\PY{p}{)} \PY{k}{if} \PY{n}{x}\PY{o}{\PYZpc{}}\PY{k}{17}==0) 
        
        \PY{c+c1}{\PYZsh{} on va calculer tous les carrés de multiples de 17 }
        \PY{c+c1}{\PYZsh{} plus petits que 10**10 et dont les 4 derniers chiffres sont 1316}
        \PY{n}{recherche} \PY{o}{=} \PY{n+nb}{set}\PY{p}{(}\PY{p}{)}
        
        \PY{c+c1}{\PYZsh{} le point important, c\PYZsq{}est qu\PYZsq{}on n\PYZsq{}a pas besoin de }
        \PY{c+c1}{\PYZsh{} créer une liste de 10**18 éléments }
        \PY{c+c1}{\PYZsh{} qui serait beaucoup trop grosse pour la mettre dans la mémoire vive}
        
        \PY{c+c1}{\PYZsh{} avec un générateur, on ne paie que ce qu\PYZsq{}on utilise...}
        \PY{k}{for} \PY{n}{x} \PY{o+ow}{in} \PY{n}{generator}\PY{p}{:}
            \PY{k}{if} \PY{n}{x} \PY{o}{\PYZgt{}} \PY{l+m+mi}{10}\PY{o}{*}\PY{o}{*}\PY{l+m+mi}{10}\PY{p}{:}
                \PY{k}{break}
            \PY{k}{elif} \PY{n+nb}{str}\PY{p}{(}\PY{n}{x}\PY{p}{)}\PY{p}{[}\PY{o}{\PYZhy{}}\PY{l+m+mi}{4}\PY{p}{:}\PY{p}{]} \PY{o}{==} \PY{l+s+s1}{\PYZsq{}}\PY{l+s+s1}{1316}\PY{l+s+s1}{\PYZsq{}}\PY{p}{:}
                \PY{n}{recherche}\PY{o}{.}\PY{n}{add}\PY{p}{(}\PY{n}{x}\PY{p}{)}
        \PY{n+nb}{print}\PY{p}{(}\PY{n}{recherche}\PY{p}{)}
\end{Verbatim}


    \begin{Verbatim}[commandchars=\\\{\},frame=single,framerule=0.3mm,rulecolor=\color{cellframecolor}]
\{617721316, 4536561316, 3617541316, 311381316\}
\end{Verbatim}

    \hypertarget{compluxe9ment---niveau-intermuxe9diaire}{%
\subsection{Complément - niveau
intermédiaire}\label{compluxe9ment---niveau-intermuxe9diaire}}

    \hypertarget{compruxe9hension-vs-expression-guxe9nuxe9ratrice}{%
\subsubsection{\texorpdfstring{Compréhension \emph{vs} expression
génératrice}{Compréhension vs expression génératrice}}\label{compruxe9hension-vs-expression-guxe9nuxe9ratrice}}

    \hypertarget{digression-liste-vs-ituxe9rateur}{%
\paragraph{\texorpdfstring{Digression~: liste \emph{vs}
itérateur}{Digression~: liste vs itérateur}}\label{digression-liste-vs-ituxe9rateur}}

    En Python 3, nous avons déjà rencontré la fonction \texttt{range} qui
retourne les premiers entiers.

Ou plutôt, c'est \textbf{comme si} elle retournait les premiers entiers
lorsqu'on fait une boucle \texttt{for}

    \begin{Verbatim}[commandchars=\\\{\},frame=single,framerule=0.3mm,rulecolor=\color{cellframecolor}]
{\color{incolor}In [{\color{incolor}6}]:} \PY{c+c1}{\PYZsh{} on peut parcourir un range comme si c\PYZsq{}était une liste}
        \PY{k}{for} \PY{n}{i} \PY{o+ow}{in} \PY{n+nb}{range}\PY{p}{(}\PY{l+m+mi}{4}\PY{p}{)}\PY{p}{:}
            \PY{n+nb}{print}\PY{p}{(}\PY{n}{i}\PY{p}{)}
\end{Verbatim}


    \begin{Verbatim}[commandchars=\\\{\},frame=single,framerule=0.3mm,rulecolor=\color{cellframecolor}]
0
1
2
3
\end{Verbatim}

    mais en réalité le résultat de \texttt{range} exhibe un comportement un
peu étrange, en ce sens que~:

    \begin{Verbatim}[commandchars=\\\{\},frame=single,framerule=0.3mm,rulecolor=\color{cellframecolor}]
{\color{incolor}In [{\color{incolor}7}]:} \PY{c+c1}{\PYZsh{} mais en fait la fonction range ne renvoie PAS une liste (depuis Python 3)}
        \PY{n+nb}{range}\PY{p}{(}\PY{l+m+mi}{4}\PY{p}{)}
\end{Verbatim}


\begin{Verbatim}[commandchars=\\\{\},frame=single,framerule=0.3mm,rulecolor=\color{cellframecolor}]
{\color{outcolor}Out[{\color{outcolor}7}]:} range(0, 4)
\end{Verbatim}
            
    \begin{Verbatim}[commandchars=\\\{\},frame=single,framerule=0.3mm,rulecolor=\color{cellframecolor}]
{\color{incolor}In [{\color{incolor}8}]:} \PY{c+c1}{\PYZsh{} et en effet ce n\PYZsq{}est pas une liste}
        \PY{n+nb}{isinstance}\PY{p}{(}\PY{n+nb}{range}\PY{p}{(}\PY{l+m+mi}{4}\PY{p}{)}\PY{p}{,} \PY{n+nb}{list}\PY{p}{)}
\end{Verbatim}


\begin{Verbatim}[commandchars=\\\{\},frame=single,framerule=0.3mm,rulecolor=\color{cellframecolor}]
{\color{outcolor}Out[{\color{outcolor}8}]:} False
\end{Verbatim}
            
    La raison de fond pour ceci, c'est que \textbf{le fait de construire une
liste} est une opération relativement coûteuse - toutes proportions
gardées - car il est nécessaire d'allouer de la mémoire pour
\textbf{stocker tous les éléments} de la liste à un instant donné~;
alors qu'en fait dans l'immense majorité des cas, on n'a \textbf{pas
réellement besoin} de cette place mémoire, tout ce dont on a besoin
c'est d'itérer sur un certain nombre de valeurs mais \textbf{qui peuvent
être calculées} au fur et à mesure que l'on parcourt la liste.

    \hypertarget{compruxe9hension-et-expression-guxe9nuxe9ratrice}{%
\paragraph{Compréhension et expression
génératrice}\label{compruxe9hension-et-expression-guxe9nuxe9ratrice}}

    À la lumière de ce qui vient d'être dit, on peut voir qu'une
compréhension n'est \textbf{pas toujours le bon choix}, car par
définition elle \textbf{construit une liste} de résultats - de la
fonction appliquée successivement aux entrées.

Or dans les cas où, comme pour \texttt{range}, on n'a pas réellement
besoin de cette liste \textbf{en tant que telle} mais seulement de cet
artefact pour pouvoir itérer sur la liste des résultats, il est
préférable d'utiliser une \textbf{expression génératrice}.

Voyons tout de suite sur un exemple à quoi cela ressemblerait.

    \begin{Verbatim}[commandchars=\\\{\},frame=single,framerule=0.3mm,rulecolor=\color{cellframecolor}]
{\color{incolor}In [{\color{incolor}9}]:} \PY{n}{depart} \PY{o}{=} \PY{p}{(}\PY{o}{\PYZhy{}}\PY{l+m+mi}{5}\PY{p}{,} \PY{o}{\PYZhy{}}\PY{l+m+mi}{3}\PY{p}{,} \PY{l+m+mi}{0}\PY{p}{,} \PY{l+m+mi}{3}\PY{p}{,} \PY{l+m+mi}{5}\PY{p}{,} \PY{l+m+mi}{10}\PY{p}{)}
        \PY{c+c1}{\PYZsh{} dans le premier calcul de arrivee }
        \PY{c+c1}{\PYZsh{} pour rappel, la compréhension est entre []}
        \PY{c+c1}{\PYZsh{} arrivee = [x**2 for x in depart]}
        
        \PY{c+c1}{\PYZsh{} on peut écrire presque la même chose avec des () à la place }
        \PY{n}{arrivee2} \PY{o}{=} \PY{p}{(}\PY{n}{x}\PY{o}{*}\PY{o}{*}\PY{l+m+mi}{2} \PY{k}{for} \PY{n}{x} \PY{o+ow}{in} \PY{n}{depart}\PY{p}{)}
        \PY{n}{arrivee2}
\end{Verbatim}


\begin{Verbatim}[commandchars=\\\{\},frame=single,framerule=0.3mm,rulecolor=\color{cellframecolor}]
{\color{outcolor}Out[{\color{outcolor}9}]:} <generator object <genexpr> at 0x10e411138>
\end{Verbatim}
            
    Comme pour \texttt{range}, le résultat de l'expression génératrice ne se
laisse pas regarder avec \texttt{print}, mais comme pour \texttt{range},
on peut itérer sur le résultat~:

    \begin{Verbatim}[commandchars=\\\{\},frame=single,framerule=0.3mm,rulecolor=\color{cellframecolor}]
{\color{incolor}In [{\color{incolor}10}]:} \PY{k}{for} \PY{n}{x}\PY{p}{,} \PY{n}{y} \PY{o+ow}{in} \PY{n+nb}{zip}\PY{p}{(}\PY{n}{depart}\PY{p}{,} \PY{n}{arrivee2}\PY{p}{)}\PY{p}{:}
             \PY{n+nb}{print}\PY{p}{(}\PY{n}{f}\PY{l+s+s2}{\PYZdq{}}\PY{l+s+s2}{x=}\PY{l+s+si}{\PYZob{}x\PYZcb{}}\PY{l+s+s2}{ =\PYZgt{} y=}\PY{l+s+si}{\PYZob{}y\PYZcb{}}\PY{l+s+s2}{\PYZdq{}}\PY{p}{)}
\end{Verbatim}


    \begin{Verbatim}[commandchars=\\\{\},frame=single,framerule=0.3mm,rulecolor=\color{cellframecolor}]
x=-5 => y=25
x=-3 => y=9
x=0 => y=0
x=3 => y=9
x=5 => y=25
x=10 => y=100
\end{Verbatim}

    Il n'est pas \textbf{toujours} possible de remplacer une compréhension
par une expression génératrice, mais c'est \textbf{souvent souhaitable},
car de cette façon on peut faire de substantielles économies en matière
de performances. On peut le faire dès lors que l'on a seulement besoin
d'itérer sur les résultats.

    Il faut juste un peu se méfier, car comme on parle ici d'itérateurs,
comme toujours si on essaie de faire plusieurs fois une boucle sur le
même itérateur, il ne se passe plus rien, car l'itérateur a été épuisé~:

    \begin{Verbatim}[commandchars=\\\{\},frame=single,framerule=0.3mm,rulecolor=\color{cellframecolor}]
{\color{incolor}In [{\color{incolor}11}]:} \PY{k}{for} \PY{n}{x}\PY{p}{,} \PY{n}{y} \PY{o+ow}{in} \PY{n+nb}{zip}\PY{p}{(}\PY{n}{depart}\PY{p}{,} \PY{n}{arrivee2}\PY{p}{)}\PY{p}{:}
             \PY{n+nb}{print}\PY{p}{(}\PY{n}{f}\PY{l+s+s2}{\PYZdq{}}\PY{l+s+s2}{x=}\PY{l+s+si}{\PYZob{}x\PYZcb{}}\PY{l+s+s2}{ =\PYZgt{} y=}\PY{l+s+si}{\PYZob{}y\PYZcb{}}\PY{l+s+s2}{\PYZdq{}}\PY{p}{)}
\end{Verbatim}


    \hypertarget{pour-aller-plus-loin}{%
\subsubsection{Pour aller plus loin}\label{pour-aller-plus-loin}}

    Vous pouvez regarder
\href{http://python-history.blogspot.fr/2010/06/from-list-comprehensions-to-generator.html}{cette
intéressante discussion de Guido van Rossum} sur les compréhensions et
les expressions génératrices.


    % Add a bibliography block to the postdoc
    
    
    
