    
    
    
    

    

    \hypertarget{introduction-aux-classes}{%
\section{Introduction aux classes}\label{introduction-aux-classes}}

    \hypertarget{compluxe9ment---niveau-basique}{%
\subsection{Complément - niveau
basique}\label{compluxe9ment---niveau-basique}}

    On définit une classe lorsqu'on a besoin de créer un type spécifique au
contexte de l'application. Il faut donc voir une classe au même niveau
qu'un type \emph{built-in} comme \texttt{list} ou \texttt{dict}.

    \hypertarget{un-exemple-simpliste}{%
\subsubsection{Un exemple simpliste}\label{un-exemple-simpliste}}

    Par exemple, imaginons qu'on a besoin de manipuler des matrices
\(2\times 2\)

\(A = \left( \begin{array}{cc} a_{11} & a_{12} \\ a_{21} & a_{22}\end{array} \right)\)

Et en guise d'illustration, nous allons utiliser le déterminant~; c'est
juste un prétexte pour implémenter une méthode sur cette classe, ne vous
inquiétez pas si le terme ne vous dit rien, ou vous rappelle de mauvais
souvenirs. Tout ce qu'on a besoin de savoir c'est que, sur une matrice
de ce type, le déterminant vaut~:

\(det (A) = a_{11}.a_{22} - a_{12}.a_{21}\)

    Dans la pratique, on utiliserait la classe \texttt{matrix} de
\href{http://www.numpy.org/}{\texttt{numpy}} qui est une bibliothèque de
calcul scientifique très populaire et largement utilisée. Mais comme
premier exemple de classe, nous allons écrire \textbf{notre propre
classe \texttt{Matrix2}} pour mettre en action les mécanismes de base
des classes de python. Naturellement, il s'agit d'une implémentation
jouet.

    \begin{Verbatim}[commandchars=\\\{\},frame=single,framerule=0.3mm,rulecolor=\color{cellframecolor}]
{\color{incolor}In [{\color{incolor}1}]:} \PY{k}{class} \PY{n+nc}{Matrix2}\PY{p}{:}
            \PY{l+s+s2}{\PYZdq{}}\PY{l+s+s2}{Une implémentation sommaire de matrice carrée 2x2}\PY{l+s+s2}{\PYZdq{}}
        
            \PY{k}{def} \PY{n+nf}{\PYZus{}\PYZus{}init\PYZus{}\PYZus{}}\PY{p}{(}\PY{n+nb+bp}{self}\PY{p}{,} \PY{n}{a11}\PY{p}{,} \PY{n}{a12}\PY{p}{,} \PY{n}{a21}\PY{p}{,} \PY{n}{a22}\PY{p}{)}\PY{p}{:}
                \PY{l+s+s2}{\PYZdq{}}\PY{l+s+s2}{construit une matrice à partir des 4 coefficients}\PY{l+s+s2}{\PYZdq{}}
                \PY{n+nb+bp}{self}\PY{o}{.}\PY{n}{a11} \PY{o}{=} \PY{n}{a11}
                \PY{n+nb+bp}{self}\PY{o}{.}\PY{n}{a12} \PY{o}{=} \PY{n}{a12}
                \PY{n+nb+bp}{self}\PY{o}{.}\PY{n}{a21} \PY{o}{=} \PY{n}{a21}
                \PY{n+nb+bp}{self}\PY{o}{.}\PY{n}{a22} \PY{o}{=} \PY{n}{a22}
                
            \PY{k}{def} \PY{n+nf}{determinant}\PY{p}{(}\PY{n+nb+bp}{self}\PY{p}{)}\PY{p}{:}
                \PY{l+s+s2}{\PYZdq{}}\PY{l+s+s2}{renvoie le déterminant de la matrice}\PY{l+s+s2}{\PYZdq{}}
                \PY{k}{return} \PY{n+nb+bp}{self}\PY{o}{.}\PY{n}{a11} \PY{o}{*} \PY{n+nb+bp}{self}\PY{o}{.}\PY{n}{a22} \PY{o}{\PYZhy{}} \PY{n+nb+bp}{self}\PY{o}{.}\PY{n}{a12} \PY{o}{*} \PY{n+nb+bp}{self}\PY{o}{.}\PY{n}{a21}
\end{Verbatim}


    \hypertarget{la-premiuxe8re-version-de-matrix2}{%
\subsubsection{\texorpdfstring{La première version de
\texttt{Matrix2}}{La première version de Matrix2}}\label{la-premiuxe8re-version-de-matrix2}}

    \hypertarget{une-classe-peut-avoir-un-docstring}{%
\subparagraph{\texorpdfstring{Une classe peut avoir un
\emph{docstring}}{Une classe peut avoir un docstring}}\label{une-classe-peut-avoir-un-docstring}}

    Pour commencer, vous remarquez qu'on peut attacher à cette classe un
\emph{docstring} comme pour les fonctions

    \begin{Verbatim}[commandchars=\\\{\},frame=single,framerule=0.3mm,rulecolor=\color{cellframecolor}]
{\color{incolor}In [{\color{incolor}2}]:} \PY{n}{help}\PY{p}{(}\PY{n}{Matrix2}\PY{p}{)}
\end{Verbatim}


    \begin{Verbatim}[commandchars=\\\{\},frame=single,framerule=0.3mm,rulecolor=\color{cellframecolor}]
Help on class Matrix2 in module \_\_main\_\_:

class Matrix2(builtins.object)
 |  Matrix2(a11, a12, a21, a22)
 |  
 |  Une implémentation sommaire de matrice carrée 2x2
 |  
 |  Methods defined here:
 |  
 |  \_\_init\_\_(self, a11, a12, a21, a22)
 |      construit une matrice à partir des 4 coefficients
 |  
 |  determinant(self)
 |      renvoie le déterminant de la matrice
 |  
 |  ----------------------------------------------------------------------
 |  Data descriptors defined here:
 |  
 |  \_\_dict\_\_
 |      dictionary for instance variables (if defined)
 |  
 |  \_\_weakref\_\_
 |      list of weak references to the object (if defined)
\end{Verbatim}

    La classe définit donc deux méthodes, nommées \texttt{\_\_init\_\_} et
\texttt{determinant}.

    \hypertarget{la-muxe9thode-__init__}{%
\subparagraph{\texorpdfstring{La méthode
\texttt{\_\_init\_\_}}{La méthode \_\_init\_\_}}\label{la-muxe9thode-__init__}}

    La méthode \textbf{\texttt{\_\_init\_\_}}, comme toutes celles qui ont
un nom en \texttt{\_\_}\emph{nom}\texttt{\_\_}, est une \textbf{méthode
spéciale}. En l'occurrence, il s'agit de ce qu'on appelle le
\textbf{constructeur} de la classe, c'est-à-dire le code qui va être
appelé lorsqu'on crée une instance. Voyons cela tout de suite sur un
exemple.

    \begin{Verbatim}[commandchars=\\\{\},frame=single,framerule=0.3mm,rulecolor=\color{cellframecolor}]
{\color{incolor}In [{\color{incolor}3}]:} \PY{n}{matrice} \PY{o}{=} \PY{n}{Matrix2}\PY{p}{(}\PY{l+m+mi}{1}\PY{p}{,} \PY{l+m+mi}{2}\PY{p}{,} \PY{l+m+mi}{2}\PY{p}{,} \PY{l+m+mi}{1}\PY{p}{)}
        \PY{n+nb}{print}\PY{p}{(}\PY{n}{matrice}\PY{p}{)}
\end{Verbatim}


    \begin{Verbatim}[commandchars=\\\{\},frame=single,framerule=0.3mm,rulecolor=\color{cellframecolor}]
<\_\_main\_\_.Matrix2 object at 0x1089c9e10>
\end{Verbatim}

    Vous remarquez tout d'abord que \texttt{\_\_init\_\_} s'attend à
recevoir \emph{5 arguments}, mais que nous appelons \texttt{Matrix2}
avec seulement \emph{4 arguments}.

L'argument surnuméraire, le \textbf{premier} de ceux qui sont déclarés
dans la méthode, correspond à l'\textbf{instance qui vient d'être créée}
et qui est automatiquement passée par l'interpréteur python à la méthode
\texttt{\_\_init\_\_}. En ce sens, le terme constructeur est impropre
puisque la méthode \texttt{\_\_init\_\_} ne crée pas l'instance, elle ne
fait que l'initialiser, mais c'est un abus de langage très répandu. Nous
reviendrons sur le processus de création des objets lorsque nous
parlerons des métaclasses en dernière semaine.

La \textbf{convention} est de nommer le premier argument de ce
constructeur \textbf{\texttt{self}}, nous y reviendrons un peu plus
loin.

    On voit également que le constructeur se contente de mémoriser, à
l'intérieur de l'instance, les arguments qu'on lui passe, sous la forme
d'\textbf{attributs} de l'\textbf{instance} \texttt{self}.

C'est un cas extrêmement fréquent~; de manière générale, il est
recommandé d'écrire des constructeurs passifs de ce genre~; dit
autrement, on évite de faire trop de traitements dans le constructeur.

    \hypertarget{la-muxe9thode-determinant}{%
\subparagraph{\texorpdfstring{La méthode
\texttt{determinant}}{La méthode determinant}}\label{la-muxe9thode-determinant}}

    La classe définit aussi la méthode \texttt{determinant}, qu'on
utiliserait comme ceci~:

    \begin{Verbatim}[commandchars=\\\{\},frame=single,framerule=0.3mm,rulecolor=\color{cellframecolor}]
{\color{incolor}In [{\color{incolor}4}]:} \PY{n}{matrice}\PY{o}{.}\PY{n}{determinant}\PY{p}{(}\PY{p}{)}
\end{Verbatim}


\begin{Verbatim}[commandchars=\\\{\},frame=single,framerule=0.3mm,rulecolor=\color{cellframecolor}]
{\color{outcolor}Out[{\color{outcolor}4}]:} -3
\end{Verbatim}
            
    Vous voyez que la \textbf{syntaxe} pour appeler une méthode sur un objet
est \textbf{identique} à celle que nous avons utilisée jusqu'ici avec
\textbf{les types de base}. Nous verrons très bientôt comment on peut
pousser beaucoup plus loin la similitude, pour pouvoir par exemple
calculer la \textbf{somme} de deux objets de la classe \texttt{Matrix2}
avec l'opérateur \texttt{+}, mais n'anticipons pas.

    Vous voyez aussi que, ici encore, la méthode définie dans la classe
attend \textbf{\emph{1 argument} \texttt{self}}, alors qu'apparemment
nous ne lui en passons \textbf{aucun}. Comme tout à l'heure avec le
constructeur, le premier argument passé automatiquement par
l'interpréteur python à \texttt{determinant} est l'objet
\texttt{matrice} lui-même.

En fait on aurait pu aussi bien écrire, de manière parfaitement
équivalente~:

    \begin{Verbatim}[commandchars=\\\{\},frame=single,framerule=0.3mm,rulecolor=\color{cellframecolor}]
{\color{incolor}In [{\color{incolor}5}]:} \PY{n}{Matrix2}\PY{o}{.}\PY{n}{determinant}\PY{p}{(}\PY{n}{matrice}\PY{p}{)}
\end{Verbatim}


\begin{Verbatim}[commandchars=\\\{\},frame=single,framerule=0.3mm,rulecolor=\color{cellframecolor}]
{\color{outcolor}Out[{\color{outcolor}5}]:} -3
\end{Verbatim}
            
    qui n'est presque jamais utilisé en pratique, mais qui illustre bien ce
qui se passe lorsqu'on invoque une méthode sur un objet. En réalité,
lorsque l'on écrit \texttt{matrice.determinant()} l'interpréteur python
va essentiellement convertir cette expression en
\texttt{Matrix2.determinant(matrice)}.

    \hypertarget{compluxe9ment---niveau-intermuxe9diaire}{%
\subsection{Complément - niveau
intermédiaire}\label{compluxe9ment---niveau-intermuxe9diaire}}

    \hypertarget{uxe0-quoi-uxe7a-sert}{%
\subsubsection{À quoi ça sert~?}\label{uxe0-quoi-uxe7a-sert}}

    Ce cours n'est pas consacré à la Programmation Orientée Objet (OOP) en
tant que telle. Voici toutefois quelques-uns des avantages qui sont
généralement mis en avant~:

\begin{itemize}
\tightlist
\item
  encapsulation~;
\item
  résolution dynamique de méthode~;
\item
  héritage.
\end{itemize}

    \hypertarget{encapsulation}{%
\subparagraph{Encapsulation}\label{encapsulation}}

    L'idée de la notion d'encapsulation consiste à ce que~:

\begin{itemize}
\tightlist
\item
  une classe définit son \textbf{interface}, c'est-à-dire les méthodes
  par lesquelles on peut utiliser ce code,
\item
  mais reste tout à fait libre de modifier son \textbf{implémentation},
  et tant que cela n'impacte pas l'interface, \textbf{aucun changement}
  n'est requis dans les \textbf{codes utilisateurs}.
\end{itemize}

    Nous verrons plus bas une deuxième implémentation de \texttt{Matrix2}
qui est plus générale que notre première version, mais qui utilise la
même interface, donc qui fonctionne exactement de la même manière pour
le code utilisateur.

    La notion d'encapsulation peut paraître à première vue banale~; il ne
faut pas s'y fier, c'est de cette manière qu'on peut efficacement
découper un gros logiciel en petits morceaux indépendants, et réellement
découplés les uns des autres, et ainsi casser, réduire la complexité.

La programmation objet est une des techniques permettant d'atteindre
cette bonne propriété d'encapsulation. Il faut reconnaître que certains
langages comme Java et C++ ont des mécanismes plus sophistiqués, mais
aussi plus complexes, pour garantir une bonne étanchéité entre
l'interface publique et les détails d'implémentation. Les choix faits en
la matière en Python reviennent, une fois encore, à privilégier la
simplicité.

Aussi, il n'existe pas en Python l'équivalent des notions d'interface
\texttt{public}, \texttt{private} et \texttt{protected} qu'on trouve en
C++ et en Java. Il existe tout au plus une convention, selon laquelle
les attributs commençant par un underscore (le tiret bas \texttt{\_})
sont privés et ne \emph{devraient} pas être utilisés par un code tiers,
mais le langage ne fait rien pour garantir le bon usage de cette
convention.

Si vous désirez creuser ce point nous vous conseillons de lire~:

\begin{itemize}
\tightlist
\item
  \href{https://docs.python.org/3/reference/lexical_analysis.html\#reserved-classes-of-identifiers}{\emph{Reserved
  classes of identifiers}} où l'on décrit également les noms privés à
  une classe (les noms de variables en \texttt{\_\_}\emph{nom})~;
\item
  \href{https://docs.python.org/3/tutorial/classes.html\#tut-private}{\emph{Private
  Variables and Class-local References}}, qui en donne une illustration.
\end{itemize}

Malgré cette simplicité revendiquée, les classes de python permettent
d'implémenter en pratique une encapsulation tout à fait acceptable, on
peut en juger rien que par le nombre de bibliothèques tierces existantes
dans l'écosystème python.

    \hypertarget{ruxe9solution-dynamique-de-muxe9thode}{%
\subparagraph{Résolution dynamique de
méthode}\label{ruxe9solution-dynamique-de-muxe9thode}}

    Le deuxième atout de OOP, c'est le fait que l'envoi de méthode est
résolu lors de l'exécution (\emph{run-time}) et non pas lors de la
compilation (\emph{compile-time}). Ceci signifie que l'on peut écrire du
code générique, qui pourra fonctionner avec des objets non connus
\emph{a priori}. Nous allons en voir un exemple tout de suite, en
redéfinissant le comportement de \texttt{print} dans la deuxième
implémentation de \texttt{Matrix2}.

    \hypertarget{huxe9ritage}{%
\subparagraph{Héritage}\label{huxe9ritage}}

    L'héritage est le concept qui permet de~:

\begin{itemize}
\tightlist
\item
  dupliquer une classe presque à l'identique, mais en redéfinissant une
  ou quelques méthodes seulement (héritage simple)~;
\item
  composer plusieurs classes en une seule, pour réaliser en quelque
  sorte l'union des propriétés de ces classes (héritage multiple).
\end{itemize}

    \hypertarget{illustration}{%
\subsubsection{Illustration}\label{illustration}}

    Nous revenons sur l'héritage dans une prochaine vidéo. Dans l'immédiat,
nous allons voir une seconde implémentation de la classe
\texttt{Matrix2}, qui illustre l'encapsulation et l'envoi dynamique de
méthodes.

    Pour une raison ou pour une autre, disons que l'on décide de remplacer
les 4 attributs nommés \texttt{self.a11}, \texttt{self.a12}, etc., qui
n'étaient pas très extensibles, par un seul attribut \texttt{a} qui
regroupe tous les coefficients de la matrice dans un seul tuple.

    \begin{Verbatim}[commandchars=\\\{\},frame=single,framerule=0.3mm,rulecolor=\color{cellframecolor}]
{\color{incolor}In [{\color{incolor}6}]:} \PY{k}{class} \PY{n+nc}{Matrix2}\PY{p}{:}
            \PY{l+s+sd}{\PYZdq{}\PYZdq{}\PYZdq{}Une deuxième implémentation, tout aussi}
        \PY{l+s+sd}{    sommaire, mais différente, de matrice carrée 2x2\PYZdq{}\PYZdq{}\PYZdq{}}
            
            \PY{k}{def} \PY{n+nf}{\PYZus{}\PYZus{}init\PYZus{}\PYZus{}}\PY{p}{(}\PY{n+nb+bp}{self}\PY{p}{,} \PY{n}{a11}\PY{p}{,} \PY{n}{a12}\PY{p}{,} \PY{n}{a21}\PY{p}{,} \PY{n}{a22}\PY{p}{)}\PY{p}{:}
                \PY{l+s+s2}{\PYZdq{}}\PY{l+s+s2}{construit une matrice à partir des 4 coefficients}\PY{l+s+s2}{\PYZdq{}}
                \PY{c+c1}{\PYZsh{} on décide d\PYZsq{}utiliser un tuple plutôt que de ranger}
                \PY{c+c1}{\PYZsh{} les coefficients individuellement}
                \PY{n+nb+bp}{self}\PY{o}{.}\PY{n}{a} \PY{o}{=} \PY{p}{(}\PY{n}{a11}\PY{p}{,} \PY{n}{a12}\PY{p}{,} \PY{n}{a21}\PY{p}{,} \PY{n}{a22}\PY{p}{)}
                
            \PY{k}{def} \PY{n+nf}{determinant}\PY{p}{(}\PY{n+nb+bp}{self}\PY{p}{)}\PY{p}{:}
                \PY{l+s+s2}{\PYZdq{}}\PY{l+s+s2}{le déterminant de la matrice}\PY{l+s+s2}{\PYZdq{}}
                \PY{k}{return} \PY{n+nb+bp}{self}\PY{o}{.}\PY{n}{a}\PY{p}{[}\PY{l+m+mi}{0}\PY{p}{]} \PY{o}{*} \PY{n+nb+bp}{self}\PY{o}{.}\PY{n}{a}\PY{p}{[}\PY{l+m+mi}{3}\PY{p}{]} \PY{o}{\PYZhy{}} \PY{n+nb+bp}{self}\PY{o}{.}\PY{n}{a}\PY{p}{[}\PY{l+m+mi}{1}\PY{p}{]} \PY{o}{*} \PY{n+nb+bp}{self}\PY{o}{.}\PY{n}{a}\PY{p}{[}\PY{l+m+mi}{2}\PY{p}{]}
            
            \PY{k}{def} \PY{n+nf}{\PYZus{}\PYZus{}repr\PYZus{}\PYZus{}}\PY{p}{(}\PY{n+nb+bp}{self}\PY{p}{)}\PY{p}{:}
                \PY{l+s+s2}{\PYZdq{}}\PY{l+s+s2}{comment présenter une matrice dans un print()}\PY{l+s+s2}{\PYZdq{}}
                \PY{k}{return} \PY{n}{f}\PY{l+s+s2}{\PYZdq{}}\PY{l+s+s2}{\PYZlt{}\PYZlt{}mat\PYZhy{}2x2 }\PY{l+s+si}{\PYZob{}self.a\PYZcb{}}\PY{l+s+s2}{\PYZgt{}\PYZgt{}}\PY{l+s+s2}{\PYZdq{}}
\end{Verbatim}


    Grâce à l'\textbf{encapsulation}, on peut continuer à utiliser la classe
exactement de la même manière~:

    \begin{Verbatim}[commandchars=\\\{\},frame=single,framerule=0.3mm,rulecolor=\color{cellframecolor}]
{\color{incolor}In [{\color{incolor}7}]:} \PY{n}{matrice} \PY{o}{=} \PY{n}{Matrix2}\PY{p}{(}\PY{l+m+mi}{1}\PY{p}{,} \PY{l+m+mi}{2}\PY{p}{,} \PY{l+m+mi}{2}\PY{p}{,} \PY{l+m+mi}{1}\PY{p}{)}
        \PY{n+nb}{print}\PY{p}{(}\PY{l+s+s2}{\PYZdq{}}\PY{l+s+s2}{Determinant =}\PY{l+s+s2}{\PYZdq{}}\PY{p}{,} \PY{n}{matrice}\PY{o}{.}\PY{n}{determinant}\PY{p}{(}\PY{p}{)}\PY{p}{)}
\end{Verbatim}


    \begin{Verbatim}[commandchars=\\\{\},frame=single,framerule=0.3mm,rulecolor=\color{cellframecolor}]
Determinant = -3
\end{Verbatim}

    Et en prime, grâce à la \textbf{résolution dynamique de méthode}, et
parce que dans cette seconde implémentation on a défini une autre
méthode spéciale \texttt{\_\_repr\_\_}, nous avons maintenant une
impression beaucoup plus lisible de l'objet \texttt{matrice}~:

    \begin{Verbatim}[commandchars=\\\{\},frame=single,framerule=0.3mm,rulecolor=\color{cellframecolor}]
{\color{incolor}In [{\color{incolor}8}]:} \PY{n+nb}{print}\PY{p}{(}\PY{n}{matrice}\PY{p}{)}
\end{Verbatim}


    \begin{Verbatim}[commandchars=\\\{\},frame=single,framerule=0.3mm,rulecolor=\color{cellframecolor}]
<<mat-2x2 (1, 2, 2, 1)>>
\end{Verbatim}

    Ce format d'impression reste d'ailleurs valable dans l'impression
d'objets plus compliqués, comme par exemple~:

    \begin{Verbatim}[commandchars=\\\{\},frame=single,framerule=0.3mm,rulecolor=\color{cellframecolor}]
{\color{incolor}In [{\color{incolor}9}]:} \PY{c+c1}{\PYZsh{} on profite de ce nouveau format d\PYZsq{}impression même si on met}
        \PY{c+c1}{\PYZsh{} par exemple un objet Matrix2 à l\PYZsq{}intérieur d\PYZsq{}une liste}
        \PY{n}{composite} \PY{o}{=} \PY{p}{[}\PY{n}{matrice}\PY{p}{,} \PY{k+kc}{None}\PY{p}{,} \PY{n}{Matrix2}\PY{p}{(}\PY{l+m+mi}{1}\PY{p}{,} \PY{l+m+mi}{0}\PY{p}{,} \PY{l+m+mi}{0}\PY{p}{,} \PY{l+m+mi}{1}\PY{p}{)}\PY{p}{]}
        \PY{n+nb}{print}\PY{p}{(}\PY{n}{f}\PY{l+s+s2}{\PYZdq{}}\PY{l+s+s2}{composite=}\PY{l+s+si}{\PYZob{}composite\PYZcb{}}\PY{l+s+s2}{\PYZdq{}}\PY{p}{)}
\end{Verbatim}


    \begin{Verbatim}[commandchars=\\\{\},frame=single,framerule=0.3mm,rulecolor=\color{cellframecolor}]
composite=[<<mat-2x2 (1, 2, 2, 1)>>, None, <<mat-2x2 (1, 0, 0, 1)>>]
\end{Verbatim}

    Cela est possible parce que le code de \texttt{print} envoie la méthode
\texttt{\_\_repr\_\_} sur les objets qu'elle parcourt. Le langage
fournit une façon de faire par défaut, comme on l'a vu plus haut avec la
première implémentation de \texttt{Matrix2}~; et en définissant notre
propre méthode \texttt{\_\_repr\_\_} nous pouvons surcharger ce
comportement, et définir notre format d'impression.

    Nous reviendrons sur les notions de surcharge et d'héritage dans les
prochaines séquences vidéos.

    \hypertarget{la-convention-dutiliser-self}{%
\subsubsection{\texorpdfstring{La convention d'utiliser
\texttt{self}}{La convention d'utiliser self}}\label{la-convention-dutiliser-self}}

    Avant de conclure, revenons rapidement sur le nom \texttt{self} qui est
utilisé comme nom pour le premier argument des méthodes habituelles
(nous verrons en semaine 9 d'autres sortes de méthodes, les méthodes
statiques et de classe, qui ne reçoivent pas l'instance comme premier
argument).

Comme nous l'avons dit plus haut, le premier argument d'une méthode
s'appelle \texttt{self} \textbf{par convention}. Cette pratique est
particulièrement bien suivie, mais ce n'est qu'une convention, en ce
sens qu'on aurait pu utiliser n'importe quel identificateur~; pour le
langage \texttt{self} n'a aucun sens particulier, ce n'est pas un mot
clé ni une variable \emph{built-in}.

    Ceci est à mettre en contraste avec le choix fait dans d'autres
langages, comme par exemple en C++ où l'instance est référencée par le
mot-clé \texttt{this}, qui n'est pas mentionné dans la signature de la
méthode. En Python, selon le manifeste, \emph{explicit is better than
implicit}, c'est pourquoi on mentionne l'instance dans la signature,
sous le nom \texttt{self}.


    % Add a bibliography block to the postdoc
    
    
    
