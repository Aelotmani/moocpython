    
    
    
    

    

    \hypertarget{comparaison-de-fonctions}{%
\section{Comparaison de fonctions}\label{comparaison-de-fonctions}}

    \hypertarget{exercice---niveau-avancuxe9}{%
\subsection{Exercice - niveau
avancé}\label{exercice---niveau-avancuxe9}}

    \begin{Verbatim}[commandchars=\\\{\}]
{\color{incolor}In [{\color{incolor}1}]:} \PY{c+c1}{\PYZsh{} Pour charger l\PYZsq{}exercice}
        \PY{k+kn}{from} \PY{n+nn}{corrections}\PY{n+nn}{.}\PY{n+nn}{exo\PYZus{}compare\PYZus{}all} \PY{k}{import} \PY{n}{exo\PYZus{}compare\PYZus{}all}
\end{Verbatim}


    \hypertarget{note}{%
\section{NOTE}\label{note}}

\hypertarget{automatic-execution-has-skipped-execution-of-this-cell}{%
\section{automatic execution has skipped execution of this
cell}\label{automatic-execution-has-skipped-execution-of-this-cell}}

À présent nous allons écrire une version très simplifiée de l'outil qui
est utilisé dans ce cours pour corriger les exercices. Vous aurez sans
doute remarqué que les fonctions de correction prennent en argument la
fonction à corriger.

Par exemple un peu plus bas, la cellule de correction fait

\begin{verbatim}
exo_compare_all.correction(compare_all)  
\end{verbatim}

dans lequel \texttt{compare\_all} est l'objet fonction que vous écrivez
en réponse à cet exercice.

    On vous demande d'écrire une fonction \texttt{compare} qui prend en
argument~:

\begin{itemize}
\tightlist
\item
  deux fonctions \texttt{f} et \texttt{g}~; imaginez que l'une d'entre
  elles fonctionne et qu'on cherche à valider l'autre~; dans cette
  version simplifiée toutes les fonctions acceptent exactement un
  argument~;
\item
  une liste d'entrées \texttt{entrees}~; vous pouvez supposer que
  chacune de ces entrées est dans le domaine de \texttt{f} et de
  \texttt{g} (dit autrement, on peut appeler \texttt{f} et \texttt{g}
  sur chacune des entrées sans craindre qu'une exception soit levée).
\end{itemize}

Le résultat attendu pour le retour de \texttt{compare} est une liste qui
contient autant de booléens que d'éléments dans \texttt{entrees}, chacun
indiquant si avec l'entrée correspondante on a pu vérifier que
\texttt{f(entree)\ ==\ g(entree)}.

Dans cette première version de l'exercice vous pouvez enfin supposer que
les entrées ne sont pas modifiées par \texttt{f} ou \texttt{g}.

    Pour information dans cet exercice~:

\begin{itemize}
\tightlist
\item
  \texttt{factorial} correspond à \texttt{math.factorial}
\item
  \texttt{fact} et \texttt{broken\_fact} sont des fonctions implémentées
  par nos soins, la première est correcte alors que la seconde retourne
  0 au lieu de 1 pour l'entrée 0.
\end{itemize}

    \begin{Verbatim}[commandchars=\\\{\}]
{\color{incolor}In [{\color{incolor}2}]:} \PY{c+c1}{\PYZsh{} par exemple}
        \PY{n}{exo\PYZus{}compare\PYZus{}all}\PY{o}{.}\PY{n}{example}\PY{p}{(}\PY{p}{)}
\end{Verbatim}


\begin{Verbatim}[commandchars=\\\{\}]
{\color{outcolor}Out[{\color{outcolor}2}]:} <IPython.core.display.HTML object>
\end{Verbatim}
            
    Ce qui, dit autrement, veut tout simplement dire que \texttt{fact} et
\texttt{factorial} coïncident sur les entrées 0, 1 et 5, alors que
\texttt{broken\_fact} et \texttt{factorial} ne renvoient pas la même
valeur avec l'entrée \texttt{0}.

    \begin{Verbatim}[commandchars=\\\{\}]
{\color{incolor}In [{\color{incolor}3}]:} \PY{c+c1}{\PYZsh{} c\PYZsq{}est à vous}
        \PY{k}{def} \PY{n+nf}{compare\PYZus{}all}\PY{p}{(}\PY{n}{f}\PY{p}{,} \PY{n}{g}\PY{p}{,} \PY{n}{entrees}\PY{p}{)}\PY{p}{:}
            \PY{l+s+s2}{\PYZdq{}}\PY{l+s+s2}{\PYZlt{}votre code\PYZgt{}}\PY{l+s+s2}{\PYZdq{}}
\end{Verbatim}


    \begin{Verbatim}[commandchars=\\\{\}]
{\color{incolor}In [{\color{incolor} }]:} \PY{c+c1}{\PYZsh{} NOTE}
        \PY{c+c1}{\PYZsh{} automatic execution has skipped execution of this cell}
        
        \PY{c+c1}{\PYZsh{} pour vérifier votre code}
        \PY{n}{exo\PYZus{}compare\PYZus{}all}\PY{o}{.}\PY{n}{correction}\PY{p}{(}\PY{n}{compare\PYZus{}all}\PY{p}{)}
\end{Verbatim}


    \hypertarget{exercice-optionnel---niveau-avancuxe9}{%
\subsection{Exercice optionnel - niveau
avancé}\label{exercice-optionnel---niveau-avancuxe9}}

    \begin{Verbatim}[commandchars=\\\{\}]
{\color{incolor}In [{\color{incolor}4}]:} \PY{c+c1}{\PYZsh{} Pour charger l\PYZsq{}exercice}
        \PY{k+kn}{from} \PY{n+nn}{corrections}\PY{n+nn}{.}\PY{n+nn}{exo\PYZus{}compare\PYZus{}args} \PY{k}{import} \PY{n}{exo\PYZus{}compare\PYZus{}args}
\end{Verbatim}


    \hypertarget{compare-revisituxe9e}{%
\subsubsection{\texorpdfstring{\texttt{compare}
revisitée}{compare revisitée}}\label{compare-revisituxe9e}}

    Nous reprenons ici la même idée que \texttt{compare}, mais en levant
l'hypothèse que les deux fonctions attendent un seul argument. Il faut
écrire une nouvelle fonction \texttt{compare\_args} qui prend en
entrée~:

\begin{itemize}
\tightlist
\item
  deux fonctions \texttt{f} et \texttt{g} comme ci-dessus~;
\item
  mais cette fois une liste (ou un tuple) \texttt{argument\_tuples} de
  \textbf{tuples} d'arguments d'entrée.
\end{itemize}

Comme ci-dessus on attend en retour une liste \texttt{retour} de
booléens, de même taille que \texttt{argument\_tuples}, telle que, si
\texttt{len(argument\_tuples)} vaut \(n\)~:

\(\forall i \in \{1,...,n\}\), si \texttt{argument\_tuples{[}i{]}} ==
{[} \(a_1\),\ldots{},\(a_j\) {]}, alors

\texttt{retour(i)\ ==\ True} \(\Longleftrightarrow\) f
(\(a_1\),\ldots{},\(a_j\)) == g (\(a_1\),\ldots{},\(a_j\))

    Pour information, dans tout cet exercice~:

\begin{itemize}
\tightlist
\item
  \texttt{factorial} correspond à \texttt{math.factorial}~;
\item
  \texttt{fact} et \texttt{broken\_fact} sont des fonctions implémentées
  par nos soins, la première est correcte alors que la seconde retourne
  0 au lieu de 1 pour l'entrée 0~;
\item
  \texttt{add} correspond à l'addition binaire \texttt{operator.add}~;
\item
  \texttt{plus} et \texttt{broken\_plus} sont des additions binaires que
  nous avons écrites, l'une étant correcte et l'autre étant fausse
  lorsque le premier argument est nul.
\end{itemize}

    \begin{Verbatim}[commandchars=\\\{\}]
{\color{incolor}In [{\color{incolor}5}]:} \PY{n}{exo\PYZus{}compare\PYZus{}args}\PY{o}{.}\PY{n}{example}\PY{p}{(}\PY{p}{)}
\end{Verbatim}


\begin{Verbatim}[commandchars=\\\{\}]
{\color{outcolor}Out[{\color{outcolor}5}]:} <IPython.core.display.HTML object>
\end{Verbatim}
            
    \begin{Verbatim}[commandchars=\\\{\}]
{\color{incolor}In [{\color{incolor}6}]:} \PY{c+c1}{\PYZsh{} ATTENTION vous devez aussi définir les arguments de la fonction}
        \PY{k}{def} \PY{n+nf}{compare\PYZus{}args}\PY{p}{(}\PY{n}{votre}\PY{p}{,} \PY{n}{signature}\PY{p}{)}\PY{p}{:}
            \PY{l+s+s2}{\PYZdq{}}\PY{l+s+s2}{\PYZlt{}votre\PYZus{}code\PYZgt{}}\PY{l+s+s2}{\PYZdq{}}
\end{Verbatim}


    \begin{Verbatim}[commandchars=\\\{\}]
{\color{incolor}In [{\color{incolor} }]:} \PY{c+c1}{\PYZsh{} NOTE}
        \PY{c+c1}{\PYZsh{} automatic execution has skipped execution of this cell}
        
        \PY{n}{exo\PYZus{}compare\PYZus{}args}\PY{o}{.}\PY{n}{correction}\PY{p}{(}\PY{n}{compare\PYZus{}args}\PY{p}{)}
\end{Verbatim}



    % Add a bibliography block to the postdoc
    
    
    
