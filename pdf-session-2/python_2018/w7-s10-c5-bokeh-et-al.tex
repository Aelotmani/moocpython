    
    
    
    

    

    \hypertarget{autres-bibliothuxe8ques-de-visualisation}{%
\section{Autres bibliothèques de
visualisation}\label{autres-bibliothuxe8ques-de-visualisation}}

    \hypertarget{compluxe9ment---niveau-basique}{%
\subsection{Complément - niveau
basique}\label{compluxe9ment---niveau-basique}}

    Pour conclure cette séquence sur les outils de visualisation, nous
allons très rapidement évoquer des alternatives à la bibliothèque
\texttt{matplotlib}, sachant qu'il existe en réalité un très grand
nombre de bibliothèques en développement dans ce domaine en pleine
expansion.

    \hypertarget{le-poids-du-passuxe9}{%
\subsubsection{Le poids du passé}\label{le-poids-du-passuxe9}}

    On a vu que \texttt{matplotlib} est un outil relativement complet.
Toutefois, on peut lui reprocher deux défauts majeurs.

\begin{itemize}
\item
  D'une part, \texttt{matplotlib} a choisi d'offrir une interface aussi
  proche que possible de ce qui existait préalablement en MatLab. C'est
  un choix tout à fait judicieux dans l'optique d'attirer la communauté
  utilisatrice de MatLab à des outils open source basés sur Python et
  numpy. Mais en contrepartie cela implique d'adopter tels quels des
  choix de conception.
\item
  Et notamment, en suivant cette approche on hérite d'un modèle mental
  qui est plus orienté vers la sortie vers du papier que vers la
  création de documents interactifs.
\end{itemize}

    Ceci, ajouté à l'explosion du domaine de l'analyse et de la
visualisation de données, explique la largeur de l'offre en matière de
bibliothèques de visualisation alternatives.

    \hypertarget{bokeh}{%
\subsubsection{\texorpdfstring{\texttt{bokeh}}{bokeh}}\label{bokeh}}

    Parmi celles-ci, nous voulons vous signaler notamment la bibliothèque
\href{https://bokeh.pydata.org/en/latest/}{\texttt{bokeh}}, qui est
développée principalement par Anaconda, dans un modèle open source.

\texttt{bokeh} présente quelques bonnes propriétés qui nous semblent
mériter d'être signalées.

    Pour commencer cette bibliothèque utilise une architecture qui permet de
\emph{penser la visualisation comme quelque chose d'interactif} (disons
une page html), et non pas de figé comme lorsqu'on pense en termes de
feuille de papier. Notamment elle permet de faire collaborer du code
Python avec du code JavaScript, qui offre immédiatement des possibilités
bien plus pertinentes lorsqu'il s'agit de créer des interactions
utilisateur qui soient attractives et efficaces. Signalons en passant, à
cet égard, qu'elle utilise \href{https://d3js.org/}{la librairie
JavaScript \texttt{d3.js}}, qui est devenu un standard de fait plus ou
moins incontournable dans le domaine de la visualisation.

En tout état de cause, elle offre une interface de programmation qui
tient compte d'environnements comme les notebooks, ce qui peut s'avérer
un atout précieux si vous utilisez massivement ce support, comme on va
le voir, précisément, dans ce notebook.

    Il peut aussi être intéressant de savoir que \texttt{bokeh} offre des
possibilités natives de
\href{https://bokeh.pydata.org/en/latest/docs/user_guide/graph.html}{visualisation
de graphes} et de
\href{https://bokeh.pydata.org/en/latest/docs/user_guide/geo.html\#}{données
géographiques}.

    Par contre à ce stade du développement, la visualisation en 3D n'est
sans doute pas le point fort de \texttt{bokeh}. C'est une option qui
reste possible (voir
\href{https://github.com/bokeh/bokeh/tree/master/examples/app/surface3d}{par
exemple ceci}), mais cela est pour l'instant considéré comme une
extension de la librairie, et donc n'est accessible qu'au prix de
l'écriture de code javascript.

    Pour une présentation plus complète, je vous renvoie à
\href{https://bokeh.pydata.org/en/latest/docs/user_guide.html}{la
documentation utilisateur}.

    \hypertarget{bokeh-dans-les-notebooks}{%
\subsubsection{\texorpdfstring{\texttt{bokeh} dans les
notebooks}{bokeh dans les notebooks}}\label{bokeh-dans-les-notebooks}}

    Nous allons rapidement illustrer ici comment \texttt{bokeh} s'interface
avec l'environnement des notebooks pour créer une visualisation
interactive. Vous remarquerez que dans le code qui suit, on n'a
\textbf{pas eu besoin de mentionner} de \emph{magic} ipython, comme
lorsqu'on avait du faire dans le complément sur les notebooks
interactifs~:

\begin{verbatim}
%matplotlib notebook
\end{verbatim}

    \begin{Verbatim}[commandchars=\\\{\}]
{\color{incolor}In [{\color{incolor}1}]:} \PY{k+kn}{import} \PY{n+nn}{numpy} \PY{k}{as} \PY{n+nn}{np}
\end{Verbatim}


    \begin{Verbatim}[commandchars=\\\{\}]
{\color{incolor}In [{\color{incolor}2}]:} \PY{c+c1}{\PYZsh{} l\PYZsq{}attirail de notebooks interactifs}
        \PY{k+kn}{from} \PY{n+nn}{ipywidgets} \PY{k}{import} \PY{n}{interact}\PY{p}{,} \PY{n}{fixed}\PY{p}{,} \PY{n}{FloatSlider}
\end{Verbatim}


    \begin{Verbatim}[commandchars=\\\{\}]
{\color{incolor}In [{\color{incolor}3}]:} \PY{c+c1}{\PYZsh{} les imports pour bokeh}
        \PY{k+kn}{from} \PY{n+nn}{bokeh}\PY{n+nn}{.}\PY{n+nn}{plotting} \PY{k}{import} \PY{n}{figure}\PY{p}{,} \PY{n}{show}
        \PY{c+c1}{\PYZsh{} dans la rubrique entrée\PYZhy{}sortie, on trouve}
        \PY{c+c1}{\PYZsh{} les outils pour produire du html}
        \PY{c+c1}{\PYZsh{}  (le mode par défaut)}
        \PY{c+c1}{\PYZsh{} ou pour interactig avec un notebook}
        \PY{k+kn}{from} \PY{n+nn}{bokeh}\PY{n+nn}{.}\PY{n+nn}{io} \PY{k}{import} \PY{n}{push\PYZus{}notebook}\PY{p}{,} \PY{n}{output\PYZus{}notebook}
\end{Verbatim}


    \begin{Verbatim}[commandchars=\\\{\}]
{\color{incolor}In [{\color{incolor}4}]:} \PY{c+c1}{\PYZsh{} c\PYZsq{}est cette déclaration qui remplace}
        \PY{c+c1}{\PYZsh{} si on veut la magic  \PYZsq{}\PYZpc{}matplotlib notebook\PYZsq{}}
        \PY{n}{output\PYZus{}notebook}\PY{p}{(}\PY{p}{)}
\end{Verbatim}


    
    
    
    
    \begin{center}\rule{0.5\linewidth}{\linethickness}\end{center}

    \begin{Verbatim}[commandchars=\\\{\}]
{\color{incolor}In [{\color{incolor}5}]:} \PY{c+c1}{\PYZsh{} on crée un objet figure}
        \PY{n}{fig1} \PY{o}{=} \PY{n}{figure}\PY{p}{(}
            \PY{n}{title}\PY{o}{=}\PY{l+s+s2}{\PYZdq{}}\PY{l+s+s2}{fonctions trigonométriques}\PY{l+s+s2}{\PYZdq{}}\PY{p}{,}
            \PY{n}{plot\PYZus{}height}\PY{o}{=}\PY{l+m+mi}{300}\PY{p}{,} \PY{n}{plot\PYZus{}width}\PY{o}{=}\PY{l+m+mi}{600}\PY{p}{,}
            \PY{c+c1}{\PYZsh{} c\PYZsq{}est là notamment qu\PYZsq{}on précise}
            \PY{c+c1}{\PYZsh{} l\PYZsq{}intervalle en y}
            \PY{n}{y\PYZus{}range}\PY{o}{=}\PY{p}{(}\PY{o}{\PYZhy{}}\PY{l+m+mi}{5}\PY{p}{,} \PY{l+m+mi}{5}\PY{p}{)}\PY{p}{,}
        \PY{p}{)}
\end{Verbatim}


    \begin{Verbatim}[commandchars=\\\{\}]
{\color{incolor}In [{\color{incolor}6}]:} \PY{c+c1}{\PYZsh{} on initialise la figure en créant}
        \PY{c+c1}{\PYZsh{} un objet courbe}
        \PY{n}{x} \PY{o}{=} \PY{n}{np}\PY{o}{.}\PY{n}{linspace}\PY{p}{(}\PY{l+m+mi}{0}\PY{p}{,} \PY{l+m+mi}{2}\PY{o}{*}\PY{n}{np}\PY{o}{.}\PY{n}{pi}\PY{p}{,} \PY{l+m+mi}{2000}\PY{p}{)}
        \PY{n}{y} \PY{o}{=} \PY{n}{np}\PY{o}{.}\PY{n}{sin}\PY{p}{(}\PY{n}{x}\PY{p}{)}
        \PY{n}{courbe\PYZus{}trigo} \PY{o}{=} \PY{n}{fig1}\PY{o}{.}\PY{n}{line}\PY{p}{(}\PY{n}{x}\PY{p}{,} \PY{n}{y}\PY{p}{,} \PY{n}{color}\PY{o}{=}\PY{l+s+s2}{\PYZdq{}}\PY{l+s+s2}{\PYZsh{}2222aa}\PY{l+s+s2}{\PYZdq{}}\PY{p}{,} \PY{n}{line\PYZus{}width}\PY{o}{=}\PY{l+m+mi}{3}\PY{p}{)}
\end{Verbatim}


    \begin{Verbatim}[commandchars=\\\{\}]
{\color{incolor}In [{\color{incolor}7}]:} \PY{c+c1}{\PYZsh{} la fonction de mise à jour, qui sera connectée}
        \PY{c+c1}{\PYZsh{} à interact}
        \PY{k}{def} \PY{n+nf}{update\PYZus{}trigo}\PY{p}{(}\PY{n}{function\PYZus{}name}\PY{p}{,} \PY{n}{frequence}\PY{o}{=}\PY{l+m+mi}{1}\PY{p}{,}
                         \PY{n}{amplitude}\PY{o}{=}\PY{l+m+mi}{1}\PY{p}{,} \PY{n}{phase}\PY{o}{=}\PY{l+m+mi}{0}\PY{p}{,}
                         \PY{c+c1}{\PYZsh{} l\PYZsq{}objet handle correspond}
                         \PY{c+c1}{\PYZsh{} à une figure à mettre à jour}
                         \PY{o}{*}\PY{p}{,} \PY{n}{handle}\PY{p}{)}\PY{p}{:}
            \PY{c+c1}{\PYZsh{} juste une astuce pour pouvoir choisir}
            \PY{c+c1}{\PYZsh{} la fonction trigonométrique, qu\PYZsq{}on recherche}
            \PY{c+c1}{\PYZsh{} dans le module numpy}
            \PY{n}{func} \PY{o}{=} \PY{n+nb}{getattr}\PY{p}{(}\PY{n}{np}\PY{p}{,} \PY{n}{function\PYZus{}name}\PY{p}{)}
            \PY{c+c1}{\PYZsh{} c\PYZsq{}est ici qu\PYZsq{}on modifie les données}
            \PY{c+c1}{\PYZsh{} utilisées pour produire la courbe}
            \PY{n}{courbe\PYZus{}trigo}\PY{o}{.}\PY{n}{data\PYZus{}source}\PY{o}{.}\PY{n}{data}\PY{p}{[}\PY{l+s+s1}{\PYZsq{}}\PY{l+s+s1}{y}\PY{l+s+s1}{\PYZsq{}}\PY{p}{]} \PY{o}{=} \PYZbs{}
               \PY{n}{amplitude} \PY{o}{*} \PY{n}{func}\PY{p}{(}\PY{n}{frequence} \PY{o}{*} \PY{n}{x} \PY{o}{+} \PY{n}{phase}\PY{p}{)}
            \PY{c+c1}{\PYZsh{} et c\PYZsq{}est ici qu\PYZsq{}on provoque la mise à jour}
            \PY{n}{push\PYZus{}notebook}\PY{p}{(}\PY{n}{handle}\PY{p}{)}
\end{Verbatim}


    \begin{Verbatim}[commandchars=\\\{\}]
{\color{incolor}In [{\color{incolor}8}]:} \PY{c+c1}{\PYZsh{} au moment où matérialise l\PYZsq{}objet figure}
        \PY{c+c1}{\PYZsh{} on récupère une `handle` qui lui correspond}
        \PY{n}{handle1} \PY{o}{=} \PY{n}{show}\PY{p}{(}\PY{n}{fig1}\PY{p}{,} \PY{n}{notebook\PYZus{}handle}\PY{o}{=}\PY{k+kc}{True}\PY{p}{)}
\end{Verbatim}


    
    
    
    
    \begin{Verbatim}[commandchars=\\\{\}]
{\color{incolor}In [{\color{incolor}9}]:} \PY{c+c1}{\PYZsh{} maintenant on peut créer un interacteur}
        \PY{n}{interact}\PY{p}{(}\PY{n}{update\PYZus{}trigo}\PY{p}{,} \PY{n}{function\PYZus{}name}\PY{o}{=}\PY{p}{[}\PY{l+s+s2}{\PYZdq{}}\PY{l+s+s2}{sin}\PY{l+s+s2}{\PYZdq{}}\PY{p}{,} \PY{l+s+s2}{\PYZdq{}}\PY{l+s+s2}{cos}\PY{l+s+s2}{\PYZdq{}}\PY{p}{,} \PY{l+s+s2}{\PYZdq{}}\PY{l+s+s2}{tan}\PY{l+s+s2}{\PYZdq{}}\PY{p}{]}\PY{p}{,}
                 \PY{n}{frequence}\PY{o}{=}\PY{p}{(}\PY{l+m+mi}{1}\PY{p}{,}\PY{l+m+mi}{20}\PY{p}{)}\PY{p}{,}
                 \PY{n}{amplitude}\PY{o}{=}\PY{p}{[}\PY{l+m+mf}{0.5}\PY{p}{,} \PY{l+m+mi}{1}\PY{p}{,} \PY{l+m+mi}{3}\PY{p}{,} \PY{l+m+mi}{5}\PY{p}{]}\PY{p}{,}
                 \PY{n}{phase}\PY{o}{=}\PY{p}{(}\PY{l+m+mi}{0}\PY{p}{,} \PY{l+m+mi}{2}\PY{o}{*}\PY{n}{np}\PY{o}{.}\PY{n}{pi}\PY{p}{,} \PY{l+m+mf}{0.05}\PY{p}{)}\PY{p}{,}
                 \PY{n}{handle}\PY{o}{=}\PY{n}{fixed}\PY{p}{(}\PY{n}{handle1}\PY{p}{)}\PY{p}{,}
                \PY{p}{)}\PY{p}{;}
\end{Verbatim}


    \begin{center}\rule{0.5\linewidth}{\linethickness}\end{center}

    \hypertarget{exercice-distribution-uniforme}{%
\subsubsection{Exercice : distribution
uniforme}\label{exercice-distribution-uniforme}}

    Voyons un deuxième exemple avec \texttt{bokeh}. Vous pouvez prendre ceci
comme un exercice, et le faire de votre côté avant de lire la suite du
notebook.

    On veut ici écrire un outil pour afficher une distribution de points
dans une ellipse, de taille et de position réglables.

Dans la solution que vous trouverez ci-dessous, le nombre de points
\texttt{N} dans la distribution est supposé constant; en fait, dans ce
code on va tirer au sort une bonne fois pour toutes \texttt{N} points
dans le cercle de rayon 1, avec une distribution uniforme, et simplement
déformer cette distribution pour occuper l'espace cible.

On se donne donc comme réglages~:

\begin{itemize}
\tightlist
\item
  \texttt{dx} et \texttt{dy}, les coordonnées du centre de l'ellipse,
\item
  \texttt{rx} et \texttt{ry} les rayons en x et en y de l'ellipse,
\item
  et enfin \texttt{alpha} l'angle de rotation de l'ellipse.
\end{itemize}

    \begin{center}\rule{0.5\linewidth}{\linethickness}\end{center}

    \begin{Verbatim}[commandchars=\\\{\}]
{\color{incolor}In [{\color{incolor}10}]:} \PY{c+c1}{\PYZsh{} petit utilitaire pour calculer la distribution}
         \PY{c+c1}{\PYZsh{} uniforme de départ}
         \PY{k}{def} \PY{n+nf}{uniform\PYZus{}distribution}\PY{p}{(}\PY{n}{N}\PY{p}{)}\PY{p}{:}
             \PY{c+c1}{\PYZsh{} on tire au hasard un rho et un rayon}
             \PY{n}{rhos} \PY{o}{=} \PY{l+m+mi}{2} \PY{o}{*} \PY{n}{np}\PY{o}{.}\PY{n}{pi} \PY{o}{*} \PY{n}{np}\PY{o}{.}\PY{n}{random}\PY{o}{.}\PY{n}{sample}\PY{p}{(}\PY{n}{N}\PY{p}{)}
             \PY{n}{rads} \PY{o}{=} \PY{n}{np}\PY{o}{.}\PY{n}{random}\PY{o}{.}\PY{n}{sample}\PY{p}{(}\PY{n}{N}\PY{p}{)}
             \PY{c+c1}{\PYZsh{} il faut prendre la racine carrée du rayon}
             \PY{c+c1}{\PYZsh{} sinon ce n\PYZsq{}est pas uniforme dans le plan}
             \PY{n}{circle\PYZus{}x} \PY{o}{=} \PY{n}{np}\PY{o}{.}\PY{n}{sqrt}\PY{p}{(}\PY{n}{rads}\PY{p}{)} \PY{o}{*} \PY{n}{np}\PY{o}{.}\PY{n}{cos}\PY{p}{(}\PY{n}{rhos}\PY{p}{)}
             \PY{n}{circle\PYZus{}y} \PY{o}{=} \PY{n}{np}\PY{o}{.}\PY{n}{sqrt}\PY{p}{(}\PY{n}{rads}\PY{p}{)} \PY{o}{*} \PY{n}{np}\PY{o}{.}\PY{n}{sin}\PY{p}{(}\PY{n}{rhos}\PY{p}{)}
             \PY{k}{return} \PY{n}{circle\PYZus{}x}\PY{p}{,} \PY{n}{circle\PYZus{}y}
\end{Verbatim}


    \hypertarget{les-grandeurs-constantes}{%
\subparagraph{Les grandeurs constantes}\label{les-grandeurs-constantes}}

    \begin{Verbatim}[commandchars=\\\{\}]
{\color{incolor}In [{\color{incolor}11}]:} \PY{c+c1}{\PYZsh{} les grandeurs constantes}
         \PY{n}{N} \PY{o}{=} \PY{l+m+mi}{1000}
\end{Verbatim}


    \begin{Verbatim}[commandchars=\\\{\}]
{\color{incolor}In [{\color{incolor}12}]:} \PY{c+c1}{\PYZsh{} on calcule la distribution initiale}
         \PY{c+c1}{\PYZsh{} (celle\PYZhy{}ci est vraiment uniforme)}
         \PY{c+c1}{\PYZsh{} dans le cercle de rayon 1}
         \PY{n}{x0}\PY{p}{,} \PY{n}{y0} \PY{o}{=} \PY{n}{uniform\PYZus{}distribution}\PY{p}{(}\PY{n}{N}\PY{p}{)}
\end{Verbatim}


    \begin{Verbatim}[commandchars=\\\{\}]
{\color{incolor}In [{\color{incolor}13}]:} \PY{c+c1}{\PYZsh{} et aussi:}
         \PY{c+c1}{\PYZsh{} pour que ce soit plus joli je tire au hasard}
         \PY{c+c1}{\PYZsh{} des couleurs, et des rayons pour les points}
         
         \PY{c+c1}{\PYZsh{} le rouge entre 50 et 250}
         \PY{n}{reds} \PY{o}{=} \PY{l+m+mi}{50} \PY{o}{+} \PY{l+m+mi}{200} \PY{o}{*} \PY{n}{np}\PY{o}{.}\PY{n}{random}\PY{o}{.}\PY{n}{random}\PY{p}{(}\PY{n}{size}\PY{o}{=}\PY{n}{N}\PY{p}{)}
         \PY{c+c1}{\PYZsh{} le vert entre 30 et 250}
         \PY{n}{greens} \PY{o}{=} \PY{l+m+mi}{30} \PY{o}{+} \PY{l+m+mi}{220} \PY{o}{*} \PY{n}{np}\PY{o}{.}\PY{n}{random}\PY{o}{.}\PY{n}{random}\PY{p}{(}\PY{n}{size}\PY{o}{=}\PY{n}{N}\PY{p}{)}
         \PY{c+c1}{\PYZsh{} la mise en forme des couleurs}
         \PY{c+c1}{\PYZsh{} le bleu est constant à 150}
         \PY{n}{colors} \PY{o}{=} \PY{p}{[}
             \PY{n}{f}\PY{l+s+s2}{\PYZdq{}}\PY{l+s+s2}{\PYZsh{}}\PY{l+s+s2}{\PYZob{}}\PY{l+s+s2}{int(red):02x\PYZcb{}}\PY{l+s+s2}{\PYZob{}}\PY{l+s+s2}{int(green):02x\PYZcb{}}\PY{l+s+si}{\PYZob{}150:02x\PYZcb{}}\PY{l+s+s2}{\PYZdq{}}
             \PY{k}{for} \PY{n}{red}\PY{p}{,} \PY{n}{green} \PY{o+ow}{in} \PY{n+nb}{zip}\PY{p}{(}\PY{n}{reds}\PY{p}{,} \PY{n}{greens}\PY{p}{)}
         \PY{p}{]}
         
         \PY{c+c1}{\PYZsh{} les rayons des points; entre 0.05 et 0.25}
         \PY{n}{radii} \PY{o}{=} \PY{l+m+mf}{0.05} \PY{o}{+} \PY{n}{np}\PY{o}{.}\PY{n}{random}\PY{o}{.}\PY{n}{random}\PY{p}{(}\PY{n}{size}\PY{o}{=}\PY{n}{N}\PY{p}{)} \PY{o}{*} \PY{o}{.}\PY{l+m+mi}{20}
\end{Verbatim}


    \hypertarget{cruxe9ation-de-la-figure-initiale}{%
\subparagraph{Création de la figure
initiale}\label{cruxe9ation-de-la-figure-initiale}}

    \begin{Verbatim}[commandchars=\\\{\}]
{\color{incolor}In [{\color{incolor}14}]:} \PY{c+c1}{\PYZsh{} c\PYZsq{}est ici qu\PYZsq{}on commence à faire du bokeh}
         \PY{c+c1}{\PYZsh{} les choix des bornes sont très arbitraires}
         \PY{n}{fig2} \PY{o}{=} \PY{n}{figure}\PY{p}{(}
             \PY{n}{title}\PY{o}{=}\PY{l+s+s2}{\PYZdq{}}\PY{l+s+s2}{distribution pseudo\PYZhy{}uniforme}\PY{l+s+s2}{\PYZdq{}}\PY{p}{,}
             \PY{n}{plot\PYZus{}height}\PY{o}{=}\PY{l+m+mi}{250}\PY{p}{,} \PY{n}{plot\PYZus{}width}\PY{o}{=}\PY{l+m+mi}{250}\PY{p}{,}
             \PY{n}{x\PYZus{}range}\PY{o}{=}\PY{p}{(}\PY{o}{\PYZhy{}}\PY{l+m+mi}{10}\PY{p}{,} \PY{l+m+mi}{10}\PY{p}{)}\PY{p}{,}
             \PY{n}{y\PYZus{}range}\PY{o}{=}\PY{p}{(}\PY{o}{\PYZhy{}}\PY{l+m+mi}{10}\PY{p}{,} \PY{l+m+mi}{10}\PY{p}{)}\PY{p}{,}
         \PY{p}{)}
\end{Verbatim}


    \begin{Verbatim}[commandchars=\\\{\}]
{\color{incolor}In [{\color{incolor}15}]:} \PY{c+c1}{\PYZsh{} on crée le nuage de points dans la figure}
         \PY{n}{cloud} \PY{o}{=} \PY{n}{fig2}\PY{o}{.}\PY{n}{circle}\PY{p}{(}
             \PY{n}{x0}\PY{p}{,} \PY{n}{y0}\PY{p}{,}
             \PY{n}{radius} \PY{o}{=} \PY{n}{radii}\PY{p}{,}
             \PY{n}{fill\PYZus{}color}\PY{o}{=}\PY{n}{colors}\PY{p}{,} \PY{n}{fill\PYZus{}alpha}\PY{o}{=}\PY{l+m+mf}{0.6}\PY{p}{,}
             \PY{n}{line\PYZus{}color}\PY{o}{=}\PY{k+kc}{None}\PY{p}{,} \PY{n}{line\PYZus{}width}\PY{o}{=}\PY{o}{.}\PY{l+m+mi}{1}
         \PY{p}{)}
\end{Verbatim}


    \hypertarget{mise-uxe0-jour-de-la-figure}{%
\subparagraph{Mise à jour de la
figure}\label{mise-uxe0-jour-de-la-figure}}

    \begin{Verbatim}[commandchars=\\\{\}]
{\color{incolor}In [{\color{incolor}16}]:} \PY{c+c1}{\PYZsh{} c\PYZsq{}est cette fonction qu\PYZsq{}on passe à interact}
         \PY{k}{def} \PY{n+nf}{update\PYZus{}cloud}\PY{p}{(}\PY{n}{rx}\PY{p}{,} \PY{n}{ry}\PY{p}{,} \PY{n}{dx}\PY{p}{,} \PY{n}{dy}\PY{p}{,} \PY{n}{alpha}\PY{p}{,} \PY{n}{handle}\PY{p}{)}\PY{p}{:}
             \PY{c+c1}{\PYZsh{} on recalcule les x et y}
             \PY{c+c1}{\PYZsh{} à partir des valeurs initiales}
             \PY{n}{s}\PY{p}{,} \PY{n}{c} \PY{o}{=} \PY{n}{np}\PY{o}{.}\PY{n}{sin}\PY{p}{(}\PY{n}{alpha}\PY{p}{)}\PY{p}{,} \PY{n}{np}\PY{o}{.}\PY{n}{cos}\PY{p}{(}\PY{n}{alpha}\PY{p}{)}
             \PY{n}{x} \PY{o}{=} \PY{n}{dx} \PY{o}{+} \PY{n}{c} \PY{o}{*} \PY{n}{rx} \PY{o}{*} \PY{n}{x0} \PY{o}{\PYZhy{}} \PY{n}{s} \PY{o}{*} \PY{n}{ry} \PY{o}{*} \PY{n}{y0}
             \PY{n}{y} \PY{o}{=} \PY{n}{dy} \PY{o}{+} \PY{n}{s} \PY{o}{*} \PY{n}{rx} \PY{o}{*} \PY{n}{x0} \PY{o}{+} \PY{n}{c} \PY{o}{*} \PY{n}{ry} \PY{o}{*} \PY{n}{y0}
             \PY{n}{cloud}\PY{o}{.}\PY{n}{data\PYZus{}source}\PY{o}{.}\PY{n}{data}\PY{p}{[}\PY{l+s+s1}{\PYZsq{}}\PY{l+s+s1}{x}\PY{l+s+s1}{\PYZsq{}}\PY{p}{]} \PY{o}{=} \PY{n}{x}
             \PY{n}{cloud}\PY{o}{.}\PY{n}{data\PYZus{}source}\PY{o}{.}\PY{n}{data}\PY{p}{[}\PY{l+s+s1}{\PYZsq{}}\PY{l+s+s1}{y}\PY{l+s+s1}{\PYZsq{}}\PY{p}{]} \PY{o}{=} \PY{n}{y}
             \PY{n}{push\PYZus{}notebook}\PY{p}{(}\PY{n}{handle}\PY{p}{)}
\end{Verbatim}


    \hypertarget{il-ny-a-plus-quuxe0}{%
\subparagraph{Il n'y a plus qu'à \ldots{}}\label{il-ny-a-plus-quuxe0}}

    \begin{Verbatim}[commandchars=\\\{\}]
{\color{incolor}In [{\color{incolor}17}]:} \PY{n}{handle2} \PY{o}{=} \PY{n}{show}\PY{p}{(}
             \PY{n}{fig2}\PY{p}{,}
             \PY{n}{notebook\PYZus{}handle}\PY{o}{=}\PY{k+kc}{True}\PY{p}{)}
\end{Verbatim}


    
    
    
    
    \begin{Verbatim}[commandchars=\\\{\}]
{\color{incolor}In [{\color{incolor}18}]:} \PY{n}{interact}\PY{p}{(}
           \PY{n}{update\PYZus{}cloud}\PY{p}{,}
           \PY{n}{rx}\PY{o}{=}\PY{n}{FloatSlider}\PY{p}{(}\PY{n+nb}{min}\PY{o}{=}\PY{o}{.}\PY{l+m+mi}{5}\PY{p}{,} \PY{n+nb}{max}\PY{o}{=}\PY{l+m+mi}{8}\PY{p}{,}
                          \PY{n}{step}\PY{o}{=}\PY{o}{.}\PY{l+m+mi}{1}\PY{p}{,} \PY{n}{value}\PY{o}{=}\PY{l+m+mf}{1.}\PY{p}{)}\PY{p}{,}
           \PY{n}{ry}\PY{o}{=}\PY{n}{FloatSlider}\PY{p}{(}\PY{n+nb}{min}\PY{o}{=}\PY{o}{.}\PY{l+m+mi}{5}\PY{p}{,} \PY{n+nb}{max}\PY{o}{=}\PY{l+m+mi}{8}\PY{p}{,}
                          \PY{n}{step}\PY{o}{=}\PY{o}{.}\PY{l+m+mi}{1}\PY{p}{,} \PY{n}{value}\PY{o}{=}\PY{l+m+mf}{1.}\PY{p}{)}\PY{p}{,}
           \PY{n}{dx}\PY{o}{=}\PY{p}{(}\PY{o}{\PYZhy{}}\PY{l+m+mi}{3}\PY{p}{,} \PY{o}{+}\PY{l+m+mi}{3}\PY{p}{,} \PY{o}{.}\PY{l+m+mi}{2}\PY{p}{)}\PY{p}{,}
           \PY{n}{dy}\PY{o}{=}\PY{p}{(}\PY{o}{\PYZhy{}}\PY{l+m+mi}{3}\PY{p}{,} \PY{o}{+}\PY{l+m+mi}{3}\PY{p}{,} \PY{o}{.}\PY{l+m+mi}{2}\PY{p}{)}\PY{p}{,}
           \PY{n}{alpha}\PY{o}{=}\PY{n}{FloatSlider}\PY{p}{(}
             \PY{n+nb}{min}\PY{o}{=}\PY{l+m+mf}{0.}\PY{p}{,} \PY{n+nb}{max}\PY{o}{=}\PY{n}{np}\PY{o}{.}\PY{n}{pi}\PY{p}{,}
             \PY{n}{step}\PY{o}{=}\PY{o}{.}\PY{l+m+mi}{05}\PY{p}{,} \PY{n}{value}\PY{o}{=}\PY{l+m+mf}{0.}\PY{p}{)}\PY{p}{,}
           \PY{n}{handle}\PY{o}{=}\PY{n}{fixed}\PY{p}{(}\PY{n}{handle2}\PY{p}{)}
         \PY{p}{)}\PY{p}{;}
\end{Verbatim}


    \hypertarget{autres-bibliothuxe8ques}{%
\subsubsection{Autres bibliothèques}\label{autres-bibliothuxe8ques}}

    Pour terminer cette digression sur les solutions alternatives à
\texttt{matplotlib}, j'aimerais vous signaler enfin rapidement
\href{https://plot.ly/}{la bibliothèque \texttt{plotly}}.

Cette bibliothèque est disponible en open source, et l'offre commerciale
de plotly est tournée vers le conseil autour de cette technologie. Comme
pour \texttt{bokeh}, elle est conçue comme un hybride entre Python et
JavaScript, au dessus de \texttt{d3.js}. En réalité, elle présente même
la particularité d'offrir une API unique disponible depuis Python,
JavaScript, et R.

    Comme on l'a dit en introduction, l'offre dans ce domaine est
pléthorique, aussi si vous avez un témoignage à apporter sur une
expérience que vous avez eue dans ce domaine, nous serons ravis de vous
voir la partager dans le forum du cours.


    % Add a bibliography block to the postdoc
    
    
    
