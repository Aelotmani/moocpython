    \hypertarget{yield-from-pour-cascader-deux-guxe9nuxe9rateurs}{%
\section{\texorpdfstring{\texttt{yield\ from} pour cascader deux
générateurs}{yield from pour cascader deux générateurs}}\label{yield-from-pour-cascader-deux-guxe9nuxe9rateurs}}

    Dans ce notebook nous allons voir comment fabriquer une fonction
génératrice qui appelle elle-même une autre fonction génératrice.

    \hypertarget{compluxe9ment---niveau-avancuxe9}{%
\subsection{Complément - niveau
avancé}\label{compluxe9ment---niveau-avancuxe9}}

    \hypertarget{une-fonction-guxe9nuxe9ratrice}{%
\subsubsection{Une fonction
génératrice}\label{une-fonction-guxe9nuxe9ratrice}}

    Commençons à nous définir une fonction génératrice ; par exemple ici
nous listons les diviseurs d'un entier, en excluant 1 et l'entier
lui-même~:

    \begin{Verbatim}[commandchars=\\\{\}]
{\color{incolor}In [{\color{incolor}1}]:} \PY{k}{def} \PY{n+nf}{divs}\PY{p}{(}\PY{n}{n}\PY{p}{,} \PY{n}{verbose}\PY{o}{=}\PY{k+kc}{False}\PY{p}{)}\PY{p}{:}
            \PY{k}{for} \PY{n}{i} \PY{o+ow}{in} \PY{n+nb}{range}\PY{p}{(}\PY{l+m+mi}{2}\PY{p}{,} \PY{n}{n}\PY{p}{)}\PY{p}{:}
                \PY{k}{if} \PY{n}{n} \PY{o}{\PYZpc{}} \PY{n}{i} \PY{o}{==} \PY{l+m+mi}{0}\PY{p}{:}
                    \PY{k}{if} \PY{n}{verbose}\PY{p}{:} 
                        \PY{n+nb}{print}\PY{p}{(}\PY{n}{f}\PY{l+s+s1}{\PYZsq{}}\PY{l+s+s1}{trouvé diviseur }\PY{l+s+si}{\PYZob{}i\PYZcb{}}\PY{l+s+s1}{ de }\PY{l+s+si}{\PYZob{}n\PYZcb{}}\PY{l+s+s1}{\PYZsq{}}\PY{p}{)}
                    \PY{k}{yield} \PY{n}{i}
\end{Verbatim}


    Comme attendu, l'appel direct à cette fonction ne donne rien d'utile :

    \begin{Verbatim}[commandchars=\\\{\}]
{\color{incolor}In [{\color{incolor}2}]:} \PY{n}{divs}\PY{p}{(}\PY{l+m+mi}{28}\PY{p}{)}
\end{Verbatim}


\begin{Verbatim}[commandchars=\\\{\}]
{\color{outcolor}Out[{\color{outcolor}2}]:} <generator object divs at 0x04F34660>
\end{Verbatim}
            
    Mais lorsqu'on l'utilise dans une boucle \texttt{for}:

    \begin{Verbatim}[commandchars=\\\{\}]
{\color{incolor}In [{\color{incolor}3}]:} \PY{k}{for} \PY{n}{d} \PY{o+ow}{in} \PY{n}{divs}\PY{p}{(}\PY{l+m+mi}{28}\PY{p}{)}\PY{p}{:}
            \PY{n+nb}{print}\PY{p}{(}\PY{n}{d}\PY{p}{)}
\end{Verbatim}


    \begin{Verbatim}[commandchars=\\\{\}]
2
4
7
14

    \end{Verbatim}

    \hypertarget{une-fonction-guxe9nuxe9ratrice-qui-appelle-une-autre-fonction-guxe9nuxe9ratrice}{%
\subsubsection{Une fonction génératrice qui appelle une autre fonction
génératrice}\label{une-fonction-guxe9nuxe9ratrice-qui-appelle-une-autre-fonction-guxe9nuxe9ratrice}}

    Bien, jusqu'ici c'est clair. Maintenant supposons que je veuille écrire
une fonction génératrice qui énumère tous les diviseurs de tous les
diviseurs d'un entier. Il s'agit donc, en sorte, d'écrire une fonction
génératrice qui en appelle une autre - ici elle même.

    \hypertarget{premiuxe8re-iduxe9e}{%
\subparagraph{Première idée\\\\}\label{premiuxe8re-iduxe9e}}

    Première idée naïve pour faire cela, mais qui ne marche pas~:

    \begin{Verbatim}[commandchars=\\\{\}]
{\color{incolor}In [{\color{incolor}4}]:} \PY{k}{def} \PY{n+nf}{divdivs}\PY{p}{(}\PY{n}{n}\PY{p}{)}\PY{p}{:}
            \PY{k}{for} \PY{n}{i} \PY{o+ow}{in} \PY{n}{divs}\PY{p}{(}\PY{n}{n}\PY{p}{)}\PY{p}{:}
                \PY{n}{divs}\PY{p}{(}\PY{n}{i}\PY{p}{)}
\end{Verbatim}


    \begin{Verbatim}[commandchars=\\\{\}]
{\color{incolor}In [{\color{incolor}5}]:} \PY{k}{try}\PY{p}{:}
            \PY{k}{for} \PY{n}{i} \PY{o+ow}{in} \PY{n}{divdivs}\PY{p}{(}\PY{l+m+mi}{28}\PY{p}{)}\PY{p}{:}
                \PY{n+nb}{print}\PY{p}{(}\PY{n}{i}\PY{p}{)}
        \PY{k}{except} \PY{n+ne}{Exception} \PY{k}{as} \PY{n}{e}\PY{p}{:}
            \PY{n+nb}{print}\PY{p}{(}\PY{n}{f}\PY{l+s+s2}{\PYZdq{}}\PY{l+s+s2}{OOPS }\PY{l+s+si}{\PYZob{}e\PYZcb{}}\PY{l+s+s2}{\PYZdq{}}\PY{p}{)}
\end{Verbatim}


    \begin{Verbatim}[commandchars=\\\{\}]
OOPS 'NoneType' object is not iterable

    \end{Verbatim}

    Ce qui se passe ici, c'est que \texttt{divdivs} est perçue comme une
fonction normale, lorsqu'on l'appelle elle ne retourne rien, donc
\texttt{None} ; et c'est sur ce \texttt{None} qu'on essaie de faire la
boucle \texttt{for} (à l'interieur du \texttt{try}), qui donc échoue.

    \hypertarget{deuxiuxe8me-iduxe9e}{%
\subparagraph{Deuxième idée\\\\}\label{deuxiuxe8me-iduxe9e}}

    Si on utilise juste \texttt{yield}, ça ne fait pas du tout ce qu'on
veut~:

    \begin{Verbatim}[commandchars=\\\{\}]
{\color{incolor}In [{\color{incolor}6}]:} \PY{k}{def} \PY{n+nf}{divdivs}\PY{p}{(}\PY{n}{n}\PY{p}{)}\PY{p}{:}
            \PY{k}{for} \PY{n}{i} \PY{o+ow}{in} \PY{n}{divs}\PY{p}{(}\PY{n}{n}\PY{p}{)}\PY{p}{:}
                \PY{k}{yield} \PY{n}{divs}\PY{p}{(}\PY{n}{i}\PY{p}{)}
\end{Verbatim}


    \begin{Verbatim}[commandchars=\\\{\}]
{\color{incolor}In [{\color{incolor}7}]:} \PY{k}{try}\PY{p}{:}
            \PY{k}{for} \PY{n}{i} \PY{o+ow}{in} \PY{n}{divdivs}\PY{p}{(}\PY{l+m+mi}{28}\PY{p}{)}\PY{p}{:}
                \PY{n+nb}{print}\PY{p}{(}\PY{n}{i}\PY{p}{)}
        \PY{k}{except} \PY{n+ne}{Exception} \PY{k}{as} \PY{n}{e}\PY{p}{:}
            \PY{n+nb}{print}\PY{p}{(}\PY{n}{f}\PY{l+s+s2}{\PYZdq{}}\PY{l+s+s2}{OOPS }\PY{l+s+si}{\PYZob{}e\PYZcb{}}\PY{l+s+s2}{\PYZdq{}}\PY{p}{)}
\end{Verbatim}


    \begin{Verbatim}[commandchars=\\\{\}]
<generator object divs at 0x051F8330>
<generator object divs at 0x051F8360>
<generator object divs at 0x051F8330>
<generator object divs at 0x051F8360>

    \end{Verbatim}

    En effet, c'est logique, chaque \texttt{yield} dans \texttt{divdivs()}
correspond à une itération de la boucle. Bref, il nous manque quelque
chose dans le langage pour arriver à faire ce qu'on veut.

    \hypertarget{yield-from}{%
\subparagraph{\texorpdfstring{\texttt{yield\ from}}{yield from}\\\\}\label{yield-from}}

    La construction du langage qui permet de faire ceci s'appelle
\texttt{yield\ from};

    \begin{Verbatim}[commandchars=\\\{\}]
{\color{incolor}In [{\color{incolor}8}]:} \PY{k}{def} \PY{n+nf}{divdivs}\PY{p}{(}\PY{n}{n}\PY{p}{)}\PY{p}{:}
            \PY{k}{for} \PY{n}{i} \PY{o+ow}{in} \PY{n}{divs}\PY{p}{(}\PY{n}{n}\PY{p}{)}\PY{p}{:}
                \PY{k}{yield from} \PY{n}{divs}\PY{p}{(}\PY{n}{i}\PY{p}{,} \PY{n}{verbose}\PY{o}{=}\PY{k+kc}{True}\PY{p}{)}
\end{Verbatim}


    \begin{Verbatim}[commandchars=\\\{\}]
{\color{incolor}In [{\color{incolor}9}]:} \PY{k}{try}\PY{p}{:}
            \PY{k}{for} \PY{n}{i} \PY{o+ow}{in} \PY{n}{divdivs}\PY{p}{(}\PY{l+m+mi}{28}\PY{p}{)}\PY{p}{:}
                \PY{n+nb}{print}\PY{p}{(}\PY{n}{i}\PY{p}{)}
        \PY{k}{except} \PY{n+ne}{Exception} \PY{k}{as} \PY{n}{e}\PY{p}{:}
            \PY{n+nb}{print}\PY{p}{(}\PY{n}{f}\PY{l+s+s2}{\PYZdq{}}\PY{l+s+s2}{OOPS }\PY{l+s+si}{\PYZob{}e\PYZcb{}}\PY{l+s+s2}{\PYZdq{}}\PY{p}{)}
\end{Verbatim}


    \begin{Verbatim}[commandchars=\\\{\}]
trouvé diviseur 2 de 4
2
trouvé diviseur 2 de 14
2
trouvé diviseur 7 de 14
7

    \end{Verbatim}

    Avec \texttt{yield\ from}, on peut indiquer que \texttt{divdivs} est une
fonction génératrice, et qu'il faut évaluer \texttt{divs(..)} comme un
générateur; ici l'interpréteur va empiler un second appel à
\texttt{divdivs}, et énumérer tous les résultats que cette fonction va
énumérer avec \texttt{yield}.