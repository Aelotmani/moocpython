    
    
    
    

    

    \hypertarget{les-outils-de-base-sur-les-chauxeenes-de-caractuxe8res-str}{%
\section{\texorpdfstring{Les outils de base sur les chaînes de
caractères
(\texttt{str})}{Les outils de base sur les chaînes de caractères (str)}}\label{les-outils-de-base-sur-les-chauxeenes-de-caractuxe8res-str}}

    \hypertarget{compluxe9ment---niveau-intermuxe9diaire}{%
\subsection{Complément - niveau
intermédiaire}\label{compluxe9ment---niveau-intermuxe9diaire}}

    \hypertarget{lire-la-documentation}{%
\subsubsection{Lire la documentation}\label{lire-la-documentation}}

    Même après des années de pratique, il est difficile de se souvenir de
toutes les méthodes travaillant sur les chaînes de caractères. Aussi il
est toujours utile de recourir à la documentation embarquée

    \begin{Verbatim}[commandchars=\\\{\}]
{\color{incolor}In [{\color{incolor}1}]:} \PY{n}{help}\PY{p}{(}\PY{n+nb}{str}\PY{p}{)}
\end{Verbatim}


    \begin{Verbatim}[commandchars=\\\{\}]
Help on class str in module builtins:

class str(object)
 |  str(object='') -> str
 |  str(bytes\_or\_buffer[, encoding[, errors]]) -> str
 |  
 |  Create a new string object from the given object. If encoding or
 |  errors is specified, then the object must expose a data buffer
 |  that will be decoded using the given encoding and error handler.
 |  Otherwise, returns the result of object.\_\_str\_\_() (if defined)
 |  or repr(object).
 |  encoding defaults to sys.getdefaultencoding().
 |  errors defaults to 'strict'.
 |  
 |  Methods defined here:
 |  
 |  \_\_add\_\_(self, value, /)
 |      Return self+value.
 |  
 |  \_\_contains\_\_(self, key, /)
 |      Return key in self.
 |  
 |  \_\_eq\_\_(self, value, /)
 |      Return self==value.
 |  
 |  \_\_format\_\_(self, format\_spec, /)
 |      Return a formatted version of the string as described by format\_spec.
 |  
 |  \_\_ge\_\_(self, value, /)
 |      Return self>=value.
 |  
 |  \_\_getattribute\_\_(self, name, /)
 |      Return getattr(self, name).
 |  
 |  \_\_getitem\_\_(self, key, /)
 |      Return self[key].
 |  
 |  \_\_getnewargs\_\_({\ldots})
 |  
 |  \_\_gt\_\_(self, value, /)
 |      Return self>value.
 |  
 |  \_\_hash\_\_(self, /)
 |      Return hash(self).
 |  
 |  \_\_iter\_\_(self, /)
 |      Implement iter(self).
 |  
 |  \_\_le\_\_(self, value, /)
 |      Return self<=value.
 |  
 |  \_\_len\_\_(self, /)
 |      Return len(self).
 |  
 |  \_\_lt\_\_(self, value, /)
 |      Return self<value.
 |  
 |  \_\_mod\_\_(self, value, /)
 |      Return self\%value.
 |  
 |  \_\_mul\_\_(self, value, /)
 |      Return self*value.
 |  
 |  \_\_ne\_\_(self, value, /)
 |      Return self!=value.
 |  
 |  \_\_repr\_\_(self, /)
 |      Return repr(self).
 |  
 |  \_\_rmod\_\_(self, value, /)
 |      Return value\%self.
 |  
 |  \_\_rmul\_\_(self, value, /)
 |      Return value*self.
 |  
 |  \_\_sizeof\_\_(self, /)
 |      Return the size of the string in memory, in bytes.
 |  
 |  \_\_str\_\_(self, /)
 |      Return str(self).
 |  
 |  capitalize(self, /)
 |      Return a capitalized version of the string.
 |      
 |      More specifically, make the first character have upper case and the rest lower
 |      case.
 |  
 |  casefold(self, /)
 |      Return a version of the string suitable for caseless comparisons.
 |  
 |  center(self, width, fillchar=' ', /)
 |      Return a centered string of length width.
 |      
 |      Padding is done using the specified fill character (default is a space).
 |  
 |  count({\ldots})
 |      S.count(sub[, start[, end]]) -> int
 |      
 |      Return the number of non-overlapping occurrences of substring sub in
 |      string S[start:end].  Optional arguments start and end are
 |      interpreted as in slice notation.
 |  
 |  encode(self, /, encoding='utf-8', errors='strict')
 |      Encode the string using the codec registered for encoding.
 |      
 |      encoding
 |        The encoding in which to encode the string.
 |      errors
 |        The error handling scheme to use for encoding errors.
 |        The default is 'strict' meaning that encoding errors raise a
 |        UnicodeEncodeError.  Other possible values are 'ignore', 'replace' and
 |        'xmlcharrefreplace' as well as any other name registered with
 |        codecs.register\_error that can handle UnicodeEncodeErrors.
 |  
 |  endswith({\ldots})
 |      S.endswith(suffix[, start[, end]]) -> bool
 |      
 |      Return True if S ends with the specified suffix, False otherwise.
 |      With optional start, test S beginning at that position.
 |      With optional end, stop comparing S at that position.
 |      suffix can also be a tuple of strings to try.
 |  
 |  expandtabs(self, /, tabsize=8)
 |      Return a copy where all tab characters are expanded using spaces.
 |      
 |      If tabsize is not given, a tab size of 8 characters is assumed.
 |  
 |  find({\ldots})
 |      S.find(sub[, start[, end]]) -> int
 |      
 |      Return the lowest index in S where substring sub is found,
 |      such that sub is contained within S[start:end].  Optional
 |      arguments start and end are interpreted as in slice notation.
 |      
 |      Return -1 on failure.
 |  
 |  format({\ldots})
 |      S.format(*args, **kwargs) -> str
 |      
 |      Return a formatted version of S, using substitutions from args and kwargs.
 |      The substitutions are identified by braces ('\{' and '\}').
 |  
 |  format\_map({\ldots})
 |      S.format\_map(mapping) -> str
 |      
 |      Return a formatted version of S, using substitutions from mapping.
 |      The substitutions are identified by braces ('\{' and '\}').
 |  
 |  index({\ldots})
 |      S.index(sub[, start[, end]]) -> int
 |      
 |      Return the lowest index in S where substring sub is found, 
 |      such that sub is contained within S[start:end].  Optional
 |      arguments start and end are interpreted as in slice notation.
 |      
 |      Raises ValueError when the substring is not found.
 |  
 |  isalnum(self, /)
 |      Return True if the string is an alpha-numeric string, False otherwise.
 |      
 |      A string is alpha-numeric if all characters in the string are alpha-numeric and
 |      there is at least one character in the string.
 |  
 |  isalpha(self, /)
 |      Return True if the string is an alphabetic string, False otherwise.
 |      
 |      A string is alphabetic if all characters in the string are alphabetic and there
 |      is at least one character in the string.
 |  
 |  isascii(self, /)
 |      Return True if all characters in the string are ASCII, False otherwise.
 |      
 |      ASCII characters have code points in the range U+0000-U+007F.
 |      Empty string is ASCII too.
 |  
 |  isdecimal(self, /)
 |      Return True if the string is a decimal string, False otherwise.
 |      
 |      A string is a decimal string if all characters in the string are decimal and
 |      there is at least one character in the string.
 |  
 |  isdigit(self, /)
 |      Return True if the string is a digit string, False otherwise.
 |      
 |      A string is a digit string if all characters in the string are digits and there
 |      is at least one character in the string.
 |  
 |  isidentifier(self, /)
 |      Return True if the string is a valid Python identifier, False otherwise.
 |      
 |      Use keyword.iskeyword() to test for reserved identifiers such as "def" and
 |      "class".
 |  
 |  islower(self, /)
 |      Return True if the string is a lowercase string, False otherwise.
 |      
 |      A string is lowercase if all cased characters in the string are lowercase and
 |      there is at least one cased character in the string.
 |  
 |  isnumeric(self, /)
 |      Return True if the string is a numeric string, False otherwise.
 |      
 |      A string is numeric if all characters in the string are numeric and there is at
 |      least one character in the string.
 |  
 |  isprintable(self, /)
 |      Return True if the string is printable, False otherwise.
 |      
 |      A string is printable if all of its characters are considered printable in
 |      repr() or if it is empty.
 |  
 |  isspace(self, /)
 |      Return True if the string is a whitespace string, False otherwise.
 |      
 |      A string is whitespace if all characters in the string are whitespace and there
 |      is at least one character in the string.
 |  
 |  istitle(self, /)
 |      Return True if the string is a title-cased string, False otherwise.
 |      
 |      In a title-cased string, upper- and title-case characters may only
 |      follow uncased characters and lowercase characters only cased ones.
 |  
 |  isupper(self, /)
 |      Return True if the string is an uppercase string, False otherwise.
 |      
 |      A string is uppercase if all cased characters in the string are uppercase and
 |      there is at least one cased character in the string.
 |  
 |  join(self, iterable, /)
 |      Concatenate any number of strings.
 |      
 |      The string whose method is called is inserted in between each given string.
 |      The result is returned as a new string.
 |      
 |      Example: '.'.join(['ab', 'pq', 'rs']) -> 'ab.pq.rs'
 |  
 |  ljust(self, width, fillchar=' ', /)
 |      Return a left-justified string of length width.
 |      
 |      Padding is done using the specified fill character (default is a space).
 |  
 |  lower(self, /)
 |      Return a copy of the string converted to lowercase.
 |  
 |  lstrip(self, chars=None, /)
 |      Return a copy of the string with leading whitespace removed.
 |      
 |      If chars is given and not None, remove characters in chars instead.
 |  
 |  partition(self, sep, /)
 |      Partition the string into three parts using the given separator.
 |      
 |      This will search for the separator in the string.  If the separator is found,
 |      returns a 3-tuple containing the part before the separator, the separator
 |      itself, and the part after it.
 |      
 |      If the separator is not found, returns a 3-tuple containing the original string
 |      and two empty strings.
 |  
 |  replace(self, old, new, count=-1, /)
 |      Return a copy with all occurrences of substring old replaced by new.
 |      
 |        count
 |          Maximum number of occurrences to replace.
 |          -1 (the default value) means replace all occurrences.
 |      
 |      If the optional argument count is given, only the first count occurrences are
 |      replaced.
 |  
 |  rfind({\ldots})
 |      S.rfind(sub[, start[, end]]) -> int
 |      
 |      Return the highest index in S where substring sub is found,
 |      such that sub is contained within S[start:end].  Optional
 |      arguments start and end are interpreted as in slice notation.
 |      
 |      Return -1 on failure.
 |  
 |  rindex({\ldots})
 |      S.rindex(sub[, start[, end]]) -> int
 |      
 |      Return the highest index in S where substring sub is found,
 |      such that sub is contained within S[start:end].  Optional
 |      arguments start and end are interpreted as in slice notation.
 |      
 |      Raises ValueError when the substring is not found.
 |  
 |  rjust(self, width, fillchar=' ', /)
 |      Return a right-justified string of length width.
 |      
 |      Padding is done using the specified fill character (default is a space).
 |  
 |  rpartition(self, sep, /)
 |      Partition the string into three parts using the given separator.
 |      
 |      This will search for the separator in the string, starting at the end. If
 |      the separator is found, returns a 3-tuple containing the part before the
 |      separator, the separator itself, and the part after it.
 |      
 |      If the separator is not found, returns a 3-tuple containing two empty strings
 |      and the original string.
 |  
 |  rsplit(self, /, sep=None, maxsplit=-1)
 |      Return a list of the words in the string, using sep as the delimiter string.
 |      
 |        sep
 |          The delimiter according which to split the string.
 |          None (the default value) means split according to any whitespace,
 |          and discard empty strings from the result.
 |        maxsplit
 |          Maximum number of splits to do.
 |          -1 (the default value) means no limit.
 |      
 |      Splits are done starting at the end of the string and working to the front.
 |  
 |  rstrip(self, chars=None, /)
 |      Return a copy of the string with trailing whitespace removed.
 |      
 |      If chars is given and not None, remove characters in chars instead.
 |  
 |  split(self, /, sep=None, maxsplit=-1)
 |      Return a list of the words in the string, using sep as the delimiter string.
 |      
 |      sep
 |        The delimiter according which to split the string.
 |        None (the default value) means split according to any whitespace,
 |        and discard empty strings from the result.
 |      maxsplit
 |        Maximum number of splits to do.
 |        -1 (the default value) means no limit.
 |  
 |  splitlines(self, /, keepends=False)
 |      Return a list of the lines in the string, breaking at line boundaries.
 |      
 |      Line breaks are not included in the resulting list unless keepends is given and
 |      true.
 |  
 |  startswith({\ldots})
 |      S.startswith(prefix[, start[, end]]) -> bool
 |      
 |      Return True if S starts with the specified prefix, False otherwise.
 |      With optional start, test S beginning at that position.
 |      With optional end, stop comparing S at that position.
 |      prefix can also be a tuple of strings to try.
 |  
 |  strip(self, chars=None, /)
 |      Return a copy of the string with leading and trailing whitespace remove.
 |      
 |      If chars is given and not None, remove characters in chars instead.
 |  
 |  swapcase(self, /)
 |      Convert uppercase characters to lowercase and lowercase characters to uppercase.
 |  
 |  title(self, /)
 |      Return a version of the string where each word is titlecased.
 |      
 |      More specifically, words start with uppercased characters and all remaining
 |      cased characters have lower case.
 |  
 |  translate(self, table, /)
 |      Replace each character in the string using the given translation table.
 |      
 |        table
 |          Translation table, which must be a mapping of Unicode ordinals to
 |          Unicode ordinals, strings, or None.
 |      
 |      The table must implement lookup/indexing via \_\_getitem\_\_, for instance a
 |      dictionary or list.  If this operation raises LookupError, the character is
 |      left untouched.  Characters mapped to None are deleted.
 |  
 |  upper(self, /)
 |      Return a copy of the string converted to uppercase.
 |  
 |  zfill(self, width, /)
 |      Pad a numeric string with zeros on the left, to fill a field of the given width.
 |      
 |      The string is never truncated.
 |  
 |  ----------------------------------------------------------------------
 |  Static methods defined here:
 |  
 |  \_\_new\_\_(*args, **kwargs) from builtins.type
 |      Create and return a new object.  See help(type) for accurate signature.
 |  
 |  maketrans(x, y=None, z=None, /)
 |      Return a translation table usable for str.translate().
 |      
 |      If there is only one argument, it must be a dictionary mapping Unicode
 |      ordinals (integers) or characters to Unicode ordinals, strings or None.
 |      Character keys will be then converted to ordinals.
 |      If there are two arguments, they must be strings of equal length, and
 |      in the resulting dictionary, each character in x will be mapped to the
 |      character at the same position in y. If there is a third argument, it
 |      must be a string, whose characters will be mapped to None in the result.


    \end{Verbatim}

    Nous allons tenter ici de citer les méthodes les plus utilisées. Nous
n'avons le temps que de les utiliser de manière très simple, mais bien
souvent il est possible de passer en argument des options permettant de
ne travailler que sur une sous-chaîne, ou sur la première ou dernière
occurrence d'une sous-chaîne. Nous vous renvoyons à la documentation
pour obtenir toutes les précisions utiles.

    \hypertarget{duxe9coupage---assemblage-split-et-join}{%
\subsubsection{\texorpdfstring{Découpage - assemblage~: \texttt{split}
et
\texttt{join}}{Découpage - assemblage~: split et join}}\label{duxe9coupage---assemblage-split-et-join}}

    Les méthodes \texttt{split} et \texttt{join} permettent de découper une
chaîne selon un séparateur pour obtenir une liste, et à l'inverse de
reconstruire une chaîne à partir d'une liste.

    \texttt{split} permet donc de découper~:

    \begin{Verbatim}[commandchars=\\\{\}]
{\color{incolor}In [{\color{incolor}2}]:} \PY{l+s+s1}{\PYZsq{}}\PY{l+s+s1}{abc=:=def=:=ghi=:=jkl}\PY{l+s+s1}{\PYZsq{}}\PY{o}{.}\PY{n}{split}\PY{p}{(}\PY{l+s+s1}{\PYZsq{}}\PY{l+s+s1}{=:=}\PY{l+s+s1}{\PYZsq{}}\PY{p}{)}
\end{Verbatim}


\begin{Verbatim}[commandchars=\\\{\}]
{\color{outcolor}Out[{\color{outcolor}2}]:} ['abc', 'def', 'ghi', 'jkl']
\end{Verbatim}
            
    Et à l'inverse~:

    \begin{Verbatim}[commandchars=\\\{\}]
{\color{incolor}In [{\color{incolor}3}]:} \PY{l+s+s2}{\PYZdq{}}\PY{l+s+s2}{=:=}\PY{l+s+s2}{\PYZdq{}}\PY{o}{.}\PY{n}{join}\PY{p}{(}\PY{p}{[}\PY{l+s+s1}{\PYZsq{}}\PY{l+s+s1}{abc}\PY{l+s+s1}{\PYZsq{}}\PY{p}{,} \PY{l+s+s1}{\PYZsq{}}\PY{l+s+s1}{def}\PY{l+s+s1}{\PYZsq{}}\PY{p}{,} \PY{l+s+s1}{\PYZsq{}}\PY{l+s+s1}{ghi}\PY{l+s+s1}{\PYZsq{}}\PY{p}{,} \PY{l+s+s1}{\PYZsq{}}\PY{l+s+s1}{jkl}\PY{l+s+s1}{\PYZsq{}}\PY{p}{]}\PY{p}{)}
\end{Verbatim}


\begin{Verbatim}[commandchars=\\\{\}]
{\color{outcolor}Out[{\color{outcolor}3}]:} 'abc=:=def=:=ghi=:=jkl'
\end{Verbatim}
            
    Attention toutefois si le séparateur est un terminateur, la liste
résultat contient alors une dernière chaîne vide. En pratique, on
utilisera la méthode \texttt{strip}, que nous allons voir ci-dessous,
avant la méthode \texttt{split} pour éviter ce problème.

    \begin{Verbatim}[commandchars=\\\{\}]
{\color{incolor}In [{\color{incolor}4}]:} \PY{l+s+s1}{\PYZsq{}}\PY{l+s+s1}{abc;def;ghi;jkl;}\PY{l+s+s1}{\PYZsq{}}\PY{o}{.}\PY{n}{split}\PY{p}{(}\PY{l+s+s1}{\PYZsq{}}\PY{l+s+s1}{;}\PY{l+s+s1}{\PYZsq{}}\PY{p}{)}
\end{Verbatim}


\begin{Verbatim}[commandchars=\\\{\}]
{\color{outcolor}Out[{\color{outcolor}4}]:} ['abc', 'def', 'ghi', 'jkl', '']
\end{Verbatim}
            
    Qui s'inverse correctement cependant~:

    \begin{Verbatim}[commandchars=\\\{\}]
{\color{incolor}In [{\color{incolor}5}]:} \PY{l+s+s2}{\PYZdq{}}\PY{l+s+s2}{;}\PY{l+s+s2}{\PYZdq{}}\PY{o}{.}\PY{n}{join}\PY{p}{(}\PY{p}{[}\PY{l+s+s1}{\PYZsq{}}\PY{l+s+s1}{abc}\PY{l+s+s1}{\PYZsq{}}\PY{p}{,} \PY{l+s+s1}{\PYZsq{}}\PY{l+s+s1}{def}\PY{l+s+s1}{\PYZsq{}}\PY{p}{,} \PY{l+s+s1}{\PYZsq{}}\PY{l+s+s1}{ghi}\PY{l+s+s1}{\PYZsq{}}\PY{p}{,} \PY{l+s+s1}{\PYZsq{}}\PY{l+s+s1}{jkl}\PY{l+s+s1}{\PYZsq{}}\PY{p}{,} \PY{l+s+s1}{\PYZsq{}}\PY{l+s+s1}{\PYZsq{}}\PY{p}{]}\PY{p}{)}
\end{Verbatim}


\begin{Verbatim}[commandchars=\\\{\}]
{\color{outcolor}Out[{\color{outcolor}5}]:} 'abc;def;ghi;jkl;'
\end{Verbatim}
            
    \hypertarget{remplacement-replace}{%
\subsubsection{\texorpdfstring{Remplacement~:
\texttt{replace}}{Remplacement~: replace}}\label{remplacement-replace}}

    \texttt{replace} est très pratique pour remplacer une sous-chaîne par
une autre, avec une limite éventuelle sur le nombre de remplacements~:

    \begin{Verbatim}[commandchars=\\\{\}]
{\color{incolor}In [{\color{incolor}6}]:} \PY{l+s+s2}{\PYZdq{}}\PY{l+s+s2}{abcdefabcdefabcdef}\PY{l+s+s2}{\PYZdq{}}\PY{o}{.}\PY{n}{replace}\PY{p}{(}\PY{l+s+s2}{\PYZdq{}}\PY{l+s+s2}{abc}\PY{l+s+s2}{\PYZdq{}}\PY{p}{,} \PY{l+s+s2}{\PYZdq{}}\PY{l+s+s2}{zoo}\PY{l+s+s2}{\PYZdq{}}\PY{p}{)}
\end{Verbatim}


\begin{Verbatim}[commandchars=\\\{\}]
{\color{outcolor}Out[{\color{outcolor}6}]:} 'zoodefzoodefzoodef'
\end{Verbatim}
            
    \begin{Verbatim}[commandchars=\\\{\}]
{\color{incolor}In [{\color{incolor}7}]:} \PY{l+s+s2}{\PYZdq{}}\PY{l+s+s2}{abcdefabcdefabcdef}\PY{l+s+s2}{\PYZdq{}}\PY{o}{.}\PY{n}{replace}\PY{p}{(}\PY{l+s+s2}{\PYZdq{}}\PY{l+s+s2}{abc}\PY{l+s+s2}{\PYZdq{}}\PY{p}{,} \PY{l+s+s2}{\PYZdq{}}\PY{l+s+s2}{zoo}\PY{l+s+s2}{\PYZdq{}}\PY{p}{,} \PY{l+m+mi}{2}\PY{p}{)}
\end{Verbatim}


\begin{Verbatim}[commandchars=\\\{\}]
{\color{outcolor}Out[{\color{outcolor}7}]:} 'zoodefzoodefabcdef'
\end{Verbatim}
            
    Plusieurs appels à \texttt{replace} peuvent être chaînés comme ceci~:

    \begin{Verbatim}[commandchars=\\\{\}]
{\color{incolor}In [{\color{incolor}8}]:} \PY{l+s+s2}{\PYZdq{}}\PY{l+s+s2}{les [x] qui disent [y]}\PY{l+s+s2}{\PYZdq{}}\PY{o}{.}\PY{n}{replace}\PY{p}{(}\PY{l+s+s2}{\PYZdq{}}\PY{l+s+s2}{[x]}\PY{l+s+s2}{\PYZdq{}}\PY{p}{,} \PY{l+s+s2}{\PYZdq{}}\PY{l+s+s2}{chevaliers}\PY{l+s+s2}{\PYZdq{}}\PY{p}{)}\PY{o}{.}\PY{n}{replace}\PY{p}{(}\PY{l+s+s2}{\PYZdq{}}\PY{l+s+s2}{[y]}\PY{l+s+s2}{\PYZdq{}}\PY{p}{,} \PY{l+s+s2}{\PYZdq{}}\PY{l+s+s2}{Ni}\PY{l+s+s2}{\PYZdq{}}\PY{p}{)}
\end{Verbatim}


\begin{Verbatim}[commandchars=\\\{\}]
{\color{outcolor}Out[{\color{outcolor}8}]:} 'les chevaliers qui disent Ni'
\end{Verbatim}
            
    \hypertarget{nettoyage-strip}{%
\subsubsection{\texorpdfstring{Nettoyage~:
\texttt{strip}}{Nettoyage~: strip}}\label{nettoyage-strip}}

    On pourrait par exemple utiliser \texttt{replace} pour enlever les
espaces dans une chaîne, ce qui peut être utile pour ``nettoyer'' comme
ceci~:

    \begin{Verbatim}[commandchars=\\\{\}]
{\color{incolor}In [{\color{incolor}9}]:} \PY{l+s+s2}{\PYZdq{}}\PY{l+s+s2}{ abc:def:ghi }\PY{l+s+s2}{\PYZdq{}}\PY{o}{.}\PY{n}{replace}\PY{p}{(}\PY{l+s+s2}{\PYZdq{}}\PY{l+s+s2}{ }\PY{l+s+s2}{\PYZdq{}}\PY{p}{,} \PY{l+s+s2}{\PYZdq{}}\PY{l+s+s2}{\PYZdq{}}\PY{p}{)}
\end{Verbatim}


\begin{Verbatim}[commandchars=\\\{\}]
{\color{outcolor}Out[{\color{outcolor}9}]:} 'abc:def:ghi'
\end{Verbatim}
            
    Toutefois bien souvent on préfère utiliser \texttt{strip} qui ne
s'occupe que du début et de la fin de la chaîne, et gère aussi les
tabulations et autres retour à la ligne~:

    \begin{Verbatim}[commandchars=\\\{\}]
{\color{incolor}In [{\color{incolor}10}]:} \PY{l+s+s2}{\PYZdq{}}\PY{l+s+s2}{ }\PY{l+s+se}{\PYZbs{}t}\PY{l+s+s2}{une chaîne avec des trucs qui dépassent }\PY{l+s+se}{\PYZbs{}n}\PY{l+s+s2}{\PYZdq{}}\PY{o}{.}\PY{n}{strip}\PY{p}{(}\PY{p}{)}
\end{Verbatim}


\begin{Verbatim}[commandchars=\\\{\}]
{\color{outcolor}Out[{\color{outcolor}10}]:} 'une chaîne avec des trucs qui dépassent'
\end{Verbatim}
            
    On peut appliquer \texttt{strip} avant \texttt{split} pour éviter le
problème du dernier élément vide~:

    \begin{Verbatim}[commandchars=\\\{\}]
{\color{incolor}In [{\color{incolor}11}]:} \PY{l+s+s1}{\PYZsq{}}\PY{l+s+s1}{abc;def;ghi;jkl;}\PY{l+s+s1}{\PYZsq{}}\PY{o}{.}\PY{n}{strip}\PY{p}{(}\PY{l+s+s1}{\PYZsq{}}\PY{l+s+s1}{;}\PY{l+s+s1}{\PYZsq{}}\PY{p}{)}\PY{o}{.}\PY{n}{split}\PY{p}{(}\PY{l+s+s1}{\PYZsq{}}\PY{l+s+s1}{;}\PY{l+s+s1}{\PYZsq{}}\PY{p}{)}
\end{Verbatim}


\begin{Verbatim}[commandchars=\\\{\}]
{\color{outcolor}Out[{\color{outcolor}11}]:} ['abc', 'def', 'ghi', 'jkl']
\end{Verbatim}
            
    \hypertarget{rechercher-une-sous-chauxeene}{%
\subsubsection{Rechercher une
sous-chaîne}\label{rechercher-une-sous-chauxeene}}

    Plusieurs outils permettent de chercher une sous-chaîne. Il existe
\texttt{find} qui renvoie le plus petit index où on trouve la
sous-chaîne~:

    \begin{Verbatim}[commandchars=\\\{\}]
{\color{incolor}In [{\color{incolor}12}]:} \PY{c+c1}{\PYZsh{} l\PYZsq{}indice du début de la première occurrence}
         \PY{l+s+s2}{\PYZdq{}}\PY{l+s+s2}{abcdefcdefghefghijk}\PY{l+s+s2}{\PYZdq{}}\PY{o}{.}\PY{n}{find}\PY{p}{(}\PY{l+s+s2}{\PYZdq{}}\PY{l+s+s2}{def}\PY{l+s+s2}{\PYZdq{}}\PY{p}{)}
\end{Verbatim}


\begin{Verbatim}[commandchars=\\\{\}]
{\color{outcolor}Out[{\color{outcolor}12}]:} 3
\end{Verbatim}
            
    \begin{Verbatim}[commandchars=\\\{\}]
{\color{incolor}In [{\color{incolor}13}]:} \PY{c+c1}{\PYZsh{} ou \PYZhy{}1 si la chaîne n\PYZsq{}est pas présente}
         \PY{l+s+s2}{\PYZdq{}}\PY{l+s+s2}{abcdefcdefghefghijk}\PY{l+s+s2}{\PYZdq{}}\PY{o}{.}\PY{n}{find}\PY{p}{(}\PY{l+s+s2}{\PYZdq{}}\PY{l+s+s2}{zoo}\PY{l+s+s2}{\PYZdq{}}\PY{p}{)}
\end{Verbatim}


\begin{Verbatim}[commandchars=\\\{\}]
{\color{outcolor}Out[{\color{outcolor}13}]:} -1
\end{Verbatim}
            
    \texttt{rfind} fonctionne comme \texttt{find} mais en partant de la fin
de la chaîne~:

    \begin{Verbatim}[commandchars=\\\{\}]
{\color{incolor}In [{\color{incolor}14}]:} \PY{c+c1}{\PYZsh{} en partant de la fin}
         \PY{l+s+s2}{\PYZdq{}}\PY{l+s+s2}{abcdefcdefghefghijk}\PY{l+s+s2}{\PYZdq{}}\PY{o}{.}\PY{n}{rfind}\PY{p}{(}\PY{l+s+s2}{\PYZdq{}}\PY{l+s+s2}{fgh}\PY{l+s+s2}{\PYZdq{}}\PY{p}{)}
\end{Verbatim}


\begin{Verbatim}[commandchars=\\\{\}]
{\color{outcolor}Out[{\color{outcolor}14}]:} 13
\end{Verbatim}
            
    \begin{Verbatim}[commandchars=\\\{\}]
{\color{incolor}In [{\color{incolor}15}]:} \PY{c+c1}{\PYZsh{} notez que le résultat correspond}
         \PY{c+c1}{\PYZsh{} tout de même toujours au début de la chaîne}
         
         \PY{c+c1}{\PYZsh{} NB: en python les indices commencent à zéro}
         \PY{c+c1}{\PYZsh{} donc la notation ma\PYZus{}chaine[n] }
         \PY{c+c1}{\PYZsh{} permet d\PYZsq{}accèder au n+1 ème caractère de la chaine}
         \PY{l+s+s2}{\PYZdq{}}\PY{l+s+s2}{abcdefcdefghefghijk}\PY{l+s+s2}{\PYZdq{}}\PY{p}{[}\PY{l+m+mi}{13}\PY{p}{]}
\end{Verbatim}


\begin{Verbatim}[commandchars=\\\{\}]
{\color{outcolor}Out[{\color{outcolor}15}]:} 'f'
\end{Verbatim}
            
    La méthode \texttt{index} se comporte comme \texttt{find}, mais en cas
d'absence elle lève une \textbf{exception} (nous verrons ce concept plus
tard) plutôt que de renvoyer \texttt{-1}~:

    \begin{Verbatim}[commandchars=\\\{\}]
{\color{incolor}In [{\color{incolor}16}]:} \PY{l+s+s2}{\PYZdq{}}\PY{l+s+s2}{abcdefcdefghefghijk}\PY{l+s+s2}{\PYZdq{}}\PY{o}{.}\PY{n}{index}\PY{p}{(}\PY{l+s+s2}{\PYZdq{}}\PY{l+s+s2}{def}\PY{l+s+s2}{\PYZdq{}}\PY{p}{)}
\end{Verbatim}


\begin{Verbatim}[commandchars=\\\{\}]
{\color{outcolor}Out[{\color{outcolor}16}]:} 3
\end{Verbatim}
            
    \begin{Verbatim}[commandchars=\\\{\}]
{\color{incolor}In [{\color{incolor}17}]:} \PY{k}{try}\PY{p}{:}
             \PY{l+s+s2}{\PYZdq{}}\PY{l+s+s2}{abcdefcdefghefghijk}\PY{l+s+s2}{\PYZdq{}}\PY{o}{.}\PY{n}{index}\PY{p}{(}\PY{l+s+s2}{\PYZdq{}}\PY{l+s+s2}{zoo}\PY{l+s+s2}{\PYZdq{}}\PY{p}{)}
         \PY{k}{except} \PY{n+ne}{Exception} \PY{k}{as} \PY{n}{e}\PY{p}{:}
             \PY{n+nb}{print}\PY{p}{(}\PY{l+s+s2}{\PYZdq{}}\PY{l+s+s2}{OOPS}\PY{l+s+s2}{\PYZdq{}}\PY{p}{,} \PY{n+nb}{type}\PY{p}{(}\PY{n}{e}\PY{p}{)}\PY{p}{,} \PY{n}{e}\PY{p}{)}
\end{Verbatim}


    \begin{Verbatim}[commandchars=\\\{\}]
OOPS <class 'ValueError'> substring not found

    \end{Verbatim}

    Mais le plus simple pour chercher si une sous-chaîne est dans une autre
chaîne est d'utiliser l'instruction \texttt{in} sur laquelle nous
reviendrons lorsque nous parlerons des séquences~:

    \begin{Verbatim}[commandchars=\\\{\}]
{\color{incolor}In [{\color{incolor}18}]:} \PY{l+s+s2}{\PYZdq{}}\PY{l+s+s2}{def}\PY{l+s+s2}{\PYZdq{}} \PY{o+ow}{in} \PY{l+s+s2}{\PYZdq{}}\PY{l+s+s2}{abcdefcdefghefghijk}\PY{l+s+s2}{\PYZdq{}}
\end{Verbatim}


\begin{Verbatim}[commandchars=\\\{\}]
{\color{outcolor}Out[{\color{outcolor}18}]:} True
\end{Verbatim}
            
    La méthode \texttt{count} compte le nombre d'occurrences d'une
sous-chaîne~:

    \begin{Verbatim}[commandchars=\\\{\}]
{\color{incolor}In [{\color{incolor}19}]:} \PY{l+s+s2}{\PYZdq{}}\PY{l+s+s2}{abcdefcdefghefghijk}\PY{l+s+s2}{\PYZdq{}}\PY{o}{.}\PY{n}{count}\PY{p}{(}\PY{l+s+s2}{\PYZdq{}}\PY{l+s+s2}{ef}\PY{l+s+s2}{\PYZdq{}}\PY{p}{)}
\end{Verbatim}


\begin{Verbatim}[commandchars=\\\{\}]
{\color{outcolor}Out[{\color{outcolor}19}]:} 3
\end{Verbatim}
            
    Signalons enfin les méthodes de commodité suivantes~:

    \begin{Verbatim}[commandchars=\\\{\}]
{\color{incolor}In [{\color{incolor}20}]:} \PY{l+s+s2}{\PYZdq{}}\PY{l+s+s2}{abcdefcdefghefghijk}\PY{l+s+s2}{\PYZdq{}}\PY{o}{.}\PY{n}{startswith}\PY{p}{(}\PY{l+s+s2}{\PYZdq{}}\PY{l+s+s2}{abcd}\PY{l+s+s2}{\PYZdq{}}\PY{p}{)}
\end{Verbatim}


\begin{Verbatim}[commandchars=\\\{\}]
{\color{outcolor}Out[{\color{outcolor}20}]:} True
\end{Verbatim}
            
    \begin{Verbatim}[commandchars=\\\{\}]
{\color{incolor}In [{\color{incolor}21}]:} \PY{l+s+s2}{\PYZdq{}}\PY{l+s+s2}{abcdefcdefghefghijk}\PY{l+s+s2}{\PYZdq{}}\PY{o}{.}\PY{n}{endswith}\PY{p}{(}\PY{l+s+s2}{\PYZdq{}}\PY{l+s+s2}{ghijk}\PY{l+s+s2}{\PYZdq{}}\PY{p}{)}
\end{Verbatim}


\begin{Verbatim}[commandchars=\\\{\}]
{\color{outcolor}Out[{\color{outcolor}21}]:} True
\end{Verbatim}
            
    S'agissant des deux dernières, remarquons que~:

    \texttt{chaine.startswith(sous\_chaine)} \(\Longleftrightarrow\)
\texttt{chaine.find(sous\_chaine)\ ==\ 0}

\texttt{chaine.endswith(sous\_chaine)} \(\Longleftrightarrow\)
\texttt{chaine.rfind(sous\_chaine)\ ==\ (len(chaine)\ -\ len(sous\_chaine))}

    On remarque ici la supériorité en terme d'expressivité des méthodes
pythoniques \texttt{startswith} et \texttt{endswith}.

    \hypertarget{changement-de-casse}{%
\subsubsection{Changement de casse}\label{changement-de-casse}}

    Voici pour conclure quelques méthodes utiles qui parlent d'elles-mêmes~:

    \begin{Verbatim}[commandchars=\\\{\}]
{\color{incolor}In [{\color{incolor}22}]:} \PY{l+s+s2}{\PYZdq{}}\PY{l+s+s2}{monty PYTHON}\PY{l+s+s2}{\PYZdq{}}\PY{o}{.}\PY{n}{upper}\PY{p}{(}\PY{p}{)}
\end{Verbatim}


\begin{Verbatim}[commandchars=\\\{\}]
{\color{outcolor}Out[{\color{outcolor}22}]:} 'MONTY PYTHON'
\end{Verbatim}
            
    \begin{Verbatim}[commandchars=\\\{\}]
{\color{incolor}In [{\color{incolor}23}]:} \PY{l+s+s2}{\PYZdq{}}\PY{l+s+s2}{monty PYTHON}\PY{l+s+s2}{\PYZdq{}}\PY{o}{.}\PY{n}{lower}\PY{p}{(}\PY{p}{)}
\end{Verbatim}


\begin{Verbatim}[commandchars=\\\{\}]
{\color{outcolor}Out[{\color{outcolor}23}]:} 'monty python'
\end{Verbatim}
            
    \begin{Verbatim}[commandchars=\\\{\}]
{\color{incolor}In [{\color{incolor}24}]:} \PY{l+s+s2}{\PYZdq{}}\PY{l+s+s2}{monty PYTHON}\PY{l+s+s2}{\PYZdq{}}\PY{o}{.}\PY{n}{swapcase}\PY{p}{(}\PY{p}{)}
\end{Verbatim}


\begin{Verbatim}[commandchars=\\\{\}]
{\color{outcolor}Out[{\color{outcolor}24}]:} 'MONTY python'
\end{Verbatim}
            
    \begin{Verbatim}[commandchars=\\\{\}]
{\color{incolor}In [{\color{incolor}25}]:} \PY{l+s+s2}{\PYZdq{}}\PY{l+s+s2}{monty PYTHON}\PY{l+s+s2}{\PYZdq{}}\PY{o}{.}\PY{n}{capitalize}\PY{p}{(}\PY{p}{)}
\end{Verbatim}


\begin{Verbatim}[commandchars=\\\{\}]
{\color{outcolor}Out[{\color{outcolor}25}]:} 'Monty python'
\end{Verbatim}
            
    \begin{Verbatim}[commandchars=\\\{\}]
{\color{incolor}In [{\color{incolor}26}]:} \PY{l+s+s2}{\PYZdq{}}\PY{l+s+s2}{monty PYTHON}\PY{l+s+s2}{\PYZdq{}}\PY{o}{.}\PY{n}{title}\PY{p}{(}\PY{p}{)}
\end{Verbatim}


\begin{Verbatim}[commandchars=\\\{\}]
{\color{outcolor}Out[{\color{outcolor}26}]:} 'Monty Python'
\end{Verbatim}
            
    \hypertarget{pour-en-savoir-plus}{%
\subsubsection{Pour en savoir plus}\label{pour-en-savoir-plus}}

    Tous ces outils sont
\href{https://docs.python.org/3/library/stdtypes.html\#string-methods}{documentés
en détail ici (en anglais)}.


    % Add a bibliography block to the postdoc
    
    
    
