    
    
    
    

    

    \hypertarget{visibilituxe9-des-variables-de-boucle}{%
\section{Visibilité des variables de
boucle}\label{visibilituxe9-des-variables-de-boucle}}

    \hypertarget{compluxe9ment---niveau-basique}{%
\subsection{Complément - niveau
basique}\label{compluxe9ment---niveau-basique}}

    \hypertarget{une-astuce}{%
\subsubsection{Une astuce}\label{une-astuce}}

    Dans ce complément, nous allons beaucoup jouer avec le fait qu'une
variable soit définie ou non. Pour nous simplifier la vie, et surtout
rendre les cellules plus indépendantes les unes des autres si vous devez
les rejouer, nous allons utiliser la formule un peu magique suivante~:

    \begin{Verbatim}[commandchars=\\\{\}]
{\color{incolor}In [{\color{incolor}1}]:} \PY{c+c1}{\PYZsh{} on détruit la variable i si elle existe}
        \PY{k}{if} \PY{l+s+s1}{\PYZsq{}}\PY{l+s+s1}{i}\PY{l+s+s1}{\PYZsq{}} \PY{o+ow}{in} \PY{n+nb}{locals}\PY{p}{(}\PY{p}{)}\PY{p}{:} 
            \PY{k}{del} \PY{n}{i}      
\end{Verbatim}


    qui repose d'une part sur l'instruction \texttt{del} que nous avons déjà
vue, et sur la fonction \emph{built-in} \texttt{locals} que nous verrons
plus tard~; cette formule a l'avantage qu'on peut l'exécuter dans
n'importe quel contexte, que \texttt{i} soit définie ou non.

    \hypertarget{une-variable-de-boucle-reste-duxe9finie-au-deluxe0-de-la-boucle}{%
\subsubsection{Une variable de boucle reste définie au-delà de la
boucle}\label{une-variable-de-boucle-reste-duxe9finie-au-deluxe0-de-la-boucle}}

    Une variable de boucle est définie (assignée) dans la boucle et
\textbf{reste \emph{visible}} une fois la boucle terminée. Le plus
simple est de le voir sur un exemple~:

    \begin{Verbatim}[commandchars=\\\{\}]
{\color{incolor}In [{\color{incolor}2}]:} \PY{c+c1}{\PYZsh{} La variable \PYZsq{}i\PYZsq{} n\PYZsq{}est pas définie}
        \PY{k}{try}\PY{p}{:}
            \PY{n}{i}
        \PY{k}{except} \PY{n+ne}{NameError} \PY{k}{as} \PY{n}{e}\PY{p}{:}
            \PY{n+nb}{print}\PY{p}{(}\PY{l+s+s1}{\PYZsq{}}\PY{l+s+s1}{OOPS}\PY{l+s+s1}{\PYZsq{}}\PY{p}{,} \PY{n}{e}\PY{p}{)}
\end{Verbatim}


    \begin{Verbatim}[commandchars=\\\{\}]
OOPS name 'i' is not defined

    \end{Verbatim}

    \begin{Verbatim}[commandchars=\\\{\}]
{\color{incolor}In [{\color{incolor}3}]:} \PY{c+c1}{\PYZsh{} si à présent on fait une boucle}
        \PY{c+c1}{\PYZsh{} avec i comme variable de boucle}
        \PY{k}{for} \PY{n}{i} \PY{o+ow}{in} \PY{p}{[}\PY{l+m+mi}{0}\PY{p}{]}\PY{p}{:}
            \PY{k}{pass}
        
        \PY{c+c1}{\PYZsh{} alors maintenant i est définie}
        \PY{n}{i}
\end{Verbatim}


\begin{Verbatim}[commandchars=\\\{\}]
{\color{outcolor}Out[{\color{outcolor}3}]:} 0
\end{Verbatim}
            
    On dit que la variable \emph{fuite} (en anglais ``\emph{leak}''), dans
ce sens qu'elle continue d'exister au delà du bloc de la boucle à
proprement parler.

    On peut être tenté de tirer profit de ce trait, en lisant la valeur de
la variable après la boucle~; l'objet de ce complément est de vous
inciter à la prudence, et d'attirer votre attention sur certains points
qui peuvent être sources d'erreur.

    \hypertarget{attention-aux-boucles-vides}{%
\subsubsection{Attention aux boucles
vides}\label{attention-aux-boucles-vides}}

    Tout d'abord, il faut faire attention à ne pas écrire du code qui
dépende de ce trait \textbf{si la boucle peut être vide}. En effet, si
la boucle ne s'exécute pas du tout, la variable n'est \textbf{pas
affectée} et donc elle n'est \textbf{pas définie}. C'est évident, mais
ça peut l'être moins quand on lit du code réel, comme par exemple~:

    \begin{Verbatim}[commandchars=\\\{\}]
{\color{incolor}In [{\color{incolor}4}]:} \PY{c+c1}{\PYZsh{} on détruit la variable i si elle existe}
        \PY{k}{if} \PY{l+s+s1}{\PYZsq{}}\PY{l+s+s1}{i}\PY{l+s+s1}{\PYZsq{}} \PY{o+ow}{in} \PY{n+nb}{locals}\PY{p}{(}\PY{p}{)}\PY{p}{:} 
            \PY{k}{del} \PY{n}{i}   
\end{Verbatim}


    \begin{Verbatim}[commandchars=\\\{\}]
{\color{incolor}In [{\color{incolor}5}]:} \PY{c+c1}{\PYZsh{} une façon très scabreuse de calculer la longueur de l}
        \PY{k}{def} \PY{n+nf}{length}\PY{p}{(}\PY{n}{l}\PY{p}{)}\PY{p}{:}
            \PY{k}{for} \PY{n}{i}\PY{p}{,} \PY{n}{x} \PY{o+ow}{in} \PY{n+nb}{enumerate}\PY{p}{(}\PY{n}{l}\PY{p}{)}\PY{p}{:}
                \PY{k}{pass}
            \PY{k}{return} \PY{n}{i} \PY{o}{+} \PY{l+m+mi}{1}
        
        \PY{n}{length}\PY{p}{(}\PY{p}{[}\PY{l+m+mi}{1}\PY{p}{,} \PY{l+m+mi}{2}\PY{p}{,} \PY{l+m+mi}{3}\PY{p}{]}\PY{p}{)}
\end{Verbatim}


\begin{Verbatim}[commandchars=\\\{\}]
{\color{outcolor}Out[{\color{outcolor}5}]:} 3
\end{Verbatim}
            
    Ça a l'air correct, sauf que~:

    \begin{Verbatim}[commandchars=\\\{\}]
{\color{incolor}In [{\color{incolor} }]:} \PY{c+c1}{\PYZsh{} NOTE}
        \PY{c+c1}{\PYZsh{} auto\PYZhy{}exec\PYZhy{}for\PYZhy{}latex has skipped execution of this cell}
        
        \PY{c+c1}{\PYZsh{} ceci provoque une UnboundLocalError}
        \PY{n}{length}\PY{p}{(}\PY{p}{[}\PY{p}{]}\PY{p}{)}
\end{Verbatim}


    Ce résultat mérite une explication. Nous allons voir très bientôt
l'exception \texttt{UnboundLocalError}, mais pour le moment sachez
qu'elle se produit lorsqu'on a dans une fonction une variable locale et
une variable globale de même nom. Alors, pourquoi l'appel
\texttt{length({[}1,\ 2,\ 3{]})} retourne-t-il sans encombre, alors que
pour l'appel \texttt{length({[}{]})} il y a une exception~? Cela est lié
à la manière dont python détermine qu'une variable est locale.

Une variable est locale dans une fonction si elle est assignée dans la
fonction explicitement (avec une opération d'affectation) ou
implicitement (par exemple avec une boucle \texttt{for} comme ici)~;
nous reviendrons sur ce point un peu plus tard. Mais pour les fonctions,
pour une raison d'efficacité, une variable est définie comme locale à la
phase de pré-compilation, c'est-à-dire \emph{avant} l'exécution du code.
Le pré-compilateur ne peut pas savoir quel sera l'argument passé à la
fonction, il peut simplement savoir qu'il y a une boucle \texttt{for}
utilisant la variable \texttt{i}, il en conclut que \texttt{i} est
locale pour toute la fonction.

Lors du premier appel, on passe une liste à la fonction, liste qui est
parcourue par la boucle \texttt{for}. En sortie de boucle, on a bien une
variable locale \texttt{i} qui vaut 3. Lors du deuxième appel par
contre, on passe une liste vide à la fonction, la boucle \texttt{for} ne
peut rien parcourir, donc elle termine immédiatement. Lorsque l'on
arrive à la ligne \texttt{return\ i\ +\ 1} de la fonction, la variable
\texttt{i} n'a pas de valeur (on doit donc chercher \texttt{i} dans le
module), mais \texttt{i} a été définie par le pré-compilateur comme
étant locale, on a donc dans la même fonction une variable \texttt{i}
locale et une référence à une variable \texttt{i} globale, ce qui
provoque l'exception \texttt{UnboundLocalError}.

    \hypertarget{comment-faire-alors}{%
\subsubsection{Comment faire alors ?}\label{comment-faire-alors}}

    \hypertarget{utiliser-une-autre-variable}{%
\subparagraph{Utiliser une autre
variable}\label{utiliser-une-autre-variable}}

    La première voie consiste à déclarer une variable externe à la boucle et
à l'affecter à l'intérieur de la boucle, c'est-à-dire~:

    \begin{Verbatim}[commandchars=\\\{\}]
{\color{incolor}In [{\color{incolor}6}]:} \PY{c+c1}{\PYZsh{} on veut chercher le premier de ces nombres qui vérifie une condition}
        \PY{n}{candidates} \PY{o}{=} \PY{p}{[}\PY{l+m+mi}{3}\PY{p}{,} \PY{o}{\PYZhy{}}\PY{l+m+mi}{15}\PY{p}{,} \PY{l+m+mi}{1}\PY{p}{,} \PY{l+m+mi}{8}\PY{p}{]}
        
        \PY{c+c1}{\PYZsh{} pour fixer les idées disons qu\PYZsq{}on cherche un multiple de 5, peu importe}
        \PY{k}{def} \PY{n+nf}{checks}\PY{p}{(}\PY{n}{candidate}\PY{p}{)}\PY{p}{:}
            \PY{k}{return} \PY{n}{candidate} \PY{o}{\PYZpc{}} \PY{l+m+mi}{5} \PY{o}{==} \PY{l+m+mi}{0}
\end{Verbatim}


    \begin{Verbatim}[commandchars=\\\{\}]
{\color{incolor}In [{\color{incolor}7}]:} \PY{c+c1}{\PYZsh{} plutôt que de faire ceci}
        \PY{k}{for} \PY{n}{item} \PY{o+ow}{in} \PY{n}{candidates}\PY{p}{:}
            \PY{k}{if} \PY{n}{checks}\PY{p}{(}\PY{n}{item}\PY{p}{)}\PY{p}{:}
                \PY{k}{break}
        \PY{n+nb}{print}\PY{p}{(}\PY{l+s+s1}{\PYZsq{}}\PY{l+s+s1}{trouvé solution}\PY{l+s+s1}{\PYZsq{}}\PY{p}{,} \PY{n}{item}\PY{p}{)}
\end{Verbatim}


    \begin{Verbatim}[commandchars=\\\{\}]
trouvé solution -15

    \end{Verbatim}

    \begin{Verbatim}[commandchars=\\\{\}]
{\color{incolor}In [{\color{incolor}8}]:} \PY{c+c1}{\PYZsh{} il vaut mieux faire ceci}
        \PY{n}{solution} \PY{o}{=} \PY{k+kc}{None}
        \PY{k}{for} \PY{n}{item} \PY{o+ow}{in} \PY{n}{candidates}\PY{p}{:}
            \PY{k}{if} \PY{n}{checks}\PY{p}{(}\PY{n}{item}\PY{p}{)}\PY{p}{:}
                \PY{n}{solution} \PY{o}{=} \PY{n}{item}
                \PY{k}{break}
        
        \PY{n+nb}{print}\PY{p}{(}\PY{l+s+s1}{\PYZsq{}}\PY{l+s+s1}{trouvé solution}\PY{l+s+s1}{\PYZsq{}}\PY{p}{,} \PY{n}{solution}\PY{p}{)}
\end{Verbatim}


    \begin{Verbatim}[commandchars=\\\{\}]
trouvé solution -15

    \end{Verbatim}

    \hypertarget{au-minimum-initialiser-la-variable}{%
\subparagraph{Au minimum initialiser la
variable}\label{au-minimum-initialiser-la-variable}}

    Au minimum, si vous utilisez la variable de boucle après la boucle, il
est vivement conseillé de l'\textbf{initialiser} explicitement
\textbf{avant} la boucle, pour vous prémunir contre les boucles vides,
comme ceci~:

    \begin{Verbatim}[commandchars=\\\{\}]
{\color{incolor}In [{\color{incolor}9}]:} \PY{c+c1}{\PYZsh{} la fonction length de tout à l\PYZsq{}heure}
        \PY{k}{def} \PY{n+nf}{length1}\PY{p}{(}\PY{n}{l}\PY{p}{)}\PY{p}{:}
            \PY{k}{for} \PY{n}{i}\PY{p}{,} \PY{n}{x} \PY{o+ow}{in} \PY{n+nb}{enumerate}\PY{p}{(}\PY{n}{l}\PY{p}{)}\PY{p}{:}
                \PY{k}{pass}
            \PY{k}{return} \PY{n}{i} \PY{o}{+} \PY{l+m+mi}{1}
\end{Verbatim}


    \begin{Verbatim}[commandchars=\\\{\}]
{\color{incolor}In [{\color{incolor}10}]:} \PY{c+c1}{\PYZsh{} une version plus robuste }
         \PY{k}{def} \PY{n+nf}{length2}\PY{p}{(}\PY{n}{l}\PY{p}{)}\PY{p}{:}
             \PY{c+c1}{\PYZsh{} on initialise i explicitement}
             \PY{c+c1}{\PYZsh{} pour le cas où l est vide}
             \PY{n}{i} \PY{o}{=} \PY{o}{\PYZhy{}}\PY{l+m+mi}{1}
             \PY{k}{for} \PY{n}{i}\PY{p}{,} \PY{n}{x} \PY{o+ow}{in} \PY{n+nb}{enumerate}\PY{p}{(}\PY{n}{l}\PY{p}{)}\PY{p}{:}
                 \PY{k}{pass}
             \PY{c+c1}{\PYZsh{} comme cela i est toujours déclarée}
             \PY{k}{return} \PY{n}{i} \PY{o}{+} \PY{l+m+mi}{1}
\end{Verbatim}


    \begin{Verbatim}[commandchars=\\\{\}]
{\color{incolor}In [{\color{incolor} }]:} \PY{c+c1}{\PYZsh{} NOTE}
        \PY{c+c1}{\PYZsh{} auto\PYZhy{}exec\PYZhy{}for\PYZhy{}latex has skipped execution of this cell}
        
        \PY{c+c1}{\PYZsh{} comme ci\PYZhy{}dessus: UnboundLocalError}
        \PY{n}{length1}\PY{p}{(}\PY{p}{[}\PY{p}{]}\PY{p}{)}
\end{Verbatim}


    \begin{Verbatim}[commandchars=\\\{\}]
{\color{incolor}In [{\color{incolor}11}]:} \PY{n}{length2}\PY{p}{(}\PY{p}{[}\PY{p}{]}\PY{p}{)}
\end{Verbatim}


\begin{Verbatim}[commandchars=\\\{\}]
{\color{outcolor}Out[{\color{outcolor}11}]:} 0
\end{Verbatim}
            
    \hypertarget{les-compruxe9hensions}{%
\subsubsection{Les compréhensions}\label{les-compruxe9hensions}}

    Notez bien que par contre, les variables de compréhension \textbf{ne
fuient pas} (contrairement à ce qui se passait en Python 2)~:

    \begin{Verbatim}[commandchars=\\\{\}]
{\color{incolor}In [{\color{incolor}12}]:} \PY{c+c1}{\PYZsh{} on détruit la variable i si elle existe}
         \PY{k}{if} \PY{l+s+s1}{\PYZsq{}}\PY{l+s+s1}{i}\PY{l+s+s1}{\PYZsq{}} \PY{o+ow}{in} \PY{n+nb}{locals}\PY{p}{(}\PY{p}{)}\PY{p}{:} 
             \PY{k}{del} \PY{n}{i}   
\end{Verbatim}


    \begin{Verbatim}[commandchars=\\\{\}]
{\color{incolor}In [{\color{incolor}13}]:} \PY{c+c1}{\PYZsh{} en Python 3, les variables de compréhension ne fuitent pas}
         \PY{p}{[}\PY{n}{i}\PY{o}{*}\PY{o}{*}\PY{l+m+mi}{2} \PY{k}{for} \PY{n}{i} \PY{o+ow}{in} \PY{n+nb}{range}\PY{p}{(}\PY{l+m+mi}{3}\PY{p}{)}\PY{p}{]}
\end{Verbatim}


\begin{Verbatim}[commandchars=\\\{\}]
{\color{outcolor}Out[{\color{outcolor}13}]:} [0, 1, 4]
\end{Verbatim}
            
    \begin{Verbatim}[commandchars=\\\{\}]
{\color{incolor}In [{\color{incolor}14}]:} \PY{c+c1}{\PYZsh{} ici i est à nouveau indéfinie}
         \PY{k}{try}\PY{p}{:}
             \PY{n}{i}
         \PY{k}{except} \PY{n+ne}{NameError} \PY{k}{as} \PY{n}{e}\PY{p}{:}
             \PY{n+nb}{print}\PY{p}{(}\PY{l+s+s2}{\PYZdq{}}\PY{l+s+s2}{OOPS}\PY{l+s+s2}{\PYZdq{}}\PY{p}{,} \PY{n}{e}\PY{p}{)}
\end{Verbatim}


    \begin{Verbatim}[commandchars=\\\{\}]
OOPS name 'i' is not defined

    \end{Verbatim}


    % Add a bibliography block to the postdoc
    
    
    
