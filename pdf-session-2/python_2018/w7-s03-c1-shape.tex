    
    
    
    

    

    \hypertarget{forme-dun-tableau-numpy}{%
\section{\texorpdfstring{Forme d'un tableau
\texttt{numpy}}{Forme d'un tableau numpy}}\label{forme-dun-tableau-numpy}}

    Nous allons voir dans ce complément comment créer des tableaux en
plusieurs dimensions et manipuler la forme (\texttt{shape}) des
tableaux.

    \begin{Verbatim}[commandchars=\\\{\}]
{\color{incolor}In [{\color{incolor}1}]:} \PY{k+kn}{import} \PY{n+nn}{numpy} \PY{k}{as} \PY{n+nn}{np}
\end{Verbatim}


    \hypertarget{un-exemple}{%
\subsubsection{Un exemple}\label{un-exemple}}

    Nous avons vu précédemment comment créer un tableau \texttt{numpy} de
dimension 1 à partir d'un simple itérable, nous allons à présent créer
un tableau à 2 dimensions, et pour cela nous allons utiliser une liste
imbriquée~:

    \begin{Verbatim}[commandchars=\\\{\}]
{\color{incolor}In [{\color{incolor}2}]:} \PY{n}{d2} \PY{o}{=} \PY{n}{np}\PY{o}{.}\PY{n}{array}\PY{p}{(}\PY{p}{[}\PY{p}{[}\PY{l+m+mi}{11}\PY{p}{,} \PY{l+m+mi}{12}\PY{p}{,} \PY{l+m+mi}{13}\PY{p}{]}\PY{p}{,} \PY{p}{[}\PY{l+m+mi}{21}\PY{p}{,} \PY{l+m+mi}{22}\PY{p}{,} \PY{l+m+mi}{23}\PY{p}{]}\PY{p}{]}\PY{p}{)}
        \PY{n}{d2}
\end{Verbatim}


\begin{Verbatim}[commandchars=\\\{\}]
{\color{outcolor}Out[{\color{outcolor}2}]:} array([[11, 12, 13],
               [21, 22, 23]])
\end{Verbatim}
            
    Ce premier exemple va nous permettre de voir les différents attributs de
tous les tableaux \texttt{numpy}.

    \hypertarget{lattribut-shape}{%
\subsubsection{\texorpdfstring{L'attribut
\texttt{shape}}{L'attribut shape}}\label{lattribut-shape}}

    Tous les tableaux \texttt{numpy} possèdent un attribut \texttt{shape}
qui retourne, sous la forme d'un tuple, les dimensions du tableau~:

    \begin{Verbatim}[commandchars=\\\{\}]
{\color{incolor}In [{\color{incolor}3}]:} \PY{c+c1}{\PYZsh{} la forme (les dimensions) du tableau}
        \PY{n}{d2}\PY{o}{.}\PY{n}{shape}
\end{Verbatim}


\begin{Verbatim}[commandchars=\\\{\}]
{\color{outcolor}Out[{\color{outcolor}3}]:} (2, 3)
\end{Verbatim}
            
    Dans le cas d'un tableau en 2 dimensions, cela correspond donc à
\textbf{lignes} x \textbf{colonnes}.

    \hypertarget{on-peut-facilement-changer-de-forme}{%
\subsubsection{On peut facilement changer de
forme}\label{on-peut-facilement-changer-de-forme}}

    Comme on l'a vu dans la vidéo, un tableau est en fait une vue vers un
bloc de données. Aussi il est facile de changer la dimension d'un
tableau - ou plutôt, de créer une autre vue vers les mêmes données~:

    \begin{Verbatim}[commandchars=\\\{\}]
{\color{incolor}In [{\color{incolor}4}]:} \PY{c+c1}{\PYZsh{} l\PYZsq{}argument qu\PYZsq{}on passe à reshape est le tuple}
        \PY{c+c1}{\PYZsh{} qui décrit la nouvelle *shape*}
        \PY{n}{v2} \PY{o}{=} \PY{n}{d2}\PY{o}{.}\PY{n}{reshape}\PY{p}{(}\PY{p}{(}\PY{l+m+mi}{3}\PY{p}{,} \PY{l+m+mi}{2}\PY{p}{)}\PY{p}{)}
        \PY{n}{v2}
\end{Verbatim}


\begin{Verbatim}[commandchars=\\\{\}]
{\color{outcolor}Out[{\color{outcolor}4}]:} array([[11, 12],
               [13, 21],
               [22, 23]])
\end{Verbatim}
            
    Et donc, ces deux tableaux sont deux vues vers la même zone de données~;
ce qui fait qu'une modification sur l'un se répercute dans l'autre~:

    \begin{Verbatim}[commandchars=\\\{\}]
{\color{incolor}In [{\color{incolor}5}]:} \PY{c+c1}{\PYZsh{} je change un tableau}
        \PY{n}{d2}\PY{p}{[}\PY{l+m+mi}{0}\PY{p}{]}\PY{p}{[}\PY{l+m+mi}{0}\PY{p}{]} \PY{o}{=} \PY{l+m+mi}{100}
        \PY{n}{d2}
\end{Verbatim}


\begin{Verbatim}[commandchars=\\\{\}]
{\color{outcolor}Out[{\color{outcolor}5}]:} array([[100,  12,  13],
               [ 21,  22,  23]])
\end{Verbatim}
            
    \begin{Verbatim}[commandchars=\\\{\}]
{\color{incolor}In [{\color{incolor}6}]:} \PY{c+c1}{\PYZsh{} ça se répercute dans l\PYZsq{}autre}
        \PY{n}{v2}
\end{Verbatim}


\begin{Verbatim}[commandchars=\\\{\}]
{\color{outcolor}Out[{\color{outcolor}6}]:} array([[100,  12],
               [ 13,  21],
               [ 22,  23]])
\end{Verbatim}
            
    \hypertarget{les-attributs-liuxe9s-uxe0-la-forme}{%
\subsubsection{Les attributs liés à la
forme}\label{les-attributs-liuxe9s-uxe0-la-forme}}

    Signalons par commodité les attributs suivants, qui se dérivent de
\texttt{shape}~:

    \begin{Verbatim}[commandchars=\\\{\}]
{\color{incolor}In [{\color{incolor}7}]:} \PY{c+c1}{\PYZsh{} le nombre de dimensions}
        \PY{n}{d2}\PY{o}{.}\PY{n}{ndim}
\end{Verbatim}


\begin{Verbatim}[commandchars=\\\{\}]
{\color{outcolor}Out[{\color{outcolor}7}]:} 2
\end{Verbatim}
            
    \begin{Verbatim}[commandchars=\\\{\}]
{\color{incolor}In [{\color{incolor}8}]:} \PY{c+c1}{\PYZsh{} vrai pour tous les tableaux}
        \PY{n+nb}{len}\PY{p}{(}\PY{n}{d2}\PY{o}{.}\PY{n}{shape}\PY{p}{)} \PY{o}{==} \PY{n}{d2}\PY{o}{.}\PY{n}{ndim}
\end{Verbatim}


\begin{Verbatim}[commandchars=\\\{\}]
{\color{outcolor}Out[{\color{outcolor}8}]:} True
\end{Verbatim}
            
    \begin{Verbatim}[commandchars=\\\{\}]
{\color{incolor}In [{\color{incolor}9}]:} \PY{c+c1}{\PYZsh{} le nombre de cellules}
        \PY{n}{d2}\PY{o}{.}\PY{n}{size}
\end{Verbatim}


\begin{Verbatim}[commandchars=\\\{\}]
{\color{outcolor}Out[{\color{outcolor}9}]:} 6
\end{Verbatim}
            
    \begin{Verbatim}[commandchars=\\\{\}]
{\color{incolor}In [{\color{incolor}10}]:} \PY{c+c1}{\PYZsh{} vrai pour tous les tableaux}
         \PY{c+c1}{\PYZsh{} une façon compliquée de dire}
         \PY{c+c1}{\PYZsh{} une chose toute simple :}
         \PY{c+c1}{\PYZsh{} la taille est le produit}
         \PY{c+c1}{\PYZsh{} des dimensions}
         \PY{k+kn}{from} \PY{n+nn}{operator} \PY{k}{import} \PY{n}{mul}
         \PY{k+kn}{from} \PY{n+nn}{functools} \PY{k}{import} \PY{n}{reduce}
         \PY{n}{d2}\PY{o}{.}\PY{n}{size} \PY{o}{==} \PY{n}{reduce}\PY{p}{(}\PY{n}{mul}\PY{p}{,} \PY{n}{d2}\PY{o}{.}\PY{n}{shape}\PY{p}{,} \PY{l+m+mi}{1}\PY{p}{)}
\end{Verbatim}


\begin{Verbatim}[commandchars=\\\{\}]
{\color{outcolor}Out[{\color{outcolor}10}]:} True
\end{Verbatim}
            
    Lorsqu'on utilise \texttt{reshape}, il faut bien sûr que la nouvelle
forme soit compatible~:

    \begin{Verbatim}[commandchars=\\\{\}]
{\color{incolor}In [{\color{incolor}11}]:} \PY{k}{try}\PY{p}{:}
             \PY{n}{d2}\PY{o}{.}\PY{n}{reshape}\PY{p}{(}\PY{p}{(}\PY{l+m+mi}{3}\PY{p}{,} \PY{l+m+mi}{4}\PY{p}{)}\PY{p}{)}
         \PY{k}{except} \PY{n+ne}{Exception} \PY{k}{as} \PY{n}{e}\PY{p}{:}
             \PY{n+nb}{print}\PY{p}{(}\PY{n}{f}\PY{l+s+s2}{\PYZdq{}}\PY{l+s+s2}{OOPS }\PY{l+s+s2}{\PYZob{}}\PY{l+s+s2}{type(e)\PYZcb{} }\PY{l+s+si}{\PYZob{}e\PYZcb{}}\PY{l+s+s2}{\PYZdq{}}\PY{p}{)}
\end{Verbatim}


    \begin{Verbatim}[commandchars=\\\{\}]
OOPS <class 'ValueError'> cannot reshape array of size 6 into shape (3,4)

    \end{Verbatim}

    \hypertarget{dimensions-supuxe9rieures}{%
\subsubsection{Dimensions supérieures}\label{dimensions-supuxe9rieures}}

    Vous pouvez donc deviner comment on construit des tableaux en dimensions
supérieures à 2, il suffit d'utiliser un attribut \texttt{shape} plus
élaboré~:

    \begin{Verbatim}[commandchars=\\\{\}]
{\color{incolor}In [{\color{incolor}12}]:} \PY{n}{shape} \PY{o}{=} \PY{p}{(}\PY{l+m+mi}{2}\PY{p}{,} \PY{l+m+mi}{3}\PY{p}{,} \PY{l+m+mi}{4}\PY{p}{)}
         \PY{n}{size} \PY{o}{=} \PY{n}{reduce}\PY{p}{(}\PY{n}{mul}\PY{p}{,} \PY{n}{shape}\PY{p}{)}
         
         \PY{c+c1}{\PYZsh{} vous vous souvenez de arange}
         \PY{n}{data} \PY{o}{=} \PY{n}{np}\PY{o}{.}\PY{n}{arange}\PY{p}{(}\PY{n}{size}\PY{p}{)}
\end{Verbatim}


    \begin{Verbatim}[commandchars=\\\{\}]
{\color{incolor}In [{\color{incolor}13}]:} \PY{n}{d3} \PY{o}{=} \PY{n}{data}\PY{o}{.}\PY{n}{reshape}\PY{p}{(}\PY{n}{shape}\PY{p}{)}
         \PY{n}{d3}
\end{Verbatim}


\begin{Verbatim}[commandchars=\\\{\}]
{\color{outcolor}Out[{\color{outcolor}13}]:} array([[[ 0,  1,  2,  3],
                 [ 4,  5,  6,  7],
                 [ 8,  9, 10, 11]],
         
                [[12, 13, 14, 15],
                 [16, 17, 18, 19],
                 [20, 21, 22, 23]]])
\end{Verbatim}
            
    Cet exemple vous permet de voir qu'en dimensions supérieures la forme
est toujours~:

n1 x n2 x \ldots{} x \textbf{lignes} x \textbf{colonnes}

    Enfin, ce que je viens de dire est arbitraire, dans le sens où, bien
entendu, vous pouvez décider d'interpréter les tableaux comme vous
voulez.

Mais en termes au moins de l'impression par \texttt{print}, il est
logique de voir que l'algorithme d'impression balaye le tableau de
manière mécanique comme ceci~:

\begin{Shaded}
\begin{Highlighting}[]
\ControlFlowTok{for}\NormalTok{ i }\KeywordTok{in} \BuiltInTok{range}\NormalTok{(}\DecValTok{2}\NormalTok{):}
    \ControlFlowTok{for}\NormalTok{ j }\KeywordTok{in} \BuiltInTok{range}\NormalTok{(}\DecValTok{3}\NormalTok{):}
        \ControlFlowTok{for}\NormalTok{ k }\KeywordTok{in} \BuiltInTok{range}\NormalTok{(}\DecValTok{4}\NormalTok{):}
\NormalTok{            array[i][j][k]}
\end{Highlighting}
\end{Shaded}

    Et c'est pourquoi vous obtenez la présentation suivante avec des
tableaux de dimensions plus grandes~:

    \begin{Verbatim}[commandchars=\\\{\}]
{\color{incolor}In [{\color{incolor}14}]:} \PY{c+c1}{\PYZsh{} la même chose avec plus de dimensions}
         \PY{n}{shape} \PY{o}{=} \PY{p}{(}\PY{l+m+mi}{2}\PY{p}{,} \PY{l+m+mi}{3}\PY{p}{,} \PY{l+m+mi}{4}\PY{p}{,} \PY{l+m+mi}{5}\PY{p}{)}
         \PY{n}{size} \PY{o}{=} \PY{n}{reduce}\PY{p}{(}\PY{n}{mul}\PY{p}{,} \PY{n}{shape}\PY{p}{)} \PY{c+c1}{\PYZsh{} le produit des 4 nombres dans shape}
         \PY{n}{size}
\end{Verbatim}


\begin{Verbatim}[commandchars=\\\{\}]
{\color{outcolor}Out[{\color{outcolor}14}]:} 120
\end{Verbatim}
            
    \begin{Verbatim}[commandchars=\\\{\}]
{\color{incolor}In [{\color{incolor}15}]:} \PY{n}{data} \PY{o}{=} \PY{n}{np}\PY{o}{.}\PY{n}{arange}\PY{p}{(}\PY{n}{size}\PY{p}{)}
         
         \PY{c+c1}{\PYZsh{} ce tableau est visualisé}
         \PY{c+c1}{\PYZsh{} à base de briques de base}
         \PY{c+c1}{\PYZsh{} de 4 lignes et 5 colonnes}
         \PY{n}{d4} \PY{o}{=} \PY{n}{data}\PY{o}{.}\PY{n}{reshape}\PY{p}{(}\PY{n}{shape}\PY{p}{)}
         \PY{n}{d4}
\end{Verbatim}


\begin{Verbatim}[commandchars=\\\{\}]
{\color{outcolor}Out[{\color{outcolor}15}]:} array([[[[  0,   1,   2,   3,   4],
                  [  5,   6,   7,   8,   9],
                  [ 10,  11,  12,  13,  14],
                  [ 15,  16,  17,  18,  19]],
         
                 [[ 20,  21,  22,  23,  24],
                  [ 25,  26,  27,  28,  29],
                  [ 30,  31,  32,  33,  34],
                  [ 35,  36,  37,  38,  39]],
         
                 [[ 40,  41,  42,  43,  44],
                  [ 45,  46,  47,  48,  49],
                  [ 50,  51,  52,  53,  54],
                  [ 55,  56,  57,  58,  59]]],
         
         
                [[[ 60,  61,  62,  63,  64],
                  [ 65,  66,  67,  68,  69],
                  [ 70,  71,  72,  73,  74],
                  [ 75,  76,  77,  78,  79]],
         
                 [[ 80,  81,  82,  83,  84],
                  [ 85,  86,  87,  88,  89],
                  [ 90,  91,  92,  93,  94],
                  [ 95,  96,  97,  98,  99]],
         
                 [[100, 101, 102, 103, 104],
                  [105, 106, 107, 108, 109],
                  [110, 111, 112, 113, 114],
                  [115, 116, 117, 118, 119]]]])
\end{Verbatim}
            
    Vous voyez donc qu'avec la forme~:

\begin{verbatim}
2, 3, 4, 5
\end{verbatim}

cela vous donne l'impression que vous avez comme brique de base des
tableaux qui ont~:

\begin{verbatim}
4 lignes
5 colonnes
\end{verbatim}

    Et souvenez-vous que vous pouvez toujours insérer un \texttt{1}
n'importe où dans la forme, puisque ça ne modifie pas la taille qui est
le produit des dimensions~:

    \begin{Verbatim}[commandchars=\\\{\}]
{\color{incolor}In [{\color{incolor}16}]:} \PY{n}{d2}\PY{o}{.}\PY{n}{shape}
\end{Verbatim}


\begin{Verbatim}[commandchars=\\\{\}]
{\color{outcolor}Out[{\color{outcolor}16}]:} (2, 3)
\end{Verbatim}
            
    \begin{Verbatim}[commandchars=\\\{\}]
{\color{incolor}In [{\color{incolor}17}]:} \PY{n}{d2}
\end{Verbatim}


\begin{Verbatim}[commandchars=\\\{\}]
{\color{outcolor}Out[{\color{outcolor}17}]:} array([[100,  12,  13],
                [ 21,  22,  23]])
\end{Verbatim}
            
    \begin{Verbatim}[commandchars=\\\{\}]
{\color{incolor}In [{\color{incolor}18}]:} \PY{n}{d2}\PY{o}{.}\PY{n}{reshape}\PY{p}{(}\PY{l+m+mi}{2}\PY{p}{,} \PY{l+m+mi}{1}\PY{p}{,} \PY{l+m+mi}{3}\PY{p}{)}
\end{Verbatim}


\begin{Verbatim}[commandchars=\\\{\}]
{\color{outcolor}Out[{\color{outcolor}18}]:} array([[[100,  12,  13]],
         
                [[ 21,  22,  23]]])
\end{Verbatim}
            
    \begin{Verbatim}[commandchars=\\\{\}]
{\color{incolor}In [{\color{incolor}19}]:} \PY{n}{d2}\PY{o}{.}\PY{n}{reshape}\PY{p}{(}\PY{l+m+mi}{2}\PY{p}{,} \PY{l+m+mi}{3}\PY{p}{,} \PY{l+m+mi}{1}\PY{p}{)}
\end{Verbatim}


\begin{Verbatim}[commandchars=\\\{\}]
{\color{outcolor}Out[{\color{outcolor}19}]:} array([[[100],
                 [ 12],
                 [ 13]],
         
                [[ 21],
                 [ 22],
                 [ 23]]])
\end{Verbatim}
            
    Ou même~:

    \begin{Verbatim}[commandchars=\\\{\}]
{\color{incolor}In [{\color{incolor}20}]:} \PY{n}{d2}\PY{o}{.}\PY{n}{reshape}\PY{p}{(}\PY{p}{(}\PY{l+m+mi}{1}\PY{p}{,} \PY{l+m+mi}{2}\PY{p}{,} \PY{l+m+mi}{3}\PY{p}{)}\PY{p}{)}
\end{Verbatim}


\begin{Verbatim}[commandchars=\\\{\}]
{\color{outcolor}Out[{\color{outcolor}20}]:} array([[[100,  12,  13],
                 [ 21,  22,  23]]])
\end{Verbatim}
            
    \begin{Verbatim}[commandchars=\\\{\}]
{\color{incolor}In [{\color{incolor}21}]:} \PY{n}{d2}\PY{o}{.}\PY{n}{reshape}\PY{p}{(}\PY{p}{(}\PY{l+m+mi}{1}\PY{p}{,} \PY{l+m+mi}{1}\PY{p}{,} \PY{l+m+mi}{1}\PY{p}{,} \PY{l+m+mi}{1}\PY{p}{,} \PY{l+m+mi}{2}\PY{p}{,} \PY{l+m+mi}{3}\PY{p}{)}\PY{p}{)}
\end{Verbatim}


\begin{Verbatim}[commandchars=\\\{\}]
{\color{outcolor}Out[{\color{outcolor}21}]:} array([[[[[[100,  12,  13],
                    [ 21,  22,  23]]]]]])
\end{Verbatim}
            
    \hypertarget{ruxe9sumuxe9-des-attributs}{%
\subsubsection{Résumé des attributs}\label{ruxe9sumuxe9-des-attributs}}

    Voici un résumé des attributs des tableaux \texttt{numpy}~:

    \begin{longtable}[]{@{}lll@{}}
\toprule
\emph{attribut} & \emph{signification} & \emph{exemple}\tabularnewline
\midrule
\endhead
\texttt{shape} & tuple des dimensions &
\texttt{(3,\ 5,\ 7)}\tabularnewline
\texttt{ndim} & nombre dimensions & \texttt{3}\tabularnewline
\texttt{size} & nombre d'éléments &
\texttt{3\ *\ 5\ *\ 7}\tabularnewline
\texttt{dtype} & type de chaque élément &
\texttt{np.float64}\tabularnewline
\texttt{itemsize} & taille en octets d'un élément &
\texttt{8}\tabularnewline
\bottomrule
\end{longtable}

    \hypertarget{divers}{%
\subsubsection{Divers}\label{divers}}

    Je vous signale enfin, à titre totalement anecdotique cette fois,
l'existence de la méthode \texttt{ravel} qui vous permet d'aplatir
n'importe quel tableau~:

    \begin{Verbatim}[commandchars=\\\{\}]
{\color{incolor}In [{\color{incolor}22}]:} \PY{n}{d2}
\end{Verbatim}


\begin{Verbatim}[commandchars=\\\{\}]
{\color{outcolor}Out[{\color{outcolor}22}]:} array([[100,  12,  13],
                [ 21,  22,  23]])
\end{Verbatim}
            
    \begin{Verbatim}[commandchars=\\\{\}]
{\color{incolor}In [{\color{incolor}23}]:} \PY{n}{d2}\PY{o}{.}\PY{n}{ravel}\PY{p}{(}\PY{p}{)}
\end{Verbatim}


\begin{Verbatim}[commandchars=\\\{\}]
{\color{outcolor}Out[{\color{outcolor}23}]:} array([100,  12,  13,  21,  22,  23])
\end{Verbatim}
            
    \begin{Verbatim}[commandchars=\\\{\}]
{\color{incolor}In [{\color{incolor}24}]:} \PY{c+c1}{\PYZsh{} il y a d\PYZsq{}ailleurs aussi flatten qui fait}
         \PY{c+c1}{\PYZsh{} quelque chose de semblable}
         \PY{n}{d2}\PY{o}{.}\PY{n}{flatten}\PY{p}{(}\PY{p}{)}
\end{Verbatim}


\begin{Verbatim}[commandchars=\\\{\}]
{\color{outcolor}Out[{\color{outcolor}24}]:} array([100,  12,  13,  21,  22,  23])
\end{Verbatim}
            

    % Add a bibliography block to the postdoc
    
    
    
