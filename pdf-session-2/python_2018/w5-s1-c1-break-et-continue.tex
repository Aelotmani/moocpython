    
    
    
    

    

    \hypertarget{les-instructions-break-et-continue}{%
\section{\texorpdfstring{Les instructions \texttt{break} et
\texttt{continue}}{Les instructions break et continue}}\label{les-instructions-break-et-continue}}

    \hypertarget{compluxe9ment---niveau-basique}{%
\subsection{Complément - niveau
basique}\label{compluxe9ment---niveau-basique}}

    \hypertarget{break-et-continue}{%
\subsubsection{\texorpdfstring{\texttt{break} et
\texttt{continue}}{break et continue}}\label{break-et-continue}}

    En guise de rappel de ces deux notions que nous avons déjà rencontrées
dans la séquence consacrée aux boucles \texttt{while} la semaine passée,
python propose deux instructions très pratiques permettant de contrôler
l'exécution à l'intérieur des boucles de répétition, et ceci s'applique
indifféremment aux boucles \texttt{for} ou \texttt{while}~:

\begin{itemize}
\tightlist
\item
  \texttt{continue}~: pour abandonner l'itération courante, et passer à
  la suivante, en \textbf{restant dans la boucle}~;
\item
  \texttt{break}~: pour abandonner \textbf{complètement} la boucle.
\end{itemize}

Voici un exemple simple d'utilisation de ces deux instructions~:

    \begin{Verbatim}[commandchars=\\\{\},frame=single,framerule=0.3mm,rulecolor=\color{cellframecolor}]
{\color{incolor}In [{\color{incolor}1}]:} \PY{k}{for} \PY{n}{entier} \PY{o+ow}{in} \PY{n+nb}{range}\PY{p}{(}\PY{l+m+mi}{1000}\PY{p}{)}\PY{p}{:}
            \PY{c+c1}{\PYZsh{} on ignore les nombres non multiples de 10}
            \PY{k}{if} \PY{n}{entier} \PY{o}{\PYZpc{}} \PY{l+m+mi}{10} \PY{o}{!=} \PY{l+m+mi}{0}\PY{p}{:}
                \PY{k}{continue}
            \PY{n+nb}{print}\PY{p}{(}\PY{n}{f}\PY{l+s+s2}{\PYZdq{}}\PY{l+s+s2}{on traite l}\PY{l+s+s2}{\PYZsq{}}\PY{l+s+s2}{entier }\PY{l+s+si}{\PYZob{}entier\PYZcb{}}\PY{l+s+s2}{\PYZdq{}}\PY{p}{)}
            \PY{c+c1}{\PYZsh{} on s\PYZsq{}arrête à 50}
            \PY{k}{if} \PY{n}{entier} \PY{o}{\PYZgt{}}\PY{o}{=} \PY{l+m+mi}{50}\PY{p}{:}
                \PY{k}{break}
        \PY{n+nb}{print}\PY{p}{(}\PY{l+s+s2}{\PYZdq{}}\PY{l+s+s2}{on est sorti de la boucle}\PY{l+s+s2}{\PYZdq{}}\PY{p}{)}
\end{Verbatim}


    \begin{Verbatim}[commandchars=\\\{\},frame=single,framerule=0.3mm,rulecolor=\color{cellframecolor}]
on traite l'entier 0
on traite l'entier 10
on traite l'entier 20
on traite l'entier 30
on traite l'entier 40
on traite l'entier 50
on est sorti de la boucle
\end{Verbatim}

    Pour aller plus loin, vous pouvez lire
\href{https://docs.python.org/3/tutorial/controlflow.html?highlight=break\#break-and-continue-statements-and-else-clauses-on-loops}{cette
documentation}.


    % Add a bibliography block to the postdoc
    
    
    
