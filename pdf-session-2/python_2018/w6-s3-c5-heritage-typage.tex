    \hypertarget{huxe9ritage-typage}{%
\section{Héritage, typage}\label{huxe9ritage-typage}}

    \hypertarget{compluxe9ment---niveau-avancuxe9}{%
\subsection{Complément - niveau
avancé}\label{compluxe9ment---niveau-avancuxe9}}

    Dans ce complément, nous allons revenir sur la notion de \emph{duck
typing}, et attirer votre attention sur cette différence assez
essentielle entre python et les langages statiquement typés. On
s'adresse ici principalement à ceux d'entre vous qui sont habitués à C++
et/ou Java.

    \hypertarget{type-concret-et-type-abstrait}{%
\subsubsection{Type concret et type
abstrait}\label{type-concret-et-type-abstrait}}

    Revenons sur la notion de type et remarquons que les types peuvent jouer
plusieurs rôles, comme on l'a évoqué rapidement en première semaine~; et
pour reprendre des notions standard en langages de programmation nous
allons distinguer deux types.

\begin{enumerate}
\def\labelenumi{\arabic{enumi}.}
\tightlist
\item
  \textbf{type concret~:} d'une part, la notion de type a bien entendu à
  voir avec l'implémentation~; par exemple, un compilateur C a besoin de
  savoir très précisément quel espace allouer à une variable, et
  l'interpréteur python sous-traite à la classe le soin d'initialiser un
  objet~;
\item
  \textbf{type abstrait~:} d'autre part, les types sont cruciaux dans
  les systèmes de vérification statique, au sens large, dont le but est
  de trouver un maximum de défauts à la seule lecture du code (par
  opposition aux techniques qui nécessitent de le faire tourner).
\end{enumerate}

    \hypertarget{duck-typing}{%
\subsubsection{\texorpdfstring{\emph{Duck
typing}}{Duck typing}}\label{duck-typing}}

    En python, ces deux aspects du typage sont relativement décorrélés.\\

    Pour la seconde dimension du typage, le système de types abstraits de
python est connu sous le nom de
\href{http://en.wikipedia.org/wiki/Duck_typing}{\emph{duck typing}}, une
appellation qui fait référence à cette phrase~:

\begin{verbatim}
When I see a bird that walks like a duck and swims like a duck and quacks like
a duck, I call that bird a duck.
\end{verbatim}

    \hypertarget{lexemple-des-ituxe9rables}{%
\subsubsection{L'exemple des
itérables}\label{lexemple-des-ituxe9rables}}

    Pour prendre l'exemple sans doute le plus représentatif, la notion
d'\emph{itérable} est un type abstrait, en ce sens que, pour que le
fragment~:

\begin{verbatim}
for item in container:
    do_something(item)
    
\end{verbatim}

ait un sens, il faut et il suffit que \texttt{container} soit un
itérable. Et vous connaissez maintenant plein d'exemples très différents
d'objets itérables, a minima parmi les \emph{built-in} \texttt{str},
\texttt{list}, \texttt{tuple}, \texttt{range}\ldots{}\\

    Dans un langage typé statiquement, pour pouvoir donner un type à cette
construction, on serait \textbf{obligé} de définir un type - qu'on
appellerait logiquement une classe abstraite - dont ces trois types
seraient des descendants.\\

    En python, et c'est le point que nous voulons souligner dans ce
complément, il n'existe pas dans le système python d'objet de type
\texttt{type} qui matérialise l'ensemble des \texttt{iterable}s. Si on
regarde les superclasses de nos types concrets itérables, on voit que
leur seul ancêtre commun est la classe \texttt{object}~:

    \begin{Verbatim}[commandchars=\\\{\}]
{\color{incolor}In [{\color{incolor}1}]:} \PY{n+nb}{str}\PY{o}{.}\PY{n+nv+vm}{\PYZus{}\PYZus{}bases\PYZus{}\PYZus{}}
\end{Verbatim}


\begin{Verbatim}[commandchars=\\\{\}]
{\color{outcolor}Out[{\color{outcolor}1}]:} (object,)
\end{Verbatim}
            
    \begin{Verbatim}[commandchars=\\\{\}]
{\color{incolor}In [{\color{incolor}2}]:} \PY{n+nb}{list}\PY{o}{.}\PY{n+nv+vm}{\PYZus{}\PYZus{}bases\PYZus{}\PYZus{}}
\end{Verbatim}


\begin{Verbatim}[commandchars=\\\{\}]
{\color{outcolor}Out[{\color{outcolor}2}]:} (object,)
\end{Verbatim}
            
    \begin{Verbatim}[commandchars=\\\{\}]
{\color{incolor}In [{\color{incolor}3}]:} \PY{n+nb}{tuple}\PY{o}{.}\PY{n+nv+vm}{\PYZus{}\PYZus{}bases\PYZus{}\PYZus{}}
\end{Verbatim}


\begin{Verbatim}[commandchars=\\\{\}]
{\color{outcolor}Out[{\color{outcolor}3}]:} (object,)
\end{Verbatim}
            
    \begin{Verbatim}[commandchars=\\\{\}]
{\color{incolor}In [{\color{incolor}4}]:} \PY{n+nb}{range}\PY{o}{.}\PY{n+nv+vm}{\PYZus{}\PYZus{}bases\PYZus{}\PYZus{}}
\end{Verbatim}


\begin{Verbatim}[commandchars=\\\{\}]
{\color{outcolor}Out[{\color{outcolor}4}]:} (object,)
\end{Verbatim}
            
    \hypertarget{un-autre-exemple}{%
\subsubsection{Un autre exemple}\label{un-autre-exemple}}

    Pour prendre un exemple plus simple, si je considère~:

\begin{verbatim}
def foo(graphic):
    ...
    graphic.draw()
\end{verbatim}

pour que l'expression \texttt{graphic.draw()} ait un sens, il faut et il
suffit que l'objet \texttt{graphic} ait une méthode \texttt{draw}.\\

    À nouveau, dans un langage typé statiquement, on serait amené à définir
une classe abstraite \texttt{Graphic}. En python ce n'est \textbf{pas
requis}~; vous pouvez utiliser ce code tel quel avec deux classes
\texttt{Rectangle} et \texttt{Texte} qui n'ont pas de rapport entre
elles - autres que, à nouveau, d'avoir \texttt{object} comme ancêtre
commun - pourvu qu'elles aient toutes les deux une méthode
\texttt{draw}.

    \hypertarget{huxe9ritage-et-type-abstrait}{%
\subsubsection{Héritage et type
abstrait}\label{huxe9ritage-et-type-abstrait}}

    Pour résumer, en python comme dans les langages typés statiquement, on a
bien entendu la bonne propriété que si, par exemple, la classe
\texttt{Spam} est itérable, alors la classe \texttt{Eggs} qui hérite de
\texttt{Spam} est itérable.\\

Mais dans l'autre sens, si \texttt{Foo} et \texttt{Bar} sont itérables,
il n'y a pas forcément une superclasse commune qui représente l'ensemble
des objets itérables.

    \hypertarget{isinstance-sur-stuxe9rouxefdes}{%
\subsubsection{\texorpdfstring{\texttt{isinstance} sur
stéroïdes}{isinstance sur stéroïdes}}\label{isinstance-sur-stuxe9rouxefdes}}

    D'un autre côté, c'est très utile d'exposer au programmeur un moyen de
vérifier si un objet a un \emph{type} donné - dans un sens
volontairement vague ici.\\

On a déjà parlé en Semaine 4 de l'intérêt qu'il peut y avoir à tester le
type d'un argument avec \texttt{isinstance} dans une fonction, pour
parvenir à faire l'équivalent de la surcharge en C++ (la surcharge en
C++, c'est quand vous définissez plusieurs fonctions qui ont le même nom
mais des types d'arguments différents).\\

C'est pourquoi, quand on a cherché à exposer au programmeur des
propriétés comme ``cet objet est-il iterable~?'', on a choisi d'étendre
\emph{isinstance} au travers de
\href{http://legacy.python.org/dev/peps/pep-3119/}{cette initiative}.
C'est ainsi qu'on peut faire par exemple~:

    \begin{Verbatim}[commandchars=\\\{\}]
{\color{incolor}In [{\color{incolor}5}]:} \PY{k+kn}{from} \PY{n+nn}{collections}\PY{n+nn}{.}\PY{n+nn}{abc} \PY{k}{import} \PY{n}{Iterable}
\end{Verbatim}


    \begin{Verbatim}[commandchars=\\\{\}]
{\color{incolor}In [{\color{incolor}6}]:} \PY{n+nb}{isinstance}\PY{p}{(}\PY{l+s+s1}{\PYZsq{}}\PY{l+s+s1}{ab}\PY{l+s+s1}{\PYZsq{}}\PY{p}{,} \PY{n}{Iterable}\PY{p}{)}
\end{Verbatim}


\begin{Verbatim}[commandchars=\\\{\}]
{\color{outcolor}Out[{\color{outcolor}6}]:} True
\end{Verbatim}
            
    \begin{Verbatim}[commandchars=\\\{\}]
{\color{incolor}In [{\color{incolor}7}]:} \PY{n+nb}{isinstance}\PY{p}{(}\PY{p}{[}\PY{l+m+mi}{1}\PY{p}{,} \PY{l+m+mi}{2}\PY{p}{]}\PY{p}{,} \PY{n}{Iterable}\PY{p}{)}
\end{Verbatim}


\begin{Verbatim}[commandchars=\\\{\}]
{\color{outcolor}Out[{\color{outcolor}7}]:} True
\end{Verbatim}
            
    \begin{Verbatim}[commandchars=\\\{\}]
{\color{incolor}In [{\color{incolor}8}]:} \PY{c+c1}{\PYZsh{} comme on l\PYZsq{}a vu, un objet qui a des méthodes}
        \PY{c+c1}{\PYZsh{} \PYZus{}\PYZus{}iter\PYZus{}\PYZus{}() et \PYZus{}\PYZus{}next\PYZus{}\PYZus{}() }
        \PY{c+c1}{\PYZsh{} est considéré comme un itérable}
        \PY{k}{class} \PY{n+nc}{Foo}\PY{p}{:}
            \PY{k}{def} \PY{n+nf}{\PYZus{}\PYZus{}iter\PYZus{}\PYZus{}}\PY{p}{(}\PY{n+nb+bp}{self}\PY{p}{)}\PY{p}{:}
                \PY{k}{return} \PY{n+nb+bp}{self}
            \PY{k}{def} \PY{n+nf}{\PYZus{}\PYZus{}next\PYZus{}\PYZus{}}\PY{p}{(}\PY{n+nb+bp}{self}\PY{p}{)}\PY{p}{:}
                \PY{c+c1}{\PYZsh{} ceci naturellement est bidon}
                \PY{k}{return} 
\end{Verbatim}


    \begin{Verbatim}[commandchars=\\\{\}]
{\color{incolor}In [{\color{incolor}9}]:} \PY{n}{foo} \PY{o}{=} \PY{n}{Foo}\PY{p}{(}\PY{p}{)}
        \PY{n+nb}{isinstance}\PY{p}{(}\PY{n}{foo}\PY{p}{,} \PY{n}{Iterable}\PY{p}{)}
\end{Verbatim}


\begin{Verbatim}[commandchars=\\\{\}]
{\color{outcolor}Out[{\color{outcolor}9}]:} True
\end{Verbatim}
            
    L'implémentation du module \texttt{abc} donne l'\textbf{illusion} que
\texttt{Iterable} est un objet dans la hiérarchie de classes, et que
tous ces \emph{classes} \texttt{str}, \texttt{list}, et \texttt{Foo} lui
sont asujetties, mais ce n'est pas le cas en réalité~; comme on l'a vu
plus tôt, ces trois types ne sont pas comparables dans la hiérarchie de
classes, ils n'ont pas de plus petit (ou plus grand) élément à part
\texttt{object}.\\

    Je signale pour finir, à propos de \texttt{isinstance} et du module
\texttt{collections}, que la définition du symbole \texttt{Hashable} est
à mon avis beaucoup moins convaincante que \texttt{Iterable}~; si vous
vous souvenez qu'en Semaine 3, Séquence ``les dictionnaires'', on avait
vu que les clés doivent être globalement immuables. C'est une
caractéristique qui est assez difficile à écrire, et en tous cas ceci de
mon point de vue ne remplit pas la fonction~:

    \begin{Verbatim}[commandchars=\\\{\}]
{\color{incolor}In [{\color{incolor}10}]:} \PY{k+kn}{from} \PY{n+nn}{collections}\PY{n+nn}{.}\PY{n+nn}{abc} \PY{k}{import} \PY{n}{Hashable}
\end{Verbatim}


    \begin{Verbatim}[commandchars=\\\{\}]
{\color{incolor}In [{\color{incolor}11}]:} \PY{c+c1}{\PYZsh{} un tuple qui contient une liste ne convient }
         \PY{c+c1}{\PYZsh{} pas comme clé dans un dictionnaire}
         \PY{c+c1}{\PYZsh{} et pourtant}
         \PY{n+nb}{isinstance} \PY{p}{(}\PY{p}{(}\PY{p}{[}\PY{l+m+mi}{1}\PY{p}{]}\PY{p}{,} \PY{p}{[}\PY{l+m+mi}{2}\PY{p}{]}\PY{p}{)}\PY{p}{,} \PY{n}{Hashable}\PY{p}{)}
\end{Verbatim}


\begin{Verbatim}[commandchars=\\\{\}]
{\color{outcolor}Out[{\color{outcolor}11}]:} True
\end{Verbatim}
            
    \hypertarget{python-et-les-classes-abstraites}{%
\subsubsection{python et les classes
abstraites}\label{python-et-les-classes-abstraites}}

    Les points à retenir de ce complément un peu digressif sont~:

\begin{itemize}
\tightlist
\item
  en python, on hérite des \textbf{implémentations} et pas des
  \textbf{spécifications}~;
\item
  et le langage n'est pas taillé pour tirer profit de \textbf{classes
  abstraites} - même si rien ne vous interdit d'écrire, pour des raisons
  documentaires, une classe qui résume l'interface qui est attendue par
  tel ou tel système de plugin.
\end{itemize}

    Venant de C++ ou de Java, cela peut prendre du temps d'arriver à se
débarrasser de l'espèce de réflexe qui fait qu'on pense d'abord classe
abstraite, puis implémentations.

    \hypertarget{pour-aller-plus-loin}{%
\subsubsection{Pour aller plus loin}\label{pour-aller-plus-loin}}

    La
\href{https://docs.python.org/3/library/collections.abc.html}{documentation
du module \texttt{collections.abc}} contient la liste de tous les
symboles exposés par ce module, dont par exemple en vrac~:

\begin{itemize}
\tightlist
\item
  \texttt{Iterable}
\item
  \texttt{Iterator}
\item
  \texttt{Hashable}
\item
  \texttt{Generator}
\item
  \texttt{Coroutine} (rendez-vous semaine 8)
\end{itemize}

et de nombreux autres.

    \hypertarget{avertissement}{%
\subsubsection{Avertissement}\label{avertissement}}

    Prenez garde enfin que ces symboles n'ont - à ce stade du moins - pas de
relation forte avec ceux
\href{https://docs.python.org/3/library/typing.html}{du module
\texttt{typing}} dont on a parlé lorsqu'on a vu les \emph{type hints}.