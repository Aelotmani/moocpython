    
    
    
    

    

    \hypertarget{outils-puxe9riphuxe9riques}{%
\section{Outils périphériques}\label{outils-puxe9riphuxe9riques}}

    \hypertarget{compluxe9ments---niveau-intermuxe9diaire}{%
\subsection{Compléments - niveau
intermédiaire}\label{compluxe9ments---niveau-intermuxe9diaire}}

    Pour conclure le tronc commun de ce cours Python, nous allons très
rapidement citer quelques outils qui ne sont pas nécessairement dans la
bibliothèque standard, mais qui sont très largement utilisés dans
l'écosystème python.

Il s'agit d'une liste non exhaustive bien entendu.

    \hypertarget{distribution-et-packaging}{%
\subsubsection{Distribution et
packaging}\label{distribution-et-packaging}}

    On l'a rapidement mentionné, il existe une infrastructure globale pour
la distribution de librairies écrites en python. Celle-ci repose sur

\begin{itemize}
\tightlist
\item
  un site web \url{https://pypi.org/} où l'on peut consulter les - très
  nombreuses - bibliothèques diffusées, avec leurs historiques et
  révisions,
\item
  un outil en ligne de commande \texttt{pip}, pour installer et mettre à
  jour ces bibliothèques (utiliser \texttt{pip3} comme \texttt{python3}
  si vous avez python2 installé en parallèle),
\item
  et enfin un système de packaging à destination des éditeurs.
\end{itemize}

Je vous signale, par rapport à ce dernier point, que la bibliothèque
standard vient avec un outil qui s'appelle \texttt{distutils}, qui est
essentiellement obsolète, et qui n'est conservé que pour des questions
de compatibilité. Si vous devez commencer depuis une feuille blanche le
packaging d'une nouvelle librairie, je vous recommande d'utiliser plutôt
\texttt{setuptools} qui est le standard de fait dans le domaine.

    Dans une domaine très voisin, l'outil \texttt{virtualenv} est très
populaire également ; il permet de créer sur une seul machine, plusieurs
environnements python avec des versions et contenus différents.

C'est très utile si vous travaillez sur plusieurs projets, dont l'un a
besoin de python-3.5 avec numpy et sans pandas, et Django-1.11, et un
second avec python-3.6 sans numpy et avec Django-2.0.

    Pour finir, on ne peut pas parler de packaging sans citer
\texttt{conda}, l'outil de référence pour la packaging et les
environnements virtuels en data science. Quelques références sur conda :

\begin{itemize}
\tightlist
\item
  une documentation complète de conda \url{https://conda.io/docs/}
\item
  une excellente discussion (en anglais) sur le positionnement de
  \texttt{pip} et \texttt{conda}
  \url{http://jakevdp.github.io/blog/2016/08/25/conda-myths-and-misconceptions/}
\end{itemize}

    \hypertarget{debugging}{%
\subsubsection{Debugging}\label{debugging}}

    Pour le debugging, la bibliothèque standard s'appelle \texttt{pdb}.
Typiquement pour mettre un \emph{breakpoint} on écrit~:

\begin{Shaded}
\begin{Highlighting}[]
\KeywordTok{def}\NormalTok{ foo(n):}
\NormalTok{    n }\OperatorTok{=}\NormalTok{ n }\OperatorTok{**} \DecValTok{2}
    \CommentTok{# pour mettre un point d'arrêt}
    \ImportTok{import}\NormalTok{ pdb}
\NormalTok{    pdb.set_trace()}
    \CommentTok{# la suite de foo()}
    \ControlFlowTok{return}\NormalTok{ n }\OperatorTok{/} \DecValTok{10}
\end{Highlighting}
\end{Shaded}

    Je vous signale d'ailleurs qu'à partir de Python 3.7, il est recommandé
d'utiliser la nouvelle fonction \emph{built-in} \emph{breakpoint()} qui
rend le même service.

    \hypertarget{tests}{%
\subsubsection{Tests}\label{tests}}

    Le module \texttt{unittest} de la bibliothèque standard fournit des
fonctionnalités de base pour écrire des tests unitaires.

Je vous signale par ailleurs des outils comme \texttt{nosetests} et
\texttt{pytest}, qui ne sont pas dans la distribution standard, mais qui
enrichissent les capacités de \texttt{unittest} pour en rendre
l'utilisation quotidienne plus fluide.

    \hypertarget{documentation}{%
\subsubsection{Documentation}\label{documentation}}

    Le standard de fait dans ce domaine est clairement une combinaison basée
sur

\begin{itemize}
\tightlist
\item
  l'outil \texttt{sphinx}, qui permet de générer la documentation à
  partir du source, avec

  \begin{itemize}
  \tightlist
  \item
    des plugins pour divers sous-formats dans les docstrings,
  \item
    un système de templating,
  \item
    et de nombreuses autres possibilités ;
  \end{itemize}
\item
  \texttt{readthedocs.io} qui est une plateforme ouverte pour
  l'hébergement des documentations ; elle-même facilement intégrable
  avec un repository type \texttt{github.io},
\end{itemize}

Pour vous donner une idée du résultat, je vous invite à consulter un
module de ma facture~:

\begin{itemize}
\tightlist
\item
  les sources sur github sur
  \url{https://github.com/parmentelat/asynciojobs}, et notamment le
  sous-répertoire \texttt{sphinx},
\item
  et la documentation en ligne sur
  \url{http://asynciojobs.readthedocs.io/}.
\end{itemize}

    \hypertarget{linter}{%
\subsubsection{Linter}\label{linter}}

    Au delà de la vérification automatique de la présentation du code
(PEP8), il existe un outil \texttt{pylint} qui fait de l'analyse de code
source en vue de détecter des erreurs le plus tôt possible dans le cycle
de développement.

En quelques mots, ma recommandation à ce sujet est que~:

\begin{itemize}
\tightlist
\item
  tout d'abord, et comme dans tous les langages en fait, il est
  \textbf{très utile} de faire passer systématiquement son code dans un
  linter de ce genre ;
\item
  idéalement on ne devrait commiter que du code qui passe cette étape ;
\item
  cependant, il y a un petit travail de filtrage à faire au démarrage,
  car pylint détecte plusieurs centaines de sortes d'erreurs, du coup il
  convient de passer un moment à configurer l'outil pour qu'il en ignore
  certaines.
\end{itemize}

    Dès que vous commencez à travailler sur des projets sérieux, vous devez
utiliser un éditeur qui intègre et exécute automatiquement
\texttt{pylint}. On peut notamment recommander
\href{https://www.jetbrains.com/pycharm/}{PyCharm}.

    \hypertarget{type-hints}{%
\subsubsection{Type hints}\label{type-hints}}

    Je voudrais citer enfin l'outil \texttt{mypy} qui est un complément
crucial dans la mise en oeuvre des \emph{type hints}.

Comme on l'a vu en Semaine 4 dans la séquence consacrée aux type hints,
et en tous cas jusque Python-3.6, les annotations de typage que vous
insérez éventuellement dans votre code sont complètement ignorées de
l'interpréteur.

Elles sont par contre analysées par l'outil \texttt{mypy} qui fournit
une sorte de couche supplémentaire de \emph{linter} et permet de
détecter, ici encore, les éventuelles erreurs qui peuvent résulter
notamment d'une mauvaise utilisation de telle ou telle librairie.

    \hypertarget{conclusion}{%
\subsubsection{Conclusion}\label{conclusion}}

    À nouveau cette liste n'est pas exhaustive, elle s'efforce simplement de
guider vos premiers pas dans cet écosystème.

Je vous invite à creuser de votre coté les différents aspects qui, parmi
cette liste, vous semblent les plus intéressants pour votre usage.


    % Add a bibliography block to the postdoc
    
    
    
