    
    
    
    

    

    \hypertarget{ituxe9rateurs}{%
\section{Itérateurs}\label{ituxe9rateurs}}

    \hypertarget{compluxe9ment---niveau-intermuxe9diaire}{%
\subsection{Complément - niveau
intermédiaire}\label{compluxe9ment---niveau-intermuxe9diaire}}

    Dans ce complément nous allons dire quelques mots du module
\texttt{itertools} qui fournit sous forme d'itérateurs des utilitaires
communs qui peuvent être très utiles. On vous rappelle que l'intérêt
premier des itérateurs est de parcourir des données sans créer de
structure de données temporaire, donc à coût mémoire faible et constant.

    \hypertarget{le-module-itertools}{%
\subsubsection{\texorpdfstring{Le module
\texttt{itertools}}{Le module itertools}}\label{le-module-itertools}}

    À ce stade, j'espère que vous savez trouver
\href{https://docs.python.org/3/library/itertools.html}{la documentation
du module} que je vous invite à avoir sous la main.

    \begin{Verbatim}[commandchars=\\\{\},frame=single,framerule=0.3mm,rulecolor=\color{cellframecolor}]
{\color{incolor}In [{\color{incolor}1}]:} \PY{k+kn}{import} \PY{n+nn}{itertools}
\end{Verbatim}


    Comme vous le voyez dans la doc, les fonctionnalités de
\texttt{itertools} tombent dans 3 catégories~:

\begin{itemize}
\tightlist
\item
  des itérateurs infinis, comme par exemple \texttt{cycle}~;
\item
  des itérateurs pour énumérer les combinatoires usuelles en
  mathématiques, comme les permutations, les combinaisons, le produit
  cartésien, etc.~;
\item
  et enfin des itérateurs correspondants à des traits que nous avons
  déjà rencontrés, mais implémentés sous forme d'itérateurs.
\end{itemize}

À nouveau, toutes ces fonctionnalités sont offertes \textbf{sous la
forme d'itérateurs}.

    Pour détailler un tout petit peu cette dernière famille, signalons~:

    \begin{itemize}
\tightlist
\item
  \texttt{chain} qui permet de \textbf{concaténer} plusieurs itérables
  sous la forme d'un \textbf{itérateur}~:
\end{itemize}

    \begin{Verbatim}[commandchars=\\\{\},frame=single,framerule=0.3mm,rulecolor=\color{cellframecolor}]
{\color{incolor}In [{\color{incolor}2}]:} \PY{k}{for} \PY{n}{x} \PY{o+ow}{in} \PY{n}{itertools}\PY{o}{.}\PY{n}{chain}\PY{p}{(}\PY{p}{(}\PY{l+m+mi}{1}\PY{p}{,} \PY{l+m+mi}{2}\PY{p}{)}\PY{p}{,} \PY{p}{[}\PY{l+m+mi}{3}\PY{p}{,} \PY{l+m+mi}{4}\PY{p}{]}\PY{p}{)}\PY{p}{:}
            \PY{n+nb}{print}\PY{p}{(}\PY{n}{x}\PY{p}{)}
\end{Verbatim}


    \begin{Verbatim}[commandchars=\\\{\},frame=single,framerule=0.3mm,rulecolor=\color{cellframecolor}]
1
2
3
4
\end{Verbatim}

    \begin{itemize}
\tightlist
\item
  \texttt{islice} qui fournit un itérateur sur un slice d'un itérable.
  On peut le voir comme une généralisation de \texttt{range} qui
  parcourt n'importe quel itérable.
\end{itemize}

    \begin{Verbatim}[commandchars=\\\{\},frame=single,framerule=0.3mm,rulecolor=\color{cellframecolor}]
{\color{incolor}In [{\color{incolor}3}]:} \PY{k+kn}{import} \PY{n+nn}{string}
        \PY{n}{support} \PY{o}{=} \PY{n}{string}\PY{o}{.}\PY{n}{ascii\PYZus{}lowercase}
        \PY{n+nb}{print}\PY{p}{(}\PY{n}{f}\PY{l+s+s1}{\PYZsq{}}\PY{l+s+s1}{support=}\PY{l+s+si}{\PYZob{}support\PYZcb{}}\PY{l+s+s1}{\PYZsq{}}\PY{p}{)}
\end{Verbatim}


    \begin{Verbatim}[commandchars=\\\{\},frame=single,framerule=0.3mm,rulecolor=\color{cellframecolor}]
support=abcdefghijklmnopqrstuvwxyz
\end{Verbatim}

    \begin{Verbatim}[commandchars=\\\{\},frame=single,framerule=0.3mm,rulecolor=\color{cellframecolor}]
{\color{incolor}In [{\color{incolor}4}]:} \PY{c+c1}{\PYZsh{} range}
        \PY{k}{for} \PY{n}{x} \PY{o+ow}{in} \PY{n+nb}{range}\PY{p}{(}\PY{l+m+mi}{3}\PY{p}{,} \PY{l+m+mi}{8}\PY{p}{)}\PY{p}{:}
            \PY{n+nb}{print}\PY{p}{(}\PY{n}{x}\PY{p}{)}
\end{Verbatim}


    \begin{Verbatim}[commandchars=\\\{\},frame=single,framerule=0.3mm,rulecolor=\color{cellframecolor}]
3
4
5
6
7
\end{Verbatim}

    \begin{Verbatim}[commandchars=\\\{\},frame=single,framerule=0.3mm,rulecolor=\color{cellframecolor}]
{\color{incolor}In [{\color{incolor}5}]:} \PY{c+c1}{\PYZsh{} islice}
        \PY{k}{for} \PY{n}{x} \PY{o+ow}{in} \PY{n}{itertools}\PY{o}{.}\PY{n}{islice}\PY{p}{(}\PY{n}{support}\PY{p}{,} \PY{l+m+mi}{3}\PY{p}{,} \PY{l+m+mi}{8}\PY{p}{)}\PY{p}{:}
            \PY{n+nb}{print}\PY{p}{(}\PY{n}{x}\PY{p}{)}
\end{Verbatim}


    \begin{Verbatim}[commandchars=\\\{\},frame=single,framerule=0.3mm,rulecolor=\color{cellframecolor}]
d
e
f
g
h
\end{Verbatim}


    % Add a bibliography block to the postdoc
    
    
    
