    
    
    
    

    

    \hypertarget{une-limite-de-la-boucle-for}{%
\section{\texorpdfstring{Une limite de la boucle
\texttt{for}}{Une limite de la boucle for}}\label{une-limite-de-la-boucle-for}}

    \hypertarget{compluxe9ment---niveau-basique}{%
\subsection{Complément - niveau
basique}\label{compluxe9ment---niveau-basique}}

    Pour ceux qui veulent suivre le cours au niveau basique, retenez
seulement que dans une boucle \texttt{for} sur un objet mutable,
\textbf{il ne faut pas modifier le sujet} de la boucle.

Ainsi par exemple il ne \textbf{faut pas faire} quelque chose comme
ceci~:

    \begin{Verbatim}[commandchars=\\\{\}]
{\color{incolor}In [{\color{incolor}1}]:} \PY{c+c1}{\PYZsh{} on veut enlever de l\PYZsq{}ensemble toutes les chaînes }
        \PY{c+c1}{\PYZsh{} qui ne contiennent pas \PYZsq{}bert\PYZsq{}}
        \PY{n}{ensemble} \PY{o}{=} \PY{p}{\PYZob{}}\PY{l+s+s1}{\PYZsq{}}\PY{l+s+s1}{marc}\PY{l+s+s1}{\PYZsq{}}\PY{p}{,} \PY{l+s+s1}{\PYZsq{}}\PY{l+s+s1}{albert}\PY{l+s+s1}{\PYZsq{}}\PY{p}{\PYZcb{}}
\end{Verbatim}


    \begin{Verbatim}[commandchars=\\\{\}]
{\color{incolor}In [{\color{incolor} }]:} \PY{c+c1}{\PYZsh{} NOTE}
        \PY{c+c1}{\PYZsh{} automatic execution has skipped execution of this cell}
        
        \PY{c+c1}{\PYZsh{} ceci semble une bonne idée mais ne fonctionne pas}
        \PY{c+c1}{\PYZsh{} provoque RuntimeError: Set changed size during iteration}
        
        \PY{k}{for} \PY{n}{valeur} \PY{o+ow}{in} \PY{n}{ensemble}\PY{p}{:}
            \PY{k}{if} \PY{l+s+s1}{\PYZsq{}}\PY{l+s+s1}{bert}\PY{l+s+s1}{\PYZsq{}} \PY{o+ow}{not} \PY{o+ow}{in} \PY{n}{valeur}\PY{p}{:}
                \PY{n}{ensemble}\PY{o}{.}\PY{n}{discard}\PY{p}{(}\PY{n}{valeur}\PY{p}{)}
\end{Verbatim}


    \hypertarget{comment-faire-alors}{%
\subsubsection{Comment faire alors~?}\label{comment-faire-alors}}

    Première remarque, votre premier réflexe pourrait être de penser à une
compréhension d'ensemble~:

    \begin{Verbatim}[commandchars=\\\{\}]
{\color{incolor}In [{\color{incolor}2}]:} \PY{n}{ensemble2} \PY{o}{=} \PY{p}{\PYZob{}}\PY{n}{valeur} \PY{k}{for} \PY{n}{valeur} \PY{o+ow}{in} \PY{n}{ensemble} \PY{k}{if} \PY{l+s+s1}{\PYZsq{}}\PY{l+s+s1}{bert}\PY{l+s+s1}{\PYZsq{}} \PY{o+ow}{in} \PY{n}{valeur}\PY{p}{\PYZcb{}}
        \PY{n}{ensemble2}
\end{Verbatim}


\begin{Verbatim}[commandchars=\\\{\}]
{\color{outcolor}Out[{\color{outcolor}2}]:} \{'albert'\}
\end{Verbatim}
            
    C'est sans doute la meilleure solution. Par contre, évidemment, on n'a
pas modifié l'objet ensemble initial, on a créé un nouvel objet. En
supposant que l'on veuille modifier l'objet initial, il nous faut faire
la boucle sur une \emph{shallow copy} de cet objet. Notez qu'ici, il ne
s'agit d'économiser de la mémoire, puisque l'on fait une \emph{shallow
copy}.

    \begin{Verbatim}[commandchars=\\\{\}]
{\color{incolor}In [{\color{incolor}3}]:} \PY{k+kn}{from} \PY{n+nn}{copy} \PY{k}{import} \PY{n}{copy}
        \PY{c+c1}{\PYZsh{} on veut enlever de l\PYZsq{}ensemble toutes les chaînes }
        \PY{c+c1}{\PYZsh{} qui ne contiennent pas \PYZsq{}bert\PYZsq{}}
        \PY{n}{ensemble} \PY{o}{=} \PY{p}{\PYZob{}}\PY{l+s+s1}{\PYZsq{}}\PY{l+s+s1}{marc}\PY{l+s+s1}{\PYZsq{}}\PY{p}{,} \PY{l+s+s1}{\PYZsq{}}\PY{l+s+s1}{albert}\PY{l+s+s1}{\PYZsq{}}\PY{p}{\PYZcb{}}
        
        \PY{c+c1}{\PYZsh{} si on fait d\PYZsq{}abord une copie tout va bien}
        \PY{k}{for} \PY{n}{valeur} \PY{o+ow}{in} \PY{n}{copy}\PY{p}{(}\PY{n}{ensemble}\PY{p}{)}\PY{p}{:}
            \PY{k}{if} \PY{l+s+s1}{\PYZsq{}}\PY{l+s+s1}{bert}\PY{l+s+s1}{\PYZsq{}} \PY{o+ow}{not} \PY{o+ow}{in} \PY{n}{valeur}\PY{p}{:}
                \PY{n}{ensemble}\PY{o}{.}\PY{n}{discard}\PY{p}{(}\PY{n}{valeur}\PY{p}{)}
                
        \PY{n+nb}{print}\PY{p}{(}\PY{n}{ensemble}\PY{p}{)}
\end{Verbatim}


    \begin{Verbatim}[commandchars=\\\{\}]
\{'albert'\}

    \end{Verbatim}

    \hypertarget{avertissement}{%
\subsubsection{Avertissement}\label{avertissement}}

    Dans l'exemple ci-dessus, on voit que l'interpréteur se rend compte que
l'on est en train de modifier l'objet de la boucle, et nous le signifie.

Ne vous fiez pas forcément à cet exemple, il existe des cas -- nous en
verrons plus loin dans ce document -- où l'interpréteur peut accepter
votre code alors qu'il n'obéit pas à cette règle, et du coup
essentiellement se mettre à faire n'importe quoi.

    \hypertarget{pruxe9cisons-bien-la-limite}{%
\subsubsection{Précisons bien la
limite}\label{pruxe9cisons-bien-la-limite}}

    Pour être tout à fait clair, lorsqu'on dit qu'il ne faut pas modifier
l'objet de la boucle \texttt{for}, il ne s'agit que du premier niveau.

On ne doit pas modifier la \textbf{composition de l'objet en tant
qu'itérable}, mais on peut sans souci modifier chacun des objets qui
constitue l'itération.

Ainsi cette construction par contre est tout à fait valide~:

    \begin{Verbatim}[commandchars=\\\{\}]
{\color{incolor}In [{\color{incolor}4}]:} \PY{n}{liste} \PY{o}{=} \PY{p}{[}\PY{p}{[}\PY{l+m+mi}{1}\PY{p}{]}\PY{p}{,} \PY{p}{[}\PY{l+m+mi}{2}\PY{p}{]}\PY{p}{,} \PY{p}{[}\PY{l+m+mi}{3}\PY{p}{]}\PY{p}{]}
        \PY{n+nb}{print}\PY{p}{(}\PY{l+s+s1}{\PYZsq{}}\PY{l+s+s1}{avant}\PY{l+s+s1}{\PYZsq{}}\PY{p}{,} \PY{n}{liste}\PY{p}{)}
\end{Verbatim}


    \begin{Verbatim}[commandchars=\\\{\}]
avant [[1], [2], [3]]

    \end{Verbatim}

    \begin{Verbatim}[commandchars=\\\{\}]
{\color{incolor}In [{\color{incolor}5}]:} \PY{k}{for} \PY{n}{sous\PYZus{}liste} \PY{o+ow}{in} \PY{n}{liste}\PY{p}{:}
            \PY{n}{sous\PYZus{}liste}\PY{o}{.}\PY{n}{append}\PY{p}{(}\PY{l+m+mi}{100}\PY{p}{)}
        \PY{n+nb}{print}\PY{p}{(}\PY{l+s+s1}{\PYZsq{}}\PY{l+s+s1}{après}\PY{l+s+s1}{\PYZsq{}}\PY{p}{,} \PY{n}{liste}\PY{p}{)}
\end{Verbatim}


    \begin{Verbatim}[commandchars=\\\{\}]
après [[1, 100], [2, 100], [3, 100]]

    \end{Verbatim}

    Dans cet exemple, les modifications ont lieu sur les éléments de
\texttt{liste}, et non sur l'objet \texttt{liste} lui-même, c'est donc
tout à fait légal.

    \hypertarget{compluxe9ment---niveau-intermuxe9diaire}{%
\subsection{Complément - niveau
intermédiaire}\label{compluxe9ment---niveau-intermuxe9diaire}}

    Pour bien comprendre la nature de cette limitation, il faut bien voir
que cela soulève deux types de problèmes distincts.

    \hypertarget{difficultuxe9-dordre-suxe9mantique}{%
\subsubsection{Difficulté d'ordre
sémantique}\label{difficultuxe9-dordre-suxe9mantique}}

    D'un point de vue sémantique, si l'on voulait autoriser ce genre de
choses, il faudrait définir très précisément le comportement attendu.

Considérons par exemple la situation d'une liste qui a 10 éléments, sur
laquelle on ferait une boucle et que, par exemple au 5ème élément, on
enlève le 8ème élément. Quel serait le comportement attendu dans ce
cas~? Faut-il ou non que la boucle envisage alors le 8-ème élément~?

La situation serait encore pire pour les dictionnaires et ensembles pour
lesquels l'ordre de parcours n'est pas spécifié~; ainsi on pourrait
écrire du code totalement indéterministe si le parcours d'un ensemble
essayait~:

\begin{itemize}
\tightlist
\item
  d'enlever l'élément \emph{b} lorsqu'on parcourt l'élément \emph{a}~;
\item
  d'enlever l'élément \emph{a} lorsqu'on parcourt l'élément \emph{b}.
\end{itemize}

On le voit, il n'est déjà pas très simple d'expliciter sans ambiguïté le
comportement attendu d'une boucle \texttt{for} qui serait autorisée à
modifier son propre sujet.

    \hypertarget{difficultuxe9-dimpluxe9mentation}{%
\subsubsection{Difficulté
d'implémentation}\label{difficultuxe9-dimpluxe9mentation}}

    Voyons maintenant un exemple de code qui ne respecte pas la règle, et
qui modifie le sujet de la boucle en lui ajoutant des valeurs

    \begin{verbatim}
# cette boucle ne termine pas
liste = [1, 2, 3]
for c in liste:
    if c == 3:
        liste.append(c)
\end{verbatim}

    Nous avons volontairement mis ce code \textbf{dans une cellule de texte}
et non de code~: vous \textbf{ne pouvez pas l'exécuter} dans le
notebook. Si vous essayez de l'exécuter sur votre ordinateur vous
constaterez que la boucle ne termine pas : en fait à chaque itération on
ajoute un nouvel élément dans la liste, et du coup la boucle a un
élément de plus à balayer~; ce programme ne termine théoriquement
jamais. En pratique, ce sera le cas quand votre système n'aura plus de
mémoire disponible (sauvegardez vos documents avant d'essayer !).


    % Add a bibliography block to the postdoc
    
    
    
