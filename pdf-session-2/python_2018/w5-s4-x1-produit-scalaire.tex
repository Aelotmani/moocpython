    
    
    
    

    

    \hypertarget{les-boucles-for}{%
\section{\texorpdfstring{Les boucles
\texttt{for}}{Les boucles for}}\label{les-boucles-for}}

    \begin{center}\rule{0.5\linewidth}{\linethickness}\end{center}

    \hypertarget{exercice---niveau-intermuxe9diaire}{%
\subsection{Exercice - niveau
intermédiaire}\label{exercice---niveau-intermuxe9diaire}}

    \hypertarget{produit-scalaire}{%
\subsubsection{Produit scalaire}\label{produit-scalaire}}

    \begin{Verbatim}[commandchars=\\\{\}]
{\color{incolor}In [{\color{incolor}1}]:} \PY{c+c1}{\PYZsh{} Pour charger l\PYZsq{}exercice}
        \PY{k+kn}{from} \PY{n+nn}{corrections}\PY{n+nn}{.}\PY{n+nn}{exo\PYZus{}produit\PYZus{}scalaire} \PY{k}{import} \PY{n}{exo\PYZus{}produit\PYZus{}scalaire}
\end{Verbatim}


    On veut écrire une fonction qui retourne le produit scalaire de deux
vecteurs. Pour ceci on va matérialiser les deux vecteurs en entrée par
deux listes que l'on suppose de même taille.

On rappelle que le produit de X et Y vaut~\(\sum_{i} X_i * Y_i\).

On posera que le produit scalaire de deux listes vides vaut \texttt{0}.

    Naturellement puisque le sujet de la séquence est les expressions
génératrices, on vous demande d'utiliser ce trait pour résoudre cet
exercice.

\textbf{NOTE} remarquez bien qu'on a dit \textbf{expression} génératrice
et pas nécessairement \textbf{fonction génératrice}.

    \begin{Verbatim}[commandchars=\\\{\}]
{\color{incolor}In [{\color{incolor}2}]:} \PY{c+c1}{\PYZsh{} un petit exemple}
        \PY{n}{exo\PYZus{}produit\PYZus{}scalaire}\PY{o}{.}\PY{n}{example}\PY{p}{(}\PY{p}{)}
\end{Verbatim}


\begin{Verbatim}[commandchars=\\\{\}]
{\color{outcolor}Out[{\color{outcolor}2}]:} <IPython.core.display.HTML object>
\end{Verbatim}
            
    Vous devez donc écrire~:

    \begin{Verbatim}[commandchars=\\\{\}]
{\color{incolor}In [{\color{incolor}3}]:} \PY{k}{def} \PY{n+nf}{produit\PYZus{}scalaire}\PY{p}{(}\PY{n}{X}\PY{p}{,} \PY{n}{Y}\PY{p}{)}\PY{p}{:} 
            \PY{l+s+sd}{\PYZdq{}\PYZdq{}\PYZdq{}retourne le produit scalaire de deux listes de même taille\PYZdq{}\PYZdq{}\PYZdq{}}
            \PY{l+s+s2}{\PYZdq{}}\PY{l+s+s2}{\PYZlt{}votre\PYZus{}code\PYZgt{}}\PY{l+s+s2}{\PYZdq{}}
\end{Verbatim}


    \begin{Verbatim}[commandchars=\\\{\}]
{\color{incolor}In [{\color{incolor} }]:} \PY{c+c1}{\PYZsh{} NOTE}
        \PY{c+c1}{\PYZsh{} auto\PYZhy{}exec\PYZhy{}for\PYZhy{}latex has skipped execution of this cell}
        
        \PY{c+c1}{\PYZsh{} pour vérifier votre code}
        \PY{n}{exo\PYZus{}produit\PYZus{}scalaire}\PY{o}{.}\PY{n}{correction}\PY{p}{(}\PY{n}{produit\PYZus{}scalaire}\PY{p}{)}
\end{Verbatim}



    % Add a bibliography block to the postdoc
    
    
    
