    
    
    
    

    

    \hypertarget{linstruction-if}{%
\section{\texorpdfstring{L'instruction
\texttt{if}}{L'instruction if}}\label{linstruction-if}}

    \hypertarget{exercice---niveau-basique}{%
\subsection{Exercice - niveau basique}\label{exercice---niveau-basique}}

    \hypertarget{ruxe9partiteur-1}{%
\subsubsection{Répartiteur (1)}\label{ruxe9partiteur-1}}

    \begin{Verbatim}[commandchars=\\\{\}]
{\color{incolor}In [{\color{incolor}1}]:} \PY{c+c1}{\PYZsh{} on charge l\PYZsq{}exercice}
        \PY{k+kn}{from} \PY{n+nn}{corrections}\PY{n+nn}{.}\PY{n+nn}{exo\PYZus{}dispatch} \PY{k}{import} \PY{n}{exo\PYZus{}dispatch1}
\end{Verbatim}


    On vous demande d'écrire une fonction \texttt{dispatch1}, qui prend en
argument deux entiers \texttt{a} et \texttt{b}, et qui renvoie selon les
cas~:

\[
\begin{array}{c|c|c}
\ & a\  pair & a\ impair \\
\hline
b\ pair & a^2+b^2 & (a-1)*b\\
\hline
b\ impair & a*(b-1)& a^2-b^2\\
\end{array}
\]

    \begin{Verbatim}[commandchars=\\\{\}]
{\color{incolor}In [{\color{incolor}2}]:} \PY{c+c1}{\PYZsh{} un petit exemple}
        \PY{n}{exo\PYZus{}dispatch1}\PY{o}{.}\PY{n}{example}\PY{p}{(}\PY{p}{)}
\end{Verbatim}


\begin{Verbatim}[commandchars=\\\{\}]
{\color{outcolor}Out[{\color{outcolor}2}]:} <IPython.core.display.HTML object>
\end{Verbatim}
            
    \begin{Verbatim}[commandchars=\\\{\}]
{\color{incolor}In [{\color{incolor}3}]:} \PY{k}{def} \PY{n+nf}{dispatch1}\PY{p}{(}\PY{n}{a}\PY{p}{,} \PY{n}{b}\PY{p}{)}\PY{p}{:}
            \PY{l+s+s2}{\PYZdq{}}\PY{l+s+s2}{\PYZlt{}votre\PYZus{}code\PYZgt{}}\PY{l+s+s2}{\PYZdq{}}
\end{Verbatim}


    \begin{Verbatim}[commandchars=\\\{\}]
{\color{incolor}In [{\color{incolor} }]:} \PY{c+c1}{\PYZsh{} NOTE}
        \PY{c+c1}{\PYZsh{} auto\PYZhy{}exec\PYZhy{}for\PYZhy{}latex has skipped execution of this cell}
        
        \PY{c+c1}{\PYZsh{} pour vérifier votre code}
        \PY{n}{exo\PYZus{}dispatch1}\PY{o}{.}\PY{n}{correction}\PY{p}{(}\PY{n}{dispatch1}\PY{p}{)}
\end{Verbatim}


    \hypertarget{exercice---niveau-basique}{%
\subsection{Exercice - niveau basique}\label{exercice---niveau-basique}}

    \hypertarget{ruxe9partiteur-2}{%
\subsubsection{Répartiteur (2)}\label{ruxe9partiteur-2}}

    \begin{Verbatim}[commandchars=\\\{\}]
{\color{incolor}In [{\color{incolor}4}]:} \PY{c+c1}{\PYZsh{} chargement de l\PYZsq{}exercice}
        \PY{k+kn}{from} \PY{n+nn}{corrections}\PY{n+nn}{.}\PY{n+nn}{exo\PYZus{}dispatch} \PY{k}{import} \PY{n}{exo\PYZus{}dispatch2}
\end{Verbatim}


    Dans une seconde version de cet exercice, on vous demande d'écrire une
fonction \texttt{dispatch2} qui prend en arguments~:

\begin{itemize}
\tightlist
\item
  \texttt{a} et \texttt{b} deux entiers
\item
  \texttt{A} et \texttt{B} deux ensembles (chacun pouvant être
  matérialisé par un ensemble, une liste ou un tuple)
\end{itemize}

et qui renvoie selon les cas~:

\[
\begin{array}{c|c|c}
\ & a \in A & a\notin A \\
\hline
b\in B & a^2+b^2 & (a-1)*b\\
\hline
b\notin B & a*(b-1)& a^2+b^2\\
\end{array}
\]

    \begin{Verbatim}[commandchars=\\\{\}]
{\color{incolor}In [{\color{incolor}5}]:} \PY{k}{def} \PY{n+nf}{dispatch2}\PY{p}{(}\PY{n}{a}\PY{p}{,} \PY{n}{b}\PY{p}{,} \PY{n}{A}\PY{p}{,} \PY{n}{B}\PY{p}{)}\PY{p}{:}
            \PY{l+s+s2}{\PYZdq{}}\PY{l+s+s2}{\PYZlt{}votre\PYZus{}code\PYZgt{}}\PY{l+s+s2}{\PYZdq{}}
\end{Verbatim}


    \begin{Verbatim}[commandchars=\\\{\}]
{\color{incolor}In [{\color{incolor} }]:} \PY{c+c1}{\PYZsh{} NOTE}
        \PY{c+c1}{\PYZsh{} auto\PYZhy{}exec\PYZhy{}for\PYZhy{}latex has skipped execution of this cell}
        
        \PY{c+c1}{\PYZsh{} pour vérifier votre code}
        \PY{n}{exo\PYZus{}dispatch2}\PY{o}{.}\PY{n}{correction}\PY{p}{(}\PY{n}{dispatch2}\PY{p}{)}
\end{Verbatim}



    % Add a bibliography block to the postdoc
    
    
    
