    
    
    
    

    

    \hypertarget{les-diffuxe9rentes-copies}{%
\section{Les différentes copies}\label{les-diffuxe9rentes-copies}}

    \begin{Verbatim}[commandchars=\\\{\}]
{\color{incolor}In [{\color{incolor}1}]:} \PY{o}{\PYZpc{}}\PY{k}{load\PYZus{}ext} ipythontutor
\end{Verbatim}


    \hypertarget{compluxe9ment---niveau-basique}{%
\subsection{Complément - niveau
basique}\label{compluxe9ment---niveau-basique}}

    \hypertarget{deux-types-de-copie}{%
\subsubsection{Deux types de copie}\label{deux-types-de-copie}}

    Pour résumer les deux grands types de copie que l'on a vus dans la
vidéo~:

\begin{itemize}
\tightlist
\item
  La \emph{shallow copy} - de l'anglais \emph{shallow} qui signifie
  superficiel~;
\item
  La \emph{deep copy} - de \emph{deep} qui signifie profond.
\end{itemize}

    \hypertarget{le-module-copy}{%
\subsubsection{\texorpdfstring{Le module
\texttt{copy}}{Le module copy}}\label{le-module-copy}}

    Pour réaliser une copie, la méthode la plus simple, en ceci qu'elle
fonctionne avec tous les types de manière identique, consiste à utiliser
\href{https://docs.python.org/3/library/copy.html}{le module standard
\texttt{copy}}, et notamment~:

\begin{itemize}
\tightlist
\item
  \texttt{copy.copy} pour une copie superficielle~;
\item
  \texttt{copy.deepcopy} pour une copie en profondeur.
\end{itemize}

    \begin{Verbatim}[commandchars=\\\{\}]
{\color{incolor}In [{\color{incolor}2}]:} \PY{k+kn}{import} \PY{n+nn}{copy}
        \PY{c+c1}{\PYZsh{}help(copy.copy)}
        \PY{c+c1}{\PYZsh{}help(copy.deepcopy)}
\end{Verbatim}


    \hypertarget{un-exemple}{%
\subsubsection{Un exemple}\label{un-exemple}}

    Nous allons voir le résultat des deux formes de copie sur un même sujet
de départ.

    \hypertarget{la-copie-superficielle-shallow-copie-copy.copy}{%
\paragraph{\texorpdfstring{La copie superficielle / \emph{shallow} copie
/
\texttt{copy.copy}}{La copie superficielle / shallow copie / copy.copy}}\label{la-copie-superficielle-shallow-copie-copy.copy}}

    N'oubliez pas de cliquer le bouton \texttt{Forward} dans la fenêtre
pythontutor~:

    \begin{Verbatim}[commandchars=\\\{\}]
{\color{incolor}In [{\color{incolor} }]:} \PY{c+c1}{\PYZsh{} NOTE}
        \PY{c+c1}{\PYZsh{} automatic execution has skipped execution of this cell}
        
        \PY{o}{\PYZpc{}\PYZpc{}}\PY{k}{ipythontutor} height=410 curInstr=6
        import copy
        \PYZsh{} On se donne un objet de départ
        source = [
            [1, 2, 3],  \PYZsh{} une liste
            \PYZob{}1, 2, 3\PYZcb{},  \PYZsh{} un ensemble
            (1, 2, 3),  \PYZsh{} un tuple
            \PYZsq{}123\PYZsq{},       \PYZsh{} un string
            123,         \PYZsh{} un entier
        ]
        \PYZsh{} une copie simple renvoie ceci
        shallow\PYZus{}copy = copy.copy(source)
\end{Verbatim}


    Vous remarquez que~:

\begin{itemize}
\tightlist
\item
  la source et la copie partagent tous leurs (sous-)éléments, et
  notamment la liste \texttt{source{[}0{]}} et l'ensemble
  \texttt{source{[}1{]}}~;
\item
  ainsi, après cette copie, on peut modifier l'un de ces deux objets (la
  liste ou l'ensemble), et ainsi modifier la source \textbf{et} la
  copie.
\end{itemize}

    On rappelle aussi que, la source étant une liste, on aurait pu aussi
bien faire la copie superficielle avec

\begin{Shaded}
\begin{Highlighting}[]
\NormalTok{shallow2 }\OperatorTok{=}\NormalTok{ source[:]}
\end{Highlighting}
\end{Shaded}

    \hypertarget{la-copie-profonde-deep-copie-copy.deepcopy}{%
\paragraph{\texorpdfstring{La copie profonde / \emph{deep} copie /
\texttt{copy.deepcopy}}{La copie profonde / deep copie / copy.deepcopy}}\label{la-copie-profonde-deep-copie-copy.deepcopy}}

    Sur le même objet de départ, voici ce que fait la copie profonde~:

    \begin{Verbatim}[commandchars=\\\{\}]
{\color{incolor}In [{\color{incolor} }]:} \PY{c+c1}{\PYZsh{} NOTE}
        \PY{c+c1}{\PYZsh{} automatic execution has skipped execution of this cell}
        
        \PY{o}{\PYZpc{}\PYZpc{}}\PY{k}{ipythontutor} height=410 curInstr=6
        import copy
        \PYZsh{} On se donne un objet de départ
        source = [
            [1, 2, 3],  \PYZsh{} une liste
            \PYZob{}1, 2, 3\PYZcb{},  \PYZsh{} un ensemble
            (1, 2, 3),  \PYZsh{} un tuple
            \PYZsq{}123\PYZsq{},       \PYZsh{} un string
            123,         \PYZsh{} un entier
        ]
        \PYZsh{} une copie profonde renvoie ceci
        deep\PYZus{}copy = copy.deepcopy(source)
\end{Verbatim}


    Ici, il faut remarquer que~:

\begin{itemize}
\tightlist
\item
  les deux objets mutables accessibles via \texttt{source}, c'est-à-dire
  \textbf{la liste} \texttt{source{[}0{]}} et \textbf{l'ensemble
  \texttt{source{[}1{]}}}, ont été tous deux dupliqués~;
\item
  \textbf{le tuple} correspondant à \texttt{source{[}2{]}} n'est
  \textbf{pas dupliqué}, mais comme il n'est \textbf{pas mutable} on ne
  peut pas modifier la copie au travers de la source~;
\item
  de manière générale, on a la bonne propriété que la source et sa copie
  ne partagent rien qui soit modifiable~;
\item
  et donc on ne peut pas modifier l'un au travers de l'autre.
\end{itemize}

On retrouve donc à nouveau l'optimisation qui est mise en place dans
python pour implémenter les types immuables comme des singletons lorsque
c'est possible. Cela a été vu en détail dans le complément consacré à
l'opérateur \texttt{is}.

    \hypertarget{compluxe9ment---niveau-intermuxe9diaire}{%
\subsection{Complément - niveau
intermédiaire}\label{compluxe9ment---niveau-intermuxe9diaire}}

    \begin{Verbatim}[commandchars=\\\{\}]
{\color{incolor}In [{\color{incolor}3}]:} \PY{c+c1}{\PYZsh{} on répète car le code précédent a seulement été exposé à pythontutor}
        \PY{k+kn}{import} \PY{n+nn}{copy}
        \PY{n}{source} \PY{o}{=} \PY{p}{[}
            \PY{p}{[}\PY{l+m+mi}{1}\PY{p}{,} \PY{l+m+mi}{2}\PY{p}{,} \PY{l+m+mi}{3}\PY{p}{]}\PY{p}{,}  \PY{c+c1}{\PYZsh{} une liste}
            \PY{p}{\PYZob{}}\PY{l+m+mi}{1}\PY{p}{,} \PY{l+m+mi}{2}\PY{p}{,} \PY{l+m+mi}{3}\PY{p}{\PYZcb{}}\PY{p}{,}  \PY{c+c1}{\PYZsh{} un ensemble}
            \PY{p}{(}\PY{l+m+mi}{1}\PY{p}{,} \PY{l+m+mi}{2}\PY{p}{,} \PY{l+m+mi}{3}\PY{p}{)}\PY{p}{,}  \PY{c+c1}{\PYZsh{} un tuple}
            \PY{l+s+s1}{\PYZsq{}}\PY{l+s+s1}{123}\PY{l+s+s1}{\PYZsq{}}\PY{p}{,}       \PY{c+c1}{\PYZsh{} un string}
            \PY{l+m+mi}{123}\PY{p}{,}         \PY{c+c1}{\PYZsh{} un entier}
        \PY{p}{]}
        \PY{n}{shallow\PYZus{}copy} \PY{o}{=} \PY{n}{copy}\PY{o}{.}\PY{n}{copy}\PY{p}{(}\PY{n}{source}\PY{p}{)}
        \PY{n}{deep\PYZus{}copy} \PY{o}{=} \PY{n}{copy}\PY{o}{.}\PY{n}{deepcopy}\PY{p}{(}\PY{n}{source}\PY{p}{)}
\end{Verbatim}


    \hypertarget{objets-uxe9gaux-au-sens-logique}{%
\subsubsection{\texorpdfstring{Objets \emph{égaux} au sens
logique}{Objets égaux au sens logique}}\label{objets-uxe9gaux-au-sens-logique}}

    Bien sûr ces trois objets se ressemblent si on fait une comparaison
\emph{logique} avec \texttt{==}~:

    \begin{Verbatim}[commandchars=\\\{\}]
{\color{incolor}In [{\color{incolor}4}]:} \PY{n+nb}{print}\PY{p}{(}\PY{l+s+s1}{\PYZsq{}}\PY{l+s+s1}{source == shallow\PYZus{}copy:}\PY{l+s+s1}{\PYZsq{}}\PY{p}{,} \PY{n}{source} \PY{o}{==} \PY{n}{shallow\PYZus{}copy}\PY{p}{)}
        \PY{n+nb}{print}\PY{p}{(}\PY{l+s+s1}{\PYZsq{}}\PY{l+s+s1}{source == deep\PYZus{}copy:}\PY{l+s+s1}{\PYZsq{}}\PY{p}{,} \PY{n}{source} \PY{o}{==} \PY{n}{deep\PYZus{}copy}\PY{p}{)}
\end{Verbatim}


    \begin{Verbatim}[commandchars=\\\{\}]
source == shallow\_copy: True
source == deep\_copy: True

    \end{Verbatim}

    \hypertarget{inspectons-les-objets-de-premier-niveau}{%
\subsubsection{Inspectons les objets de premier
niveau}\label{inspectons-les-objets-de-premier-niveau}}

    Mais par contre si on compare \textbf{l'identité} des objets de premier
niveau, on voit que \texttt{source} et \texttt{shallow\_copy} partagent
leurs objets~:

    \begin{Verbatim}[commandchars=\\\{\}]
{\color{incolor}In [{\color{incolor}5}]:} \PY{c+c1}{\PYZsh{} voir la cellule ci\PYZhy{}dessous si ceci vous parait peu clair}
        \PY{k}{for} \PY{n}{i}\PY{p}{,} \PY{p}{(}\PY{n}{source\PYZus{}item}\PY{p}{,} \PY{n}{copy\PYZus{}item}\PY{p}{)} \PY{o+ow}{in} \PY{n+nb}{enumerate}\PY{p}{(}\PY{n+nb}{zip}\PY{p}{(}\PY{n}{source}\PY{p}{,} \PY{n}{shallow\PYZus{}copy}\PY{p}{)}\PY{p}{)}\PY{p}{:}
            \PY{n}{compare} \PY{o}{=} \PY{n}{source\PYZus{}item} \PY{o+ow}{is} \PY{n}{copy\PYZus{}item}
            \PY{n+nb}{print}\PY{p}{(}\PY{n}{f}\PY{l+s+s2}{\PYZdq{}}\PY{l+s+s2}{source[}\PY{l+s+si}{\PYZob{}i\PYZcb{}}\PY{l+s+s2}{] is shallow\PYZus{}copy[}\PY{l+s+si}{\PYZob{}i\PYZcb{}}\PY{l+s+s2}{] \PYZhy{}\PYZgt{} }\PY{l+s+si}{\PYZob{}compare\PYZcb{}}\PY{l+s+s2}{\PYZdq{}}\PY{p}{)}
\end{Verbatim}


    \begin{Verbatim}[commandchars=\\\{\}]
source[0] is shallow\_copy[0] -> True
source[1] is shallow\_copy[1] -> True
source[2] is shallow\_copy[2] -> True
source[3] is shallow\_copy[3] -> True
source[4] is shallow\_copy[4] -> True

    \end{Verbatim}

    \begin{Verbatim}[commandchars=\\\{\}]
{\color{incolor}In [{\color{incolor}6}]:} \PY{c+c1}{\PYZsh{} rappel au sujet de zip et enumerate}
        \PY{c+c1}{\PYZsh{} la cellule ci\PYZhy{}dessous est essentiellement équivalente à}
        \PY{k}{for} \PY{n}{i} \PY{o+ow}{in} \PY{n+nb}{range}\PY{p}{(}\PY{n+nb}{len}\PY{p}{(}\PY{n}{source}\PY{p}{)}\PY{p}{)}\PY{p}{:}
            \PY{n}{compare} \PY{o}{=} \PY{n}{source}\PY{p}{[}\PY{n}{i}\PY{p}{]} \PY{o+ow}{is} \PY{n}{shallow\PYZus{}copy}\PY{p}{[}\PY{n}{i}\PY{p}{]}
            \PY{n+nb}{print}\PY{p}{(}\PY{n}{f}\PY{l+s+s2}{\PYZdq{}}\PY{l+s+s2}{source[}\PY{l+s+si}{\PYZob{}i\PYZcb{}}\PY{l+s+s2}{] is shallow\PYZus{}copy[}\PY{l+s+si}{\PYZob{}i\PYZcb{}}\PY{l+s+s2}{] \PYZhy{}\PYZgt{} }\PY{l+s+si}{\PYZob{}compare\PYZcb{}}\PY{l+s+s2}{\PYZdq{}}\PY{p}{)}
\end{Verbatim}


    \begin{Verbatim}[commandchars=\\\{\}]
source[0] is shallow\_copy[0] -> True
source[1] is shallow\_copy[1] -> True
source[2] is shallow\_copy[2] -> True
source[3] is shallow\_copy[3] -> True
source[4] is shallow\_copy[4] -> True

    \end{Verbatim}

    Alors que naturellement ce \textbf{n'est pas le cas} avec la copie en
profondeur~:

    \begin{Verbatim}[commandchars=\\\{\}]
{\color{incolor}In [{\color{incolor}7}]:} \PY{k}{for} \PY{n}{i}\PY{p}{,} \PY{p}{(}\PY{n}{source\PYZus{}item}\PY{p}{,} \PY{n}{deep\PYZus{}item}\PY{p}{)} \PY{o+ow}{in} \PY{n+nb}{enumerate}\PY{p}{(}\PY{n+nb}{zip}\PY{p}{(}\PY{n}{source}\PY{p}{,} \PY{n}{deep\PYZus{}copy}\PY{p}{)}\PY{p}{)}\PY{p}{:}
            \PY{n}{compare} \PY{o}{=} \PY{n}{source\PYZus{}item} \PY{o+ow}{is} \PY{n}{deep\PYZus{}item}
            \PY{n+nb}{print}\PY{p}{(}\PY{n}{f}\PY{l+s+s2}{\PYZdq{}}\PY{l+s+s2}{source[}\PY{l+s+si}{\PYZob{}i\PYZcb{}}\PY{l+s+s2}{] is deep\PYZus{}copy[}\PY{l+s+si}{\PYZob{}i\PYZcb{}}\PY{l+s+s2}{] \PYZhy{}\PYZgt{} }\PY{l+s+si}{\PYZob{}compare\PYZcb{}}\PY{l+s+s2}{\PYZdq{}}\PY{p}{)}
\end{Verbatim}


    \begin{Verbatim}[commandchars=\\\{\}]
source[0] is deep\_copy[0] -> False
source[1] is deep\_copy[1] -> False
source[2] is deep\_copy[2] -> True
source[3] is deep\_copy[3] -> True
source[4] is deep\_copy[4] -> True

    \end{Verbatim}

    On retrouve ici ce qu'on avait déjà remarqué sous pythontutor, à savoir
que les trois derniers objets - immuables - n'ont pas été dupliqués
comme on aurait pu s'y attendre.

    \hypertarget{on-modifie-la-source}{%
\subsubsection{On modifie la source}\label{on-modifie-la-source}}

    Il doit être clair à présent que, précisément parce que
\texttt{deep\_copy} est une copie en profondeur, on peut modifier
\texttt{source} sans impacter du tout \texttt{deep\_copy}.

    S'agissant de \texttt{shallow\_copy}, par contre, seuls les éléments de
premier niveau ont été copiés. Aussi si on fait une modification par
exemple \textbf{à l'intérieur} de la liste qui est le premier fils de
\texttt{source}, cela sera \textbf{répercuté} dans
\texttt{shallow\_copy}~:

    \begin{Verbatim}[commandchars=\\\{\}]
{\color{incolor}In [{\color{incolor}8}]:} \PY{n+nb}{print}\PY{p}{(}\PY{l+s+s2}{\PYZdq{}}\PY{l+s+s2}{avant, source      }\PY{l+s+s2}{\PYZdq{}}\PY{p}{,} \PY{n}{source}\PY{p}{)}
        \PY{n+nb}{print}\PY{p}{(}\PY{l+s+s2}{\PYZdq{}}\PY{l+s+s2}{avant, shallow\PYZus{}copy}\PY{l+s+s2}{\PYZdq{}}\PY{p}{,} \PY{n}{shallow\PYZus{}copy}\PY{p}{)}
        \PY{n}{source}\PY{p}{[}\PY{l+m+mi}{0}\PY{p}{]}\PY{o}{.}\PY{n}{append}\PY{p}{(}\PY{l+m+mi}{4}\PY{p}{)}
        \PY{n+nb}{print}\PY{p}{(}\PY{l+s+s2}{\PYZdq{}}\PY{l+s+s2}{après, source      }\PY{l+s+s2}{\PYZdq{}}\PY{p}{,} \PY{n}{source}\PY{p}{)}
        \PY{n+nb}{print}\PY{p}{(}\PY{l+s+s2}{\PYZdq{}}\PY{l+s+s2}{après, shallow\PYZus{}copy}\PY{l+s+s2}{\PYZdq{}}\PY{p}{,} \PY{n}{shallow\PYZus{}copy}\PY{p}{)}
\end{Verbatim}


    \begin{Verbatim}[commandchars=\\\{\}]
avant, source       [[1, 2, 3], \{1, 2, 3\}, (1, 2, 3), '123', 123]
avant, shallow\_copy [[1, 2, 3], \{1, 2, 3\}, (1, 2, 3), '123', 123]
après, source       [[1, 2, 3, 4], \{1, 2, 3\}, (1, 2, 3), '123', 123]
après, shallow\_copy [[1, 2, 3, 4], \{1, 2, 3\}, (1, 2, 3), '123', 123]

    \end{Verbatim}

    Si par contre on remplace complètement un élément de premier niveau dans
la source, cela ne sera pas répercuté dans la copie superficielle~:

    \begin{Verbatim}[commandchars=\\\{\}]
{\color{incolor}In [{\color{incolor}9}]:} \PY{n+nb}{print}\PY{p}{(}\PY{l+s+s2}{\PYZdq{}}\PY{l+s+s2}{avant, source      }\PY{l+s+s2}{\PYZdq{}}\PY{p}{,} \PY{n}{source}\PY{p}{)}
        \PY{n+nb}{print}\PY{p}{(}\PY{l+s+s2}{\PYZdq{}}\PY{l+s+s2}{avant, shallow\PYZus{}copy}\PY{l+s+s2}{\PYZdq{}}\PY{p}{,} \PY{n}{shallow\PYZus{}copy}\PY{p}{)}
        \PY{n}{source}\PY{p}{[}\PY{l+m+mi}{0}\PY{p}{]} \PY{o}{=} \PY{l+s+s1}{\PYZsq{}}\PY{l+s+s1}{remplacement}\PY{l+s+s1}{\PYZsq{}}
        \PY{n+nb}{print}\PY{p}{(}\PY{l+s+s2}{\PYZdq{}}\PY{l+s+s2}{après, source      }\PY{l+s+s2}{\PYZdq{}}\PY{p}{,} \PY{n}{source}\PY{p}{)}
        \PY{n+nb}{print}\PY{p}{(}\PY{l+s+s2}{\PYZdq{}}\PY{l+s+s2}{après, shallow\PYZus{}copy}\PY{l+s+s2}{\PYZdq{}}\PY{p}{,} \PY{n}{shallow\PYZus{}copy}\PY{p}{)}
\end{Verbatim}


    \begin{Verbatim}[commandchars=\\\{\}]
avant, source       [[1, 2, 3, 4], \{1, 2, 3\}, (1, 2, 3), '123', 123]
avant, shallow\_copy [[1, 2, 3, 4], \{1, 2, 3\}, (1, 2, 3), '123', 123]
après, source       ['remplacement', \{1, 2, 3\}, (1, 2, 3), '123', 123]
après, shallow\_copy [[1, 2, 3, 4], \{1, 2, 3\}, (1, 2, 3), '123', 123]

    \end{Verbatim}

    \hypertarget{copie-et-circularituxe9}{%
\subsubsection{Copie et circularité}\label{copie-et-circularituxe9}}

    Le module \texttt{copy} est capable de copier - même en profondeur - des
objets contenant des références circulaires.

    \begin{Verbatim}[commandchars=\\\{\}]
{\color{incolor}In [{\color{incolor}10}]:} \PY{n}{l} \PY{o}{=} \PY{p}{[}\PY{k+kc}{None}\PY{p}{]}
         \PY{n}{l}\PY{p}{[}\PY{l+m+mi}{0}\PY{p}{]} \PY{o}{=} \PY{n}{l}
         \PY{n}{l}
\end{Verbatim}


\begin{Verbatim}[commandchars=\\\{\}]
{\color{outcolor}Out[{\color{outcolor}10}]:} [[{\ldots}]]
\end{Verbatim}
            
    \begin{Verbatim}[commandchars=\\\{\}]
{\color{incolor}In [{\color{incolor}11}]:} \PY{n}{copy}\PY{o}{.}\PY{n}{copy}\PY{p}{(}\PY{n}{l}\PY{p}{)}
\end{Verbatim}


\begin{Verbatim}[commandchars=\\\{\}]
{\color{outcolor}Out[{\color{outcolor}11}]:} [[[{\ldots}]]]
\end{Verbatim}
            
    \begin{Verbatim}[commandchars=\\\{\}]
{\color{incolor}In [{\color{incolor}12}]:} \PY{n}{copy}\PY{o}{.}\PY{n}{deepcopy}\PY{p}{(}\PY{n}{l}\PY{p}{)}
\end{Verbatim}


\begin{Verbatim}[commandchars=\\\{\}]
{\color{outcolor}Out[{\color{outcolor}12}]:} [[{\ldots}]]
\end{Verbatim}
            
    \hypertarget{pour-en-savoir-plus}{%
\subsubsection{Pour en savoir plus}\label{pour-en-savoir-plus}}

    On peut se reporter à
\href{https://docs.python.org/3/library/copy.html}{la section sur le
module \texttt{copy}} dans la documentation Python.


    % Add a bibliography block to the postdoc
    
    
    
