    \hypertarget{notions-sur-la-pruxe9cision-des-calculs-flottants}{%
\section{Notions sur la précision des calculs
flottants}\label{notions-sur-la-pruxe9cision-des-calculs-flottants}}

    \hypertarget{compluxe9ment---niveau-avancuxe9}{%
\subsection{Complément - niveau
avancé}\label{compluxe9ment---niveau-avancuxe9}}

    \hypertarget{le-probluxe8me}{%
\subsubsection{Le problème}\label{le-probluxe8me}}

    Comme pour les entiers, les calculs sur les flottants sont,
naturellement, réalisés par le processeur. Cependant contrairement au
cas des entiers où les calculs sont toujours exacts, les flottants
posent un problème de précision. Cela n'est pas propre au langage
Python, mais est dû à la technique de codage des nombres flottants sous
forme binaire.\\

    Voyons tout d'abord comment se matérialise le problème~:

    \begin{Verbatim}[commandchars=\\\{\}]
{\color{incolor}In [{\color{incolor}1}]:} \PY{l+m+mf}{0.2} \PY{o}{+} \PY{l+m+mf}{0.4}
\end{Verbatim}


\begin{Verbatim}[commandchars=\\\{\}]
{\color{outcolor}Out[{\color{outcolor}1}]:} 0.6000000000000001
\end{Verbatim}
            
    Il faut retenir que lorsqu'on écrit un nombre flottant sous forme
décimale, la valeur utilisée en mémoire pour représenter ce nombre,
parce que cette valeur est codée en binaire, ne représente \textbf{pas
toujours exactement} le nombre entré.

    \begin{Verbatim}[commandchars=\\\{\}]
{\color{incolor}In [{\color{incolor}2}]:} \PY{c+c1}{\PYZsh{} du coup cette expression est fausse, à cause de l\PYZsq{}erreur d\PYZsq{}arrondi}
        \PY{l+m+mf}{0.3} \PY{o}{\PYZhy{}} \PY{l+m+mf}{0.1} \PY{o}{==} \PY{l+m+mf}{0.2}
\end{Verbatim}


\begin{Verbatim}[commandchars=\\\{\}]
{\color{outcolor}Out[{\color{outcolor}2}]:} False
\end{Verbatim}
            
    Aussi, comme on le voit, les différentes erreurs d'arrondi qui se
produisent à chaque étape du calcul s'accumulent et produisent un
résultat parfois surprenant. De nouveau, ce problème n'est pas
spécifique à Python, il existe pour tous les langages, et il est bien
connu des numériciens.\\

    Dans une grande majorité des cas, ces erreurs d'arrondi ne sont pas
pénalisantes. Il faut toutefois en être conscient car cela peut
expliquer des comportements curieux.

    \hypertarget{une-solution-penser-en-termes-de-nombres-rationnels}{%
\subsubsection{Une solution~: penser en termes de nombres
rationnels}\label{une-solution-penser-en-termes-de-nombres-rationnels}}

    Tout d'abord si votre problème se pose bien en termes de nombres
rationnels, il est alors tout à fait possible de le résoudre avec
exactitude.\\

    Alors qu'il n'est pas possible d'écrire exactement \(3/10\) en base 2,
ni d'ailleurs \(1/3\) en base 10, on peut représenter
\textbf{exactement} ces nombres dès lors qu'on les considère comme des
fractions et qu'on les encode avec deux nombres entiers.\\

    Python fournit en standard le module \texttt{fractions} qui permet de
résoudre le problème. Voici comment on pourrait l'utiliser pour
vérifier, cette fois avec succès, que \(0.3 - 0.1\) vaut bien \(0.2\).
Ce code anticipe sur l'utilisation des modules et des classes en Python,
ici nous créons des objets de type \texttt{Fraction}~:

    \begin{Verbatim}[commandchars=\\\{\}]
{\color{incolor}In [{\color{incolor}3}]:} \PY{c+c1}{\PYZsh{} on importe le module fractions, qui lui\PYZhy{}même définit le symbole Fraction}
        \PY{k+kn}{from} \PY{n+nn}{fractions} \PY{k}{import} \PY{n}{Fraction}
        
        \PY{c+c1}{\PYZsh{} et cette fois, les calculs sont exacts, et l\PYZsq{}expression retourne bien True}
        \PY{n}{Fraction}\PY{p}{(}\PY{l+m+mi}{3}\PY{p}{,} \PY{l+m+mi}{10}\PY{p}{)} \PY{o}{\PYZhy{}} \PY{n}{Fraction}\PY{p}{(}\PY{l+m+mi}{1}\PY{p}{,} \PY{l+m+mi}{10}\PY{p}{)} \PY{o}{==} \PY{n}{Fraction}\PY{p}{(}\PY{l+m+mi}{2}\PY{p}{,} \PY{l+m+mi}{10}\PY{p}{)}
\end{Verbatim}


\begin{Verbatim}[commandchars=\\\{\}]
{\color{outcolor}Out[{\color{outcolor}3}]:} True
\end{Verbatim}
            
    Ou encore d'ailleurs, équivalent et plus lisible~:

    \begin{Verbatim}[commandchars=\\\{\}]
{\color{incolor}In [{\color{incolor}4}]:} \PY{n}{Fraction}\PY{p}{(}\PY{l+s+s1}{\PYZsq{}}\PY{l+s+s1}{0.3}\PY{l+s+s1}{\PYZsq{}}\PY{p}{)} \PY{o}{\PYZhy{}} \PY{n}{Fraction}\PY{p}{(}\PY{l+s+s1}{\PYZsq{}}\PY{l+s+s1}{0.1}\PY{l+s+s1}{\PYZsq{}}\PY{p}{)} \PY{o}{==} \PY{n}{Fraction}\PY{p}{(}\PY{l+s+s1}{\PYZsq{}}\PY{l+s+s1}{2/10}\PY{l+s+s1}{\PYZsq{}}\PY{p}{)}
\end{Verbatim}


\begin{Verbatim}[commandchars=\\\{\}]
{\color{outcolor}Out[{\color{outcolor}4}]:} True
\end{Verbatim}
            
    \hypertarget{une-autre-solution-le-module-decimal}{%
\subsubsection{Une autre solution~: le module
decimal}\label{une-autre-solution-le-module-decimal}}

    Si par contre vous ne manipulez pas des nombres rationnels et que du
coup la représentation sous forme de fractions ne peut pas convenir dans
votre cas, signalons l'existence du module standard \texttt{decimal} qui
offre des fonctionnalités très voisines du type \texttt{float}, tout en
éliminant la plupart des inconvénients, au prix naturellement d'une
consommation mémoire supérieure.\\

    Pour reprendre l'exemple de départ, mais en utilisant le module decimal,
on écrirait alors~:

    \begin{Verbatim}[commandchars=\\\{\}]
{\color{incolor}In [{\color{incolor}5}]:} \PY{k+kn}{from} \PY{n+nn}{decimal} \PY{k}{import} \PY{n}{Decimal}
        
        \PY{n}{Decimal}\PY{p}{(}\PY{l+s+s1}{\PYZsq{}}\PY{l+s+s1}{0.3}\PY{l+s+s1}{\PYZsq{}}\PY{p}{)} \PY{o}{\PYZhy{}} \PY{n}{Decimal}\PY{p}{(}\PY{l+s+s1}{\PYZsq{}}\PY{l+s+s1}{0.1}\PY{l+s+s1}{\PYZsq{}}\PY{p}{)} \PY{o}{==} \PY{n}{Decimal}\PY{p}{(}\PY{l+s+s1}{\PYZsq{}}\PY{l+s+s1}{0.2}\PY{l+s+s1}{\PYZsq{}}\PY{p}{)}
\end{Verbatim}


\begin{Verbatim}[commandchars=\\\{\}]
{\color{outcolor}Out[{\color{outcolor}5}]:} True
\end{Verbatim}
            
    \hypertarget{pour-aller-plus-loin}{%
\subsubsection{Pour aller plus loin}\label{pour-aller-plus-loin}}

    Tous ces documents sont en anglais~:

\begin{itemize}
\tightlist
\item
  un
  \href{https://docs.python.org/3/tutorial/floatingpoint.html}{tutoriel
  sur les nombres flottants}~;
\item
  la
  \href{https://docs.python.org/3/library/fractions.html}{documentation
  sur la classe Fraction}~;
\item
  la \href{https://docs.python.org/3/library/decimal.html}{documentation
  sur la classe Decimal}.
\end{itemize}