    
    
    
    

    

    \hypertarget{exercice---niveau-intermuxe9diaire}{%
\section{Exercice - niveau
intermédiaire}\label{exercice---niveau-intermuxe9diaire}}

    \begin{Verbatim}[commandchars=\\\{\},frame=single,framerule=0.3mm,rulecolor=\color{cellframecolor}]
{\color{incolor}In [{\color{incolor}1}]:} \PY{k+kn}{import} \PY{n+nn}{numpy} \PY{k}{as} \PY{n+nn}{np}
\end{Verbatim}


    \begin{Verbatim}[commandchars=\\\{\},frame=single,framerule=0.3mm,rulecolor=\color{cellframecolor}]
{\color{incolor}In [{\color{incolor}2}]:} \PY{k+kn}{from} \PY{n+nn}{corrections}\PY{n+nn}{.}\PY{n+nn}{exo\PYZus{}stairs} \PY{k}{import} \PY{n}{exo\PYZus{}stairs}
\end{Verbatim}


    On vous demande d'écrire une fonction \texttt{stairs} qui crée un
tableau \texttt{numpy}.

La fonction prend en argument un entier \texttt{k} et construit un
tableau carré de taille \(2*k+1\).

Aux quatre coins du tableau on trouve la valeur \(0\). Dans la case
centrale on trouve la valeur \(2*k\).

Si vous partez de n'importe quelle case et que vous vous déplacez d'une
case horizontalement ou verticalement vers une cas plus proche du
centre, vous incrémentez la valeur du tableau de \texttt{1}.

    \begin{Verbatim}[commandchars=\\\{\},frame=single,framerule=0.3mm,rulecolor=\color{cellframecolor}]
{\color{incolor}In [{\color{incolor}3}]:} \PY{c+c1}{\PYZsh{} voici deux exemples pour la fonction stairs}
        \PY{n}{exo\PYZus{}stairs}\PY{o}{.}\PY{n}{example}\PY{p}{(}\PY{p}{)}
\end{Verbatim}


\begin{Verbatim}[commandchars=\\\{\},frame=single,framerule=0.3mm,rulecolor=\color{cellframecolor}]
{\color{outcolor}Out[{\color{outcolor}3}]:} <IPython.core.display.HTML object>
\end{Verbatim}
            
    \begin{Verbatim}[commandchars=\\\{\},frame=single,framerule=0.3mm,rulecolor=\color{cellframecolor}]
{\color{incolor}In [{\color{incolor}4}]:} \PY{c+c1}{\PYZsh{} à vous de jouer}
        \PY{k}{def} \PY{n+nf}{stairs}\PY{p}{(}\PY{n}{k}\PY{p}{)}\PY{p}{:}
            \PY{k}{return} \PY{l+s+s2}{\PYZdq{}}\PY{l+s+s2}{votre code}\PY{l+s+s2}{\PYZdq{}}
        
        \PY{c+c1}{\PYZsh{} NOTE:}
        \PY{c+c1}{\PYZsh{} auto\PYZhy{}exec\PYZhy{}for\PYZhy{}latex has used instead:}
        \PY{c+c1}{\PYZsh{}\PYZsh{}\PYZsh{}\PYZsh{}\PYZsh{}\PYZsh{}\PYZsh{}\PYZsh{}\PYZsh{}\PYZsh{}}
        \PY{n}{stairs}\PY{o}{=}\PY{n}{exo\PYZus{}stairs}\PY{o}{.}\PY{n}{solution}
        \PY{c+c1}{\PYZsh{}\PYZsh{}\PYZsh{}\PYZsh{}\PYZsh{}\PYZsh{}\PYZsh{}\PYZsh{}\PYZsh{}\PYZsh{}}
\end{Verbatim}


    \begin{Verbatim}[commandchars=\\\{\},frame=single,framerule=0.3mm,rulecolor=\color{cellframecolor}]
{\color{incolor}In [{\color{incolor} }]:} \PY{c+c1}{\PYZsh{} NOTE}
        \PY{c+c1}{\PYZsh{} auto\PYZhy{}exec\PYZhy{}for\PYZhy{}latex has skipped execution of this cell}
        
        \PY{c+c1}{\PYZsh{} pour corriger votre code}
        \PY{n}{exo\PYZus{}stairs}\PY{o}{.}\PY{n}{correction}\PY{p}{(}\PY{n}{stairs}\PY{p}{)}
\end{Verbatim}


    \hypertarget{visualisation}{%
\subsubsection{Visualisation}\label{visualisation}}

    \begin{Verbatim}[commandchars=\\\{\},frame=single,framerule=0.3mm,rulecolor=\color{cellframecolor}]
{\color{incolor}In [{\color{incolor}5}]:} \PY{k+kn}{import} \PY{n+nn}{matplotlib}\PY{n+nn}{.}\PY{n+nn}{pyplot} \PY{k}{as} \PY{n+nn}{plt}
        \PY{o}{\PYZpc{}}\PY{k}{matplotlib} inline
        \PY{n}{plt}\PY{o}{.}\PY{n}{ion}\PY{p}{(}\PY{p}{)}
\end{Verbatim}


    L'exercice est terminé, mais si vous avez réussi et que vous voulez
visualisez le résultat, voici comment vous pouvez aussi voir ce type de
tableau~:

    \begin{Verbatim}[commandchars=\\\{\},frame=single,framerule=0.3mm,rulecolor=\color{cellframecolor}]
{\color{incolor}In [{\color{incolor}6}]:} \PY{n}{squares} \PY{o}{=} \PY{n}{stairs}\PY{p}{(}\PY{l+m+mi}{100}\PY{p}{)}
\end{Verbatim}


    Pour le voir comme une image avec un niveau de gris comme code de
couleurs (noir = 0, blanc = maximum = 201 dans notre cas)~:

    \begin{Verbatim}[commandchars=\\\{\},frame=single,framerule=0.3mm,rulecolor=\color{cellframecolor}]
{\color{incolor}In [{\color{incolor}7}]:} \PY{c+c1}{\PYZsh{} convertir en flottant pour imshow}
        \PY{n}{squares} \PY{o}{=} \PY{n}{squares}\PY{o}{.}\PY{n}{astype}\PY{p}{(}\PY{n}{np}\PY{o}{.}\PY{n}{float}\PY{p}{)}
        \PY{c+c1}{\PYZsh{} afficher avec une colormap \PYZsq{}gray\PYZsq{}}
        \PY{n}{plt}\PY{o}{.}\PY{n}{imshow}\PY{p}{(}\PY{n}{squares}\PY{p}{,} \PY{n}{cmap}\PY{o}{=}\PY{l+s+s1}{\PYZsq{}}\PY{l+s+s1}{gray}\PY{l+s+s1}{\PYZsq{}}\PY{p}{)}\PY{p}{;}
\end{Verbatim}


    \begin{center}
    \adjustimage{max size={0.9\linewidth}{0.9\paperheight}}{w7-s05-x1-stairs_files/w7-s05-x1-stairs_13_0.png}
    \end{center}
    { \hspace*{\fill} \\}
    

    % Add a bibliography block to the postdoc
    
    
    
