    
    
    
    

    

    \hypertarget{notebooks-interactifs}{%
\section{Notebooks interactifs}\label{notebooks-interactifs}}

    \begin{Verbatim}[commandchars=\\\{\},frame=single,framerule=0.3mm,rulecolor=\color{cellframecolor}]
{\color{incolor}In [{\color{incolor}1}]:} \PY{k+kn}{import} \PY{n+nn}{matplotlib}\PY{n+nn}{.}\PY{n+nn}{pyplot} \PY{k}{as} \PY{n+nn}{plt}
        \PY{k+kn}{import} \PY{n+nn}{numpy} \PY{k}{as} \PY{n+nn}{np}
        \PY{n}{plt}\PY{o}{.}\PY{n}{ion}\PY{p}{(}\PY{p}{)}
\end{Verbatim}


    \hypertarget{compluxe9ment---niveau-basique}{%
\subsection{Complément - niveau
basique}\label{compluxe9ment---niveau-basique}}

    Pour conclure cette série sur les outils de visualisation, nous allons
voir quelques fonctionnalités disponibles uniquement dans
l'environnement des notebooks, et qui offrent des possibilités
supplémentaires par rapport aux visualisations que l'on a vues jusqu'à
maintenant.

    Pour commencer et en guise d'exemple, je vous invite à consulter un
\href{http://www.nature.com/news/ipython-interactive-demo-7.21492}{exemple
de notebook publié par la célèbre revue \emph{Nature}}, qui pourra vous
donner une idée de ce qu'il est possible de faire de cette façon~;
essayez de bien penser à cliquer sur \texttt{Expand\ to\ fullscreen}
pour un meilleur confort.

Je vous signale d'ailleurs que
\href{https://github.com/jupyter/nature-demo}{ce notebook est disponible
sur github ici}.

    \hypertarget{une-visualisation-interactive-simple-interact}{%
\subsubsection{\texorpdfstring{Une visualisation interactive simple~:
\texttt{interact}}{Une visualisation interactive simple~: interact}}\label{une-visualisation-interactive-simple-interact}}

    Pour refaire de notre coté quelque chose d'analogue, nous allons
commencer par animer la fonction sinus, avec un bouton pour régler la
fréquence. Pour cela nous allons utiliser la fonction
\texttt{interact}~; à nouveau c'est un utilitaire qui fait partie de
l'écosystème des notebooks, et plus précisément du module
\texttt{ipywidgets}~:

    \begin{Verbatim}[commandchars=\\\{\},frame=single,framerule=0.3mm,rulecolor=\color{cellframecolor}]
{\color{incolor}In [{\color{incolor}2}]:} \PY{k+kn}{from} \PY{n+nn}{ipywidgets} \PY{k}{import} \PY{n}{interact}
\end{Verbatim}


    Dans un premier temps, j'écris une fonction qui prend en paramètre la
fréquence, et qui dessine la fonction sinus sur un intervalle fixe de 0.
à \(4\pi\)~:

    \begin{Verbatim}[commandchars=\\\{\},frame=single,framerule=0.3mm,rulecolor=\color{cellframecolor}]
{\color{incolor}In [{\color{incolor}3}]:} \PY{k}{def} \PY{n+nf}{sinus}\PY{p}{(}\PY{n}{freq}\PY{p}{)}\PY{p}{:}
            \PY{n}{X} \PY{o}{=} \PY{n}{np}\PY{o}{.}\PY{n}{linspace}\PY{p}{(}\PY{l+m+mf}{0.}\PY{p}{,} \PY{l+m+mi}{4}\PY{o}{*}\PY{n}{np}\PY{o}{.}\PY{n}{pi}\PY{p}{,} \PY{l+m+mi}{200}\PY{p}{)}
            \PY{n}{Y} \PY{o}{=} \PY{n}{np}\PY{o}{.}\PY{n}{sin}\PY{p}{(}\PY{n}{freq}\PY{o}{*}\PY{n}{X}\PY{p}{)}
            \PY{n}{plt}\PY{o}{.}\PY{n}{plot}\PY{p}{(}\PY{n}{X}\PY{p}{,} \PY{n}{Y}\PY{p}{)}
\end{Verbatim}


    \begin{Verbatim}[commandchars=\\\{\},frame=single,framerule=0.3mm,rulecolor=\color{cellframecolor}]
{\color{incolor}In [{\color{incolor}4}]:} \PY{n}{sinus}\PY{p}{(}\PY{l+m+mi}{1}\PY{p}{)}
\end{Verbatim}


    \begin{center}
    \adjustimage{max size={0.9\linewidth}{0.9\paperheight}}{w7-s10-c3-notebooks-interactifs_files/w7-s10-c3-notebooks-interactifs_11_0.png}
    \end{center}
    { \hspace*{\fill} \\}
    
    \begin{Verbatim}[commandchars=\\\{\},frame=single,framerule=0.3mm,rulecolor=\color{cellframecolor}]
{\color{incolor}In [{\color{incolor}5}]:} \PY{n}{sinus}\PY{p}{(}\PY{l+m+mf}{0.5}\PY{p}{)}
\end{Verbatim}


    \begin{center}
    \adjustimage{max size={0.9\linewidth}{0.9\paperheight}}{w7-s10-c3-notebooks-interactifs_files/w7-s10-c3-notebooks-interactifs_12_0.png}
    \end{center}
    { \hspace*{\fill} \\}
    
    Maintenant, plutôt que de tracer individuellement les courbes une à une,
j'utilise \texttt{interact} qui va m'afficher une réglette pour changer
le paramètre \texttt{freq}. Ça se présente comme ceci~:

    \begin{Verbatim}[commandchars=\\\{\},frame=single,framerule=0.3mm,rulecolor=\color{cellframecolor}]
{\color{incolor}In [{\color{incolor}6}]:} \PY{c+c1}{\PYZsh{} je change maintenant la taille des visualisations}
        \PY{n}{plt}\PY{o}{.}\PY{n}{figure}\PY{p}{(}\PY{n}{figsize}\PY{o}{=}\PY{p}{(}\PY{l+m+mi}{12}\PY{p}{,} \PY{l+m+mi}{4}\PY{p}{)}\PY{p}{)}\PY{p}{;}
\end{Verbatim}


    
    \begin{verbatim}
<Figure size 864x288 with 0 Axes>
    \end{verbatim}

    
    \begin{Verbatim}[commandchars=\\\{\},frame=single,framerule=0.3mm,rulecolor=\color{cellframecolor}]
{\color{incolor}In [{\color{incolor}7}]:} \PY{n}{interact}\PY{p}{(}\PY{n}{sinus}\PY{p}{,} \PY{n}{freq}\PY{o}{=}\PY{p}{(}\PY{l+m+mf}{0.5}\PY{p}{,} \PY{l+m+mf}{10.}\PY{p}{,} \PY{l+m+mf}{0.25}\PY{p}{)}\PY{p}{)}\PY{p}{;}
\end{Verbatim}


    \begin{center}
    \adjustimage{max size={0.9\linewidth}{0.9\paperheight}}{w7-s10-c3-notebooks-interactifs_files/w7-s10-c3-notebooks-interactifs_15_0.png}
    \end{center}
    { \hspace*{\fill} \\}
    
    \hypertarget{muxe9canisme-dinteract}{%
\subsection{\texorpdfstring{Mécanisme
d'\texttt{interact}}{Mécanisme d'interact}}\label{muxe9canisme-dinteract}}

    La fonction \texttt{interact} s'attend à recevoir~:

\begin{itemize}
\tightlist
\item
  en premier argument~: une fonction \texttt{f}~;
\item
  et ensuite autant d'arguments nommés supplémentaires que de paramètres
  attendus par \texttt{f}.
\end{itemize}

Comme dans mon cas la fonction \texttt{sinus} attend un paramètre nommé
\texttt{freq}, le deuxième argument de \texttt{interact} lui est passé
aussi avec le nom \texttt{freq}.

    \hypertarget{les-objets-slider}{%
\subsubsection{\texorpdfstring{Les objets
\texttt{Slider}}{Les objets Slider}}\label{les-objets-slider}}

    Chacun des arguments à \texttt{interact} (en plus de la fonction)
correspond à un objet de type \texttt{Slider} (dans la ménagerie de
\texttt{ipywidget}). Ici en passant juste le tuple
\texttt{(0.5,\ 10.,\ 0.25)} j'utilise un raccourci pour dire que je veux
pouvoir régler le paramètre \texttt{freq} sur une plage allant de
\texttt{0.5} à \texttt{10} avec un pas de \texttt{0.25}.

    Mon premier exemple avec \texttt{interact} est en réalité équivalent à
ceci~:

    \begin{Verbatim}[commandchars=\\\{\},frame=single,framerule=0.3mm,rulecolor=\color{cellframecolor}]
{\color{incolor}In [{\color{incolor}8}]:} \PY{k+kn}{from} \PY{n+nn}{ipywidgets} \PY{k}{import} \PY{n}{FloatSlider}
\end{Verbatim}


    \begin{Verbatim}[commandchars=\\\{\},frame=single,framerule=0.3mm,rulecolor=\color{cellframecolor}]
{\color{incolor}In [{\color{incolor}9}]:} \PY{c+c1}{\PYZsh{} exactement équivalent à la version ci\PYZhy{}dessus}
        \PY{n}{interact}\PY{p}{(}\PY{n}{sinus}\PY{p}{,} \PY{n}{freq}\PY{o}{=}\PY{n}{FloatSlider}\PY{p}{(}\PY{n+nb}{min}\PY{o}{=}\PY{l+m+mf}{0.5}\PY{p}{,} \PY{n+nb}{max}\PY{o}{=}\PY{l+m+mf}{10.}\PY{p}{,} \PY{n}{step}\PY{o}{=}\PY{l+m+mf}{0.25}\PY{p}{)}\PY{p}{)}\PY{p}{;}
\end{Verbatim}


    \begin{center}
    \adjustimage{max size={0.9\linewidth}{0.9\paperheight}}{w7-s10-c3-notebooks-interactifs_files/w7-s10-c3-notebooks-interactifs_22_0.png}
    \end{center}
    { \hspace*{\fill} \\}
    
    Mais en utilisant la forme bavarde, je peux choisir davantage d'options,
comme notamment~:

\begin{itemize}
\tightlist
\item
  mettre \texttt{continuous\_update\ =\ False}~; l'effet de ce réglage,
  c'est que l'on met à jour la figure seulement lorsque je lâche la
  réglette~; c'est utile lorsque les calculs sont un peu lents, comme
  ici avec l'infrastructure notebook qui est à distance~;
\item
  mettre \texttt{value=1.} pour choisir la valeur initiale~:
\end{itemize}

    \begin{Verbatim}[commandchars=\\\{\},frame=single,framerule=0.3mm,rulecolor=\color{cellframecolor}]
{\color{incolor}In [{\color{incolor}10}]:} \PY{c+c1}{\PYZsh{} exactement équivalent à la version ci\PYZhy{}dessus}
         \PY{c+c1}{\PYZsh{} sauf qu\PYZsq{}on ne redessine que lorsque la réglette}
         \PY{c+c1}{\PYZsh{} est relâchée}
         \PY{n}{interact}\PY{p}{(}\PY{n}{sinus}\PY{p}{,} \PY{n}{freq}\PY{o}{=}\PY{n}{FloatSlider}\PY{p}{(}\PY{n+nb}{min}\PY{o}{=}\PY{l+m+mf}{0.5}\PY{p}{,} \PY{n+nb}{max}\PY{o}{=}\PY{l+m+mf}{10.}\PY{p}{,} 
                                          \PY{n}{step}\PY{o}{=}\PY{l+m+mf}{0.25}\PY{p}{,} \PY{n}{value}\PY{o}{=}\PY{l+m+mf}{1.}\PY{p}{,}
                                          \PY{n}{continuous\PYZus{}update}\PY{o}{=}\PY{k+kc}{False}\PY{p}{)}\PY{p}{)}\PY{p}{;}
\end{Verbatim}


    \begin{center}
    \adjustimage{max size={0.9\linewidth}{0.9\paperheight}}{w7-s10-c3-notebooks-interactifs_files/w7-s10-c3-notebooks-interactifs_24_0.png}
    \end{center}
    { \hspace*{\fill} \\}
    
    \hypertarget{plusieurs-paramuxe8tres}{%
\subsubsection{Plusieurs paramètres}\label{plusieurs-paramuxe8tres}}

    Voyons tout de suite un exemple avec deux paramètres, je vais écrire
maintenant une fonction qui me permet de changer aussi la phase~:

    \begin{Verbatim}[commandchars=\\\{\},frame=single,framerule=0.3mm,rulecolor=\color{cellframecolor}]
{\color{incolor}In [{\color{incolor}11}]:} \PY{k}{def} \PY{n+nf}{sinus2}\PY{p}{(}\PY{n}{freq}\PY{p}{,} \PY{n}{phase}\PY{p}{)}\PY{p}{:}
             \PY{n}{X} \PY{o}{=} \PY{n}{np}\PY{o}{.}\PY{n}{linspace}\PY{p}{(}\PY{l+m+mf}{0.}\PY{p}{,} \PY{l+m+mi}{4}\PY{o}{*}\PY{n}{np}\PY{o}{.}\PY{n}{pi}\PY{p}{,} \PY{l+m+mi}{200}\PY{p}{)}
             \PY{n}{Y} \PY{o}{=} \PY{n}{np}\PY{o}{.}\PY{n}{sin}\PY{p}{(}\PY{n}{freq}\PY{o}{*}\PY{p}{(}\PY{n}{X}\PY{o}{+}\PY{n}{phase}\PY{p}{)}\PY{p}{)}
             \PY{n}{plt}\PY{o}{.}\PY{n}{plot}\PY{p}{(}\PY{n}{X}\PY{p}{,} \PY{n}{Y}\PY{p}{)}
\end{Verbatim}


    Et donc maintenant je passe à \texttt{interact} un troisième paramètre~:

    \begin{Verbatim}[commandchars=\\\{\},frame=single,framerule=0.3mm,rulecolor=\color{cellframecolor}]
{\color{incolor}In [{\color{incolor}12}]:} \PY{n}{interact}\PY{p}{(}\PY{n}{sinus2}\PY{p}{,}
                  \PY{n}{freq}\PY{o}{=}\PY{n}{FloatSlider}\PY{p}{(}\PY{n+nb}{min}\PY{o}{=}\PY{l+m+mf}{0.5}\PY{p}{,} \PY{n+nb}{max}\PY{o}{=}\PY{l+m+mf}{10.}\PY{p}{,} \PY{n}{step}\PY{o}{=}\PY{l+m+mf}{0.5}\PY{p}{,}
                                   \PY{n}{continuous\PYZus{}update}\PY{o}{=}\PY{k+kc}{False}\PY{p}{)}\PY{p}{,}
                  \PY{n}{phase}\PY{o}{=}\PY{n}{FloatSlider}\PY{p}{(}\PY{n+nb}{min}\PY{o}{=}\PY{l+m+mf}{0.}\PY{p}{,} \PY{n+nb}{max}\PY{o}{=}\PY{l+m+mi}{2}\PY{o}{*}\PY{n}{np}\PY{o}{.}\PY{n}{pi}\PY{p}{,} \PY{n}{step}\PY{o}{=}\PY{n}{np}\PY{o}{.}\PY{n}{pi}\PY{o}{/}\PY{l+m+mi}{6}\PY{p}{,} 
                                    \PY{n}{continuous\PYZus{}update}\PY{o}{=}\PY{k+kc}{False}\PY{p}{)}\PY{p}{,}
                 \PY{p}{)}\PY{p}{;}
\end{Verbatim}


    \begin{center}
    \adjustimage{max size={0.9\linewidth}{0.9\paperheight}}{w7-s10-c3-notebooks-interactifs_files/w7-s10-c3-notebooks-interactifs_29_0.png}
    \end{center}
    { \hspace*{\fill} \\}
    
    \hypertarget{bouche-trou-fixed}{%
\subsection{\texorpdfstring{Bouche-trou~:
\texttt{fixed}}{Bouche-trou~: fixed}}\label{bouche-trou-fixed}}

    Si j'ai une fonction qui prend plus de paramètres que je ne veux montrer
de réglettes, je peux fixer un des paramètres par exemple comme ceci~:

    \begin{Verbatim}[commandchars=\\\{\},frame=single,framerule=0.3mm,rulecolor=\color{cellframecolor}]
{\color{incolor}In [{\color{incolor}13}]:} \PY{k+kn}{from} \PY{n+nn}{ipywidgets} \PY{k}{import} \PY{n}{fixed}
\end{Verbatim}


    \begin{Verbatim}[commandchars=\\\{\},frame=single,framerule=0.3mm,rulecolor=\color{cellframecolor}]
{\color{incolor}In [{\color{incolor}14}]:} \PY{c+c1}{\PYZsh{} avec une fonction à deux argument,}
         \PY{c+c1}{\PYZsh{} je peux en fixer un, et n\PYZsq{}avoir qu\PYZsq{}une réglette}
         \PY{c+c1}{\PYZsh{} pour fixer celui qui est libre}
         \PY{n}{interact}\PY{p}{(}\PY{n}{sinus2}\PY{p}{,} \PY{n}{freq}\PY{o}{=}\PY{n}{fixed}\PY{p}{(}\PY{l+m+mf}{1.}\PY{p}{)}\PY{p}{,}
                  \PY{n}{phase}\PY{o}{=}\PY{n}{FloatSlider}\PY{p}{(}\PY{n+nb}{min}\PY{o}{=}\PY{l+m+mf}{0.}\PY{p}{,} \PY{n+nb}{max}\PY{o}{=}\PY{l+m+mi}{2}\PY{o}{*}\PY{n}{np}\PY{o}{.}\PY{n}{pi}\PY{p}{,} \PY{n}{step}\PY{o}{=}\PY{n}{np}\PY{o}{.}\PY{n}{pi}\PY{o}{/}\PY{l+m+mi}{6}\PY{p}{)}\PY{p}{,}
                 \PY{p}{)}\PY{p}{;}
\end{Verbatim}


    \begin{center}
    \adjustimage{max size={0.9\linewidth}{0.9\paperheight}}{w7-s10-c3-notebooks-interactifs_files/w7-s10-c3-notebooks-interactifs_33_0.png}
    \end{center}
    { \hspace*{\fill} \\}
    
    \hypertarget{widgets}{%
\subsection{Widgets}\label{widgets}}

    Il existe toute une famille de widgets, dont \texttt{FloatSlider} est
l'exemple le plus courant, mais vous pouvez aussi~:

\begin{itemize}
\tightlist
\item
  créer des radio bouton pour entrer un paramètre booléen~;
\item
  ou une saisie de texte pour entre un paramètre de type \texttt{str}~;
\item
  ou une liste à choix multiples\ldots{}
\end{itemize}

Bref, vous pouvez créer une mini interface-utilisateur avec des objets
graphiques simples choisis dans une palette assez complète pour ce type
d'application.

Voyez
\href{http://ipywidgets.readthedocs.io/en/latest/examples/Using\%20Interact.html}{les
détails complets sur \texttt{readthedocs.io}}

    \begin{Verbatim}[commandchars=\\\{\},frame=single,framerule=0.3mm,rulecolor=\color{cellframecolor}]
{\color{incolor}In [{\color{incolor}15}]:} \PY{c+c1}{\PYZsh{} de même qu\PYZsq{}un tuple était ci\PYZhy{}dessus un raccourci pour un FloatSlider}
         \PY{c+c1}{\PYZsh{} une liste ou un dictionnaire est transformé(e) en un Dropdown}
         \PY{n}{interact}\PY{p}{(}\PY{n}{sinus}\PY{p}{,} \PY{n}{freq}\PY{o}{=}\PY{p}{\PYZob{}}\PY{l+s+s1}{\PYZsq{}}\PY{l+s+s1}{rapide}\PY{l+s+s1}{\PYZsq{}}\PY{p}{:} \PY{l+m+mf}{10.}\PY{p}{,} \PY{l+s+s1}{\PYZsq{}}\PY{l+s+s1}{moyenne}\PY{l+s+s1}{\PYZsq{}}\PY{p}{:} \PY{l+m+mf}{1.}\PY{p}{,} \PY{l+s+s1}{\PYZsq{}}\PY{l+s+s1}{lente}\PY{l+s+s1}{\PYZsq{}}\PY{p}{:} \PY{l+m+mf}{0.1}\PY{p}{\PYZcb{}}\PY{p}{)}\PY{p}{;}
\end{Verbatim}


    \begin{center}
    \adjustimage{max size={0.9\linewidth}{0.9\paperheight}}{w7-s10-c3-notebooks-interactifs_files/w7-s10-c3-notebooks-interactifs_36_0.png}
    \end{center}
    { \hspace*{\fill} \\}
    
    Voyez la
\href{http://ipywidgets.readthedocs.io/en/latest/examples/Widget\%20List.html}{liste
complète des widgets ici}.

    \hypertarget{dashboards}{%
\section{Dashboards}\label{dashboards}}

    Lorsqu'on a besoin de faire une interface un peu plus soignée, on peut
créer sa propre disposition de boutons et autres réglages.

    Voici un exemple de dashboard, uniquement pour vous donner une meilleure
idée, qui pour changer agit sur une visualisation réalisée avec plot.ly
plutôt que matplotlib~:

    \begin{Verbatim}[commandchars=\\\{\},frame=single,framerule=0.3mm,rulecolor=\color{cellframecolor}]
{\color{incolor}In [{\color{incolor}16}]:} \PY{k+kn}{import} \PY{n+nn}{plotly}
         \PY{n}{plotly}\PY{o}{.}\PY{n}{\PYZus{}\PYZus{}version\PYZus{}\PYZus{}}
\end{Verbatim}


\begin{Verbatim}[commandchars=\\\{\},frame=single,framerule=0.3mm,rulecolor=\color{cellframecolor}]
{\color{outcolor}Out[{\color{outcolor}16}]:} '3.1.1'
\end{Verbatim}
            
    \begin{Verbatim}[commandchars=\\\{\},frame=single,framerule=0.3mm,rulecolor=\color{cellframecolor}]
{\color{incolor}In [{\color{incolor}17}]:} \PY{c+c1}{\PYZsh{} on importe la bibliothèque plot.ly}
         \PY{k+kn}{import} \PY{n+nn}{plotly}\PY{n+nn}{.}\PY{n+nn}{plotly} \PY{k}{as} \PY{n+nn}{py}
         \PY{k+kn}{import} \PY{n+nn}{plotly}\PY{n+nn}{.}\PY{n+nn}{graph\PYZus{}objs} \PY{k}{as} \PY{n+nn}{go}
\end{Verbatim}


    \begin{Verbatim}[commandchars=\\\{\},frame=single,framerule=0.3mm,rulecolor=\color{cellframecolor}]
{\color{incolor}In [{\color{incolor}18}]:} \PY{c+c1}{\PYZsh{} il est impératif d\PYZsq{}utiliser plot.ly en mode \PYZsq{}offline\PYZsq{} }
         \PY{c+c1}{\PYZsh{} pour in mode interactif, }
         \PY{c+c1}{\PYZsh{} car sinon les affichages sont beaucoup trop lents}
         \PY{k+kn}{import} \PY{n+nn}{plotly}\PY{n+nn}{.}\PY{n+nn}{offline} \PY{k}{as} \PY{n+nn}{pyoff}
         
         \PY{n}{pyoff}\PY{o}{.}\PY{n}{init\PYZus{}notebook\PYZus{}mode}\PY{p}{(}\PY{p}{)}
\end{Verbatim}


    
    
    \begin{Verbatim}[commandchars=\\\{\},frame=single,framerule=0.3mm,rulecolor=\color{cellframecolor}]
{\color{incolor}In [{\color{incolor}19}]:} \PY{c+c1}{\PYZsh{} les widgets pour construire le tableau de bord}
         \PY{k+kn}{from} \PY{n+nn}{ipywidgets} \PY{k}{import} \PY{p}{(}\PY{n}{interactive\PYZus{}output}\PY{p}{,}
                                 \PY{n}{IntSlider}\PY{p}{,} \PY{n}{Dropdown}\PY{p}{,} \PY{n}{Layout}\PY{p}{,} \PY{n}{HBox}\PY{p}{,} \PY{n}{VBox}\PY{p}{,} \PY{n}{Text}\PY{p}{)}
         \PY{k+kn}{from} \PY{n+nn}{IPython}\PY{n+nn}{.}\PY{n+nn}{display} \PY{k}{import} \PY{n}{display}
\end{Verbatim}


    \begin{Verbatim}[commandchars=\\\{\},frame=single,framerule=0.3mm,rulecolor=\color{cellframecolor}]
{\color{incolor}In [{\color{incolor}20}]:} \PY{c+c1}{\PYZsh{} une fonction sinus à 4 réglages}
         \PY{c+c1}{\PYZsh{} qu\PYZsq{}on réalise pour changer avec plot.ly}
         \PY{c+c1}{\PYZsh{} et non pas avec matplotlib}
         \PY{k}{def} \PY{n+nf}{sinus4}\PY{p}{(}\PY{n}{freq}\PY{p}{,} \PY{n}{phase}\PY{p}{,} \PY{n}{amplitude}\PY{p}{,} \PY{n}{domain}\PY{p}{)}\PY{p}{:}
         
             \PY{n}{X} \PY{o}{=} \PY{n}{np}\PY{o}{.}\PY{n}{linspace}\PY{p}{(}\PY{l+m+mf}{0.}\PY{p}{,} \PY{n}{domain}\PY{o}{*}\PY{n}{np}\PY{o}{.}\PY{n}{pi}\PY{p}{,} \PY{l+m+mi}{500}\PY{p}{)}
             \PY{n}{Y} \PY{o}{=} \PY{n}{amplitude} \PY{o}{*} \PY{n}{np}\PY{o}{.}\PY{n}{sin}\PY{p}{(}\PY{n}{freq}\PY{o}{*}\PY{p}{(}\PY{n}{X}\PY{o}{+}\PY{n}{phase}\PY{p}{)}\PY{p}{)}
         
             \PY{n}{data} \PY{o}{=} \PY{p}{[} \PY{n}{go}\PY{o}{.}\PY{n}{Scatter}\PY{p}{(}\PY{n}{x}\PY{o}{=}\PY{n}{X}\PY{p}{,} \PY{n}{y}\PY{o}{=}\PY{n}{Y}\PY{p}{,} \PY{n}{mode}\PY{o}{=}\PY{l+s+s1}{\PYZsq{}}\PY{l+s+s1}{lines}\PY{l+s+s1}{\PYZsq{}}\PY{p}{,} \PY{n}{name}\PY{o}{=}\PY{l+s+s1}{\PYZsq{}}\PY{l+s+s1}{sinus}\PY{l+s+s1}{\PYZsq{}}\PY{p}{)} \PY{p}{]}
             \PY{c+c1}{\PYZsh{} je fixe l\PYZsq{}amplitude à 10 pour que les animations}
             \PY{c+c1}{\PYZsh{} soient plus parlantes}
             \PY{n}{layout} \PY{o}{=} \PY{n}{go}\PY{o}{.}\PY{n}{Layout}\PY{p}{(}
                 \PY{n}{yaxis} \PY{o}{=} \PY{p}{\PYZob{}}\PY{l+s+s1}{\PYZsq{}}\PY{l+s+s1}{range}\PY{l+s+s1}{\PYZsq{}} \PY{p}{:} \PY{p}{[}\PY{o}{\PYZhy{}}\PY{l+m+mi}{10}\PY{p}{,} \PY{l+m+mi}{10}\PY{p}{]}\PY{p}{\PYZcb{}}\PY{p}{,}
                 \PY{n}{title}\PY{o}{=}\PY{l+s+s2}{\PYZdq{}}\PY{l+s+s2}{Exemple de graphique interactif avec dashboard}\PY{l+s+s2}{\PYZdq{}}\PY{p}{,}
             \PY{p}{)}
             \PY{n}{figure} \PY{o}{=} \PY{n}{go}\PY{o}{.}\PY{n}{Figure}\PY{p}{(}\PY{n}{data} \PY{o}{=} \PY{n}{data}\PY{p}{,} \PY{n}{layout}\PY{o}{=}\PY{n}{layout}\PY{p}{)}
             \PY{n}{pyoff}\PY{o}{.}\PY{n}{iplot}\PY{p}{(}\PY{n}{figure}\PY{p}{)}
\end{Verbatim}


    \begin{Verbatim}[commandchars=\\\{\},frame=single,framerule=0.3mm,rulecolor=\color{cellframecolor}]
{\color{incolor}In [{\color{incolor}21}]:} \PY{k}{def} \PY{n+nf}{my\PYZus{}dashboard}\PY{p}{(}\PY{p}{)}\PY{p}{:}
             \PY{l+s+sd}{\PYZdq{}\PYZdq{}\PYZdq{}}
         \PY{l+s+sd}{    create and display a dashboard}
         \PY{l+s+sd}{    return a dictionary name\PYZhy{}\PYZgt{}widget suitable for interactive\PYZus{}output}
         \PY{l+s+sd}{    \PYZdq{}\PYZdq{}\PYZdq{}}
             \PY{c+c1}{\PYZsh{} dashboard pieces as widgets}
             \PY{n}{l\PYZus{}75} \PY{o}{=} \PY{n}{Layout}\PY{p}{(}\PY{n}{width}\PY{o}{=}\PY{l+s+s1}{\PYZsq{}}\PY{l+s+s1}{75}\PY{l+s+s1}{\PYZpc{}}\PY{l+s+s1}{\PYZsq{}}\PY{p}{)}
             \PY{n}{l\PYZus{}50} \PY{o}{=} \PY{n}{Layout}\PY{p}{(}\PY{n}{width}\PY{o}{=}\PY{l+s+s1}{\PYZsq{}}\PY{l+s+s1}{50}\PY{l+s+s1}{\PYZpc{}}\PY{l+s+s1}{\PYZsq{}}\PY{p}{)}
             \PY{n}{l\PYZus{}25} \PY{o}{=} \PY{n}{Layout}\PY{p}{(}\PY{n}{width}\PY{o}{=}\PY{l+s+s1}{\PYZsq{}}\PY{l+s+s1}{25}\PY{l+s+s1}{\PYZpc{}}\PY{l+s+s1}{\PYZsq{}}\PY{p}{)}
         
             \PY{n}{w\PYZus{}freq} \PY{o}{=} \PY{n}{Dropdown}\PY{p}{(}\PY{n}{options}\PY{o}{=}\PY{n+nb}{list}\PY{p}{(}\PY{n+nb}{range}\PY{p}{(}\PY{l+m+mi}{1}\PY{p}{,} \PY{l+m+mi}{10}\PY{p}{)}\PY{p}{)}\PY{p}{,}
                               \PY{n}{value} \PY{o}{=} \PY{l+m+mi}{1}\PY{p}{,}
                               \PY{n}{description} \PY{o}{=} \PY{l+s+s2}{\PYZdq{}}\PY{l+s+s2}{fréquence}\PY{l+s+s2}{\PYZdq{}}\PY{p}{,}
                               \PY{n}{layout}\PY{o}{=}\PY{n}{l\PYZus{}50}\PY{p}{)}
             \PY{n}{w\PYZus{}phase} \PY{o}{=} \PY{n}{FloatSlider}\PY{p}{(}\PY{n+nb}{min}\PY{o}{=}\PY{l+m+mf}{0.}\PY{p}{,} \PY{n+nb}{max} \PY{o}{=} \PY{l+m+mi}{2}\PY{o}{*}\PY{n}{np}\PY{o}{.}\PY{n}{pi}\PY{p}{,} \PY{n}{step}\PY{o}{=}\PY{n}{np}\PY{o}{.}\PY{n}{pi}\PY{o}{/}\PY{l+m+mi}{12}\PY{p}{,}
                                   \PY{n}{description}\PY{o}{=}\PY{l+s+s2}{\PYZdq{}}\PY{l+s+s2}{phase}\PY{l+s+s2}{\PYZdq{}}\PY{p}{,}
                                   \PY{n}{value}\PY{o}{=}\PY{l+m+mf}{0.}\PY{p}{,} \PY{n}{layout}\PY{o}{=}\PY{n}{l\PYZus{}75}\PY{p}{)}
             \PY{n}{w\PYZus{}amplitude} \PY{o}{=} \PY{n}{Dropdown}\PY{p}{(}\PY{n}{options}\PY{o}{=}\PY{p}{\PYZob{}}\PY{l+s+s2}{\PYZdq{}}\PY{l+s+s2}{micro}\PY{l+s+s2}{\PYZdq{}} \PY{p}{:} \PY{o}{.}\PY{l+m+mi}{1}\PY{p}{,}
                                             \PY{l+s+s2}{\PYZdq{}}\PY{l+s+s2}{mini}\PY{l+s+s2}{\PYZdq{}} \PY{p}{:} \PY{o}{.}\PY{l+m+mi}{5}\PY{p}{,}
                                             \PY{l+s+s2}{\PYZdq{}}\PY{l+s+s2}{normal}\PY{l+s+s2}{\PYZdq{}} \PY{p}{:} \PY{l+m+mf}{1.}\PY{p}{,}
                                             \PY{l+s+s2}{\PYZdq{}}\PY{l+s+s2}{grand}\PY{l+s+s2}{\PYZdq{}} \PY{p}{:} \PY{l+m+mf}{3.}\PY{p}{,}
                                             \PY{l+s+s2}{\PYZdq{}}\PY{l+s+s2}{énorme}\PY{l+s+s2}{\PYZdq{}} \PY{p}{:} \PY{l+m+mf}{10.}\PY{p}{\PYZcb{}}\PY{p}{,}
                                    \PY{n}{value} \PY{o}{=} \PY{l+m+mf}{3.}\PY{p}{,}
                                    \PY{n}{description} \PY{o}{=} \PY{l+s+s2}{\PYZdq{}}\PY{l+s+s2}{amplitude}\PY{l+s+s2}{\PYZdq{}}\PY{p}{,}
                                    \PY{n}{layout} \PY{o}{=} \PY{n}{l\PYZus{}25}\PY{p}{)}
             \PY{n}{w\PYZus{}domain} \PY{o}{=} \PY{n}{IntSlider}\PY{p}{(}\PY{n+nb}{min}\PY{o}{=}\PY{l+m+mi}{1}\PY{p}{,} \PY{n+nb}{max}\PY{o}{=}\PY{l+m+mi}{10}\PY{p}{,} \PY{n}{description}\PY{o}{=}\PY{l+s+s2}{\PYZdq{}}\PY{l+s+s2}{dom. n * π}\PY{l+s+s2}{\PYZdq{}}\PY{p}{,} \PY{n}{layout}\PY{o}{=}\PY{n}{l\PYZus{}50}\PY{p}{)}
         
             \PY{c+c1}{\PYZsh{} make up a dashboard}
             \PY{n}{dashboard} \PY{o}{=} \PY{n}{VBox}\PY{p}{(}\PY{p}{[}\PY{n}{HBox}\PY{p}{(}\PY{p}{[}\PY{n}{w\PYZus{}amplitude}\PY{p}{,} \PY{n}{w\PYZus{}phase}\PY{p}{]}\PY{p}{)}\PY{p}{,}
                               \PY{n}{HBox}\PY{p}{(}\PY{p}{[}\PY{n}{w\PYZus{}domain}\PY{p}{,} \PY{n}{w\PYZus{}freq}\PY{p}{]}\PY{p}{)}\PY{p}{,}
                              \PY{p}{]}\PY{p}{)}
             \PY{n}{display}\PY{p}{(}\PY{n}{dashboard}\PY{p}{)}
             \PY{k}{return} \PY{n+nb}{dict}\PY{p}{(}\PY{n}{freq}\PY{o}{=}\PY{n}{w\PYZus{}freq}\PY{p}{,} \PY{n}{phase}\PY{o}{=}\PY{n}{w\PYZus{}phase}\PY{p}{,}
                         \PY{n}{amplitude}\PY{o}{=}\PY{n}{w\PYZus{}amplitude}\PY{p}{,} \PY{n}{domain}\PY{o}{=}\PY{n}{w\PYZus{}domain}\PY{p}{)}
\end{Verbatim}


    \begin{center}\rule{0.5\linewidth}{\linethickness}\end{center}

Avec tout ceci en place on peut montrer un dialogue interactif pour
changer tous les paramètres de sinus4.

    \begin{Verbatim}[commandchars=\\\{\},frame=single,framerule=0.3mm,rulecolor=\color{cellframecolor}]
{\color{incolor}In [{\color{incolor}22}]:} \PY{c+c1}{\PYZsh{} interactively call sinus4}
         \PY{c+c1}{\PYZsh{} attention il reste un bug:}
         \PY{c+c1}{\PYZsh{} au tout début rien ne s\PYZsq{}affiche,}
         \PY{c+c1}{\PYZsh{} il faut faire bouger au moins un réglage}
         \PY{n}{interactive\PYZus{}output}\PY{p}{(}\PY{n}{sinus4}\PY{p}{,} \PY{n}{my\PYZus{}dashboard}\PY{p}{(}\PY{p}{)}\PY{p}{)}
\end{Verbatim}


    
    
    
    \begin{verbatim}
Output()
    \end{verbatim}

    

    % Add a bibliography block to the postdoc
    
    
    
