    
    
    
    

    

    \hypertarget{compruxe9hensions-imbriquuxe9es}{%
\section{Compréhensions
imbriquées}\label{compruxe9hensions-imbriquuxe9es}}

    \hypertarget{compluxe9ments---niveau-intermuxe9diaire}{%
\subsection{Compléments - niveau
intermédiaire}\label{compluxe9ments---niveau-intermuxe9diaire}}

    \hypertarget{imbrications}{%
\subsubsection{Imbrications}\label{imbrications}}

    On peut également imbriquer plusieurs niveaux pour ne construire qu'une
seule liste, comme par exemple~:

    \begin{Verbatim}[commandchars=\\\{\}]
{\color{incolor}In [{\color{incolor}1}]:} \PY{p}{[}\PY{n}{n} \PY{o}{+} \PY{n}{p} \PY{k}{for} \PY{n}{n} \PY{o+ow}{in} \PY{p}{[}\PY{l+m+mi}{2}\PY{p}{,} \PY{l+m+mi}{4}\PY{p}{]} \PY{k}{for} \PY{n}{p} \PY{o+ow}{in} \PY{p}{[}\PY{l+m+mi}{10}\PY{p}{,} \PY{l+m+mi}{20}\PY{p}{,} \PY{l+m+mi}{30}\PY{p}{]}\PY{p}{]}
\end{Verbatim}


\begin{Verbatim}[commandchars=\\\{\}]
{\color{outcolor}Out[{\color{outcolor}1}]:} [12, 22, 32, 14, 24, 34]
\end{Verbatim}
            
    Bien sûr on peut aussi restreindre ces compréhensions, comme par
exemple~:

    \begin{Verbatim}[commandchars=\\\{\}]
{\color{incolor}In [{\color{incolor}2}]:} \PY{p}{[}\PY{n}{n} \PY{o}{+} \PY{n}{p} \PY{k}{for} \PY{n}{n} \PY{o+ow}{in} \PY{p}{[}\PY{l+m+mi}{2}\PY{p}{,} \PY{l+m+mi}{4}\PY{p}{]} \PY{k}{for} \PY{n}{p} \PY{o+ow}{in} \PY{p}{[}\PY{l+m+mi}{10}\PY{p}{,} \PY{l+m+mi}{20}\PY{p}{,} \PY{l+m+mi}{30}\PY{p}{]} \PY{k}{if} \PY{n}{n}\PY{o}{*}\PY{n}{p} \PY{o}{\PYZgt{}}\PY{o}{=} \PY{l+m+mi}{40}\PY{p}{]}
\end{Verbatim}


\begin{Verbatim}[commandchars=\\\{\}]
{\color{outcolor}Out[{\color{outcolor}2}]:} [22, 32, 14, 24, 34]
\end{Verbatim}
            
    Observez surtout que le résultat ci-dessus est une liste simple (de
profondeur 1), à comparer avec~:

    \begin{Verbatim}[commandchars=\\\{\}]
{\color{incolor}In [{\color{incolor}3}]:} \PY{p}{[}\PY{p}{[}\PY{n}{n} \PY{o}{+} \PY{n}{p} \PY{k}{for} \PY{n}{n} \PY{o+ow}{in} \PY{p}{[}\PY{l+m+mi}{2}\PY{p}{,} \PY{l+m+mi}{4}\PY{p}{]}\PY{p}{]} \PY{k}{for} \PY{n}{p} \PY{o+ow}{in} \PY{p}{[}\PY{l+m+mi}{10}\PY{p}{,} \PY{l+m+mi}{20}\PY{p}{,} \PY{l+m+mi}{30}\PY{p}{]}\PY{p}{]}
\end{Verbatim}


\begin{Verbatim}[commandchars=\\\{\}]
{\color{outcolor}Out[{\color{outcolor}3}]:} [[12, 14], [22, 24], [32, 34]]
\end{Verbatim}
            
    qui est de profondeur 2, et où les résultats atomiques apparaissent dans
un ordre différent.

    Un moyen mnémotechnique pour se souvenir dans quel ordre les
compréhensions imbriquées produisent leur résultat, est de penser à la
version ``naïve'' du code qui produirait le même résultat~; dans ce code
les clause \texttt{for} et \texttt{if} apparaissent \textbf{dans le même
ordre} que dans la compréhension~:

    \begin{Verbatim}[commandchars=\\\{\}]
{\color{incolor}In [{\color{incolor}4}]:} \PY{c+c1}{\PYZsh{} notre exemple :}
        \PY{c+c1}{\PYZsh{} [n + p for n in [2, 4] for p in [10, 20, 30] if n*p \PYZgt{}= 40]}
        
        \PY{c+c1}{\PYZsh{} est équivalent à ceci :}
        \PY{n}{resultat} \PY{o}{=} \PY{p}{[}\PY{p}{]}
        \PY{k}{for} \PY{n}{n} \PY{o+ow}{in} \PY{p}{[}\PY{l+m+mi}{2}\PY{p}{,} \PY{l+m+mi}{4}\PY{p}{]}\PY{p}{:}
            \PY{k}{for} \PY{n}{p} \PY{o+ow}{in} \PY{p}{[}\PY{l+m+mi}{10}\PY{p}{,} \PY{l+m+mi}{20}\PY{p}{,} \PY{l+m+mi}{30}\PY{p}{]}\PY{p}{:}
                \PY{k}{if} \PY{n}{n}\PY{o}{*}\PY{n}{p} \PY{o}{\PYZgt{}}\PY{o}{=} \PY{l+m+mi}{40}\PY{p}{:}
                    \PY{n}{resultat}\PY{o}{.}\PY{n}{append}\PY{p}{(}\PY{n}{n} \PY{o}{+} \PY{n}{p}\PY{p}{)}
        \PY{n}{resultat}
\end{Verbatim}


\begin{Verbatim}[commandchars=\\\{\}]
{\color{outcolor}Out[{\color{outcolor}4}]:} [22, 32, 14, 24, 34]
\end{Verbatim}
            
    \hypertarget{ordre-duxe9valuation-de-..-for-..-..-for-..}{%
\subsubsection{\texorpdfstring{Ordre d'évaluation de
\texttt{{[}{[}\ ..\ for\ ..\ {]}\ ..\ for\ ..\ {]}}}{Ordre d'évaluation de {[}{[} .. for .. {]} .. for .. {]}}}\label{ordre-duxe9valuation-de-..-for-..-..-for-..}}

    Pour rappel, on peut imbriquer des compréhensions de compréhensions.
Commençons par poser

    \begin{Verbatim}[commandchars=\\\{\}]
{\color{incolor}In [{\color{incolor}5}]:} \PY{n}{n} \PY{o}{=} \PY{l+m+mi}{4}
\end{Verbatim}


    On peut alors créer une liste de listes comme ceci~:

    \begin{Verbatim}[commandchars=\\\{\}]
{\color{incolor}In [{\color{incolor}6}]:} \PY{p}{[}\PY{p}{[}\PY{p}{(}\PY{n}{i}\PY{p}{,} \PY{n}{j}\PY{p}{)} \PY{k}{for} \PY{n}{i} \PY{o+ow}{in} \PY{n+nb}{range}\PY{p}{(}\PY{l+m+mi}{1}\PY{p}{,} \PY{n}{j} \PY{o}{+} \PY{l+m+mi}{1}\PY{p}{)}\PY{p}{]} \PY{k}{for} \PY{n}{j} \PY{o+ow}{in} \PY{n+nb}{range}\PY{p}{(}\PY{l+m+mi}{1}\PY{p}{,} \PY{n}{n} \PY{o}{+} \PY{l+m+mi}{1}\PY{p}{)}\PY{p}{]}
\end{Verbatim}


\begin{Verbatim}[commandchars=\\\{\}]
{\color{outcolor}Out[{\color{outcolor}6}]:} [[(1, 1)],
         [(1, 2), (2, 2)],
         [(1, 3), (2, 3), (3, 3)],
         [(1, 4), (2, 4), (3, 4), (4, 4)]]
\end{Verbatim}
            
    Et dans ce cas, très logiquement, l'évaluation se fait \textbf{en
commençant par la fin}, ou si on préfère \textbf{``par l'extérieur''},
c'est-à-dire que le code ci-dessus est équivalent à~:

    \begin{Verbatim}[commandchars=\\\{\}]
{\color{incolor}In [{\color{incolor}7}]:} \PY{c+c1}{\PYZsh{} en version bavarde, pour illustrer l\PYZsq{}ordre des \PYZdq{}for\PYZdq{}}
        \PY{n}{resultat\PYZus{}exterieur} \PY{o}{=} \PY{p}{[}\PY{p}{]}
        \PY{k}{for} \PY{n}{j} \PY{o+ow}{in} \PY{n+nb}{range}\PY{p}{(}\PY{l+m+mi}{1}\PY{p}{,} \PY{n}{n} \PY{o}{+} \PY{l+m+mi}{1}\PY{p}{)}\PY{p}{:}
            \PY{n}{resultat\PYZus{}interieur} \PY{o}{=} \PY{p}{[}\PY{p}{]}
            \PY{k}{for} \PY{n}{i} \PY{o+ow}{in} \PY{n+nb}{range}\PY{p}{(}\PY{l+m+mi}{1}\PY{p}{,} \PY{n}{j} \PY{o}{+} \PY{l+m+mi}{1}\PY{p}{)}\PY{p}{:}
                \PY{n}{resultat\PYZus{}interieur}\PY{o}{.}\PY{n}{append}\PY{p}{(}\PY{p}{(}\PY{n}{i}\PY{p}{,} \PY{n}{j}\PY{p}{)}\PY{p}{)}
            \PY{n}{resultat\PYZus{}exterieur}\PY{o}{.}\PY{n}{append}\PY{p}{(}\PY{n}{resultat\PYZus{}interieur}\PY{p}{)}
        \PY{n}{resultat\PYZus{}exterieur}
\end{Verbatim}


\begin{Verbatim}[commandchars=\\\{\}]
{\color{outcolor}Out[{\color{outcolor}7}]:} [[(1, 1)],
         [(1, 2), (2, 2)],
         [(1, 3), (2, 3), (3, 3)],
         [(1, 4), (2, 4), (3, 4), (4, 4)]]
\end{Verbatim}
            
    \hypertarget{avec-if}{%
\subsubsection{\texorpdfstring{Avec
\texttt{if}}{Avec if}}\label{avec-if}}

    Lorsqu'on assortit les compréhensions imbriquées de cette manière de
clauses \texttt{if}, l'ordre d'évaluation est tout aussi logique. Par
exemple, si on voulait se limiter - arbitrairement - aux lignes
correspondant à \texttt{j} pair, et aux diagonales où \texttt{i+j} est
pair, on écrirait~:

    \begin{Verbatim}[commandchars=\\\{\}]
{\color{incolor}In [{\color{incolor}8}]:} \PY{p}{[}\PY{p}{[}\PY{p}{(}\PY{n}{i}\PY{p}{,} \PY{n}{j}\PY{p}{)} \PY{k}{for} \PY{n}{i} \PY{o+ow}{in} \PY{n+nb}{range}\PY{p}{(}\PY{l+m+mi}{1}\PY{p}{,} \PY{n}{j} \PY{o}{+} \PY{l+m+mi}{1}\PY{p}{)} \PY{k}{if} \PY{p}{(}\PY{n}{i} \PY{o}{+} \PY{n}{j}\PY{p}{)}\PY{o}{\PYZpc{}}\PY{k}{2} == 0]
                 \PY{k}{for} \PY{n}{j} \PY{o+ow}{in} \PY{n+nb}{range}\PY{p}{(}\PY{l+m+mi}{1}\PY{p}{,} \PY{n}{n} \PY{o}{+} \PY{l+m+mi}{1}\PY{p}{)} \PY{k}{if} \PY{n}{j} \PY{o}{\PYZpc{}} \PY{l+m+mi}{2} \PY{o}{==} \PY{l+m+mi}{0}\PY{p}{]}
\end{Verbatim}


\begin{Verbatim}[commandchars=\\\{\}]
{\color{outcolor}Out[{\color{outcolor}8}]:} [[(2, 2)], [(2, 4), (4, 4)]]
\end{Verbatim}
            
    ce qui est équivalent à~:

    \begin{Verbatim}[commandchars=\\\{\}]
{\color{incolor}In [{\color{incolor}9}]:} \PY{c+c1}{\PYZsh{} en version bavarde à nouveau}
        \PY{n}{resultat\PYZus{}exterieur} \PY{o}{=} \PY{p}{[}\PY{p}{]}
        \PY{k}{for} \PY{n}{j} \PY{o+ow}{in} \PY{n+nb}{range}\PY{p}{(}\PY{l+m+mi}{1}\PY{p}{,} \PY{n}{n} \PY{o}{+} \PY{l+m+mi}{1}\PY{p}{)}\PY{p}{:}
            \PY{k}{if} \PY{n}{j} \PY{o}{\PYZpc{}} \PY{l+m+mi}{2} \PY{o}{==} \PY{l+m+mi}{0}\PY{p}{:}
                \PY{n}{resultat\PYZus{}interieur} \PY{o}{=} \PY{p}{[}\PY{p}{]}
                \PY{k}{for} \PY{n}{i} \PY{o+ow}{in} \PY{n+nb}{range}\PY{p}{(}\PY{l+m+mi}{1}\PY{p}{,} \PY{n}{j} \PY{o}{+} \PY{l+m+mi}{1}\PY{p}{)}\PY{p}{:}
                    \PY{k}{if} \PY{p}{(}\PY{n}{i} \PY{o}{+} \PY{n}{j}\PY{p}{)} \PY{o}{\PYZpc{}} \PY{l+m+mi}{2} \PY{o}{==} \PY{l+m+mi}{0}\PY{p}{:}
                        \PY{n}{resultat\PYZus{}interieur}\PY{o}{.}\PY{n}{append}\PY{p}{(}\PY{p}{(}\PY{n}{i}\PY{p}{,} \PY{n}{j}\PY{p}{)}\PY{p}{)}
                \PY{n}{resultat\PYZus{}exterieur}\PY{o}{.}\PY{n}{append}\PY{p}{(}\PY{n}{resultat\PYZus{}interieur}\PY{p}{)}
        \PY{n}{resultat\PYZus{}exterieur}
\end{Verbatim}


\begin{Verbatim}[commandchars=\\\{\}]
{\color{outcolor}Out[{\color{outcolor}9}]:} [[(2, 2)], [(2, 4), (4, 4)]]
\end{Verbatim}
            
    Le point important ici est que l'\textbf{ordre} dans lequel il faut lire
le code est \textbf{naturel}, et dicté par l'imbrication des
\texttt{{[}\ ..\ {]}}.

    \hypertarget{compluxe9ments---niveau-avancuxe9}{%
\subsection{Compléments - niveau
avancé}\label{compluxe9ments---niveau-avancuxe9}}

    \hypertarget{les-variables-de-boucle-fuient}{%
\subsubsection{\texorpdfstring{Les variables de boucle
\emph{fuient}}{Les variables de boucle fuient}}\label{les-variables-de-boucle-fuient}}

    Nous avons déjà signalé que les variables de boucle \textbf{restent
définies} après la sortie de la boucle, ainsi nous pouvons examiner~:

    \begin{Verbatim}[commandchars=\\\{\}]
{\color{incolor}In [{\color{incolor}10}]:} \PY{n}{i}\PY{p}{,} \PY{n}{j}
\end{Verbatim}


\begin{Verbatim}[commandchars=\\\{\}]
{\color{outcolor}Out[{\color{outcolor}10}]:} (4, 4)
\end{Verbatim}
            
    C'est pourquoi, afin de comparer les deux formes de compréhension
imbriquées nous allons explicitement retirer les variables \texttt{i} et
\texttt{j} de l'environnement

    \begin{Verbatim}[commandchars=\\\{\}]
{\color{incolor}In [{\color{incolor}11}]:} \PY{k}{del} \PY{n}{i}\PY{p}{,} \PY{n}{j}
\end{Verbatim}


    \hypertarget{ordre-duxe9valuation-de-..-for-..-for-..}{%
\subsubsection{\texorpdfstring{Ordre d'évaluation de
\texttt{{[}\ ..\ for\ ..\ for\ ..\ {]}}}{Ordre d'évaluation de {[} .. for .. for .. {]}}}\label{ordre-duxe9valuation-de-..-for-..-for-..}}

    Toujours pour rappel, on peut également construire une compréhension
imbriquée mais \textbf{à un seul niveau}. Dans une forme simple cela
donne~:

    \begin{Verbatim}[commandchars=\\\{\}]
{\color{incolor}In [{\color{incolor}12}]:} \PY{p}{[}\PY{p}{(}\PY{n}{x}\PY{p}{,} \PY{n}{y}\PY{p}{)} \PY{k}{for} \PY{n}{x} \PY{o+ow}{in} \PY{p}{[}\PY{l+m+mi}{1}\PY{p}{,} \PY{l+m+mi}{2}\PY{p}{]} \PY{k}{for} \PY{n}{y} \PY{o+ow}{in} \PY{p}{[}\PY{l+m+mi}{1}\PY{p}{,} \PY{l+m+mi}{2}\PY{p}{]}\PY{p}{]}
\end{Verbatim}


\begin{Verbatim}[commandchars=\\\{\}]
{\color{outcolor}Out[{\color{outcolor}12}]:} [(1, 1), (1, 2), (2, 1), (2, 2)]
\end{Verbatim}
            
    \textbf{Avertissement} méfiez-vous toutefois, car il est facile de ne
pas voir du premier coup d'oeil qu'ici on évalue les deux clauses
\texttt{for} \textbf{dans un ordre différent}.

    Pour mieux le voir, essayons de reprendre la logique de notre tout
premier exemple, mais avec une forme de double compréhension \emph{à
plat}~:

    \begin{Verbatim}[commandchars=\\\{\}]
{\color{incolor}In [{\color{incolor} }]:} \PY{c+c1}{\PYZsh{} NOTE}
        \PY{c+c1}{\PYZsh{} auto\PYZhy{}exec\PYZhy{}for\PYZhy{}latex has skipped execution of this cell}
        
        \PY{c+c1}{\PYZsh{} ceci ne fonctionne pas}
        \PY{c+c1}{\PYZsh{} NameError: name \PYZsq{}j\PYZsq{} is not defined}
        
        \PY{p}{[} \PY{p}{(}\PY{n}{i}\PY{p}{,} \PY{n}{j}\PY{p}{)} \PY{k}{for} \PY{n}{i} \PY{o+ow}{in} \PY{n+nb}{range}\PY{p}{(}\PY{l+m+mi}{1}\PY{p}{,} \PY{n}{j} \PY{o}{+} \PY{l+m+mi}{1}\PY{p}{)} \PY{k}{for} \PY{n}{j} \PY{o+ow}{in} \PY{n+nb}{range}\PY{p}{(}\PY{l+m+mi}{1}\PY{p}{,} \PY{n}{n} \PY{o}{+} \PY{l+m+mi}{1}\PY{p}{)} \PY{p}{]}
\end{Verbatim}


    On obtient une erreur, l'interpréteur se plaint à propos de la variable
\texttt{j} (c'est pourquoi nous l'avons effacée de l'environnement au
préalable).

    Ce qui se passe ici, c'est que, comme nous l'avons déjà mentionné en
semaine 3, le code que nous avons écrit est en fait équivalent à~:

    \begin{Verbatim}[commandchars=\\\{\}]
{\color{incolor}In [{\color{incolor} }]:} \PY{c+c1}{\PYZsh{} NOTE}
        \PY{c+c1}{\PYZsh{} auto\PYZhy{}exec\PYZhy{}for\PYZhy{}latex has skipped execution of this cell}
        
        \PY{c+c1}{\PYZsh{} la version bavarde de cette imbrication à plat, à nouveau :}
        \PY{c+c1}{\PYZsh{} [ (i, j) for i in range(1, j + 1) for j in range(1, n + 1) ]}
        \PY{c+c1}{\PYZsh{} serait}
        \PY{n}{resultat} \PY{o}{=} \PY{p}{[}\PY{p}{]}
        \PY{k}{for} \PY{n}{i} \PY{o+ow}{in} \PY{n+nb}{range}\PY{p}{(}\PY{l+m+mi}{1}\PY{p}{,} \PY{n}{j} \PY{o}{+} \PY{l+m+mi}{1}\PY{p}{)}\PY{p}{:}
            \PY{k}{for} \PY{n}{j} \PY{o+ow}{in} \PY{n+nb}{range}\PY{p}{(}\PY{l+m+mi}{1}\PY{p}{,} \PY{n}{n} \PY{o}{+} \PY{l+m+mi}{1}\PY{p}{)}\PY{p}{:}
                \PY{n}{resultat}\PY{o}{.}\PY{n}{append}\PY{p}{(}\PY{p}{(}\PY{n}{i}\PY{p}{,} \PY{n}{j}\PY{p}{)}\PY{p}{)}
\end{Verbatim}


    Et dans cette version * dépliée* on voit bien qu'en effet on utilise
\texttt{j} avant qu'elle ne soit définie.

    \hypertarget{conclusion}{%
\subsubsection{Conclusion}\label{conclusion}}

    La possibilité d'imbriquer des compréhensions avec plusieurs niveaux de
\texttt{for} dans la même compréhension est un trait qui peut rendre
service, car c'est une manière de simplifier la structure des entrées
(on passe essentiellement d'une liste de profondeur 2 à une liste de
profondeur 1).

Mais il faut savoir ne pas en abuser, et rester conscient de la
confusion qui peut en résulter, et en particulier être prudent et
prendre le temps de bien se relire. N'oublions pas non plus ces deux
phrases du Zen de Python~: ``\emph{Flat is better than nested}'' et
surtout ``\emph{Readability counts}''.


    % Add a bibliography block to the postdoc
    
    
    
