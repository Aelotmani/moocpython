    \hypertarget{muxe9thodes-spuxe9ciales-23}{%
\section{Méthodes spéciales (2/3)}\label{muxe9thodes-spuxe9ciales-23}}

    \hypertarget{compluxe9ment---niveau-avancuxe9}{%
\subsection{Complément - niveau
avancé}\label{compluxe9ment---niveau-avancuxe9}}

    Nous poursuivons dans ce complément la sélection de méthodes spéciales
entreprise en première partie.


    \hypertarget{contains__-__len__-__getitem__-et-apparentuxe9s}{%
\subsubsection{\texorpdfstring{\texttt{\_\_contains\_\_},
\texttt{\_\_len\_\_}, \texttt{\_\_getitem\_\_} et
apparentés}{\_\_contains\_\_, \_\_len\_\_, \_\_getitem\_\_ et apparentés}}\label{contains__-__len__-__getitem__-et-apparentuxe9s}}

    La méthode \texttt{\_\_contains\_\_} permet de donner un sens à~:

\begin{verbatim}
item in objet
\end{verbatim}

Sans grande surprise, elle prend en argument un objet et un item, et
doit renvoyer un booléen. Nous l'illustrons ci-dessous avec la classe
\texttt{DualQueue}.\\

    La méthode \texttt{\_\_len\_\_} est utilisée par la fonction
\emph{built-in} \texttt{len} pour retourner la longueur d'un objet.

    \hypertarget{la-classe-dualqueue}{%
\subparagraph{\texorpdfstring{La classe
\texttt{DualQueue}}{La classe DualQueue}\\\\}\label{la-classe-dualqueue}}

    Nous allons illustrer ceci avec un exemple de classe, un peu artificiel,
qui implémente une queue de type FIFO. Les objets sont d'abord admis
dans la file d'entrée (\texttt{add\_input}), puis déplacés dans la file
de sortie (\texttt{move\_input\_to\_output}), et enfin sortis
(\texttt{emit\_output}).\\

Clairement, cet exemple est à but uniquement pédagogique~; on veut
montrer comment une implémentation qui repose sur deux listes séparées
peut donner l'illusion d'une continuité, et se présenter comme un
container unique. De plus cette implémentation ne fait aucun contrôle
pour ne pas obscurcir le code.

    \begin{Verbatim}[commandchars=\\\{\}]
{\color{incolor}In [{\color{incolor}1}]:} \PY{k}{class} \PY{n+nc}{DualQueue}\PY{p}{:}
            \PY{l+s+sd}{\PYZdq{}\PYZdq{}\PYZdq{}Une double file d\PYZsq{}attente FIFO\PYZdq{}\PYZdq{}\PYZdq{}}
        
            \PY{k}{def} \PY{n+nf}{\PYZus{}\PYZus{}init\PYZus{}\PYZus{}}\PY{p}{(}\PY{n+nb+bp}{self}\PY{p}{)}\PY{p}{:}
                \PY{l+s+s2}{\PYZdq{}}\PY{l+s+s2}{constructeur, sans argument}\PY{l+s+s2}{\PYZdq{}}
                \PY{n+nb+bp}{self}\PY{o}{.}\PY{n}{inputs} \PY{o}{=} \PY{p}{[}\PY{p}{]}
                \PY{n+nb+bp}{self}\PY{o}{.}\PY{n}{outputs} \PY{o}{=} \PY{p}{[}\PY{p}{]}
        
            \PY{k}{def} \PY{n+nf}{\PYZus{}\PYZus{}repr\PYZus{}\PYZus{}} \PY{p}{(}\PY{n+nb+bp}{self}\PY{p}{)}\PY{p}{:}
                \PY{l+s+s2}{\PYZdq{}}\PY{l+s+s2}{affichage}\PY{l+s+s2}{\PYZdq{}}
                \PY{k}{return} \PY{n}{f}\PY{l+s+s2}{\PYZdq{}}\PY{l+s+s2}{\PYZlt{}DualQueue, inputs=}\PY{l+s+si}{\PYZob{}self.inputs\PYZcb{}}\PY{l+s+s2}{, outputs=}\PY{l+s+si}{\PYZob{}self.outputs\PYZcb{}}\PY{l+s+s2}{\PYZgt{}}\PY{l+s+s2}{\PYZdq{}}
        
            \PY{c+c1}{\PYZsh{} la partie qui nous intéresse ici}
            \PY{k}{def} \PY{n+nf}{\PYZus{}\PYZus{}contains\PYZus{}\PYZus{}}\PY{p}{(}\PY{n+nb+bp}{self}\PY{p}{,} \PY{n}{item}\PY{p}{)}\PY{p}{:}
                \PY{l+s+s2}{\PYZdq{}}\PY{l+s+s2}{appartenance d}\PY{l+s+s2}{\PYZsq{}}\PY{l+s+s2}{un objet à la queue}\PY{l+s+s2}{\PYZdq{}}
                \PY{k}{return} \PY{n}{item} \PY{o+ow}{in} \PY{n+nb+bp}{self}\PY{o}{.}\PY{n}{inputs} \PY{o+ow}{or} \PY{n}{item} \PY{o+ow}{in} \PY{n+nb+bp}{self}\PY{o}{.}\PY{n}{outputs}
            
            \PY{k}{def} \PY{n+nf}{\PYZus{}\PYZus{}len\PYZus{}\PYZus{}}\PY{p}{(}\PY{n+nb+bp}{self}\PY{p}{)}\PY{p}{:}
                \PY{l+s+s2}{\PYZdq{}}\PY{l+s+s2}{longueur de la queue}\PY{l+s+s2}{\PYZdq{}}
                \PY{k}{return} \PY{n+nb}{len}\PY{p}{(}\PY{n+nb+bp}{self}\PY{o}{.}\PY{n}{inputs}\PY{p}{)} \PY{o}{+} \PY{n+nb}{len}\PY{p}{(}\PY{n+nb+bp}{self}\PY{o}{.}\PY{n}{outputs}\PY{p}{)}        
        
            \PY{c+c1}{\PYZsh{} l\PYZsq{}interface publique de la classe}
            \PY{c+c1}{\PYZsh{} le plus simple possible et sans aucun contrôle}
            \PY{k}{def} \PY{n+nf}{add\PYZus{}input}\PY{p}{(}\PY{n+nb+bp}{self}\PY{p}{,} \PY{n}{item}\PY{p}{)}\PY{p}{:}
                \PY{l+s+s2}{\PYZdq{}}\PY{l+s+s2}{faire entrer un objet dans la queue d}\PY{l+s+s2}{\PYZsq{}}\PY{l+s+s2}{entrée}\PY{l+s+s2}{\PYZdq{}}
                \PY{n+nb+bp}{self}\PY{o}{.}\PY{n}{inputs}\PY{o}{.}\PY{n}{insert}\PY{p}{(}\PY{l+m+mi}{0}\PY{p}{,} \PY{n}{item}\PY{p}{)}
                
            \PY{k}{def} \PY{n+nf}{move\PYZus{}input\PYZus{}to\PYZus{}output} \PY{p}{(}\PY{n+nb+bp}{self}\PY{p}{)}\PY{p}{:}
                \PY{l+s+s2}{\PYZdq{}}\PY{l+s+s2}{l}\PY{l+s+s2}{\PYZsq{}}\PY{l+s+s2}{objet le plus ancien de la queue d}\PY{l+s+s2}{\PYZsq{}}\PY{l+s+s2}{entrée est promu}
                 \PY{l+s+s2}{dans la queue de sortie}\PY{l+s+s2}{\PYZdq{}}
                \PY{n+nb+bp}{self}\PY{o}{.}\PY{n}{outputs}\PY{o}{.}\PY{n}{insert}\PY{p}{(}\PY{l+m+mi}{0}\PY{p}{,} \PY{n+nb+bp}{self}\PY{o}{.}\PY{n}{inputs}\PY{o}{.}\PY{n}{pop}\PY{p}{(}\PY{p}{)}\PY{p}{)}
                
            \PY{k}{def} \PY{n+nf}{emit\PYZus{}output} \PY{p}{(}\PY{n+nb+bp}{self}\PY{p}{)}\PY{p}{:}
                \PY{l+s+s2}{\PYZdq{}}\PY{l+s+s2}{l}\PY{l+s+s2}{\PYZsq{}}\PY{l+s+s2}{objet le plus ancien de la queue de sortie est émis}\PY{l+s+s2}{\PYZdq{}}
                \PY{k}{return} \PY{n+nb+bp}{self}\PY{o}{.}\PY{n}{outputs}\PY{o}{.}\PY{n}{pop}\PY{p}{(}\PY{p}{)}
\end{Verbatim}


    \begin{Verbatim}[commandchars=\\\{\}]
{\color{incolor}In [{\color{incolor}2}]:} \PY{c+c1}{\PYZsh{} on construit une instance pour nos essais}
        \PY{n}{queue} \PY{o}{=} \PY{n}{DualQueue}\PY{p}{(}\PY{p}{)}
        \PY{n}{queue}\PY{o}{.}\PY{n}{add\PYZus{}input}\PY{p}{(}\PY{l+s+s1}{\PYZsq{}}\PY{l+s+s1}{zero}\PY{l+s+s1}{\PYZsq{}}\PY{p}{)}
        \PY{n}{queue}\PY{o}{.}\PY{n}{add\PYZus{}input}\PY{p}{(}\PY{l+s+s1}{\PYZsq{}}\PY{l+s+s1}{un}\PY{l+s+s1}{\PYZsq{}}\PY{p}{)}
        \PY{n}{queue}\PY{o}{.}\PY{n}{move\PYZus{}input\PYZus{}to\PYZus{}output}\PY{p}{(}\PY{p}{)}
        \PY{n}{queue}\PY{o}{.}\PY{n}{move\PYZus{}input\PYZus{}to\PYZus{}output}\PY{p}{(}\PY{p}{)}
        \PY{n}{queue}\PY{o}{.}\PY{n}{add\PYZus{}input}\PY{p}{(}\PY{l+s+s1}{\PYZsq{}}\PY{l+s+s1}{deux}\PY{l+s+s1}{\PYZsq{}}\PY{p}{)}
        \PY{n}{queue}\PY{o}{.}\PY{n}{add\PYZus{}input}\PY{p}{(}\PY{l+s+s1}{\PYZsq{}}\PY{l+s+s1}{trois}\PY{l+s+s1}{\PYZsq{}}\PY{p}{)}
        
        \PY{n+nb}{print}\PY{p}{(}\PY{n}{queue}\PY{p}{)}
\end{Verbatim}


    \begin{Verbatim}[commandchars=\\\{\}]
<DualQueue, inputs=['trois', 'deux'], outputs=['un', 'zero']>

    \end{Verbatim}

    \hypertarget{longueur-et-appartenance}{%
\subparagraph{Longueur et appartenance\\\\}\label{longueur-et-appartenance}}

    Avec cette première version de la classe \texttt{DualQueue} on peut
utiliser \texttt{len} et le test d'appartenance~:

    \begin{Verbatim}[commandchars=\\\{\}]
{\color{incolor}In [{\color{incolor}3}]:} \PY{n+nb}{print}\PY{p}{(}\PY{n}{f}\PY{l+s+s1}{\PYZsq{}}\PY{l+s+s1}{len() = }\PY{l+s+s1}{\PYZob{}}\PY{l+s+s1}{len(queue)\PYZcb{}}\PY{l+s+s1}{\PYZsq{}}\PY{p}{)}
        
        \PY{n+nb}{print}\PY{p}{(}\PY{n}{f}\PY{l+s+s2}{\PYZdq{}}\PY{l+s+s2}{deux appartient\PYZhy{}il ? }\PY{l+s+s2}{\PYZob{}}\PY{l+s+s2}{\PYZsq{}}\PY{l+s+s2}{deux}\PY{l+s+s2}{\PYZsq{}}\PY{l+s+s2}{ in queue\PYZcb{}}\PY{l+s+s2}{\PYZdq{}}\PY{p}{)}
        \PY{n+nb}{print}\PY{p}{(}\PY{n}{f}\PY{l+s+s2}{\PYZdq{}}\PY{l+s+s2}{1 appartient\PYZhy{}il ? }\PY{l+s+s2}{\PYZob{}}\PY{l+s+s2}{1 in queue\PYZcb{}}\PY{l+s+s2}{\PYZdq{}}\PY{p}{)}
\end{Verbatim}


    \begin{Verbatim}[commandchars=\\\{\}]
len() = 4
deux appartient-il ? True
1 appartient-il ? False

    \end{Verbatim}

    \hypertarget{accuxe8s-suxe9quentiel-accuxe8s-par-un-index-entier}{%
\subparagraph{Accès séquentiel (accès par un index
entier)\\\\}\label{accuxe8s-suxe9quentiel-accuxe8s-par-un-index-entier}}

    Lorsqu'on a la notion de longueur de l'objet avec \texttt{\_\_len\_\_},
il peut être opportun - quoique cela n'est pas imposé par le langage,
comme on vient de le voir - de proposer également un accès indexé par un
entier pour pouvoir faire~:

\begin{verbatim}
queue[1]
\end{verbatim}

\textbf{Pour ne pas répéter tout le code de la classe}, nous allons
étendre \texttt{DualQueue}~; pour cela nous définissons une fonction,
que nous affectons ensuite à \texttt{DualQueue.\_\_getitem\_\_}, comme
nous avons déjà eu l'occasion de le faire~:

    \begin{Verbatim}[commandchars=\\\{\}]
{\color{incolor}In [{\color{incolor}4}]:} \PY{c+c1}{\PYZsh{} une première version de DualQueue.\PYZus{}\PYZus{}getitem\PYZus{}\PYZus{}}
        \PY{c+c1}{\PYZsh{} pour uniquement l\PYZsq{}accès par index}
        
        \PY{c+c1}{\PYZsh{} on définit une fonction}
        \PY{k}{def} \PY{n+nf}{dual\PYZus{}queue\PYZus{}getitem} \PY{p}{(}\PY{n+nb+bp}{self}\PY{p}{,} \PY{n}{index}\PY{p}{)}\PY{p}{:}
            \PY{l+s+s2}{\PYZdq{}}\PY{l+s+s2}{redéfinit l}\PY{l+s+s2}{\PYZsq{}}\PY{l+s+s2}{accès [] séquentiel}\PY{l+s+s2}{\PYZdq{}}
        
            \PY{c+c1}{\PYZsh{} on vérifie que l\PYZsq{}index a un sens}
            \PY{k}{if} \PY{o+ow}{not} \PY{p}{(}\PY{l+m+mi}{0} \PY{o}{\PYZlt{}}\PY{o}{=} \PY{n}{index} \PY{o}{\PYZlt{}} \PY{n+nb}{len}\PY{p}{(}\PY{n+nb+bp}{self}\PY{p}{)}\PY{p}{)}\PY{p}{:}
                \PY{k}{raise} \PY{n+ne}{IndexError}\PY{p}{(}\PY{n}{f}\PY{l+s+s2}{\PYZdq{}}\PY{l+s+s2}{Mauvais indice }\PY{l+s+si}{\PYZob{}index\PYZcb{}}\PY{l+s+s2}{ pour DualQueue}\PY{l+s+s2}{\PYZdq{}}\PY{p}{)}
            \PY{c+c1}{\PYZsh{} on décide que l\PYZsq{}index 0 correspond à l\PYZsq{}élément le plus ancien}
            \PY{c+c1}{\PYZsh{} ce qui oblige à une petite gymnastique}
            \PY{n}{li} \PY{o}{=} \PY{n+nb}{len}\PY{p}{(}\PY{n+nb+bp}{self}\PY{o}{.}\PY{n}{inputs}\PY{p}{)}
            \PY{n}{lo} \PY{o}{=} \PY{n+nb}{len}\PY{p}{(}\PY{n+nb+bp}{self}\PY{o}{.}\PY{n}{outputs}\PY{p}{)}
            \PY{k}{if} \PY{n}{index} \PY{o}{\PYZlt{}} \PY{n}{lo}\PY{p}{:}
                \PY{k}{return} \PY{n+nb+bp}{self}\PY{o}{.}\PY{n}{outputs}\PY{p}{[}\PY{n}{lo} \PY{o}{\PYZhy{}} \PY{n}{index} \PY{o}{\PYZhy{}} \PY{l+m+mi}{1}\PY{p}{]}
            \PY{k}{else}\PY{p}{:}
                \PY{k}{return} \PY{n+nb+bp}{self}\PY{o}{.}\PY{n}{inputs}\PY{p}{[}\PY{n}{li} \PY{o}{\PYZhy{}} \PY{p}{(}\PY{n}{index}\PY{o}{\PYZhy{}}\PY{n}{lo}\PY{p}{)} \PY{o}{\PYZhy{}} \PY{l+m+mi}{1}\PY{p}{]}
        
        \PY{c+c1}{\PYZsh{} et on affecte cette fonction à l\PYZsq{}intérieur de la classe}
        \PY{n}{DualQueue}\PY{o}{.}\PY{n+nf+fm}{\PYZus{}\PYZus{}getitem\PYZus{}\PYZus{}} \PY{o}{=} \PY{n}{dual\PYZus{}queue\PYZus{}getitem}
\end{Verbatim}


    À présent, on peut \textbf{accéder} aux objets de la queue
\textbf{séquentiellement}~:

    \begin{Verbatim}[commandchars=\\\{\}]
{\color{incolor}In [{\color{incolor}5}]:} \PY{n+nb}{print}\PY{p}{(}\PY{n}{queue}\PY{p}{[}\PY{l+m+mi}{0}\PY{p}{]}\PY{p}{)}
\end{Verbatim}


    \begin{Verbatim}[commandchars=\\\{\}]
zero

    \end{Verbatim}

    ce qui lève la même exception qu'avec une vraie liste si on utilise un
mauvais index~:

    \begin{Verbatim}[commandchars=\\\{\}]
{\color{incolor}In [{\color{incolor}6}]:} \PY{k}{try}\PY{p}{:}
            \PY{n+nb}{print}\PY{p}{(}\PY{n}{queue}\PY{p}{[}\PY{l+m+mi}{5}\PY{p}{]}\PY{p}{)}
        \PY{k}{except} \PY{n+ne}{IndexError} \PY{k}{as} \PY{n}{e}\PY{p}{:}
            \PY{n+nb}{print}\PY{p}{(}\PY{l+s+s1}{\PYZsq{}}\PY{l+s+s1}{ERREUR}\PY{l+s+s1}{\PYZsq{}}\PY{p}{,}\PY{n}{e}\PY{p}{)}
\end{Verbatim}


    \begin{Verbatim}[commandchars=\\\{\}]
ERREUR Mauvais indice 5 pour DualQueue

    \end{Verbatim}

    \hypertarget{amuxe9lioration-accuxe8s-par-slice}{%
\subparagraph{Amélioration~: accès par
slice\\\\}\label{amuxe9lioration-accuxe8s-par-slice}}

    Si on veut aussi supporter l'accès par slice comme ceci~:

\begin{verbatim}
queue[1:3]
\end{verbatim}

il nous faut modifier la méthode \texttt{\_\_getitem\_\_}.\\

    Le second argument de \texttt{\_\_getitem\_\_} correspond naturellement
au contenu des crochets \texttt{{[}{]}}, on utilise donc
\texttt{isinstance} pour écrire un code qui s'adapte au type
d'indexation, comme ceci~:

    \begin{Verbatim}[commandchars=\\\{\}]
{\color{incolor}In [{\color{incolor}7}]:} \PY{c+c1}{\PYZsh{} une deuxième version de DualQueue.\PYZus{}\PYZus{}getitem\PYZus{}\PYZus{}}
        \PY{c+c1}{\PYZsh{} pour l\PYZsq{}accès par index et/ou par slice}
        
        \PY{k}{def} \PY{n+nf}{dual\PYZus{}queue\PYZus{}getitem} \PY{p}{(}\PY{n+nb+bp}{self}\PY{p}{,} \PY{n}{key}\PY{p}{)}\PY{p}{:}
            \PY{l+s+s2}{\PYZdq{}}\PY{l+s+s2}{redéfinit l}\PY{l+s+s2}{\PYZsq{}}\PY{l+s+s2}{accès par [] pour entiers, slices, et autres}\PY{l+s+s2}{\PYZdq{}}
        
            \PY{c+c1}{\PYZsh{} l\PYZsq{}accès par slice queue[1:3] }
            \PY{c+c1}{\PYZsh{} nous donne pour key un objet de type slice}
            \PY{k}{if} \PY{n+nb}{isinstance}\PY{p}{(}\PY{n}{key}\PY{p}{,} \PY{n+nb}{slice}\PY{p}{)}\PY{p}{:}
                \PY{c+c1}{\PYZsh{} key.indices donne les indices qui vont bien}
                \PY{k}{return} \PY{p}{[}\PY{n+nb+bp}{self}\PY{p}{[}\PY{n}{index}\PY{p}{]} \PY{k}{for} \PY{n}{index} \PY{o+ow}{in} \PY{n+nb}{range}\PY{p}{(}\PY{o}{*}\PY{n}{key}\PY{o}{.}\PY{n}{indices}\PY{p}{(}\PY{n+nb}{len}\PY{p}{(}\PY{n+nb+bp}{self}\PY{p}{)}\PY{p}{)}\PY{p}{)}\PY{p}{]}
        
            \PY{c+c1}{\PYZsh{} queue[3] nous donne pour key un entier}
            \PY{k}{elif} \PY{n+nb}{isinstance}\PY{p}{(}\PY{n}{key}\PY{p}{,} \PY{n+nb}{int}\PY{p}{)}\PY{p}{:}
                \PY{n}{index} \PY{o}{=} \PY{n}{key}
                \PY{c+c1}{\PYZsh{} on vérifie que l\PYZsq{}index a un sens}
                \PY{k}{if} \PY{n}{index} \PY{o}{\PYZlt{}} \PY{l+m+mi}{0} \PY{o+ow}{or} \PY{n}{index} \PY{o}{\PYZgt{}}\PY{o}{=} \PY{n+nb}{len}\PY{p}{(}\PY{n+nb+bp}{self}\PY{p}{)}\PY{p}{:}
                    \PY{k}{raise} \PY{n+ne}{IndexError}\PY{p}{(}\PY{n}{f}\PY{l+s+s2}{\PYZdq{}}\PY{l+s+s2}{Mauvais indice }\PY{l+s+si}{\PYZob{}index\PYZcb{}}\PY{l+s+s2}{ pour DualQueue}\PY{l+s+s2}{\PYZdq{}}\PY{p}{)}
                \PY{c+c1}{\PYZsh{} on décide que l\PYZsq{}index 0 correspond à l\PYZsq{}élément le plus ancien}
                \PY{c+c1}{\PYZsh{} ce qui oblige à une petite gymnastique}
                \PY{n}{li} \PY{o}{=} \PY{n+nb}{len}\PY{p}{(}\PY{n+nb+bp}{self}\PY{o}{.}\PY{n}{inputs}\PY{p}{)}
                \PY{n}{lo} \PY{o}{=} \PY{n+nb}{len}\PY{p}{(}\PY{n+nb+bp}{self}\PY{o}{.}\PY{n}{outputs}\PY{p}{)}
                \PY{k}{if} \PY{n}{index} \PY{o}{\PYZlt{}} \PY{n}{lo}\PY{p}{:}
                    \PY{k}{return} \PY{n+nb+bp}{self}\PY{o}{.}\PY{n}{outputs}\PY{p}{[}\PY{n}{lo}\PY{o}{\PYZhy{}}\PY{n}{index}\PY{o}{\PYZhy{}}\PY{l+m+mi}{1}\PY{p}{]}
                \PY{k}{else}\PY{p}{:}
                    \PY{k}{return} \PY{n+nb+bp}{self}\PY{o}{.}\PY{n}{inputs}\PY{p}{[}\PY{n}{li}\PY{o}{\PYZhy{}}\PY{p}{(}\PY{n}{index}\PY{o}{\PYZhy{}}\PY{n}{lo}\PY{p}{)}\PY{o}{\PYZhy{}}\PY{l+m+mi}{1}\PY{p}{]}
            \PY{c+c1}{\PYZsh{} queue [\PYZsq{}foo\PYZsq{}] n\PYZsq{}a pas de sens pour nous}
            \PY{k}{else}\PY{p}{:}
                \PY{k}{raise} \PY{n+ne}{KeyError}\PY{p}{(}\PY{n}{f}\PY{l+s+s2}{\PYZdq{}}\PY{l+s+s2}{[] avec type non reconnu }\PY{l+s+si}{\PYZob{}key\PYZcb{}}\PY{l+s+s2}{\PYZdq{}}\PY{p}{)}
        
        \PY{c+c1}{\PYZsh{} et on affecte cette fonction à l\PYZsq{}intérieur de la classe}
        \PY{n}{DualQueue}\PY{o}{.}\PY{n+nf+fm}{\PYZus{}\PYZus{}getitem\PYZus{}\PYZus{}} \PY{o}{=} \PY{n}{dual\PYZus{}queue\PYZus{}getitem}
\end{Verbatim}


    Maintenant on peut accéder par slice~:

    \begin{Verbatim}[commandchars=\\\{\}]
{\color{incolor}In [{\color{incolor}8}]:} \PY{n}{queue}\PY{p}{[}\PY{l+m+mi}{1}\PY{p}{:}\PY{l+m+mi}{3}\PY{p}{]}
\end{Verbatim}


\begin{Verbatim}[commandchars=\\\{\}]
{\color{outcolor}Out[{\color{outcolor}8}]:} ['un', 'deux']
\end{Verbatim}
            
    Et on reçoit bien une exception si on essaie d'accéder par clé~:

    \begin{Verbatim}[commandchars=\\\{\}]
{\color{incolor}In [{\color{incolor}9}]:} \PY{k}{try}\PY{p}{:}
            \PY{n}{queue}\PY{p}{[}\PY{l+s+s1}{\PYZsq{}}\PY{l+s+s1}{key}\PY{l+s+s1}{\PYZsq{}}\PY{p}{]}
        \PY{k}{except} \PY{n+ne}{KeyError} \PY{k}{as} \PY{n}{e}\PY{p}{:}
            \PY{n+nb}{print}\PY{p}{(}\PY{n}{f}\PY{l+s+s2}{\PYZdq{}}\PY{l+s+s2}{OOPS: }\PY{l+s+s2}{\PYZob{}}\PY{l+s+s2}{type(e).\PYZus{}\PYZus{}name\PYZus{}\PYZus{}\PYZcb{}: }\PY{l+s+si}{\PYZob{}e\PYZcb{}}\PY{l+s+s2}{\PYZdq{}}\PY{p}{)}
\end{Verbatim}


    \begin{Verbatim}[commandchars=\\\{\}]
OOPS: KeyError: '[] avec type non reconnu key'

    \end{Verbatim}

    \hypertarget{lobjet-est-ituxe9rable-muxeame-sans-avoir-__iter__}{%
\subparagraph{\texorpdfstring{L'objet est itérable (même sans avoir
\texttt{\_\_iter\_\_})}{L'objet est itérable (même sans avoir \_\_iter\_\_)}\\\\}\label{lobjet-est-ituxe9rable-muxeame-sans-avoir-__iter__}}

    Avec seulement \texttt{\_\_getitem\_\_}, on peut \textbf{faire une
boucle} sur l'objet queue. On l'a mentionné rapidement dans la séquence
sur les itérateurs, mais la \textbf{méthode \texttt{\_\_iter\_\_} n'est
pas la seule façon} de rendre un objet itérable~:

    \begin{Verbatim}[commandchars=\\\{\}]
{\color{incolor}In [{\color{incolor}10}]:} \PY{c+c1}{\PYZsh{} grâce à \PYZus{}\PYZus{}getitem\PYZus{}\PYZus{} on a rendu les }
         \PY{c+c1}{\PYZsh{} objets de type DualQueue itérables}
         \PY{k}{for} \PY{n}{item} \PY{o+ow}{in} \PY{n}{queue}\PY{p}{:}
            \PY{n+nb}{print}\PY{p}{(}\PY{n}{item}\PY{p}{)}
\end{Verbatim}


    \begin{Verbatim}[commandchars=\\\{\}]
zero
un
deux
trois

    \end{Verbatim}

    \hypertarget{on-peut-faire-un-test-sur-lobjet}{%
\subparagraph{On peut faire un test sur
l'objet\\\\}\label{on-peut-faire-un-test-sur-lobjet}}

    De manière similaire, même sans la méthode \texttt{\_\_bool\_\_}, cette
classe sait \textbf{faire des tests de manière correcte} grâce
uniquement à la méthode \texttt{\_\_len\_\_}~:

    \begin{Verbatim}[commandchars=\\\{\}]
{\color{incolor}In [{\color{incolor}11}]:} \PY{c+c1}{\PYZsh{} un test fait directement sur la queue}
         \PY{k}{if} \PY{n}{queue}\PY{p}{:}
             \PY{n+nb}{print}\PY{p}{(}\PY{n}{f}\PY{l+s+s2}{\PYZdq{}}\PY{l+s+s2}{La queue }\PY{l+s+si}{\PYZob{}queue\PYZcb{}}\PY{l+s+s2}{ est considérée comme True}\PY{l+s+s2}{\PYZdq{}}\PY{p}{)}
\end{Verbatim}


    \begin{Verbatim}[commandchars=\\\{\}]
La queue <DualQueue, inputs=['trois', 'deux'], outputs=['un', 'zero']> est considérée
comme True

    \end{Verbatim}

    \begin{Verbatim}[commandchars=\\\{\}]
{\color{incolor}In [{\color{incolor}12}]:} \PY{c+c1}{\PYZsh{} le même test sur une queue vide}
         \PY{n}{empty} \PY{o}{=} \PY{n}{DualQueue}\PY{p}{(}\PY{p}{)}
         
         \PY{c+c1}{\PYZsh{} maintenant le test est négatif (notez bien le *not* ici)}
         \PY{k}{if} \PY{o+ow}{not} \PY{n}{empty}\PY{p}{:}
             \PY{n+nb}{print}\PY{p}{(}\PY{n}{f}\PY{l+s+s2}{\PYZdq{}}\PY{l+s+s2}{La queue }\PY{l+s+si}{\PYZob{}empty\PYZcb{}}\PY{l+s+s2}{ est considérée comme False}\PY{l+s+s2}{\PYZdq{}}\PY{p}{)}
\end{Verbatim}


    \begin{Verbatim}[commandchars=\\\{\}]
La queue <DualQueue, inputs=[], outputs=[]> est considérée comme False

    \end{Verbatim}


    \hypertarget{call__-et-les-callables}{%
\subsubsection{\texorpdfstring{\texttt{\_\_call\_\_} et les
\emph{callables}}{\_\_call\_\_ et les callables}}\label{call__-et-les-callables}}

    Le langage introduit de manière similaire la notion de
\textbf{\emph{callable}} - littéralement, qui peut être appelé. L'idée
est très simple, on cherche à donner un sens à un fragment de code du
genre de~:

    \begin{verbatim}
# on crée une instance
objet = Classe(arguments)
\end{verbatim}

et c'est l'objet (Attention~: \textbf{l'objet, pas la classe}) qu'on
utilise comme une fonction

\begin{verbatim}
objet(arg1, arg2)
\end{verbatim}

    Le protocole ici est très simple~; cette dernière ligne a un sens en
Python dès lors que~:

\begin{itemize}
\tightlist
\item
  \texttt{objet} possède une méthode \texttt{\_\_call\_\_}~;
\item
  et que celle-ci peut être envoyée à \texttt{objet} avec les arguments
  \texttt{arg1,\ arg2}~;
\item
  et c'est ce résultat qui sera alors retourné par
  \texttt{objet(arg1,\ arg2)}~:
\end{itemize}

    \begin{verbatim}
objet(arg1, arg2) ⟺ objet.__call__(arg1, arg2)
\end{verbatim}

    Voyons cela sur un exemple~:

    \begin{Verbatim}[commandchars=\\\{\}]
{\color{incolor}In [{\color{incolor}13}]:} \PY{k}{class} \PY{n+nc}{PlusClosure}\PY{p}{:}
             \PY{l+s+sd}{\PYZdq{}\PYZdq{}\PYZdq{}Une classe callable qui permet de faire un peu comme la }
         \PY{l+s+sd}{    fonction built\PYZhy{}in sum mais en ajoutant une valeur initiale\PYZdq{}\PYZdq{}\PYZdq{}}
             \PY{k}{def} \PY{n+nf}{\PYZus{}\PYZus{}init\PYZus{}\PYZus{}}\PY{p}{(}\PY{n+nb+bp}{self}\PY{p}{,} \PY{n}{initial}\PY{p}{)}\PY{p}{:}
                 \PY{n+nb+bp}{self}\PY{o}{.}\PY{n}{initial} \PY{o}{=} \PY{n}{initial}
             \PY{k}{def} \PY{n+nf}{\PYZus{}\PYZus{}call\PYZus{}\PYZus{}}\PY{p}{(}\PY{n+nb+bp}{self}\PY{p}{,} \PY{o}{*}\PY{n}{args}\PY{p}{)}\PY{p}{:}
                 \PY{k}{return} \PY{n+nb+bp}{self}\PY{o}{.}\PY{n}{initial} \PY{o}{+} \PY{n+nb}{sum}\PY{p}{(}\PY{n}{args}\PY{p}{)}
             
         \PY{c+c1}{\PYZsh{} on crée une instance avec une valeur initiale 2 pour la somme}
         \PY{n}{plus2} \PY{o}{=} \PY{n}{PlusClosure} \PY{p}{(}\PY{l+m+mi}{2}\PY{p}{)}
\end{Verbatim}


    \begin{Verbatim}[commandchars=\\\{\}]
{\color{incolor}In [{\color{incolor}14}]:} \PY{c+c1}{\PYZsh{} on peut maintenant utiliser cet objet }
         \PY{c+c1}{\PYZsh{} comme une fonction qui fait sum(*arg)+2}
         \PY{n}{plus2}\PY{p}{(}\PY{p}{)}
\end{Verbatim}


\begin{Verbatim}[commandchars=\\\{\}]
{\color{outcolor}Out[{\color{outcolor}14}]:} 2
\end{Verbatim}
            
    \begin{Verbatim}[commandchars=\\\{\}]
{\color{incolor}In [{\color{incolor}15}]:} \PY{n}{plus2}\PY{p}{(}\PY{l+m+mi}{1}\PY{p}{)}
\end{Verbatim}


\begin{Verbatim}[commandchars=\\\{\}]
{\color{outcolor}Out[{\color{outcolor}15}]:} 3
\end{Verbatim}
            
    \begin{Verbatim}[commandchars=\\\{\}]
{\color{incolor}In [{\color{incolor}16}]:} \PY{n}{plus2}\PY{p}{(}\PY{l+m+mi}{1}\PY{p}{,} \PY{l+m+mi}{2}\PY{p}{)}
\end{Verbatim}


\begin{Verbatim}[commandchars=\\\{\}]
{\color{outcolor}Out[{\color{outcolor}16}]:} 5
\end{Verbatim}
            
    Pour ceux qui connaissent, nous avons choisi à dessein un exemple qui
s'apparente à
\href{http://en.wikipedia.org/wiki/Closure_\%28computer_programming\%29}{une
clôture}. Nous reviendrons sur cette notion de \emph{callable} lorsque
nous verrons les décorateurs en semaine 9.