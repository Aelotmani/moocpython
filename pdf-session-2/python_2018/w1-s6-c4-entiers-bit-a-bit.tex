    
    
    
    

    

    \hypertarget{opuxe9rations-bit-uxe0-bit-bitwise}{%
\section{\texorpdfstring{Opérations \emph{bit à bit}
(\emph{bitwise})}{Opérations bit à bit (bitwise)}}\label{opuxe9rations-bit-uxe0-bit-bitwise}}

    \hypertarget{compluxe9ments---niveau-avancuxe9}{%
\subsection{Compléments - niveau
avancé}\label{compluxe9ments---niveau-avancuxe9}}

    Les compléments ci-dessous expliquent des fonctions évoluées sur les
entiers. Les débutants en programmation peuvent sans souci sauter cette
partie en cas de difficultés.

    \hypertarget{opuxe9rations-logiques-et-ou-et-ou-exclusif}{%
\subsubsection{\texorpdfstring{Opérations logiques~: \emph{ET}
\texttt{\&}, \emph{OU} \texttt{\textbar{}} et \emph{OU} exclusif
\texttt{\^{}}}{Opérations logiques~: ET \&, OU \textbar{} et OU exclusif \^{}}}\label{opuxe9rations-logiques-et-ou-et-ou-exclusif}}

    Il est possible aussi de faire des opérations \emph{bit à bit} sur les
nombres entiers. Le plus simple est de penser à l'écriture du nombre en
base 2.

Considérons par exemple deux entiers constants dans cet exercice

    \begin{Verbatim}[commandchars=\\\{\},frame=single,framerule=0.3mm,rulecolor=\color{cellframecolor}]
{\color{incolor}In [{\color{incolor}1}]:} \PY{n}{x49} \PY{o}{=} \PY{l+m+mi}{49}
        \PY{n}{y81} \PY{o}{=} \PY{l+m+mi}{81}
\end{Verbatim}


    Ce qui nous donne comme décomposition binaire~:

\(\begin{array}{rcccccc} x49 & = & 49 & = & 32 + 16 + 1 & \rightarrow &(0,1,1,0,0,0,1) \\ y81 & = & 81 & = & 64 + 16 + 1 & \rightarrow &(1,0,1,0,0,0,1) \end{array}\)

    Pour comprendre comment passer de \(32 + 16 + 1\) à \((0,1,1,0,0,0,1)\)
il suffit d'observer que~:

\(32 + 16 + 1 = \textbf{0}*2^6 + \textbf{1}*2^5 + \textbf{1}*2^4 + \textbf{0}*2^3 + \textbf{0}*2^2 + \textbf{0}*2^1 + \textbf{1}*2^0\)

    \hypertarget{et-logique-opuxe9rateur}{%
\paragraph{\texorpdfstring{\emph{ET} logique~: opérateur
\texttt{\&}}{ET logique~: opérateur \&}}\label{et-logique-opuxe9rateur}}

    L'opération logique \texttt{\&} va faire un \emph{et} logique bit à bit
entre les opérandes, ainsi

    \begin{Verbatim}[commandchars=\\\{\},frame=single,framerule=0.3mm,rulecolor=\color{cellframecolor}]
{\color{incolor}In [{\color{incolor}2}]:} \PY{n}{x49} \PY{o}{\PYZam{}} \PY{n}{y81}
\end{Verbatim}


\begin{Verbatim}[commandchars=\\\{\},frame=single,framerule=0.3mm,rulecolor=\color{cellframecolor}]
{\color{outcolor}Out[{\color{outcolor}2}]:} 17
\end{Verbatim}
            
    Et en effet~:

\(\begin{array}{rcl} x49 & \rightarrow & (0,1,1,0,0,0,1) \\ y81 & \rightarrow & (1,0,1,0,0,0,1) \\ x49\ \&\ y81 & \rightarrow & (0,0,1,0,0,0,1) \rightarrow 16 + 1 \rightarrow 17 \end{array}\)

    \hypertarget{ou-logique-opuxe9rateur}{%
\paragraph{\texorpdfstring{\emph{OU} logique~: opérateur
\texttt{\textbar{}}}{OU logique~: opérateur \textbar{}}}\label{ou-logique-opuxe9rateur}}

    De même, l'opérateur logique \texttt{\textbar{}} fait simplement un
\emph{ou} logique, comme ceci~:

    \begin{Verbatim}[commandchars=\\\{\},frame=single,framerule=0.3mm,rulecolor=\color{cellframecolor}]
{\color{incolor}In [{\color{incolor}3}]:} \PY{n}{x49} \PY{o}{|} \PY{n}{y81}
\end{Verbatim}


\begin{Verbatim}[commandchars=\\\{\},frame=single,framerule=0.3mm,rulecolor=\color{cellframecolor}]
{\color{outcolor}Out[{\color{outcolor}3}]:} 113
\end{Verbatim}
            
    On s'y retrouve parce que~:

\(\begin{array}{rcl} x49 & \rightarrow & (0,1,1,0,0,0,1) \\ y81 & \rightarrow & (1,0,1,0,0,0,1) \\ x49\ |\ y81 & \rightarrow & (1,1,1,0,0,0,1) \rightarrow 64 + 32 + 16 + 1 \rightarrow 113 \end{array}\)

    \hypertarget{ou-exclusif-opuxe9rateur}{%
\paragraph{\texorpdfstring{\emph{OU} exclusif~: opérateur
\texttt{\^{}}}{OU exclusif~: opérateur \^{}}}\label{ou-exclusif-opuxe9rateur}}

    Enfin, on peut également faire la même opération à base de \emph{ou}
exclusif avec l'opérateur \texttt{\^{}}~:

    \begin{Verbatim}[commandchars=\\\{\},frame=single,framerule=0.3mm,rulecolor=\color{cellframecolor}]
{\color{incolor}In [{\color{incolor}4}]:} \PY{n}{x49} \PY{o}{\PYZca{}} \PY{n}{y81}
\end{Verbatim}


\begin{Verbatim}[commandchars=\\\{\},frame=single,framerule=0.3mm,rulecolor=\color{cellframecolor}]
{\color{outcolor}Out[{\color{outcolor}4}]:} 96
\end{Verbatim}
            
    Je vous laisse le soin de décortiquer le calcul à titre d'exercice (le
\emph{ou} exclusif de deux bits est vrai si et seulement si exactement
une des deux entrées est vraie).

    \hypertarget{duxe9calages}{%
\subsubsection{Décalages}\label{duxe9calages}}

    Un décalage \textbf{à gauche} de, par exemple, 4 positions, revient à
décaler tout le champ de bits de 4 cases à gauche (les 4 nouveaux bits
insérés sont toujours des 0). C'est donc équivalent à une
\textbf{multiplication} par \(2^4 = 16\)~:

    \begin{Verbatim}[commandchars=\\\{\},frame=single,framerule=0.3mm,rulecolor=\color{cellframecolor}]
{\color{incolor}In [{\color{incolor}5}]:} \PY{n}{x49} \PY{o}{\PYZlt{}\PYZlt{}} \PY{l+m+mi}{4}
\end{Verbatim}


\begin{Verbatim}[commandchars=\\\{\},frame=single,framerule=0.3mm,rulecolor=\color{cellframecolor}]
{\color{outcolor}Out[{\color{outcolor}5}]:} 784
\end{Verbatim}
            
    \(\begin{array}{rcl} x49 & \rightarrow & (0,1,1,0,0,0,1) \\ x49\ <<\ 4 & \rightarrow & (0,1,1,0,0,0,1,0,0,0,0) \rightarrow 512 + 256 + 16 \rightarrow 784 \end{array}\)

    De la même manière, le décalage à \textbf{droite} de \(n\) revient à une
\textbf{division} par \(2^n\) (plus précisément, le quotient de la
division)~:

    \begin{Verbatim}[commandchars=\\\{\},frame=single,framerule=0.3mm,rulecolor=\color{cellframecolor}]
{\color{incolor}In [{\color{incolor}6}]:} \PY{n}{x49} \PY{o}{\PYZgt{}\PYZgt{}} \PY{l+m+mi}{4}
\end{Verbatim}


\begin{Verbatim}[commandchars=\\\{\},frame=single,framerule=0.3mm,rulecolor=\color{cellframecolor}]
{\color{outcolor}Out[{\color{outcolor}6}]:} 3
\end{Verbatim}
            
    \(\begin{array}{rcl} x49 & \rightarrow & (0,1,1,0,0,0,1) \\ x49\ >>\ 4 & \rightarrow & (0,0,0,0,0,1,1) \rightarrow 2 + 1 \rightarrow 3 \end{array}\)

    \hypertarget{une-astuce}{%
\subsubsection{Une astuce}\label{une-astuce}}

    On peut utiliser la fonction \emph{built-in} \texttt{bin} pour calculer
la représentation binaire d'un entier. Attention, la valeur de retour
est une chaîne de caractères de type \texttt{str}~:

    \begin{Verbatim}[commandchars=\\\{\},frame=single,framerule=0.3mm,rulecolor=\color{cellframecolor}]
{\color{incolor}In [{\color{incolor}7}]:} \PY{n+nb}{bin}\PY{p}{(}\PY{n}{x49}\PY{p}{)}
\end{Verbatim}


\begin{Verbatim}[commandchars=\\\{\},frame=single,framerule=0.3mm,rulecolor=\color{cellframecolor}]
{\color{outcolor}Out[{\color{outcolor}7}]:} '0b110001'
\end{Verbatim}
            
    Dans l'autre sens, on peut aussi entrer un entier directement en base 2
comme ceci~:

    \begin{Verbatim}[commandchars=\\\{\},frame=single,framerule=0.3mm,rulecolor=\color{cellframecolor}]
{\color{incolor}In [{\color{incolor}8}]:} \PY{n}{x49bis} \PY{o}{=} \PY{l+m+mb}{0b110001}
        \PY{n}{x49bis} \PY{o}{==} \PY{n}{x49}
\end{Verbatim}


\begin{Verbatim}[commandchars=\\\{\},frame=single,framerule=0.3mm,rulecolor=\color{cellframecolor}]
{\color{outcolor}Out[{\color{outcolor}8}]:} True
\end{Verbatim}
            
    Ici, comme on le voit, \texttt{x49bis} est bien un entier.

    \hypertarget{pour-en-savoir-plus}{%
\subsubsection{Pour en savoir plus}\label{pour-en-savoir-plus}}

    \href{https://docs.python.org/3/library/stdtypes.html\#bitwise-operations-on-integer-types}{Section
de la documentation Python}.


    % Add a bibliography block to the postdoc
    
    
    
