    
    
    
    

    

    \hypertarget{exercice}{%
\section{Exercice}\label{exercice}}

    \hypertarget{niveau-basique}{%
\subsection{Niveau basique}\label{niveau-basique}}

    \begin{Verbatim}[commandchars=\\\{\},frame=single,framerule=0.3mm,rulecolor=\color{cellframecolor}]
{\color{incolor}In [{\color{incolor}1}]:} \PY{k+kn}{from} \PY{n+nn}{corrections}\PY{n+nn}{.}\PY{n+nn}{exo\PYZus{}taxes} \PY{k}{import} \PY{n}{exo\PYZus{}taxes}
\end{Verbatim}


    On se propose d'écrire une fonction \texttt{taxes} qui calcule le
montant de l'impôt sur le revenu au Royaume-Uni.

    Le barème est \href{https://www.gov.uk/income-tax-rates}{publié ici par
le gouvernement anglais}, voici les données utilisées pour l'exercice~:

    \begin{longtable}[]{@{}lll@{}}
\toprule
Tranche & Revenu imposable & Taux\tabularnewline
\midrule
\endhead
Non imposable & jusque £11.500 & 0\%\tabularnewline
Taux de base & £11.501 à £45.000 & 20\%\tabularnewline
Taux élevé & £45.001 à £150.000 & 40\%\tabularnewline
Taux supplémentaire & au delà de £150.000 & 45\%\tabularnewline
\bottomrule
\end{longtable}

    Donc naturellement il s'agit d'écrire une fonction qui prend en argument
le revenu imposable, et retourne le montant de l'impôt, \textbf{arrondi
à l'entier inférieur}.

    \begin{Verbatim}[commandchars=\\\{\},frame=single,framerule=0.3mm,rulecolor=\color{cellframecolor}]
{\color{incolor}In [{\color{incolor}2}]:} \PY{n}{exo\PYZus{}taxes}\PY{o}{.}\PY{n}{example}\PY{p}{(}\PY{p}{)}
\end{Verbatim}


\begin{Verbatim}[commandchars=\\\{\},frame=single,framerule=0.3mm,rulecolor=\color{cellframecolor}]
{\color{outcolor}Out[{\color{outcolor}2}]:} <IPython.core.display.HTML object>
\end{Verbatim}
            
    \textbf{Indices}

\begin{itemize}
\tightlist
\item
  évidemment on parle ici d'une fonction continue~;
\item
  aussi en termes de programmation, je vous encourage à séparer la
  définition des tranches de la fonction en elle-même.
\end{itemize}

    \begin{Verbatim}[commandchars=\\\{\},frame=single,framerule=0.3mm,rulecolor=\color{cellframecolor}]
{\color{incolor}In [{\color{incolor}3}]:} \PY{k}{def} \PY{n+nf}{taxes}\PY{p}{(}\PY{n}{income}\PY{p}{)}\PY{p}{:}
            \PY{c+c1}{\PYZsh{} ce n\PYZsq{}est pas la bonne réponse}
            \PY{k}{return} \PY{p}{(}\PY{n}{income}\PY{o}{\PYZhy{}}\PY{l+m+mi}{11}\PY{n}{\PYZus{}500}\PY{p}{)} \PY{o}{*} \PY{p}{(}\PY{l+m+mi}{20}\PY{o}{/}\PY{l+m+mi}{100}\PY{p}{)}
\end{Verbatim}


    \begin{Verbatim}[commandchars=\\\{\},frame=single,framerule=0.3mm,rulecolor=\color{cellframecolor}]
{\color{incolor}In [{\color{incolor} }]:} \PY{c+c1}{\PYZsh{} NOTE}
        \PY{c+c1}{\PYZsh{} auto\PYZhy{}exec\PYZhy{}for\PYZhy{}latex has skipped execution of this cell}
        
        \PY{n}{exo\PYZus{}taxes}\PY{o}{.}\PY{n}{correction}\PY{p}{(}\PY{n}{taxes}\PY{p}{)}
\end{Verbatim}


    \hypertarget{repruxe9sentation-graphique}{%
\subparagraph{Représentation
graphique}\label{repruxe9sentation-graphique}}

    Comme d'habitude vous pouvez voir la représentation graphique de votre
fonction~:

    \begin{Verbatim}[commandchars=\\\{\},frame=single,framerule=0.3mm,rulecolor=\color{cellframecolor}]
{\color{incolor}In [{\color{incolor}4}]:} \PY{k+kn}{import} \PY{n+nn}{numpy} \PY{k}{as} \PY{n+nn}{np}
        \PY{k+kn}{import} \PY{n+nn}{matplotlib}\PY{n+nn}{.}\PY{n+nn}{pyplot} \PY{k}{as} \PY{n+nn}{plt}
\end{Verbatim}


    \begin{Verbatim}[commandchars=\\\{\},frame=single,framerule=0.3mm,rulecolor=\color{cellframecolor}]
{\color{incolor}In [{\color{incolor}5}]:} \PY{o}{\PYZpc{}}\PY{k}{matplotlib} inline
        \PY{n}{plt}\PY{o}{.}\PY{n}{ion}\PY{p}{(}\PY{p}{)}
\end{Verbatim}


    \begin{Verbatim}[commandchars=\\\{\},frame=single,framerule=0.3mm,rulecolor=\color{cellframecolor}]
{\color{incolor}In [{\color{incolor}6}]:} \PY{n}{X} \PY{o}{=} \PY{n}{np}\PY{o}{.}\PY{n}{linspace}\PY{p}{(}\PY{l+m+mi}{0}\PY{p}{,} \PY{l+m+mi}{200}\PY{n}{\PYZus{}000}\PY{p}{)}
        \PY{n}{Y} \PY{o}{=} \PY{p}{[}\PY{n}{taxes}\PY{p}{(}\PY{n}{x}\PY{p}{)} \PY{k}{for} \PY{n}{x} \PY{o+ow}{in} \PY{n}{X}\PY{p}{]}
        \PY{n}{plt}\PY{o}{.}\PY{n}{plot}\PY{p}{(}\PY{n}{X}\PY{p}{,} \PY{n}{Y}\PY{p}{)}\PY{p}{;}
\end{Verbatim}


    \begin{center}
    \adjustimage{max size={0.9\linewidth}{0.9\paperheight}}{w4-s3-x2-taxes_files/w4-s3-x2-taxes_16_0.png}
    \end{center}
    { \hspace*{\fill} \\}
    
    \begin{Verbatim}[commandchars=\\\{\},frame=single,framerule=0.3mm,rulecolor=\color{cellframecolor}]
{\color{incolor}In [{\color{incolor}7}]:} \PY{c+c1}{\PYZsh{} et pour changer la taille de la figure}
        \PY{n}{plt}\PY{o}{.}\PY{n}{figure}\PY{p}{(}\PY{n}{figsize}\PY{o}{=}\PY{p}{(}\PY{l+m+mi}{10}\PY{p}{,} \PY{l+m+mi}{8}\PY{p}{)}\PY{p}{)}
        \PY{n}{plt}\PY{o}{.}\PY{n}{plot}\PY{p}{(}\PY{n}{X}\PY{p}{,} \PY{n}{Y}\PY{p}{)}\PY{p}{;}
\end{Verbatim}


    \begin{center}
    \adjustimage{max size={0.9\linewidth}{0.9\paperheight}}{w4-s3-x2-taxes_files/w4-s3-x2-taxes_17_0.png}
    \end{center}
    { \hspace*{\fill} \\}
    

    % Add a bibliography block to the postdoc
    
    
    
