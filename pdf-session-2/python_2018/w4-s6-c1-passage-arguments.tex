    
    
    
    

    

    \hypertarget{passage-darguments}{%
\section{Passage d'arguments}\label{passage-darguments}}

    \hypertarget{compluxe9ment---niveau-intermuxe9diaire}{%
\subsection{Complément - niveau
intermédiaire}\label{compluxe9ment---niveau-intermuxe9diaire}}

    \hypertarget{motivation}{%
\subsubsection{Motivation}\label{motivation}}

    Jusqu'ici nous avons développé le modèle simple qu'on trouve dans tous
les langages de programmation, à savoir qu'une fonction a un nombre
fixe, supposé connu, d'arguments. Ce modèle a cependant quelques
limitations~; les mécanismes de passage d'arguments que propose python,
et que nous venons de voir dans les vidéos, visent à lever ces
limitations.

Voyons de quelles limitations il s'agit.

    \hypertarget{nombre-darguments-non-connu-uxe0-lavance}{%
\subsubsection{Nombre d'arguments non connu à
l'avance}\label{nombre-darguments-non-connu-uxe0-lavance}}

    \hypertarget{ou-encore-introduction-uxe0-la-forme-arguments}{%
\subparagraph{\texorpdfstring{Ou encore~: introduction à la forme
\texttt{*arguments}}{Ou encore~: introduction à la forme *arguments}}\label{ou-encore-introduction-uxe0-la-forme-arguments}}

    Pour prendre un exemple aussi simple que possible, considérons la
fonction \texttt{print}, qui nous l'avons vu, accepte un nombre
quelconque d'arguments.

    \begin{Verbatim}[commandchars=\\\{\},frame=single,framerule=0.3mm,rulecolor=\color{cellframecolor}]
{\color{incolor}In [{\color{incolor}1}]:} \PY{n+nb}{print}\PY{p}{(}\PY{l+s+s2}{\PYZdq{}}\PY{l+s+s2}{la fonction}\PY{l+s+s2}{\PYZdq{}}\PY{p}{,} \PY{l+s+s2}{\PYZdq{}}\PY{l+s+s2}{print}\PY{l+s+s2}{\PYZdq{}}\PY{p}{,} \PY{l+s+s2}{\PYZdq{}}\PY{l+s+s2}{peut}\PY{l+s+s2}{\PYZdq{}}\PY{p}{,} \PY{l+s+s2}{\PYZdq{}}\PY{l+s+s2}{prendre}\PY{l+s+s2}{\PYZdq{}}\PY{p}{,} \PY{l+s+s2}{\PYZdq{}}\PY{l+s+s2}{plein}\PY{l+s+s2}{\PYZdq{}}\PY{p}{,} \PY{l+s+s2}{\PYZdq{}}\PY{l+s+s2}{d}\PY{l+s+s2}{\PYZsq{}}\PY{l+s+s2}{arguments}\PY{l+s+s2}{\PYZdq{}}\PY{p}{)}
\end{Verbatim}


    \begin{Verbatim}[commandchars=\\\{\},frame=single,framerule=0.3mm,rulecolor=\color{cellframecolor}]
la fonction print peut prendre plein d'arguments
\end{Verbatim}

    Imaginons maintenant que nous voulons implémenter une variante de
\texttt{print}, c'est-à-dire une fonction \texttt{error}, qui se
comporte exactement comme \texttt{print} sauf qu'elle ajoute en début de
ligne une balise \texttt{ERROR}.

Se posent alors deux problèmes~:

\begin{itemize}
\tightlist
\item
  D'une part il nous faut un moyen de spécifier que notre fonction prend
  un nombre quelconque d'arguments.
\item
  D'autre part il faut une syntaxe pour repasser tous ces arguments à la
  fonction \texttt{print}.
\end{itemize}

On peut faire tout cela avec la notation en \texttt{*} comme ceci~:

    \begin{Verbatim}[commandchars=\\\{\},frame=single,framerule=0.3mm,rulecolor=\color{cellframecolor}]
{\color{incolor}In [{\color{incolor}2}]:} \PY{c+c1}{\PYZsh{} accepter n\PYZsq{}importe quel nombre d\PYZsq{}arguments}
        \PY{k}{def} \PY{n+nf}{error}\PY{p}{(}\PY{o}{*}\PY{n}{print\PYZus{}args}\PY{p}{)}\PY{p}{:}
            \PY{c+c1}{\PYZsh{} et les repasser à l\PYZsq{}identique à print en plus de la balise}
            \PY{n+nb}{print}\PY{p}{(}\PY{l+s+s1}{\PYZsq{}}\PY{l+s+s1}{ERROR}\PY{l+s+s1}{\PYZsq{}}\PY{p}{,} \PY{o}{*}\PY{n}{print\PYZus{}args}\PY{p}{)}
        
        \PY{c+c1}{\PYZsh{} on peut alors l\PYZsq{}utiliser comme ceci}
        \PY{n}{error}\PY{p}{(}\PY{l+s+s2}{\PYZdq{}}\PY{l+s+s2}{problème}\PY{l+s+s2}{\PYZdq{}}\PY{p}{,} \PY{l+s+s2}{\PYZdq{}}\PY{l+s+s2}{dans}\PY{l+s+s2}{\PYZdq{}}\PY{p}{,} \PY{l+s+s2}{\PYZdq{}}\PY{l+s+s2}{la}\PY{l+s+s2}{\PYZdq{}}\PY{p}{,} \PY{l+s+s2}{\PYZdq{}}\PY{l+s+s2}{fonction}\PY{l+s+s2}{\PYZdq{}}\PY{p}{,} \PY{l+s+s2}{\PYZdq{}}\PY{l+s+s2}{foo}\PY{l+s+s2}{\PYZdq{}}\PY{p}{)}
        \PY{c+c1}{\PYZsh{} ou même sans argument}
        \PY{n}{error}\PY{p}{(}\PY{p}{)}
\end{Verbatim}


    \begin{Verbatim}[commandchars=\\\{\},frame=single,framerule=0.3mm,rulecolor=\color{cellframecolor}]
ERROR problème dans la fonction foo
ERROR
\end{Verbatim}

    \hypertarget{luxe9guxe8re-variation}{%
\subsubsection{Légère variation}\label{luxe9guxe8re-variation}}

    Pour sophistiquer un peu cet exemple, on veut maintenant imposer à la
fonction \texttt{erreur} qu'elle reçoive un argument obligatoire de type
entier qui représente un code d'erreur, plus à nouveau un nombre
quelconque d'arguments pour \texttt{print}.

Pour cela, on peut définir une signature (les paramètres de la fonction)
qui

\begin{itemize}
\tightlist
\item
  prévoit un argument traditionnel en première position, qui sera
  obligatoire lors de l'appel,
\item
  et le tuple des arguments pour \texttt{print}, comme ceci~:
\end{itemize}

    \begin{Verbatim}[commandchars=\\\{\},frame=single,framerule=0.3mm,rulecolor=\color{cellframecolor}]
{\color{incolor}In [{\color{incolor}3}]:} \PY{c+c1}{\PYZsh{} le premier argument est obligatoire}
        \PY{k}{def} \PY{n+nf}{error1}\PY{p}{(}\PY{n}{error\PYZus{}code}\PY{p}{,} \PY{o}{*}\PY{n}{print\PYZus{}args}\PY{p}{)}\PY{p}{:}
            \PY{n}{message} \PY{o}{=} \PY{n}{f}\PY{l+s+s2}{\PYZdq{}}\PY{l+s+s2}{message d}\PY{l+s+s2}{\PYZsq{}}\PY{l+s+s2}{erreur code }\PY{l+s+si}{\PYZob{}error\PYZus{}code\PYZcb{}}\PY{l+s+s2}{\PYZdq{}}
            \PY{n+nb}{print}\PY{p}{(}\PY{l+s+s2}{\PYZdq{}}\PY{l+s+s2}{ERROR}\PY{l+s+s2}{\PYZdq{}}\PY{p}{,} \PY{n}{message}\PY{p}{,} \PY{l+s+s1}{\PYZsq{}}\PY{l+s+s1}{\PYZhy{}\PYZhy{}}\PY{l+s+s1}{\PYZsq{}}\PY{p}{,} \PY{o}{*}\PY{n}{print\PYZus{}args}\PY{p}{)}
            
        \PY{c+c1}{\PYZsh{} que l\PYZsq{}on peut à présent appeler comme ceci}
        \PY{n}{error1}\PY{p}{(}\PY{l+m+mi}{100}\PY{p}{,} \PY{l+s+s2}{\PYZdq{}}\PY{l+s+s2}{un}\PY{l+s+s2}{\PYZdq{}}\PY{p}{,} \PY{l+s+s2}{\PYZdq{}}\PY{l+s+s2}{petit souci avec}\PY{l+s+s2}{\PYZdq{}}\PY{p}{,} \PY{p}{[}\PY{l+m+mi}{1}\PY{p}{,} \PY{l+m+mi}{2}\PY{p}{,} \PY{l+m+mi}{3}\PY{p}{]}\PY{p}{)}
\end{Verbatim}


    \begin{Verbatim}[commandchars=\\\{\},frame=single,framerule=0.3mm,rulecolor=\color{cellframecolor}]
ERROR message d'erreur code 100 -- un petit souci avec [1, 2, 3]
\end{Verbatim}

    Remarquons que maintenant la fonction \texttt{error1} ne peut plus être
appelée sans argument, puisqu'on a mentionné un paramètre
\textbf{obligatoire} \texttt{error\_code}.

    \hypertarget{ajout-de-fonctionnalituxe9s}{%
\subsubsection{Ajout de
fonctionnalités}\label{ajout-de-fonctionnalituxe9s}}

    \hypertarget{ou-encore-la-forme-argumentvaleur_par_defaut}{%
\subparagraph{\texorpdfstring{Ou encore~: la forme
\texttt{argument=valeur\_par\_defaut}}{Ou encore~: la forme argument=valeur\_par\_defaut}}\label{ou-encore-la-forme-argumentvaleur_par_defaut}}

    Nous envisageons à présent le cas - tout à fait indépendant de ce qui
précède - où vous avez écrit une librairie graphique, dans laquelle vous
exposez une fonction \texttt{ligne} définie comme suit. Évidemment pour
garder le code simple, nous imprimons seulement les coordonnées du
segment~:

    \begin{Verbatim}[commandchars=\\\{\},frame=single,framerule=0.3mm,rulecolor=\color{cellframecolor}]
{\color{incolor}In [{\color{incolor}4}]:} \PY{c+c1}{\PYZsh{} première version de l\PYZsq{}interface pour dessiner une ligne}
        \PY{k}{def} \PY{n+nf}{ligne}\PY{p}{(}\PY{n}{x1}\PY{p}{,} \PY{n}{y1}\PY{p}{,} \PY{n}{x2}\PY{p}{,} \PY{n}{y2}\PY{p}{)}\PY{p}{:}
            \PY{l+s+s2}{\PYZdq{}}\PY{l+s+s2}{dessine la ligne (x1, y1) \PYZhy{}\PYZgt{} (x2, y2)}\PY{l+s+s2}{\PYZdq{}}
            \PY{c+c1}{\PYZsh{} restons simple}
            \PY{n+nb}{print}\PY{p}{(}\PY{n}{f}\PY{l+s+s2}{\PYZdq{}}\PY{l+s+s2}{la ligne (}\PY{l+s+si}{\PYZob{}x1\PYZcb{}}\PY{l+s+s2}{, }\PY{l+s+si}{\PYZob{}y1\PYZcb{}}\PY{l+s+s2}{) \PYZhy{}\PYZgt{} (}\PY{l+s+si}{\PYZob{}x2\PYZcb{}}\PY{l+s+s2}{, }\PY{l+s+si}{\PYZob{}y2\PYZcb{}}\PY{l+s+s2}{)}\PY{l+s+s2}{\PYZdq{}}\PY{p}{)}
\end{Verbatim}


    Vous publiez cette librairie en version 1, vous avez des utilisateurs~;
et quelque temps plus tard vous écrivez une version 2 qui prend en
compte la couleur. Ce qui vous conduit à ajouter un paramètre pour
\texttt{ligne}.

    Si vous le faites en déclarant

\begin{Shaded}
\begin{Highlighting}[frame=lines,framerule=0.6mm,rulecolor=\color{asisframecolor}]
\KeywordTok{def}\NormalTok{ ligne(x1, y1, x2, y2, couleur):}
\NormalTok{    ...}
\end{Highlighting}
\end{Shaded}

alors tous les utilisateurs de la version 1 vont \textbf{devoir changer
leur code} - pour rester à fonctionnalité égale - en ajoutant un
cinquième argument \texttt{\textquotesingle{}noir\textquotesingle{}} à
leurs appels à \texttt{ligne}.

    Vous pouvez éviter cet inconvénient en définissant la deuxième version
de \texttt{ligne} comme ceci~:

    \begin{Verbatim}[commandchars=\\\{\},frame=single,framerule=0.3mm,rulecolor=\color{cellframecolor}]
{\color{incolor}In [{\color{incolor}5}]:} \PY{c+c1}{\PYZsh{} deuxième version de l\PYZsq{}interface pour dessiner une ligne}
        \PY{k}{def} \PY{n+nf}{ligne}\PY{p}{(}\PY{n}{x1}\PY{p}{,} \PY{n}{y1}\PY{p}{,} \PY{n}{x2}\PY{p}{,} \PY{n}{y2}\PY{p}{,} \PY{n}{couleur}\PY{o}{=}\PY{l+s+s2}{\PYZdq{}}\PY{l+s+s2}{noir}\PY{l+s+s2}{\PYZdq{}}\PY{p}{)}\PY{p}{:}
            \PY{l+s+s2}{\PYZdq{}}\PY{l+s+s2}{dessine la ligne (x1, y1) \PYZhy{}\PYZgt{} (x2, y2) dans la couleur spécifiée}\PY{l+s+s2}{\PYZdq{}}
            \PY{c+c1}{\PYZsh{} restons simple}
            \PY{n+nb}{print}\PY{p}{(}\PY{n}{f}\PY{l+s+s2}{\PYZdq{}}\PY{l+s+s2}{la ligne (}\PY{l+s+si}{\PYZob{}x1\PYZcb{}}\PY{l+s+s2}{, }\PY{l+s+si}{\PYZob{}y1\PYZcb{}}\PY{l+s+s2}{) \PYZhy{}\PYZgt{} (}\PY{l+s+si}{\PYZob{}x2\PYZcb{}}\PY{l+s+s2}{, }\PY{l+s+si}{\PYZob{}y2\PYZcb{}}\PY{l+s+s2}{) en }\PY{l+s+si}{\PYZob{}couleur\PYZcb{}}\PY{l+s+s2}{\PYZdq{}}\PY{p}{)}
\end{Verbatim}


    Avec cette nouvelle définition, on peut aussi bien

    \begin{Verbatim}[commandchars=\\\{\},frame=single,framerule=0.3mm,rulecolor=\color{cellframecolor}]
{\color{incolor}In [{\color{incolor}6}]:} \PY{c+c1}{\PYZsh{} faire fonctionner du vieux code sans le modifier}
        \PY{n}{ligne}\PY{p}{(}\PY{l+m+mi}{0}\PY{p}{,} \PY{l+m+mi}{0}\PY{p}{,} \PY{l+m+mi}{100}\PY{p}{,} \PY{l+m+mi}{100}\PY{p}{)}
        \PY{c+c1}{\PYZsh{} ou bien tirer profit du nouveau trait}
        \PY{n}{ligne}\PY{p}{(}\PY{l+m+mi}{0}\PY{p}{,} \PY{l+m+mi}{100}\PY{p}{,} \PY{l+m+mi}{100}\PY{p}{,} \PY{l+m+mi}{0}\PY{p}{,} \PY{l+s+s1}{\PYZsq{}}\PY{l+s+s1}{rouge}\PY{l+s+s1}{\PYZsq{}}\PY{p}{)}
\end{Verbatim}


    \begin{Verbatim}[commandchars=\\\{\},frame=single,framerule=0.3mm,rulecolor=\color{cellframecolor}]
la ligne (0, 0) -> (100, 100) en noir
la ligne (0, 100) -> (100, 0) en rouge
\end{Verbatim}

    \hypertarget{les-paramuxe8tres-par-duxe9faut-sont-truxe8s-utiles}{%
\subparagraph{Les paramètres par défaut sont très
utiles}\label{les-paramuxe8tres-par-duxe9faut-sont-truxe8s-utiles}}

    Notez bien que ce genre de situation peut tout aussi bien se produire
sans que vous ne publiiez de librairie, à l'intérieur d'une seule
application. Par exemple, vous pouvez être amené à ajouter un argument à
une fonction parce qu'elle doit faire face à de nouvelles situations
imprévues, et que vous n'avez pas le temps de modifier tout le code.

    Ou encore plus simplement, vous pouvez choisir d'utiliser ce passage de
paramètres dès le début de la conception~; une fonction \texttt{ligne}
réaliste présentera une interface qui précise les points concernés, la
couleur du trait, l'épaisseur du trait, le style du trait, le niveau de
transparence, etc. Il n'est vraiment pas utile que tous les appels à
\texttt{ligne} reprécisent tout ceci intégralement, aussi une bonne
partie de ces paramètres seront très constructivement déclarés avec une
valeur par défaut.

    \hypertarget{compluxe9ment---niveau-avancuxe9}{%
\subsection{Complément - niveau
avancé}\label{compluxe9ment---niveau-avancuxe9}}

    \hypertarget{uxe9crire-un-wrapper}{%
\subsubsection{Écrire un wrapper}\label{uxe9crire-un-wrapper}}

    \hypertarget{ou-encore-la-forme-keywords}{%
\subparagraph{\texorpdfstring{Ou encore~: la forme
\texttt{**keywords}}{Ou encore~: la forme **keywords}}\label{ou-encore-la-forme-keywords}}

    La notion de \emph{wrapper} - emballage, en anglais - est très répandue
en informatique, et consiste, à partir d'un morceau de code souche
existant (fonction ou classe) à définir une variante qui se comporte
comme la souche, mais avec quelques légères différences.

La fonction \texttt{error} était déjà un premier exemple de
\emph{wrapper}. Maintenant nous voulons définir un \emph{wrapper}
\texttt{ligne\_rouge}, qui sous-traite à la fonction \texttt{ligne} mais
toujours avec la couleur rouge.

Maintenant que l'on a injecté la notion de paramètre par défaut dans le
système de signature des fonctions, se repose la question de savoir
comment passer à l'identique les arguments de \texttt{ligne\_rouge} à
\texttt{ligne}.

    Évidemment, une première option consiste à regarder la signature de
\texttt{ligne}~:

\begin{verbatim}
def ligne(x1, y1, x2, y2, couleur="noir")
\end{verbatim}

    Et à en déduire une implémentation de \texttt{ligne\_rouge} comme ceci

    \begin{Verbatim}[commandchars=\\\{\},frame=single,framerule=0.3mm,rulecolor=\color{cellframecolor}]
{\color{incolor}In [{\color{incolor}7}]:} \PY{c+c1}{\PYZsh{} la version naïve \PYZhy{} non conseillée \PYZhy{} de ligne\PYZus{}rouge}
        \PY{k}{def} \PY{n+nf}{ligne\PYZus{}rouge}\PY{p}{(}\PY{n}{x1}\PY{p}{,} \PY{n}{y1}\PY{p}{,} \PY{n}{x2}\PY{p}{,} \PY{n}{y2}\PY{p}{)}\PY{p}{:}
            \PY{k}{return} \PY{n}{ligne}\PY{p}{(}\PY{n}{x1}\PY{p}{,} \PY{n}{y1}\PY{p}{,} \PY{n}{x2}\PY{p}{,} \PY{n}{y2}\PY{p}{,} \PY{n}{couleur}\PY{o}{=}\PY{l+s+s1}{\PYZsq{}}\PY{l+s+s1}{rouge}\PY{l+s+s1}{\PYZsq{}}\PY{p}{)}
        
        \PY{n}{ligne\PYZus{}rouge}\PY{p}{(}\PY{l+m+mi}{0}\PY{p}{,} \PY{l+m+mi}{0}\PY{p}{,} \PY{l+m+mi}{100}\PY{p}{,} \PY{l+m+mi}{100}\PY{p}{)}
\end{Verbatim}


    \begin{Verbatim}[commandchars=\\\{\},frame=single,framerule=0.3mm,rulecolor=\color{cellframecolor}]
la ligne (0, 0) -> (100, 100) en rouge
\end{Verbatim}

    Toutefois, avec cette implémentation, si la signature de \texttt{ligne}
venait à changer, on serait vraisemblablement amené à changer
\textbf{aussi} celle de \texttt{ligne\_rouge}, sauf à perdre en
fonctionnalité. Imaginons en effet que \texttt{ligne} devienne dans une
version suivante

    \begin{Verbatim}[commandchars=\\\{\},frame=single,framerule=0.3mm,rulecolor=\color{cellframecolor}]
{\color{incolor}In [{\color{incolor}8}]:} \PY{c+c1}{\PYZsh{} on ajoute encore une fonctionnalité à la fonction ligne}
        \PY{k}{def} \PY{n+nf}{ligne}\PY{p}{(}\PY{n}{x1}\PY{p}{,} \PY{n}{y1}\PY{p}{,} \PY{n}{x2}\PY{p}{,} \PY{n}{y2}\PY{p}{,} \PY{n}{couleur}\PY{o}{=}\PY{l+s+s2}{\PYZdq{}}\PY{l+s+s2}{noir}\PY{l+s+s2}{\PYZdq{}}\PY{p}{,} \PY{n}{epaisseur}\PY{o}{=}\PY{l+m+mi}{2}\PY{p}{)}\PY{p}{:}
            \PY{n+nb}{print}\PY{p}{(}\PY{n}{f}\PY{l+s+s2}{\PYZdq{}}\PY{l+s+s2}{la ligne (}\PY{l+s+si}{\PYZob{}x1\PYZcb{}}\PY{l+s+s2}{, }\PY{l+s+si}{\PYZob{}y1\PYZcb{}}\PY{l+s+s2}{) \PYZhy{}\PYZgt{} (}\PY{l+s+si}{\PYZob{}x2\PYZcb{}}\PY{l+s+s2}{, }\PY{l+s+si}{\PYZob{}y2\PYZcb{}}\PY{l+s+s2}{)}\PY{l+s+s2}{\PYZdq{}}
                  \PY{n}{f}\PY{l+s+s2}{\PYZdq{}}\PY{l+s+s2}{ en }\PY{l+s+si}{\PYZob{}couleur\PYZcb{}}\PY{l+s+s2}{ \PYZhy{} ep. }\PY{l+s+si}{\PYZob{}epaisseur\PYZcb{}}\PY{l+s+s2}{\PYZdq{}}\PY{p}{)}
\end{Verbatim}


    Alors le wrapper ne nous permet plus de profiter de la nouvelle
fonctionnalité. De manière générale, on cherche au maximum à se prémunir
contre de telles dépendances. Aussi, il est de beaucoup préférable
d'implémenter \texttt{ligne\_rouge} comme suit, où vous remarquerez que
\textbf{la seule hypothèse} faite sur \texttt{ligne} est qu'elle accepte
un argument nommé \texttt{couleur}.

    \begin{Verbatim}[commandchars=\\\{\},frame=single,framerule=0.3mm,rulecolor=\color{cellframecolor}]
{\color{incolor}In [{\color{incolor}9}]:} \PY{k}{def} \PY{n+nf}{ligne\PYZus{}rouge}\PY{p}{(}\PY{o}{*}\PY{n}{arguments}\PY{p}{,} \PY{o}{*}\PY{o}{*}\PY{n}{keywords}\PY{p}{)}\PY{p}{:}
            \PY{c+c1}{\PYZsh{} c\PYZsq{}est le seul endroit où on fait une hypothèse sur la fonction `ligne`}
            \PY{c+c1}{\PYZsh{} qui est qu\PYZsq{}elle accepte un argument nommé \PYZsq{}couleur\PYZsq{}}
            \PY{n}{keywords}\PY{p}{[}\PY{l+s+s1}{\PYZsq{}}\PY{l+s+s1}{couleur}\PY{l+s+s1}{\PYZsq{}}\PY{p}{]} \PY{o}{=} \PY{l+s+s2}{\PYZdq{}}\PY{l+s+s2}{rouge}\PY{l+s+s2}{\PYZdq{}}
            \PY{k}{return} \PY{n}{ligne}\PY{p}{(}\PY{o}{*}\PY{n}{arguments}\PY{p}{,} \PY{o}{*}\PY{o}{*}\PY{n}{keywords}\PY{p}{)}
\end{Verbatim}


    Ce qui permet maintenant de faire

    \begin{Verbatim}[commandchars=\\\{\},frame=single,framerule=0.3mm,rulecolor=\color{cellframecolor}]
{\color{incolor}In [{\color{incolor}10}]:} \PY{n}{ligne\PYZus{}rouge}\PY{p}{(}\PY{l+m+mi}{0}\PY{p}{,} \PY{l+m+mi}{100}\PY{p}{,} \PY{l+m+mi}{100}\PY{p}{,} \PY{l+m+mi}{0}\PY{p}{,} \PY{n}{epaisseur}\PY{o}{=}\PY{l+m+mi}{4}\PY{p}{)}
\end{Verbatim}


    \begin{Verbatim}[commandchars=\\\{\},frame=single,framerule=0.3mm,rulecolor=\color{cellframecolor}]
la ligne (0, 100) -> (100, 0) en rouge - ep. 4
\end{Verbatim}

    \hypertarget{pour-en-savoir-plus---la-forme-guxe9nuxe9rale}{%
\subsubsection{Pour en savoir plus - la forme
générale}\label{pour-en-savoir-plus---la-forme-guxe9nuxe9rale}}

    Une fois assimilé ce qui précède, vous avez de quoi comprendre une
énorme majorité (99\% au moins) du code Python.

Dans le cas général, il est possible de combiner les 4 formes
d'arguments~:

\begin{itemize}
\tightlist
\item
  arguments ``normaux'', dits positionnels
\item
  arguments nommés, comme \texttt{nom=\textless{}valeur\textgreater{}}
\item
  forme \texttt{*args}
\item
  forme \texttt{**dargs}
\end{itemize}

Vous pouvez
\href{https://docs.python.org/3/reference/expressions.html\#calls}{vous
reporter à cette page} pour une description détaillée de ce cas général.

    À l'appel d'une fonction, il faut résoudre les arguments, c'est-à-dire
associer une valeur à chaque paramètre formel (ceux qui apparaissent
dans le \texttt{def}) à partir des valeurs figurant dans l'appel.

L'idée est que pour faire cela, les arguments de l'appel ne sont pas
pris dans l'ordre où ils apparaissent, mais les arguments positionnels
sont utilisés en premier. La logique est que, naturellement les
arguments positionnels (ou ceux qui proviennent d'une
\texttt{*expression}) viennent sans nom, et donc ne peuvent pas être
utilisés pour résoudre des arguments nommés.

    Voici un tout petit exemple pour vous donner une idée de la complexité
de ce mécanisme lorsqu'on mélange toutes les 4 formes d'arguments à
l'appel de la fonction (alors qu'on a défini la fonction avec 4
paramètres positionnels)

    \begin{Verbatim}[commandchars=\\\{\},frame=single,framerule=0.3mm,rulecolor=\color{cellframecolor}]
{\color{incolor}In [{\color{incolor}11}]:} \PY{c+c1}{\PYZsh{} une fonction qui prend 4 paramètres simples}
         \PY{k}{def} \PY{n+nf}{foo}\PY{p}{(}\PY{n}{a}\PY{p}{,} \PY{n}{b}\PY{p}{,} \PY{n}{c}\PY{p}{,} \PY{n}{d}\PY{p}{)}\PY{p}{:}
             \PY{n+nb}{print}\PY{p}{(}\PY{n}{a}\PY{p}{,} \PY{n}{b}\PY{p}{,} \PY{n}{c}\PY{p}{,} \PY{n}{d}\PY{p}{)}
\end{Verbatim}


    \begin{Verbatim}[commandchars=\\\{\},frame=single,framerule=0.3mm,rulecolor=\color{cellframecolor}]
{\color{incolor}In [{\color{incolor}12}]:} \PY{c+c1}{\PYZsh{} on peut l\PYZsq{}appeler par exemple comme ceci}
         \PY{n}{foo}\PY{p}{(}\PY{l+m+mi}{1}\PY{p}{,} \PY{n}{c}\PY{o}{=}\PY{l+m+mi}{3}\PY{p}{,} \PY{o}{*}\PY{p}{(}\PY{l+m+mi}{2}\PY{p}{,}\PY{p}{)}\PY{p}{,} \PY{o}{*}\PY{o}{*}\PY{p}{\PYZob{}}\PY{l+s+s1}{\PYZsq{}}\PY{l+s+s1}{d}\PY{l+s+s1}{\PYZsq{}}\PY{p}{:}\PY{l+m+mi}{4}\PY{p}{\PYZcb{}}\PY{p}{)}
\end{Verbatim}


    \begin{Verbatim}[commandchars=\\\{\},frame=single,framerule=0.3mm,rulecolor=\color{cellframecolor}]
1 2 3 4
\end{Verbatim}

    \begin{Verbatim}[commandchars=\\\{\},frame=single,framerule=0.3mm,rulecolor=\color{cellframecolor}]
{\color{incolor}In [{\color{incolor}13}]:} \PY{c+c1}{\PYZsh{} mais pas comme cela}
         \PY{k}{try}\PY{p}{:}
             \PY{n}{foo} \PY{p}{(}\PY{l+m+mi}{1}\PY{p}{,} \PY{n}{b}\PY{o}{=}\PY{l+m+mi}{3}\PY{p}{,} \PY{o}{*}\PY{p}{(}\PY{l+m+mi}{2}\PY{p}{,}\PY{p}{)}\PY{p}{,} \PY{o}{*}\PY{o}{*}\PY{p}{\PYZob{}}\PY{l+s+s1}{\PYZsq{}}\PY{l+s+s1}{d}\PY{l+s+s1}{\PYZsq{}}\PY{p}{:}\PY{l+m+mi}{4}\PY{p}{\PYZcb{}}\PY{p}{)}
         \PY{k}{except} \PY{n+ne}{Exception} \PY{k}{as} \PY{n}{e}\PY{p}{:}
             \PY{n+nb}{print}\PY{p}{(}\PY{n}{f}\PY{l+s+s2}{\PYZdq{}}\PY{l+s+s2}{OOPS, }\PY{l+s+s2}{\PYZob{}}\PY{l+s+s2}{type(e)\PYZcb{}, }\PY{l+s+si}{\PYZob{}e\PYZcb{}}\PY{l+s+s2}{\PYZdq{}}\PY{p}{)}
\end{Verbatim}


    \begin{Verbatim}[commandchars=\\\{\},frame=single,framerule=0.3mm,rulecolor=\color{cellframecolor}]
OOPS, <class 'TypeError'>, foo() got multiple values for argument 'b'
\end{Verbatim}

    Si le problème ne vous semble pas clair, vous pouvez regarder la
\href{https://docs.python.org/3/reference/expressions.html\#calls}{documentation
python décrivant ce problème}.


    % Add a bibliography block to the postdoc
    
    
    
