    
    
    
    

    

    \hypertarget{cluxe9s-immuables}{%
\section{Clés immuables}\label{cluxe9s-immuables}}

    \hypertarget{compluxe9ment---niveau-intermuxe9diaire}{%
\subsection{Complément - niveau
intermédiaire}\label{compluxe9ment---niveau-intermuxe9diaire}}

    Nous avons vu comment manipuler un dictionnaire, il nous reste à voir un
peu plus en détail les contraintes qui sont mises par le langage sur ce
qui peut servir de clé dans un dictionnaire. On parle dans ce complément
spécifiquement des clefs construites à partir des types
\texttt{built-in}. Le cas de vos propres classes utilisées comme clefs
de dictionnaires n'est pas abordé dans ce complément.

    \hypertarget{une-cluxe9-doit-uxeatre-immuable}{%
\subsubsection{Une clé doit être
immuable}\label{une-cluxe9-doit-uxeatre-immuable}}

    Si vous vous souvenez de la vidéo sur les tables de hash, la mécanique
interne du dictionnaire repose sur le calcul, à partir de chaque clé,
d'une fonction de hachage.

C'est-à-dire que, pour simplifier, on localise la présence d'une clé en
calculant d'abord

\(f(clé) = hash\)

puis on poursuit la recherche en utilisant \(hash\) comme indice dans le
tableau contenant les couples (clé, valeur).

On le rappelle, c'est cette astuce qui permet de réaliser les opérations
sur les dictionnaires en temps constant - c'est-à-dire indépendamment du
nombre d'éléments.

    Cependant, pour que ce mécanisme fonctionne, il est indispensable que
\textbf{la valeur de la clé reste inchangée} pendant la durée de vie du
dictionnaire. Sinon, bien entendu, on pourrait avoir le scénario
suivant~:

\begin{itemize}
\tightlist
\item
  on range un tuple \texttt{(clef,\ valeur)} à un premier indice
  \(f(clef) = hash_1\)~;
\item
  on modifie la valeur de \(clef\) qui devient \(clef'\)~;
\item
  on recherche notre valeur à l'indice
  \(f(clef') = hash_2 \neq hash_1\).
\end{itemize}

et donc, avec ces hypothèses, on n'a plus la garantie de bon
fonctionnement de la logique.

    \hypertarget{une-cluxe9-doit-uxeatre-globalement-immuable}{%
\subsubsection{Une clé doit être globalement
immuable}\label{une-cluxe9-doit-uxeatre-globalement-immuable}}

    Nous avons depuis le début du cours longuement insisté sur le caractère
mutable ou immuable des différents types prédéfinis de Python. Vous
devez donc à présent avoir au moins en partie ce tableau en tête~:

\begin{longtable}[]{@{}ll@{}}
\toprule
Type & Mutable~?\tabularnewline
\midrule
\endhead
\texttt{int}, \texttt{float} & immuable\tabularnewline
\texttt{complex},\texttt{bool} & immuable\tabularnewline
\texttt{str} & immuable\tabularnewline
\texttt{list} & mutable\tabularnewline
\texttt{dict} & mutable\tabularnewline
\texttt{set} & mutable\tabularnewline
\texttt{frozenset} & immuable\tabularnewline
\bottomrule
\end{longtable}

    Le point important ici, est qu'il \textbf{ne suffit pas}, pour une clé,
d'être \textbf{de type immuable}.

On peut le voir sur un exemple très simple~; donnons-nous donc un
dictionnaire~:

    \begin{Verbatim}[commandchars=\\\{\}]
{\color{incolor}In [{\color{incolor}1}]:} \PY{n}{d} \PY{o}{=} \PY{p}{\PYZob{}}\PY{p}{\PYZcb{}}
\end{Verbatim}


    Et commençons avec un objet de type immuable, un tuple d'entiers~:

    \begin{Verbatim}[commandchars=\\\{\}]
{\color{incolor}In [{\color{incolor}2}]:} \PY{n}{bonne\PYZus{}cle} \PY{o}{=} \PY{p}{(}\PY{l+m+mi}{1}\PY{p}{,} \PY{l+m+mi}{2}\PY{p}{)}
\end{Verbatim}


    Cet objet est non seulement \textbf{de type immuable}, mais tous ses
composants et sous-composants sont \textbf{immuables}, on peut donc
l'utiliser comme clé dans le dictionnaire~:

    \begin{Verbatim}[commandchars=\\\{\}]
{\color{incolor}In [{\color{incolor}3}]:} \PY{n}{d}\PY{p}{[}\PY{n}{bonne\PYZus{}cle}\PY{p}{]} \PY{o}{=} \PY{l+s+s2}{\PYZdq{}}\PY{l+s+s2}{pas de probleme ici}\PY{l+s+s2}{\PYZdq{}}
        \PY{n+nb}{print}\PY{p}{(}\PY{n}{d}\PY{p}{)}
\end{Verbatim}


    \begin{Verbatim}[commandchars=\\\{\}]
\{(1, 2): 'pas de probleme ici'\}

    \end{Verbatim}

    Si à présent on essaie d'utiliser comme clé un tuple qui contient une
liste~:

    \begin{Verbatim}[commandchars=\\\{\}]
{\color{incolor}In [{\color{incolor}4}]:} \PY{n}{mauvaise\PYZus{}cle} \PY{o}{=} \PY{p}{(}\PY{l+m+mi}{1}\PY{p}{,} \PY{p}{[}\PY{l+m+mi}{1}\PY{p}{,} \PY{l+m+mi}{2}\PY{p}{]}\PY{p}{)}
\end{Verbatim}


    Il se trouve que cette clé, \textbf{bien que de type immuable}, peut
être \textbf{indirectement modifiée} puisque~:

    \begin{Verbatim}[commandchars=\\\{\}]
{\color{incolor}In [{\color{incolor}5}]:} \PY{n}{mauvaise\PYZus{}cle}\PY{p}{[}\PY{l+m+mi}{1}\PY{p}{]}\PY{o}{.}\PY{n}{append}\PY{p}{(}\PY{l+m+mi}{3}\PY{p}{)}
        \PY{n+nb}{print}\PY{p}{(}\PY{n}{mauvaise\PYZus{}cle}\PY{p}{)}
\end{Verbatim}


    \begin{Verbatim}[commandchars=\\\{\}]
(1, [1, 2, 3])

    \end{Verbatim}

    Et c'est pourquoi on ne peut pas utiliser cet objet comme clé dans le
dictionnaire~:

    \begin{Verbatim}[commandchars=\\\{\}]
{\color{incolor}In [{\color{incolor} }]:} \PY{c+c1}{\PYZsh{} NOTE}
        \PY{c+c1}{\PYZsh{} automatic execution has skipped execution of this cell}
        
        \PY{c+c1}{\PYZsh{} provoque une exception}
        \PY{n}{d}\PY{p}{[}\PY{n}{mauvaise\PYZus{}cle}\PY{p}{]} \PY{o}{=} \PY{l+s+s1}{\PYZsq{}}\PY{l+s+s1}{on ne peut pas faire ceci}\PY{l+s+s1}{\PYZsq{}}
\end{Verbatim}


    Pour conclure, il faut retenir qu'un objet n'est éligible pour être
utilisé comme clé que s'il est \textbf{composé de types immuables de
haut en bas} de la structure de données.

    La raison d'être principale du type \texttt{tuple}, que nous avons vu la
semaine passée, et du type \texttt{frozenset}, que nous verrons très
prochainement, est précisément de construire de tels objets globalement
immuables.

    \hypertarget{uxe9pilogue}{%
\subsubsection{Épilogue}\label{uxe9pilogue}}

    Tout ceci est valable pour les types \emph{built-in}. Nous verrons que
pour les types définis par l'utilisateur - les classes donc - que nous
effleurons à la fin de cette semaine et que nous étudions plus en
profondeur en semaine 6, c'est un autre mécanisme qui est utilisé pour
calculer la clé de hachage d'une instance de classe.


    % Add a bibliography block to the postdoc
    
    
    
